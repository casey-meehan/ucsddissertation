\begin{figure*}[h]
%%%
%VICREG
%%%
     % \centering
     \begin{subfigure}[b]{0.49\textwidth}
         \centering
         \includegraphics[width=0.49\textwidth]{figures/sample_level_simclr_param.pdf}
         \includegraphics[width=0.49\textwidth]{figures/linear_probe_simclr_param.pdf}
         \caption{SimCLR}
         \label{fig:simclr temperature}
     \end{subfigure}
     \hfill
     \begin{subfigure}[b]{0.49\textwidth}
         \centering
         \includegraphics[width=0.49\textwidth]{figures/sample_level_vicreg_param.pdf}
         \includegraphics[width=0.49\textwidth]{figures/linear_probe_vicreg_param.pdf}
         \caption{VICReg}
         \label{fig:vicreg temperature}
     \end{subfigure}
\caption[Effect of SSL hyperparameter on \dejavu memorization.]{
Effect of SSL hyperparameter on \dejavu memorization. The left plot of Figures \ref{fig:simclr temperature} and \ref{fig:vicreg temperature} show the size of the memorized set as a function of the temperature parameter for SimCLR and invariance parameter for VICReg, respectively. \Dejavu memorization is the highest within a narrow band of hyperparameters, and one can mitigate against \dejavu memorization by selecting hyperparameters outside of this band. Doing so has negligible effect on the quality of SSL embeddings as indicated by the linear probe accuracy on ImageNet validation set.
 }
\label{fig:sweep params}
% \vspace{-1.5em} 
\end{figure*}