\documentclass[12pt]{ucsddissertation}
% mathptmx is a Times Roman look-alike (don't use the times package)
% It isn't clear if Times is required. The OGS manual lists several
% "standard fonts" but never says they need to be used.
\usepackage{mathptmx}
\usepackage{natbib}
\usepackage[NoDate]{currvita}
\usepackage{array}
\usepackage{tabularx}
\usepackage{booktabs}
\usepackage{ragged2e}
\usepackage{microtype}
\usepackage{xcolor}
\usepackage[breaklinks=true,pdfborder={0 0 0}]{hyperref}
\usepackage{graphicx}
\usepackage{amsmath, amsfonts, amssymb}
\usepackage{dsfont}
\usepackage{nicefrac}
\usepackage{mathtools}
\usepackage[ruled, lined, linesnumbered, commentsnumbered, longend]{algorithm2e}
\usepackage{amsthm}
\usepackage{wrapfig}
\usepackage{geometry}
\usepackage{lipsum}
\usepackage{tikz}
\usetikzlibrary{bayesnet}
\usepackage{bbm}
\usepackage{subcaption}
\usepackage{yfonts}
\captionsetup[sub]{font=scriptsize,labelfont={bf}}
\usepackage[breakable, theorems, skins]{tcolorbox}
\usepackage[capitalize,noabbrev]{cleveref}
\AtBeginDocument{%
	\settowidth\cvlabelwidth{\cvlabelfont 0000--0000}%
}

\DeclareMathOperator*{\argmax}{arg\,max}
\DeclareMathOperator*{\argmin}{arg\,min}

% OGS recommends increasing the margins slightly.
\increasemargins{.1in}

% These are just for testing/examples, delete them
\usepackage{trace}
%\usepackage{showframe} % This package was just to see page margins
\usepackage[english]{babel}
\usepackage{blindtext}
\overfullrule5pt
% ---

%%For multi-file compilation 
%\usepackage{subfiles}

% Required information
\title{The Primacy of Applied Privacy}
\author{Casey Meehan}
\degree{Computer Science}{Doctor of Philosophy}
% Each member of the committee should be listed as Professor Foo Bar.
% If Professor is not the correct title for one, then titles should be
% omitted entirely.
\chair{Professor Kamalika Chaudhuri}
% Your committee members (other than the chairs) must be in alphabetical order
\committee{Professor Taylor Berg-Kirkpatrick}
\committee{Professor Sanjoy Dasgupta}
\committee{Professor Alon Orlitsky}
\degreeyear{2023}

% Start the document
\begin{document}
% Begin with frontmatter and so forth
\frontmatter
\maketitle
\makecopyright
\makesignature
% Optional
\begin{dedication}
\setsinglespacing
\raggedright % It would be better to use \RaggedRight from ragged2e
\parindent0pt\parskip\baselineskip
The fact that I have made something I can write a dedication for is owed all to my parents. I cannot imagine following my heart these past few years without their unrelenting support and encouragement. 
\end{dedication}

%% Optional
%\begin{epigraph}
%\vskip0pt plus.5fil
%\setsinglespacing
%%{\flushright
%%True ease in writing comes from art, not chance,\\
%%As those move easiest who have learn'd to dance.\\
%%'T is not enough to no harshness gives offence,---\\
%%The sound must seem an echo to the sense.
%%
%%\vskip\baselineskip
%%\textit{Alexander Pope}\par}
%\vfil
%\begin{center}
%Something pithy
%
%\vskip\baselineskip
%\textit{Someone smart}
%\end{center}
%\vfil
%%\noindent Writing, at its best, is a lonely life. Organizations for
%%writers palliate the writer's loneliness, but I doubt if they improve
%%his writing. He grows in public stature as he sheds his loneliness and
%%often his work deteriorates. For he does his work alone and if he is a
%%good enough writer he must face eternity, or the lack of it, each day.
%%
%%\vskip\baselineskip
%%\hskip0pt plus1fil\textit{Ernest Hemingway}\hskip0pt plus4fil\null
%
%%\vfil
%\end{epigraph}

% Next comes the table of contents, list of figures, list of tables,
% etc. If you have code listings, you can use \listoflistings (or
% \lstlistoflistings) to have it be produced here as well. Same with
% \listofalgorithms.
\tableofcontents
\listoffigures
\listoftables

%% Preface
%\begin{preface}
%Almost nothing is said in the manual about the preface. There is no
%indication about how it is to be typeset. Given that, one is forced to
%simply typeset it and hope it is accepted. It is, however, optional
%and may be omitted.
%\end{preface}

% Your fancy acks here. Keep in mind you need to ack each paper you
% use. See the examples here. In addition, each chapter ack needs to
% be repeated at the end of the relevant chapter.
\begin{acknowledgements}
Above all, I thank my advisor, Kamalika, for her mentorship over the last five years. Kamalika will do anything in her power to help her students find what success means to them, and build towards it. She puts her students' futures above all else. Part of that is mentoring her students to become thoughtful researchers. In hindsight, I see that we could have pursued much lower risk projects that made incremental gains on the hottest research directions. Instead, Kamalika guided me towards unusual and challenging problems that required me to examine my fundamentals and reconsider broad questions about the aims of data privacy. In doing so, I not only understand what solutions are effective in data privacy, but why we as a research community use them and --- in some cases --- why we have ignored them. 

I cannot overemphasize the gratitude I have to my collaborators as well. First, to my collaborators at FAIR. I thank Chuan Guo for showing me how to navigate incredibly challenging and open-ended ML problems. His Socratic approach to mentoring helped me explore my own instincts and take ownership of the project without letting me slip down the wrong path. Florian Bordes taught me the ins and outs of contemporary vision modeling. Pascal Vincent's insights on how to design deep learning experiments were truly formative; I sincerely appreciate his effort and time. I owe an enormous debt of gratitude to Ashwin Machanavajjhala at Tumult for mentoring me on how how to reason about different privacy definitions and guarantees in applied settings. 

Finally, to all of my friends. To the lab: I could not have landed in a better group. To be surrounded by such wonderful and bright companions is a win --- for them to be your colleagues as well is a gift. To all my dear pre-PhD friends (you know who you are, and you really don't have read this dissertation): you are the loves of my life! 

\pagebreak

\textbf{Chapter 1}, in full, has been submitted for publication of the material
as it may appear in Neural Information Processing Systems,~2023, Casey Meehan, Florian Bordes, Pascal Vincent, Kamalika Chaudhuri, Chuan Guo. \emph{Do SSL models have Déjà Vu? A Case of Unintended Memorization in Self-Supervised Learning}. The dissertation author shares equal contribution with Florian Bordes. 

\textbf{Chapter 2}, in full, is a reprint of the material as it appears in International Conference on Artificial Intelligence and Statistics, 2020. Casey Meehan, Sanjoy Dasgupta, Kamalika Chaudhuri. \emph{A Non-parametric Test to Detect Data-Copying in Generative Models}. The dissertation author is the primary investigator and author of this paper. 

\textbf{Chapter 3}, in full, is a reprint of the material as it appears in Proceedings of the 60th Annual Meeting of the Association for Computational Linguistics (Volume 1: Long Papers), 2022. Casey Meehan, Khalil Mrini, Kamalika Chaudhuri. \emph{Sentence-level Privacy for Document Embeddings}. The dissertation author is the primary investigator and author of this paper. 

\textbf{Chapter 4}, in full, is a reprint of the material as it appears in International Conference on Learning Representations. 2022. Casey Meehan, Amrita Roy-Chowdhury, Kamalika Chaudhuri, Somesh Jha. \emph{Privacy Implications of Shuffling}. The dissertation author is the primary investigator and author of this paper.

\textbf{Chapter 5}, in full, is a reprint of the material as it appears in International Conference on Artificial Intelligence and Statistics, 2021. Casey Meehan, Kamalika Chaudhuri. \emph{Location Trace Privacy Under Conditional Priors}. The dissertation author is the primary investigator and author of this paper. 
\end{acknowledgements}

% Stupid vita goes next
\begin{vita}
\noindent
\begin{cv}{}
\begin{cvlist}{}
\item[2015] Bachelor of Science, Brown University
\item[2018] Master of Science, Harvard University
\item[2023] Doctor of Philosophy, University of California, San Diego
\end{cvlist}
\end{cv}

%% This puts in the PUBLICATIONS header. Note that it appears inside
%% the vita environment. It is optional.
%\publications
%\noindent``Distributions of Control Points in a System for Analysis of Stress
%Distribution'' IRE Transactions of the I.R.E\@. Professional Group on
%Automatic Control, vol. AC-7, pp 272--289, September 2005

%% This puts in the FIELDS OF STUDY. Also inside vita and also
%% optional.
%\fieldsofstudy
%\noindent Major Field: Engineering (Specialization or Focused Studies)
%\vskip\baselineskip
%Studies in Applied Mathematics\par
%Professors Alpha Beta and Gamma Delta
%\vskip\baselineskip
%Studies in Mechanices\par
%Professors Epsilon Zeta and Eta Theta
%\vskip\baselineskip
%Studies in Electromagnetism\par
%Professors Iota Kappa and Lambda Mu
\end{vita}

% Put your maximum 350 word abstract here.
\begin{dissertationabstract}
%The Abstract begins here. The abstract is limited to 350 words for a
%doctoral dissertation. It should consist of a short statement of the
%problem, a brief explanation of the methods and procedures employed in
%generating the data, and a condensed summary of the findings of the
%study. The abstract may continute onto a second page if necessary. The
%text of the abstract must be double spaced.
As data collection for machine learning (ML) tasks has become more pervasive, it has also become more heterogeneous: we share our writing, images, voices, and location online every day. Naturally, the associated privacy risks are just as complex and variable. My research advances practical data privacy through two avenues: 1) drafting provable privacy definitions and mechanisms for safely sharing data in different ML domains, and 2) empirically quantifying how ML models memorize their sensitive training data and thereby risk disclosing it. This dissertation details the various data domains/tasks considered, and the corresponding privacy methods proposed. 
\end{dissertationabstract}

% This is where the main body of your dissertation goes!
\mainmatter

% Optional Introduction
\begin{dissertationintroduction}
In machine learning (ML) we learn broad trends and patterns from vast sums of data. What type of data is collected and how it is used introduces different kinds of privacy risks. Medical data allows us to link characteristics like genetic markers to adverse health conditions like cancer. However it suffers from unique correlation risks, since one's information can reliably be correlated from their familys'. Document embeddings---useful vector representations of text documents---are tremendously useful for inferring general document characteristics like topic, but can expose detailed personal information at the sentence level. Generative vision models can sample novel images, but also tend to copy and expose their training images in clever and inexact ways. 

The privacy risks and utility requirements of each of these settings and applications warrant different approaches to privacy-preserving ML. This dissertation proposes a variety of solutions to situations like those above. In doing so, I hope to illuminate the advantage of taking an application-specific approach to both measuring privacy risks and engineering private algorithms. The following five chapters are roughly organized into two parts: the first two chapters cover \emph{empirical} privacy methods, and the remaining three cover \emph{formal} privacy methods. The former includes statistical tests to quantify privacy risks of large ML models. The latter proposes provably private algorithms to satisfy different privacy definitions chosen for different ML tasks. 

\textbf{Empirical methods.} The first two chapters explore empirically measuring privacy risks in the domain of vision modeling. In both chapters, we analyze to what extent large vision models \emph{memorize} their training images, and thereby risk exposing them. In contrast with the following three chapters, we are not proposing a provably privacy-preserving algorithm. Instead we are designing methodical empirical tests to quantify memorization. 

\textbf{Provably private algorithms.} The final three chapters propose algorithms that allow us to share our data in a provably private way. We explore privacy preserving algorithms in the text, location, and interpersonal-correlated domains (\emph{e.g.} social networks or genetically linked medical data). For each of these, we study different ML tasks and privacy risks to motivate different privacy definitions and provably private algorithms. 

Taken together, this document offers a mindset towards practicable privacy methods. By directly considering the data domain and the task at hand, it is possible to efficiently measure privacy risks and propose provably private algorithms while still completing the learning task at hand .  
\end{dissertationintroduction}

%\chapter{An ordinary page}
%The purpose of this page is to illustrate an ordinary page of text in
%a doctoral dissertation or master's thesis. All pages of the doctoral
%dissertation or master's thesis must be kept within the margins of
%1.5'' on the left, 1'' on the right, 1'' on the top and 1.25'' on the
%bottom. All text must be double spaced except as indicated below.
%
%It is recommended that to increase the margins as paper can shift in a
%printer and as some photocopiers tend to increase the image being
%copied.
%
%The first line of each paragraph must be indented at least one 0.5''
%tab, as done here.
%
%This text is intended to be a part of the dissertation, for a doctoral
%student, or the thesis if you are receiving a master's degree, and now
%a quote is included here:
%\begin{quote}
%All quotes of more than six lines, even though this one is not, are to
%be indented 0.5'' from the left and 0.5'' from the right. These longer
%quotes are to be single spaced. Don't forget to adjust for proper
%spacing after the last line of the quoted material.
%\end{quote}
%The rest of the paragraph would continue as so.

\graphicspath{{./chapters/chapter1}}
\chapter{Do SSL Models Have Déjà Vu? A Case of Unintended Memorization in Self-supervised Learning} 


\newcommand{\crop}[1]{\mathrm{crop}({#1})}
\newcommand{\object}[1]{\mathrm{object}({#1})}
\newcommand{\ba}{A_i}
\newcommand{\bb}{B_i}
\newcommand{\calA}{\mathcal{A}}
\newcommand{\calB}{\mathcal{B}}
\newcommand{\calX}{\mathcal{X}}
\newcommand{\masked}[1]{\mathrm{masked}({#1})}
\newcommand{\bx}{\mathbf{x}}
\newcommand{\SSL}{\textsc{SSL}}
\newcommand{\SSLbb}{\SSL^\mathrm{back}}
\newcommand{\SSLpj}{\SSL^\mathrm{proj}}
\newcommand{\CLF}{\textsc{CLF}}
\newcommand{\CLFbb}{\CLF^\mathrm{back}}
\newcommand{\CLFpj}{\CLF^\mathrm{proj}}
\newcommand{\SUP}{\textsc{SUP}}
\newcommand{\KNN}{\textsc{KNN}}
\newcommand{\KNNset}{\textsc{KNN}^\mathrm{set}}
\newcommand{\KNNprob}{\textsc{KNN}^\mathrm{prob}}
\newcommand{\KNNcl}{\textsc{KNN}^\mathrm{cl}}
\newcommand{\KNNconf}{\textsc{KNN}^\mathrm{conf}}
\newcommand{\RCDM}{\textsc{RCDM}}
\newcommand{\cl}{\mathrm{cl}}
\newcommand{\clpred}{\tilde{\mathrm{cl}}}
\newcommand{\Abox}{\overline{\calA}}
\newcommand{\Bbox}{\overline{\calB}}
\newcommand{\dejavu}{\emph{déjà vu }}
\newcommand{\Dejavu}{\emph{Déjà vu }}

\newcommand{\citations}{{\color{green}[CITE]}}

\definecolor{part_blue}{rgb}{0.2824, 0.4706, .8157}
\definecolor{part_red}{rgb}{0.8392, 0.3725, 0.3725}
\definecolor{part_orange}{rgb}{0.9333, 0.5216, 0.2902}

\DeclareRobustCommand{\mybox}[2][gray!20]{%
\begin{tcolorbox}[   %% Adjust the following parameters at will.
        % breakable,
        left=0pt,
        right=0pt,
        top=0pt,
        bottom=0pt,
        colback=#1,
        colframe=#1,
        width=\dimexpr\columnwidth\relax, 
        % width=\textwidth, 
        enlarge left by=0mm,
        boxsep=5pt,
        arc=0pt,outer arc=0pt,
        ]
        #2
\end{tcolorbox}
}
\section{Introduction}
\label{sec:intro}
Self-supervised learning (SSL)~\citep{chen2020simclr, chen2020simsiam, zbontar2021barlow, vicreg, caron2020swav, MAE} aims to learn general representations of content-rich data without explicit labels by solving a \textit{pretext task}. In many recent works, such pretext tasks rely on joint-embedding architectures whereby randomized image augmentations are applied to create multiple views of a training sample, and the model is trained to produce similar representations for those views. When using cropping as random image augmentation, the model learns to associate objects or parts (including the background scenery) that co-occur in an image.
However, doing so also arguably exposes the training data to higher privacy risk as objects in training images can be explicitly memorized by the SSL model. For example, if the training data contains the photos of individuals, the SSL model may learn to associate the face of a person with their activity or physical location in the photo. This may allow an adversary to extract such information from the trained model for targeted individuals.

\begin{figure}[t]
    \centering
    \includegraphics[width=1.0\columnwidth]{figures/new_black_swan.pdf}
    \caption{\textbf{Left:} Reconstruction of an SSL training image from a crop containing only the background. The SSL model memorizes the association of this \emph{specific} patch of water (pink square) to this \emph{specific} foreground object (a black swan) in its embedding, which we decode to visualize the full training image. \textbf{Right:} The reconstruction technique fails on a public test image that the SSL model has not seen before.}
    \label{fig:black_swan}
\end{figure}

In this work, we aim to evaluate to what extent SSL models memorize the association of specific objects in training images or the association of objects and their specific backgrounds, and whether this memorization signal can be used to reconstruct the model's training samples. Our results demonstrate that SSL models memorize such associations beyond simple correlation. For instance, in Figure \ref{fig:black_swan} (\textbf{left}), we use the SSL representation of a \emph{training image crop containing only water} and this enables us to reconstruct the object in the foreground with remarkable specificity---in this case a black swan.
By contrast, in Figure \ref{fig:black_swan} (\textbf{right}), when using the \emph{crop from the background of a test set image} that the SSL model \emph{has not seen before}, its representation only contains enough information to infer, through correlation, that the foreground object was likely some kind of waterbird --- but not the specific one in the image.

Figure \ref{fig:black_swan} shows that SSL models suffer from the unintended memorization of images in their training data---a phenomenon we refer to as \emph{déjà vu memorization}
%\footnote{The French loanword \emph{déjà vu} means already-seen, which reflects the type of unintended memorization of objects that the SSL model saw during training.}.
\footnote{The French loanword \emph{déjà vu} means `already-seen', just as an image is seen and memorized in training.}
Beyond visualizing \emph{déjà vu} memorization through data reconstruction, we also design a series of experiments to quantify the degree of memorization for different SSL algorithms, model architectures, training set size, \emph{etc.} We observe that \emph{déjà vu} memorization is exacerbated by the atypically large number of training epochs often recommended in SSL training, as well as certain hyperparameters in the SSL training objective. Perhaps surprisingly, we show that \emph{déjà vu} memorization occurs even when the training set is large---as large as half of ImageNet~\citep{imagenet}---and can continually worsen even when standard techniques for evaluating learned representation quality (such as linear probing) do not suggest increased overfitting. Our work serves as the first systematic study of unintended memorization in SSL models and motivates future work on understanding and preventing this behavior. Specifically, we: 
\begin{itemize}
    \vspace{-0.5em}
    \item Elucidate how SSL representations memorize aspects of individual training images, what we call \emph{déjà vu} memorization;
    \item Design a novel training data reconstruction pipeline for non-generative vision models. This is in contrast to many prominent reconstruction algorithms like \citep{carlini2021extracting, google_diffusion}, which rely on the model itself to generate its own memorized samples and is not possible for SSL models or classifiers;
    \item Propose metrics to quantify the degree of \dejavu memorization committed by an SSL model. This allows us to observe how \dejavu changes with training epochs, dataset size, training criteria, model architecture and more. 
\end{itemize}

\section{Preliminaries and Related Work}
\label{sec:related}

\textbf{Self-supervised learning} (SSL) is a machine learning paradigm that leverages unlabeled data to learn representations. Many SSL algorithms rely on \emph{joint-embedding} architectures (\emph{e.g.}, SimCLR~\citep{chen2020simclr}, Barlow Twins~\citep{zbontar2021barlow}, VICReg~\citep{vicreg} and Dino~\citep{Dino}), which are trained to associate different augmented views of a given image. For example, in SimCLR, given a set of images $\calA = \{A_1,\ldots,A_n\}$ and a randomized augmentation function $\mathrm{aug}$, the model is trained to maximize the cosine similarity of draws of $\SSL(\mathrm{aug}(A_i))$ with each other and minimize their similarity with $\SSL(\mathrm{aug}(A_j))$ for $i \neq j$. The augmentation function $\mathrm{aug}$ typically consists of operations such as cropping, horizontal flipping, and color transformations to create different views that preserve an image's semantic properties. 

\paragraph{SSL representations.} Once an SSL model is trained, its learned representation can be transferred to different downstream tasks. This is often done by extracting the representation of an image from the \emph{backbone model}\footnote{SSL methods often use a trick called \emph{guillotine regularization}~\citep{Guillotine}, which decomposes the model into two parts: a \emph{backbone model} and a \emph{projector} consisting of a few fully-connected layers. Such trick is needed to handle the misalignment between the pretext SSL task and the downstream task.} and either training a linear probe on top of this representation or finetuning the backbone model with a task-specific head~\citep{Guillotine}.
%Compared to representations learned by supervised learning, SSL representations are often more robust and transferable~\citep{hendrycks2019using, ericsson2021self}, leading to state-of-the-art result on many downstream tasks. To understand the effectiveness of SSL algorithms, several prior works investigated what kind of information the SSL model has learned~\citep{jing2021understanding, ericsson2021self, kalibhat2022towards, RCDM}. In particular, \citet{RCDM} trained a conditional generative model on SSL representations and showed that they encode richer visual details about the input image compared to supervised learning. 
%However, from a privacy perspective, this may be a cause for concern as the model also has more potential to overfit and memorize precise details about the training data compared to supervised learning. We show concretely that this privacy risk can indeed be realized by defining and measuring \emph{déjà vu} memorization.
It has been shown that SSL representations encode richer visual details about input images than supervised models do \cite{RCDM}. However, from a privacy perspective, this may be a cause for concern as the model also has more potential to overfit and memorize precise details about the training data compared to supervised learning. We show concretely that this privacy risk can indeed be realized by defining and measuring \emph{déjà vu} memorization.
\vspace{-0.5em} 
% \paragraph{Privacy risks in ML.} Overfitting in ML occurs when a model memorizes information specific to its training data rather than general population-level information. When the model is trained on privacy-sensitive data, overfitting is especially harmful as an adversary can infer private information about the training data when given access to the model~\citep{yeom2018privacy, feldman2020does}. The simplest and most well-studied form of privacy risk in ML is susceptibility to \emph{membership inference attacks}~\citep{shokri2017membership, salem2018ml, sablayrolles2019white}, where the adversary infers whether an individual is part of the training set or not. More sophisticated privacy attacks include \emph{attribute inference}~\citep{fredrikson2014privacy, mehnaz2022your, jayaraman2022attribute}, where specific attributes about an individual are inferred given others, and \emph{data reconstruction}~\citep{carlini2021extracting, balle2022reconstructing, guo2022bounding}, where entire training samples are recovered from the trained model. Our study of \emph{déjà vu} memorization is similar to both attribute inference and data reconstruction, leveraging SSL representations of the training image background to infer and reconstruct the foreground object.
% \vspace{-0.5em} 
% \paragraph{Training data extraction in NLP.} Our study of \dejavu memorization in SSL models is inspired by similar work in the natural language processing (NLP) domain. \citet{carlini2019secret} first showed that language models exhibit unintended memorization, where given a context string present in its training data, the model can generate the remaining text at test time. This unintended memorization has been further exploited in \citet{carlini2021extracting} to extract training data from GPT-2~\citep{radford2019language} and, more recently, extended to extract memorized images from Stable Diffusion \citep{google_diffusion}. The way by which these works exploit unintended memorization is similar to ours: given partial information about a training sample, the model is prompted to reveal the rest of the sample. In our case, however, since the SSL model is not generative, extraction is significantly harder and requires careful design.

\paragraph{Privacy risks in ML.} When a model is overfit on privacy-sensitive data, it memorizes specific information about its training examples, allowing an adversary with access to the model to learn private information~\citep{yeom2018privacy, feldman2020does}. Privacy attacks in ML range from the simplest and best-studied \emph{membership inference attacks}~\citep{shokri2017membership, salem2018ml, sablayrolles2019white} to \emph{attribute inference}~\citep{fredrikson2014privacy, mehnaz2022your, jayaraman2022attribute} and \emph{data reconstruction}~\citep{carlini2021extracting, balle2022reconstructing, guo2022bounding} attacks. In the former, the adversary only infers whether an individual participated in the training set. Our study of \emph{déjà vu} memorization is most similar to the latter: we leverage SSL representations of the training image background to infer and reconstruct the foreground object. Our approach reflects similar work in the NLP domain \citep{carlini2019secret, carlini2021extracting}: when prompted with a context string present in the training data, a large language model is shown to generate the remainder of string at test time, revealing sensitive text like home addresses. This method was recently extended to extract memorized images from Stable Diffusion \citep{google_diffusion}.  We exploit memorization in a similar manner: given partial information about a training sample, the model is prompted to reveal the rest of the sample. In our case, however, since the SSL model is not generative, extraction is significantly harder and requires careful design.

\section{Defining \emph{Déjà Vu} Memorization}
\label{sec:definition}

\paragraph{What is \dejavu memorization?} At a high level, the objective of SSL is to learn general representations of objects that occur in nature. This is often accomplished by associating different parts of an image with one another in the learned embedding. Returning to our example in Figure \ref{fig:black_swan}, given an image whose background contains a patch of water, the model may learn that the foreground object is a water animal such as duck, pelican, otter, \emph{etc.}, by observing different images that contain water from the training set. We refer to this type of learning as \emph{correlation}: the association of objects that tend to co-occur in images from the training data distribution.

A natural question to ask is \emph{``Can the reconstruction of the black swan in Figure \ref{fig:black_swan} be reasoned as correlation?''} The intuitive answer may be no, since the reconstructed image is qualitatively very similar to the original image. However, this reasoning implicitly assumes that for a random image from the training data distribution containing a patch of water, the foreground object is unlikely to be a black swan. Mathematically, if we denote by $\mathcal{P}$ the training data distribution and $A$ the image, then
\begin{equation*}
\label{eq:p_corr}
p_\text{corr} := \mathbb{P}_{A \sim \mathcal{P}}(\mathrm{object}(A) = \texttt{black swan} ~|~ \mathrm{crop}(A) = \texttt{water})
\end{equation*}
is the probability of inferring that the foreground object is a black swan through \emph{correlation}. This probability may be naturally high due to biases in the distribution $\mathcal{P}$, \emph{e.g.}, if $\mathcal{P}$ contains no other water animal except for black swans. In fact, such correlations are often exploited to learn a model for image inpainting with great success~\citep{yu2018generative, ulyanov2018deep}.

Despite this, we argue that reconstruction of the black swan in Figure \ref{fig:black_swan} is \emph{not} due to correlation, but rather due to \emph{unintended memorization}: the association of objects unique to a single training image. As we will show in the following sections, the example in Figure \ref{fig:black_swan} is not a rare success case and can be replicated across many training samples. More importantly, failure to reconstruct the foreground object in Figure \ref{fig:black_swan} (\textbf{right}) on test images hints at inferring through correlation is unlikely to succeed---a fact that we verify quantitatively in Section \ref{sec:label inference accuracy}. Motivated by this discussion, we give a verbal definition of \dejavu memorization below, and design a testing methodology to quantify \dejavu memorization in Section \ref{sec:notation and setup}.
\mybox{\textbf{Definition:} A model exhibits \emph{déjà vu memorization} when it retains information so specific to an individual training image, that it enables recovery of aspects particular to that image given a part that does not contain them.
The recovered aspect must be beyond what can be inferred using only correlations in the data distribution.} 

% \textbf{Definition:} A model exhibits \emph{déjà vu memorization} when it retains information so specific to an individual training image, that it enables recovery of aspects particular to that image given a part that does not contain them.
% The recovered aspect must be beyond what can be inferred using only correlations in the data distribution.


 We intentionally kept the above definition broad enough to encompass different types of information that can be inferred about the training image, including but not restricted to object category, shape, color and position. For example, if one can infer that the foreground object is red given the background patch with accuracy significantly beyond correlation, we consider this an instance of \dejavu memorization as well. We mainly focus on object category to quantify \dejavu memorization in Section \ref{sec:quant} since the ground truth label can be easily obtained. We consider other types of information more qualitatively in the visual reconstruction experiments in Section \ref{sec:visualizing}.

\paragraph{Privacy implications of \dejavu memorization.} \Dejavu memorization can be a cause for concern when the training data contains privacy-sensitive information. As a motivating example, consider an SSL model trained on photos of individuals. If the model exhibits \dejavu memorization then, given the face of an individual, it may be possible to infer where the individual was or even visually reconstruct their location in the training image. Such information leakage raises privacy concerns, especially if there was no prior agreement that the trained model may reveal such information to third parties. This hypothetical scenario serves as a motivation that \dejavu memorization should be carefully examined to avoid unintended disclosure of private information in practical applications.

% \begin{figure*}[h]
%     \centering
%     \includegraphics[width = 0.85\textwidth]{figures/SSL_attack_cartoon.png}
%     \caption{We measure memorization by comparing the `target model' trained on the target image ($\SSL_A$ trained on $A_i$ in above example) with the `reference model' not trained on it ($\SSL_B$, above). \textbf{[Top Strip]} A cropping of the image disjoint from the labeled foreground object is embedded using the target model. This embedding is then labeled by a K-Nearest Neighbor (KNN) adversary built on a public set of labeled images, $X$, which it has also embedded using the target model. \textbf{[Bottom Strip]} To account for correlation, the same procedure is followed with the reference model. If the label is only extracted using the target model, it is counted as memorization. If it is extracted using either model, it is counted as correlation. We find that the KNN adversary's predictions using the target model (trained on attacked examples) are significantly more accurate than they are using the reference model, indicating routine memorization of training examples.}
%     \label{fig:ssl attack cartoon}
% \end{figure*}

\begin{figure}[t]
%%%
%SPIDER
%%%
     % \centering
     % \begin{subfigure}[b]{0.25\textwidth}
     %     \centering
     %     \includegraphics[width=\textwidth]{figures/data_split.png}
     %     % \caption{SimCLR correlated \textit{yellow garden spider} examples}
     %     \label{fig:data split}
     % \end{subfigure}
     % \hfill
     % \begin{subfigure}[b]{0.7\textwidth}
     %     \centering
     %     \includegraphics[width=\textwidth]{figures/pipeline_cartoon.png}
     %     \begin{minipage}{5cm}
     %        \vfill
     %    \end{minipage}
     %     % \caption{SimCLR memorized \textit{yellow garden spider} examples}
     %     \label{fig:pipeline cartoon}
     % \end{subfigure}
     \includegraphics[width=\textwidth]{figures/split_and_pipeline_cartoon.png}
\caption[Overview of testing methodology.]{
Overview of testing methodology. \textbf{Left:} Data is split into \emph{target set} $\calA$, \emph{reference set} $\calB$ and \emph{public set} $\calX$ that are pairwise disjoint. $\calA$ and $\calB$ are used to train two SSL models $\SSL_A$ and $\SSL_B$ in the same manner. $\calX$ is used for KNN decoding or for training an RCDM to reconstruct the input at test time. \textbf{Right:} Given a training image $A_i \in \calA$, we use $\SSL_A$ to embed $\crop{A_i}$ containing only the background, as well as the entire set $\calX$ and find the $k$-nearest neighbors of $\crop{A_i}$ in $\calX$ in the embedding space. These KNN samples can be used directly to infer the foreground object (\emph{i.e.}, class label) in $A_i$ using a KNN classifier, or their embeddings can be averaged as input to the trained RCDM to visually reconstruct the image $A_i$. For instance, the RCDM reconstruction results in Figure \ref{fig:black_swan} (left) when given $\SSL_A(\crop{A_i})$ and results in Figure \ref{fig:black_swan} (right) when given $\SSL_A(\crop{B_i})$ for an image $B_i \in \calB$.
%\textbf{Left:} illustration of the three datasets used in our tests. Two private data sets, $A$ and $B$, of equal size are used to train two SSL models, $\SSL_A$ and $\SSL_B$, respectively. A disjoint public set, $X$, is made available to the memorization test to help decode model embeddings. Memorization is only tested on examples $A_i \in A$ that are unique to set $A$. \textbf{Right:} illustration of inference pipeline used in tests. A periphery cropping that excludes the foreground object is taken from private image $A_i$. The KNN then finds the $k$ public set nearest neighbors of the periphery crop in the embedding space of $\SSL_A$. 
%The $\SSL_A$ representation of these $k$ neighbors and of the crop are used by the conditional generative model, RCDM, to reconstruct the foreground object. The labels of these $k$ neighbors are used to recover the foreground object label. (Not pictured) We repeat this process using reference model $\SSL_B$, not trained on image $A_i$, to determine whether the foreground object is still recoverable by learned correlations, e.g. if black swans were the only objects appearing near water in the data distribution. In this instance, the crop's public set neighbors in $\SSL_B$'s representation space include a variety of water animals like ducks, pelicans, and otters. Meanwhile, with $\SSL_A$, the neighbors are nearly all black swans in the same position as the swan of $A_i$.
}
\label{fig:split_and_pipeline_cartoon}
\end{figure}

\textbf{Distinguishing memorization from correlation.} When measuring \dejavu memorization, it is crucial to differentiate what the model associates through \emph{memorization} and what it associates through \emph{correlation}. Our testing methodology is based on the following intuitive definition.
\mybox{\textbf{Definition:} If an SSL model associates two parts in a training image, we say that it is due to \emph{correlation} if other SSL models trained on a similar dataset from $\mathcal{P}$ without this image would likely make the same association. Otherwise, we say that it is due to \emph{memorization}.}

Notably, such intuition forms the basis for differential privacy (DP; \cite{dwork2006calibrating, dwork2013algorithmic})---the most widely accepted notion of privacy in ML.

\subsection{Testing Methodology for Measuring \emph{Déjà Vu} Memorization}
\label{sec:notation and setup}

In this section, we use the above intuition to measure the extent of \dejavu memorization in SSL. Figure \ref{fig:split_and_pipeline_cartoon} gives an overview of our testing methodology.
\vspace{-0.75em}
\paragraph{Dataset splitting.} We focus on testing \dejavu memorization for SSL models trained on the ImageNet-1K dataset~\citep{imagenet}. Our test first splits the ImageNet training set into three independent and disjoint subsets $\calA$, $\calB$ and $\calX$. The dataset $\calA$ is called the \emph{target set} and $\calB$ is called the \emph{reference set}. The two datasets are used to train two separate SSL models, $\SSL_A$ and $\SSL_B$, called the \emph{target model} and the \emph{reference model}. Finally, the dataset set $\calX$ is used as an auxiliary public dataset to extract information from $\SSL_A$ and $\SSL_B$.
%\footnote{See Appendix \ref{sec:appx splits} for details on how the dataset splits are generated.}.
Our dataset splitting serves the purpose of distinguishing memorization from correlation in the following manner. Given a sample $A_i \in \calA$, if our test returns the same result on $\SSL_A$ and $\SSL_B$ then it is likely due to correlation because $A_i$ is not a training sample for $\SSL_B$. Otherwise, because $\calA$ and $\calB$ are drawn from the same underlying distribution, our test must have inferred some information unique to $A_i$ due to memorization. Thus, by comparing the difference in the test results for $\SSL_A$ and $\SSL_B$, we can measure the degree of \dejavu memorization\footnote{See Appendix \ref{sec:appx splits} for details on how the dataset splits are generated.}.
\vspace{-0.75em}
\paragraph{Extracting foreground and background crops.} Our testing methodology aims at measuring what can be inferred about the foreground object in an ImageNet sample given a background crop. This is made possible because ImageNet provides bounding box annotations for a subset of its training images---around 150K out of 1.3M samples. We split these annotated images equally between $\calA$ and $\calB$. Given an annotated image $A_i$, we treat everything inside the bounding box as the foreground object associated with the image label, denoted $\object{A_i}$. We take the largest possible crop that does not intersect with any bounding box as the background crop (or \emph{periphery crop}), denoted $\crop{A_i}$\footnote{We also present another heuristic in \cref{sec:appx corner crop} which takes a corner crop as the background crop, allowing our test to be run without bounding box annotations.}
%Since the labeled object tends to be at the image's center, the corner crop usually excludes it. }
%Because most images in ImageNet are object centric, an image's corner would not include the foreground object.}.
\vspace{-0.75em}
\paragraph{KNN-based test design.} Joint-embedding SSL approaches encourage the embeddings of random crops of a training image $A_i \in \calA$ to be similar. Intuitively, if the model exhibits \dejavu memorization, it is reasonable to expect that the embedding of $\crop{A_i}$ is similar to that of $\object{A_i}$ since both crops are from the same training image. In other words, $\SSL_A(\crop{A_i})$ encodes information about $\object{A_i}$ that cannot be inferred through correlation. However, decoding such information is challenging as these approaches do not learn a decoder associated with the encoder $\SSL_A$.

Here, we leverage the public set $\calX$ to decode the information contained in $\crop{A_i}$ about $\object{A_i}$. More specifically, we map images in $\calX$ to their embeddings using $\SSL_A$ and extract the $k$-nearest-neighbor (KNN) subset of $\SSL_A(\crop{A_i})$ in $\calX$. We can then decode the information contained in $\crop{A_i}$ in one of two ways:
\begin{itemize}
\item \emph{Label inference:} Since $\calX$ is a subset of ImageNet, each embedding in the KNN subset is associated with a class label. If $\crop{A_i}$ encodes information about the foreground object, its embedding will be close to samples in $\calX$ that have the same class label (\emph{i.e.}, foreground object category). We can then use a KNN classifier to infer the foreground object in $A_i$ given $\crop{A_i}$.
\item \emph{Visual reconstruction:} Following \citet{RCDM}, we train an RCDM---a conditional generative model---on $\calX$ to decode $\SSL_A$ embeddings into images. The RCDM reconstruction can recover qualitative aspects of an image remarkably well, such as recovering object color or spatial orientation using its SSL embedding. Given the KNN subset, we average their SSL embeddings and use the trained RCDM model to visually reconstruct $A_i$.
\end{itemize}
In Section \ref{sec:quant}, we focus on quantitatively measuring \dejavu memorization with label inference, and then use the RCDM reconstruction to visualize \dejavu memorization in Section \ref{sec:visualizing}.
\section{Quantifying \emph{Déjà Vu} Memorization}
\label{sec:quant}

We apply our testing methodology to quantify a specific form of \dejavu memorization: inferring the foreground object (class label) given a crop of the background.

% \paragraph{Extracting model embeddings.} We test \dejavu memorization on two popular SSL algorithms, SimCLR~\citep{chen2020simclr} and VICReg~\citep{vicreg}.
% %\footnote{We present additional SSL models in \cref{sec:appx simclr results}} 
% As described in Section \ref{sec:related}, these algorithms produce two embeddings given an input image: a \emph{backbone} embedding and a \emph{projector} embedding that is derived by applying a small fully-connected network on top of the backbone embedding. Unless otherwise noted, all SSL embeddings refer to the projector embedding.
% To understand whether \dejavu memorization is particular to SSL, we also evaluate embeddings produced by a supervised model $\CLF_A$ trained on $\calA$. We apply the same set of image augmentations as those used in SSL and train $\CLF_A$ using the cross-entropy loss to predict ground truth labels. 
\vspace{-0.75em}
\paragraph{Extracting model embeddings.} We test \dejavu memorization on a variety of popular SSL algorithms, with a focus on VICReg~\citep{vicreg}. These algorithms produce two embeddings given an input image: a \emph{backbone} embedding and a \emph{projector} embedding that is derived by applying a small fully-connected network on top of the backbone embedding. Unless otherwise noted, all SSL embeddings refer to the projector embedding. 
To understand whether \dejavu memorization is particular to SSL, we also evaluate embeddings produced by a supervised model $\CLF_A$ trained on $\calA$. We apply the same set of image augmentations as those used in SSL and train $\CLF_A$ using the cross-entropy loss to predict ground truth labels. 
\vspace{-0.75em}
\paragraph{Identifying the most memorized samples.} Prior works have shown that certain training samples can be identified as more prone to memorization than others~\citep{feldman2020does, watson2021importance, ye2021enhanced}. Similarly, we provide a heuristic to identify the most memorized samples in our label inference test using confidence of the KNN prediction.
Given a periphery crop, $\crop{A_i}$, let $\KNN_A \big( \crop{A_i} \big) \subseteq \calX$ denote its $k$-nearest neighbors in the embedding space of $\SSL_A$. From this KNN subset we can obtain: \textbf{(1)} $\KNNprob_A \big( \crop{A_i} \big)$, the vector of class probabilities (normalized counts) induced by the KNN subset, and \textbf{(2)} $\KNNconf_A \big( \crop{A_i} \big)$, the negative entropy of the probability vector $\KNNprob_A \big( \crop{A_i} \big)$, as confidence of the KNN prediction. When entropy is low, the neighbors agree on the class of $A_i$ and hence confidence is high. 
% \begin{itemize}[noitemsep, leftmargin=*, topsep=0pt]
%     \item $\KNN_A \big( \crop{A_i} \big)$: The most prevalent class in the KNN subset as prediction for the class label $\cl(A_i)$. 
%     \item $\KNNprob_A \big( \crop{A_i} \big)$: The vector of class probabilities (normalized counts) induced by the KNN subset.
%     \item $\KNNconf_A \big( \crop{A_i} \big)$: Negative entropy of the probability vector $\KNNprob_A \big( \crop{A_i} \big)$ as confidence of the KNN prediction. When entropy is low, the neighbors agree on the class of $A_i$ and hence confidence is high. 
% \end{itemize}
We can sort the confidence score $\KNNconf_A \big( \crop{A_i} \big)$ across samples $A_i$ in decreasing order to identify the most confidently predicted samples, which likely correspond to the most memorized samples when $A_i \in \calA$.

\subsection{Population-level Memorization}
\label{sec:label inference accuracy}

%ORIGINAL FIGURE SETUP IN ARXIV: 
% \input{dejavu_training_epochs.tex}
% \input{dejavu_training_set_size.tex}
%PUT ORIGINAL FIGURES SIDE BY SIDE: 
% \input{dejavu_training_epochs_set_size.tex}
%PUT IN NEW FIGURES: 

\begin{wrapfigure}{r}{0.4\textwidth} 
    \centering
    \includegraphics[width=0.4\textwidth]{figures/dejavu_main.pdf}
    \caption{Accuracy of label inference using the target model (trained on $\calA$) vs. the reference model (trained on $\calB$) on the top $\%$ most confident examples $A_i \in \calA$ using only $\crop{A_i}$. For VICReg, there is a large accuracy gap between the two models, indicating a significant degree of \dejavu memorization.}
    \label{fig:dejavu main}
    \vspace{-2ex}
\end{wrapfigure}

Our first measure of \dejavu memorization is population-level label inference accuracy: \emph{What is the average label inference accuracy over a subset of SSL training images given their periphery crops?} 
To understand how much of this accuracy is due to $\SSL_A$'s \dejavu memorization, we compare with a correlation baseline using the reference model: $\KNN_B$'s label inference accuracy on images $A_i \in \calA$. 
In principle, this inference accuracy should be significantly above chance level ($1/1000$ for ImageNet) because the periphery crop may be highly indicative of the foreground object through correlation, \emph{e.g.}, if the periphery crop is a basketball player then the foreground object is likely a basketball.

Figure \ref{fig:dejavu main} compares the accuracy of $\KNN_A$ to that of $\KNN_B$ when inferring the labels of images in $A_i \in \calA$\footnote{The sets $\calA$ and $\calB$ are exchangeable, and in practice we repeat this test on images from $\calB$ using $\SSL_B$ as the target model and $\SSL_A$ as the reference model, and average the two sets of results.} using $\crop{A_i}$.
Results are shown for VICReg and the supervised model; trends for other models are shown in Appendix \ref{sec:appx simclr results}. For both VICReg and supervised models, inferring the class of $\crop{A_i}$ using $\KNN_B$ (dashed line) through correlation achieves a reasonable accuracy that is significantly above chance level. However, for VICReg, the inference accuracy using $\KNN_A$ (solid red line) is significantly higher, and the accuracy gap between $\KNN_A$ and $\KNN_B$ indicates the degree of \dejavu memorization. We highlight two observations: 
\begin{itemize}
    \item The accuracy gap of VICReg is significantly larger than that of the supervised model. This is especially notable when accounting for the fact that the supervised model is trained to associate randomly augmented crops of images with their ground truth labels. In contrast, VICReg has no label access during training but the embedding of a periphery crop can still encode the image label. 
    \item For VICReg, inference accuracy on the $1\%$ most confident examples is nearly $95\%$, which shows that our simple confidence heuristic can effectively identify the most memorized samples. This result suggests that an adversary can use this heuristic to identify vulnerable training samples to launch a more focused privacy attack.
\end{itemize}
\vspace{-.75em}
\paragraph{The \dejavu score. }
The curves of Figure \ref{fig:dejavu main} show memorization across confidence values for a single training scenario.  To study how memorization changes with different hyperparamters, we extract a single value from these curves: the \dejavu \emph{score} at confidence level $p$. In Figure \ref{fig:dejavu main}, this is the gap between the solid red (or gray) and dashed red (or gray) where confidence ($x$-axis) equal $p\%$. In other words, given the periphery crops of set $\calA$, $\KNN_A$ and $\KNN_B$ separately select and label their top $p\%$ most confident examples, and we report the difference in their accuracy. The \dejavu score captures both the degree of memorization by the accuracy gap and the \emph{ability to identify memorized examples} by the confidence level. If the score is 10\% for $p=33\%$, $\KNN_A$ has 10\% higher accuracy on its most confident third of $\calA$ than $\KNN_B$ does on its most confident third. In the following, we set $p = 20\%$, approximately the largest gap for VICReg (red lines) in Figure \ref{fig:dejavu main}. 
% Specifically, the \dejavu \emph{score} on the top $p\%$ most confident examples is,  
% \begin{equation}
%     \mathrm{DejaVu}(p) = \mathrm{Acc}_{\SSL_A}\big( \calA_{\SSL_A, p}  \big) - \mathrm{Acc}_{\SSL_B}\big( \calA_{\SSL_B, p}  \big) \ ,
%     \label{eqn:dejavu score}
% \end{equation}
% where $\calA_{\SSL_A, p}$
% Here we introduce a DejaVu memorization metric that quantify how much a target model is able to retrieve more class information from a crop than the reference model. We define it as:
% where $p$ is a function that take the $p$ purcent most confident samples.
%Figure \ref{fig:dejavu v. training epochs} shows how \dejavu memorization changes with the number of epochs used to train the embedding model (VICReg and supervised, respectively). The training set size is fixed to 300K samples, and label inference accuracy is computed on the top $20\%$ highest confidence examples. The number of epochs has a very strong influence on the degree of memorization for VICReg as the accuracy gap widens when number of epochs increases. We note that 1000 training epochs is used in several SSL works \citep{vicreg, simclr}. Remarkably, this trend in memorization is \emph{not} reflected in the standard metric for evaluating SSL representations: linear probe accuracy. The gray line in Figure \ref{fig:dejavu v. training epochs} shows the train-test accuracy gap of a linear classifier trained on top of the VICReg embeddings. Although there is a sizeable train-test gap, it does not grow significantly beyond 500 epochs. In contrast, \dejavu memorization (blue line) continues to worsen after 500 epochs. Thus, our test can be used as an alternative to linear probe accuracy to evaluate the memorization of SSL models.
% \vspace{-.75em}

% \paragraph{Comparison with the generalization gap} A network that perform very well on a training set while performing poorly on a test set (assuming the training set and test set sampled uniformly from the same distribution) is probably memorizing the training examples without being able to generalize on the test data. One could expect that measuring the difference in accuracy between the training and test set could give us insights on the degree of \dejavu memorization. However, we show in Figure  \ref{fig:dejavu v. training epochs} and \ref{fig:dejavu v. n} that this is not the case. In fact \dejavu memorization can significantly increase while the train-test gap decrease. In our experiments, we did not find a correlation between \dejavu and generalization.
\vspace{-0.75em}
\paragraph{Comparison with the linear probe train-test gap.} A standard method for measuring SSL performance is to train a linear classifier---what we call a `linear probe'---on its embeddings and compute its performance on a held out test set. From a learning theory standpoint, one might expect the linear probe's train-test accuracy gap to be indicative of memorization: the more a model overfits, the larger is the difference between train set and test set accuracy. However, as seen in Figure \ref{fig:dejavu epochs train set size}, the linear probe gap (dark blue) fails to reveal memorization captured by the \dejavu score (red) \footnote{See section \ref{sec:mitigation} for further discussion of the \dejavu score trends of Figure \ref{fig:dejavu epochs train set size}.}.

% \paragraph{Effect of training epochs.} 
% Figure \ref{fig:dejavu v. training epochs} shows how \dejavu memorization changes with training epochs for VICReg. The training set size is fixed to 300K samples. We observe that the number of epochs has a very strong influence on the degree of memorization for VICReg. From 250 to 1000 epochs, the \dejavu score (red curve) grows threefold: from under 10\% to over 30\%. Remarkably, this trend in memorization is \emph{not} reflected in the standard metric for evaluating SSL representations: linear probe accuracy. The dark blue curve shows the train-test linear probe accuracy gap. Although there is a sizeable train-test gap, it only changes by a few percent beyond 250 epochs. %Thus, our test can be used as an alternative to linear probe accuracy to evaluate the memorization of SSL models.
% \vspace{-.75em}
\begin{figure}[ht]
\label{fig:dejavu epochs and dataset}
\begin{minipage}[t]{0.49\textwidth}
\centering
     \begin{subfigure}[b]{0.48\textwidth}
         \centering
         \includegraphics[width=\textwidth]{figures/deja_vu_vs_epochs.png}
         \vspace{-1.5em}
         \caption{\dejavu vs. epochs}
         \label{fig:dejavu v. training epochs}
     \end{subfigure}
     \begin{subfigure}[b]{0.48\textwidth}
         \centering
         \includegraphics[width=\textwidth]{figures/deja_vu_vs_n.png}
         \vspace{-1.5em}
         \caption{\dejavu vs. train set size}
         \label{fig:dejavu v. n}
     \end{subfigure}~
     \vspace{-0.5em}
    \caption{
    Effect of training epochs and train set size with VICReg on \dejavu score (red) in comparison with linear probe accuracy train-test gap (dark blue). 
    \textbf{Left:} \dejavu score increases with training epochs, indicating growing memorization while the linear probe baseline decreases significantly.  
    \textbf{Right:} \dejavu score stays roughly constant with training set size suggesting that memorization may be problematic even for large datasets. %By comparison, the baseline \emph{declines} by half, spuriously suggesting less memorization. 
    %Both trends are not captured according to the linear probe train-test gap---a common method to evaluate generalization of SSL representations.}
    }
    \label{fig:dejavu epochs train set size}
\end{minipage}
\hfill
\begin{minipage}[t]{0.49\textwidth}
\centering
     \begin{subfigure}[b]{0.48\textwidth}
         \centering
         \includegraphics[width=\textwidth]{figures/vicreg_samples_epochs.pdf}
         \vspace{-1.5em}
         \caption{\dejavu vs. epochs}
         \label{fig:per sample v. training epochs}
     \end{subfigure}
     \begin{subfigure}[b]{0.48\textwidth}
         \centering
         \includegraphics[width=\textwidth]{figures/vicreg_samples_datasets.pdf}
         \vspace{-1.5em}
         \caption{\dejavu vs. train set size}
         \label{fig:per sample v. n}
     \end{subfigure}~
     \vspace{-0.5em}
    \caption{
    \definecolor{part_blue}{rgb}{0.2824, 0.4706, .8157}
	\definecolor{part_red}{rgb}{0.8392, 0.3725, 0.3725}
	\definecolor{part_orange}{rgb}{0.9333, 0.5216, 0.2902}
    Partition of samples $A_i \in \calA$ into the four categories: {\color{gray}unassociated} (not shown), {\color{part_orange}memorized}, {\color{part_red}misrepresented} and {\color{part_blue}correlated} for VICReg. The {\color{part_orange}memorized} samples---those whose labels are predicted by $\KNN_A$ but not by $\KNN_B$---occupy a significantly larger share of the training set than the {\color{part_red}misrepresented} samples---those predicted by $\KNN_B$ but not $\KNN_A$ by chance. %At 1000 epochs, $\approx 15\%$ of the training set is {\color{part_orange}memorized}. The trends across training epochs and training set sizes are consistent with those observed in Figure \ref{fig:dejavu epochs train set size}
    }
    \label{fig:partition attack main}
    \end{minipage}
\vspace{-1em} 
\end{figure}

\iffalse

\begin{minipage}[t]{0.49\textwidth}
\centering
     \begin{subfigure}[b]{0.48\textwidth}
         \centering
         \includegraphics[width=0.95\textwidth]{figures/deja_vu_vs_parameters.png}
         \vspace{-0.4em}
         \caption{\dejavu vs. capacity}
         \label{fig:dejavu v. capacity}
     \end{subfigure}
     \hfill
     \begin{subfigure}[b]{0.48\textwidth}
          \tiny
          \centering
          \setlength{\tabcolsep}{3pt}
          \begin{tabular}{|c|c|c|}
            \hline
            Criteria & DV & Acc P/B \\
            \hline
            Supervised & 8.9 & 55.3/61.1\\
            \hline
            Byol\citep{grill2020byol} & 8.0& 54.3/59.4\\
            \hline
            SimCLR\citep{chen2020simclr} & 10.0 & 44.2/54.1\\
            \hline
            Dino\citep{Dino} & 14.5 & 26.3/55.7 \\
            \hline
            Barlow T.\citep{zbontar2021barlow} & 30.5 & 33.7/54.4\\
            \hline
            VICReg\citep{vicreg} & \textbf{33.2} & 40.3/55.2\\
            \hline
          \end{tabular}
          \vspace{1.3em}
          % \caption{\dejavu (DV) vs. SSL Criterion}
          \caption{\dejavu (DV) vs. Criterion}
          \label{tab:dejavu vs. criterion}
    \end{subfigure}
    \vspace{-0.5em}
    \caption{
    Comparison of \dejavu score for different architectures and training criteria. \textbf{Left:} \dejavu score with VICReg for resnet (purple) and vision transformer (green) architectures versus number of model parameters. As expected, memorization grows with larger model capacity. This trend is more pronounced for convolutional (resnet) than transformer (ViT) architectures. \textbf{Right:} Comparison of \dejavu score and ImageNet validation accuracy (P: using projector embeddings, B: using backbone embeddings) for various SSL criteria. \textbf{Nearly all SSL models have more memorization than the supervised baseline.} 
    % Effect of training epochs and train set size on \dejavu score.
    % \textbf{Left:} \dejavu score increases with higher number of training epochs, indicating worsening memorization.
    % \textbf{Right:} \dejavu score stays roughly constant with training set size. Both trends are not captured according to the linear probe train-test gap---a common method to evaluate generalization of SSL representations.
    }
\end{minipage}
\vspace{-2em} 
\end{figure}

\begin{figure}[ht]
\begin{minipage}[t]{0.49\textwidth}
\centering
     \begin{subfigure}[b]{0.49\textwidth}
         \centering
         \includegraphics[width=\textwidth]{figures/epochs_lb_attk_epochs_acc_top1_legend.pdf}
         \caption{\dejavu vs. epochs}
         \label{fig:dejavu v. training epochs}
     \end{subfigure}
     \begin{subfigure}[b]{0.49\textwidth}
         \centering
         \includegraphics[width=\textwidth]{figures/epochs_lb_attk_datasets_acc_top1_legend.pdf}
         \caption{\dejavu vs. train set size}
         \label{fig:dejavu v. n}
     \end{subfigure}~
     \begin{subfigure}[b]{0.32\textwidth}
         \centering
         \includegraphics[width=0.8\textwidth]{figures/dejavu_vs_parameters.pdf}
         \caption{\dejavu vs. capacity}
         \label{fig:dejavu v. n}
     \end{subfigure}
    \caption{
    Effect of training epochs and train set size on \dejavu score.
    \textbf{Left:} \dejavu score increases with higher number of training epochs, indicating worsening memorization.
    \textbf{Right:} \dejavu score stays roughly constant with training set size. Both trends are not captured according to the linear probe train-test gap---a common method to evaluate generalization of SSL representations.}
    \end{minipage}
\vspace{-1em} 
\end{figure}

\begin{table}[ht]
  \footnotesize
  \centering
  \begin{tabular}{|c|c|}
    \hline
    Supervised & 8.9\\
    \hline
    SimCLR\citep{chen2020simclr} & 10.0\\
    \hline
    Byol\citep{grill2020byol} & 8.0\\
    \hline
    Dino\citep{Dino} & 14.5\\
    \hline
    Barlow T.\citep{zbontar2021barlow} & 30.5\\
    \hline
    VICReg\citep{vicreg} & \textbf{33.2}\\
    \hline
  \end{tabular}
  \caption{DejaVu Score 20\% Conf for various SSL methods.}
  \label{tab:two-row-table}
\end{table}
\vspace{-1em} 
\fi

\iffalse
\begin{figure}[ht]
\begin{minipage}[t]{.49\textwidth}
\centering
     \begin{subfigure}[b]{0.49\textwidth}
         \centering
         \includegraphics[width=\textwidth]{figures/epochs_lb_attk_epochs_acc_top1_legend.pdf}
         \caption{\dejavu vs. epochs}
         \label{fig:dejavu v. training epochs}
     \end{subfigure}
     \hfill
     \begin{subfigure}[b]{0.49\textwidth}
         \centering
         \includegraphics[width=\textwidth]{figures/epochs_lb_attk_datasets_acc_top1_legend.pdf}
         \caption{\dejavu vs. train set size}
         \label{fig:dejavu v. n}
     \end{subfigure}
\caption{
Effect of training epochs and train set size on \dejavu score.
\textbf{Left:} \dejavu score increases with higher number of training epochs, indicating worsening memorization.
\textbf{Right:} \dejavu score stays roughly constant with training set size. Both trends are not captured according to the linear probe train-test gap---a common method to evaluate generalization of SSL representations.}
\label{fig:dejavu epochs and dataset}
\end{minipage}
\hfill
\begin{minipage}[t]{.49\textwidth}
     \centering
     \begin{subfigure}[b]{0.49\textwidth}
         \centering
         \includegraphics[width=\textwidth]{figures/criteria_epochs.pdf}
         \caption{criteria comparison}
         \label{fig:dejavu v. criteria}
     \end{subfigure}
     \hfill
     \begin{subfigure}[b]{0.49\textwidth}
         \centering
         \includegraphics[width=\textwidth]{figures/architecture_epochs.pdf}
         \caption{architecture comparison}
         \label{fig:dejavu v. arch}
     \end{subfigure}
\caption{
Effect of SSL training criteria and model architectures on \dejavu score.
%the accuracy gap between target model (trained on $\calA$) and reference model (trained on $\calB$) making predictions on their 20\% most confident examples.
\textbf{Left:} \dejavu score for various training criteria.
%Barlow and VICReg have the heaviest degree of memorization, while SimCLR and BYOL have the least. 
%Note that we show detailed reconstructions of SimCLR's training data in Section \ref{sec:visualizing} despite its relatively low degree of \dejavu. 
%Regardless, Although SimCLR and BYOL have the least, we  visualize detailed reconstructions with SimCLR in section \ref{sec:mem v corr} 
All SSL models have significantly more \dejavu than the supervised baseline. \textbf{Right:} \dejavu score versus epochs for various training architectures. As expected, lower capacity architectures (Resnet18, Resnet34) reduce \dejavu but not completely. 
}
\label{fig:dejavu criteria and architecture}
\end{minipage}
\vspace{-1em} 
\end{figure}
\fi
% %\begin{figure}[ht]
%%%
%VICREG
%%%
     \centering
     \begin{subfigure}[b]{0.49\textwidth}
         \centering
         \includegraphics[width=\textwidth]{figures/sample_level_training_epochs.pdf}
         \caption{Categories of training samples vs. number of epochs}
         \label{fig:sample level epochs}
     \end{subfigure}
     \hfill
     \begin{subfigure}[b]{0.49\textwidth}
         \centering
         \includegraphics[width=\textwidth]{figures/sample_level_training_set_size.pdf}
         \caption{Categories of training samples vs. training set size}
         \label{fig:sample level training size}
     \end{subfigure}
\caption{
\definecolor{part_blue}{rgb}{0.2824, 0.4706, .8157}
\definecolor{part_red}{rgb}{0.8392, 0.3725, 0.3725}
\definecolor{part_orange}{rgb}{0.9333, 0.5216, 0.2902}
Partition of samples $A_i \in \calA$ into the four categories: {\color{gray}unassociated} (not shown), {\color{part_orange}memorized}, {\color{part_red}misrepresented} and {\color{part_blue}correlated}. The {\color{part_orange}memorized} samples---ones whose labels are predicted by $\KNN_A$ but not by $\KNN_B$---occupy a significantly larger share for VICReg compared to the supervised model, indicating that sample-level \dejavu memorization is more prevalent in VICReg. %The trends across number of training epochs and training set sizes are consistent with those observed in Figures \ref{fig:dejavu epochs and dataset} and \ref{fig:dejavu criteria and architecture}.
}
\label{fig:partition attack main appendix}
\end{figure}
% \paragraph{Effect of training set size.} 
% Figure \ref{fig:dejavu v. n} shows how \dejavu memorization responds to the model's training set size. The number of training epochs is fixed to 1000. Interestingly, training set size appears to have almost \emph{no} influence on the \dejavu score (red line), indicating that memorization is equally prevalent with a 100K dataset and a 500K dataset (which suggests that \dejavu memorization may be detectable for larger datasets). Meanwhile, the linear probe train-test accuracy gap \emph{declines} by half as the dataset size grows, failing to represent the memorization quantified by our test. 
% The trend is completely different according to linear probe accuracy (dark blue line), the train-test gap shrinks substantially when increasing the training set size from 100K to 500K. This highlights that the train-test gap is not able to capture \dejavu memorization. %Our evidence suggests that \dejavu memorization may be detectable even for large-scale training datasets. 
%\vspace{-.75em}

\vspace{-.75em} 
\subsection{Sample-level Memorization}
\label{sec:dissection}

% Section \ref{sec:label inference accuracy} shows the \emph{average} level of \dejavu memorization on a subset of the training set $\calA$. However, this average tell us only what the attacker success rate might be without explicitly describing how much of the datatset is \dejavu memorized.
The \dejavu score shows, \emph{on average}, how much better an adversary can select and classify images when using the target model trained on them. 
This average score does not tell us how many individual images have their label successfully recovered by $\KNN_A$ but not by $\KNN_B$. In other words, how many images are exposed by virtue of \emph{being in training set} $\calA$: a risk notion foundational to differential privacy. 
% However, from the perspective of an individual image $A_i \in \calA$, it is informative to know whether it was correctly classified 
To better quantify what fraction of the dataset is at risk, we perform a sample-level analysis by fixing a sample $A_i \in \calA$ and observing the label inference result of $\KNN_A$ vs. $\KNN_B$.
To this end, we partition samples $A_i \in \calA$ based on the result of label inference into four distinct categories: {\color{gray}\textbf{Unassociated}} - label inferred with neither KNN; {\color{part_orange}\textbf{Memorized}} - label inferred only with $\KNN_A$; {\color{part_red}\textbf{Misrepresented}} - label inferred only with $\KNN_B$; {\color{part_blue}\textbf{Correlated}} - label inferred with both KNNs. 
% \begin{multicols}{2}
% \begin{itemize}
%     \vspace{-.75em}
%     \setlength\itemsep{0.15em}
%     \item {\color{gray}Unassociated}: label inferred with neither KNN   
%     \item {\color{part_orange}Memorized}: label only inferred by $\KNN_A$
%     \item {\color{part_red}Misrepresented}: label only inferred with $\KNN_B$
%     \item {\color{part_blue}Correlated}: label inferred with both KNNs
%     \vspace{-.75em}
% \end{itemize}
% \end{multicols}
Intuitively, {\color{gray}unassociated} samples are ones where the embedding of $\crop{A_i}$ does not encode information about the label. {\color{part_blue}Correlated} samples are ones where the label can be inferred from $\crop{A_i}$ using correlation, \emph{e.g.}, inferring the foreground object is basketball given a crop showing a basketball player. Ideally, the {\color{part_red}misrepresented} set should be empty but contains a small portion of examples due to chance.
\emph{Déjà vu} memorization occurs for {\color{part_orange}memorized} samples where the embedding of $\SSL_B$ does not encode the label but the embedding of $\SSL_A$ does. To measure the pervasiveness of \dejavu memorization, we compare the size of the {\color{part_orange}memorized} and {\color{part_red}misrepresented} sets.
Figure \ref{fig:partition attack main} shows how the four categories of examples change with number of training epochs and training set size. The {\color{gray}unassociated} set is not shown since the total share adds up to one. The {\color{part_red}misrepresented} set remains under $5\%$ and roughly unchanged across all settings, consistent with our explanation that it is due to chance. In comparison, VICReg's {\color{part_orange}memorized} set surpasses $15\%$ at 1000 epochs. Considering that up to 5\% of these memorized examples could also be due to chance, we conclude that \textbf{at least 10\% of VICReg's training set is \dejavu memorized.} 
%is many times larger than its {\color{part_red}misrepresented} set, indicating substantial sample-level \dejavu memorization. 
%In fact, \textbf{it is 15\% of the training set that is \dejavu memorized with VICReg.}
%The trends across different number of training epochs and training set sizes match those observed in Section \ref{sec:label inference accuracy}. % On the other hand, the supervised model's {\color{part_orange}memorized} set is only marginally larger than its {\color{part_red}misrepresented} set.

% The trends across different number of training epochs and training set sizes match those observed in Section \ref{sec:label inference accuracy}: Increasing the number of epochs increases \dejavu memorization (Figure \ref{fig:per sample v. training epochs}), while increasing the training set size does not appear to reduce \dejavu memorization (Figure \ref{fig:per sample v. n}). 
\section{Visualizing \emph{Déjà Vu} Memorization}
\label{sec:visualizing}
Beyond enabling label inference using a periphery crop, we show that \dejavu memorization allows the SSL model to encode other forms of information about a training image. Namely, we train an RCDM \citep{RCDM} on the public dataset $\calX$ and use it to visually reconstruct training images given their periphery crop.
We aim to answer the following two questions: \textbf{(1)} Can we visualize the distinction between correlation and \dejavu memorization? \textbf{(2)} What foreground object details can be extracted from the SSL model beyond class label? 
% \begin{enumerate}[noitemsep, leftmargin=*, topsep=0pt]
%     \item Can we visualize the distinction between correlation and \dejavu memorization? 
%     \item What foreground object details can be extracted from the SSL model beyond class label? 
% \end{enumerate}
\vspace{-0.5em}
\paragraph{Reconstruction pipeline.}
RCDM is a conditional generative model that is trained on the \emph{backbone embedding} of images $X_i \in \calX$ to generate an image that resembles $X_i$. All training images are first face-blurred for privacy purposes. \citet{RCDM} showed that the backbone embedding of SSL models contains more low-level information about the image, making them better suited for conditioning the RCDM.
At test time, following the pipeline in Figure \ref{fig:split_and_pipeline_cartoon}, we first use the projector embedding to find the KNN subset for the periphery crop, $\crop{A_i}$, and then average their backbone embeddings as input to the RCDM model. Ideally, when the public set contains enough representative images, the average representation of the KNN subset encodes objects present in $A_i$, and the RCDM model decodes this representation to visualize these objects.
% \begin{figure}[ht]
%%%
%VICREG
%%%
     \centering
     \begin{subfigure}[b]{0.49\textwidth}
         \centering
         \includegraphics[width=\textwidth]{figures/sample_level_training_epochs.pdf}
         \caption{Categories of training samples vs. number of epochs}
         \label{fig:sample level epochs}
     \end{subfigure}
     \hfill
     \begin{subfigure}[b]{0.49\textwidth}
         \centering
         \includegraphics[width=\textwidth]{figures/sample_level_training_set_size.pdf}
         \caption{Categories of training samples vs. training set size}
         \label{fig:sample level training size}
     \end{subfigure}
\caption{
\definecolor{part_blue}{rgb}{0.2824, 0.4706, .8157}
\definecolor{part_red}{rgb}{0.8392, 0.3725, 0.3725}
\definecolor{part_orange}{rgb}{0.9333, 0.5216, 0.2902}
Partition of samples $A_i \in \calA$ into the four categories: {\color{gray}unassociated} (not shown), {\color{part_orange}memorized}, {\color{part_red}misrepresented} and {\color{part_blue}correlated}. The {\color{part_orange}memorized} samples---ones whose labels are predicted by $\KNN_A$ but not by $\KNN_B$---occupy a significantly larger share for VICReg compared to the supervised model, indicating that sample-level \dejavu memorization is more prevalent in VICReg. %The trends across number of training epochs and training set sizes are consistent with those observed in Figures \ref{fig:dejavu epochs and dataset} and \ref{fig:dejavu criteria and architecture}.
}
\label{fig:partition attack main appendix}
\end{figure}
%\begin{figure*}[t!]
%%%
%DAM
%%%
     \centering
     \begin{subfigure}[b]{0.49\textwidth}
         \centering
         \includegraphics[width=\textwidth]{figures/dam_corr.png}
         \caption{A {\color{part_blue}correlated} dam example}
         \label{fig:dam correlated}
     \end{subfigure}
     \hfill
     \begin{subfigure}[b]{0.49\textwidth}
         \centering
         \includegraphics[width=\textwidth]{figures/dam_mem.png}
         \caption{A {\color{part_orange}memorized} dam example}
         \label{fig:dam memorized}
     \end{subfigure}
\caption{
{\color{part_blue}Correlated} and {\color{part_orange}Memorized} examples from the \emph{dam} class. Both $\SSL_A$ and $\SSL_B$ are SimCLR models.
\textbf{Left:} The periphery crop (pink square) contains a concrete structure that is often present in images of dams. Consequently, the trained RCDM can reconstruct the foreground object using representations from both $\SSL_A$ and $\SSL_B$ through this correlation.
\textbf{Right:} The periphery crop only contains a patch of water. The embedding produced by $\SSL_B$ only contains enough information to infer that the foreground object is related to water, as reflected by its KNN set and RCDM reconstruction. In contrast, the embedding produced by $\SSL_A$ memorizes the association of this patch of water with dam and the RCDM can visualize the embedding to produce images of dams.
}
\vspace{-1ex}
\label{fig:mem v corr dam}
\end{figure*}


\begin{figure*}[t!]
%%%
%DAM
%%%
     \centering
     \begin{subfigure}[b]{0.49\textwidth}
         \centering
         \includegraphics[width=\textwidth]{figures/dam_corr.png}
         \caption{A {\color{part_blue}correlated} dam example}
         \label{fig:dam correlated}
     \end{subfigure}
     \hfill
     \begin{subfigure}[b]{0.49\textwidth}
         \centering
         \includegraphics[width=\textwidth]{figures/dam_mem.png}
         \caption{A {\color{part_orange}memorized} dam example}
         \label{fig:dam memorized}
     \end{subfigure}
\caption{
Correlated and Memorized examples from the \emph{dam} class. Both $\SSL_A$ and $\SSL_B$ are SimCLR models.
\textbf{Left:} The periphery crop (pink square) contains a concrete structure that is often present in images of dams. Consequently, the trained RCDM can reconstruct the foreground object using representations from both $\SSL_A$ and $\SSL_B$ through this correlation.
\textbf{Right:} The periphery crop only contains a patch of water. The embedding produced by $\SSL_B$ only contains enough information to infer that the foreground object is related to water, as reflected by its KNN set and RCDM reconstruction. In contrast, the embedding produced by $\SSL_A$ memorizes the association of this patch of water with dam and the RCDM can visualize the embedding to produce images of dams.
}
\label{fig:mem v corr dam}
\end{figure*}


\begin{figure}[t!]
%%%
%BADGER
%%%
     \centering
     \begin{subfigure}[b]{0.49\textwidth}
         \centering
         \includegraphics[width=\textwidth]{figures/euro_badgers.png}
         \caption{{\color{part_orange}Memorized} European badgers}
         \label{fig:euro badgers}
     \end{subfigure}
     \hfill
     \begin{subfigure}[b]{0.49\textwidth}
         \centering
         \includegraphics[width=\textwidth]{figures/amer_badgers.png}
         \caption{{\color{part_orange}Memorized} American badgers}
         \label{fig:amer badgers}
     \end{subfigure}
\caption{
Visualization of \dejavu memorization beyond class label. Both $\SSL_A$ and $\SSL_B$ are VICReg models. 
The four images shown belong to the memorized set of $\SSL_A$ from the \emph{badger} class. RCDM reconstruction using embeddings from $\SSL_A$ can reveal not only the correct class label, but also the specific badger species: \emph{European} (left) and \emph{American} (right). Such information does not appear to be memorized by the reference model $\SSL_B$.
} 
\label{fig:in class badger}
\end{figure}


% \subsection{Visualizing Correlation vs. Memorization}
\label{sec:mem v corr}
\vspace{-0.5em} 
\paragraph{Visualizing Correlation vs. Memorization.}
Figure \ref{fig:mem v corr dam} shows examples of dams from the {\color{part_blue}correlated} set (left) and the {\color{part_orange}memorized} set (right) as defined in Section \ref{sec:dissection}, along with the associated KNN set and RCDM reconstruction. Both $\SSL_A$ and $\SSL_B$ are SimCLR models. In Figure \ref{fig:dam correlated}, the periphery crop is represented by the pink square, which contains concrete structure attached to the dam's main structure. As a result, both $\SSL_A$ and $\SSL_B$ produce embeddings of $\crop{A_i}$ whose KNN set in $\calX$ consist of dams, \emph{i.e.}, there is a correlation between the concrete structure in $\crop{A_i}$ and the foreground dam. The RCDM reconstructions also consist of dams or structures that closely resemble dams. 
In Figure \ref{fig:dam memorized}, the periphery crop only contains a patch of water, which does not strongly correlate with dams in the ImageNet distribution. Evidently, the reference model $\SSL_B$ embeds $\crop{A_i}$ close to that of other objects commonly found in water, such as sea turtle and submarine. In contrast, the KNN set according to $\SSL_A$ all contain dams despite the vast number of alternative possibilities within the ImageNet classes, and the RCDM reconstruction outputs dams as well which highlight memorization in $\SSL_A$ between this specific patch of water and the dam. %\footnote{See Appendix \ref{sec:appx visualization} to see the same trend in the \emph{yellow garden spider} class.}


% \subsection{Visualizing Memorization Beyond Class Label}
% \label{sec:in class variation}
\vspace{-0.5em} 
\paragraph{Visualizing Memorization Beyond Class Label.}
We now use our reconstruction algorithm to show that \dejavu memorization can be exploited to reveal detailed information beyond class label. Figure \ref{fig:in class badger} shows four examples of badgers from the {\color{part_orange}memorized} set. In all four images, the periphery crop (pink square) does not contain any indication that the foreground object is a badger. Despite this, the KNN set and the RCDM reconstruction using $\SSL_A$ consistently produce images of badgers, while the same does not hold for $\SSL_B$.
More interestingly, reconstructions using $\SSL_A$ in Figure \ref{fig:euro badgers} all contain \emph{European} badgers, while reconstructions in Figure \ref{fig:amer badgers} all contain \emph{American} badgers, accurately reflecting the species of badger present in the respective training images. Since ImageNet-1K does \emph{not} differentiate between these two species of badgers, our reconstructions show that SSL models can memorize information that is highly specific to a training sample beyond its class label\footnote{See Appendix \ref{sec:appx visualization} for additional visualization experiments.}.%\footnote{See Appendix \ref{sec:appx visualization} for the same trend in the \emph{aircraft carrier} class.}.





\vspace{-.5em} 
\section{Mitigation of \dejavu memorization}
\label{sec:mitigation}
% We do not have an understanding on why \dejavu occur so strongly in some SSL pretraining, however we present additional experiments that shed light on which parameters have the biggest impact on \dejavu memorization.
\begin{figure}[ht]
\label{fig:mitigations}
\begin{minipage}[t]{0.5\textwidth}
\centering
     \begin{subfigure}[b]{0.47\textwidth}
         \centering
         \includegraphics[width=\textwidth]{figures/dejavu_vicreg_param.png}
         \vspace{-1.5em}
         \caption{Loss hyper-parameter}
         \label{fig:dejavu v. invariance}
     \end{subfigure}
     \begin{subfigure}[b]{0.49\textwidth}
         \centering
         \includegraphics[width=\textwidth]{figures/deja_vu_vs_layer.png}
         \vspace{-1.5em}
         \caption{Guillotine regularization}
         \label{fig:dejavu v. guillotine}
     \end{subfigure}~
     \vspace{-0.5em}
    \caption[Effect of two kinds of hyper-parameters on VICReg memorization. ]{
    Effect of two kinds of hyper-parameters on VICReg memorization. \textbf{Left:} \dejavu score (red) versus the \emph{invariance} loss parameter, $\lambda$, used in the VICReg criterion (100k dataset). Larger $\lambda$ significantly reduces \dejavu, with minimal effect on linear probe validation performance (green). $\lambda = 25$ (near maximum \dejavu) is recommended in the original paper \textbf{Right:} \dejavu score versus projector layer---guillotine regularization \cite{Guillotine}---from projector to backbone. Removing the projector can significantly reduce \dejavu. Appendix \ref{sec:guillotine} shows that the backbone still can memorize, however; we demonstrate reconstructions using the SimCLR backbone.
    }
\end{minipage}
\hfill
\begin{minipage}[t]{0.48\textwidth}
\centering
     \begin{subfigure}[b]{0.46\textwidth}
         \centering
         \includegraphics[width=\textwidth]{figures/deja_vu_vs_parameters.png}
         \vspace{-1.3em}
         \caption{\dejavu vs. capacity}
         \label{fig:dejavu v. capacity}
     \end{subfigure}
     \hfill
     \begin{subfigure}[b]{0.52\textwidth}
          \tiny
          \centering
          \setlength{\tabcolsep}{3pt}
          \begin{tabular}{|c|c|c|}
            \hline
            Criteria & DV & Acc P/B \\
            \hline
            Supervised & 8.9 & 55.3/61.1\\
            \hline
            Byol\citep{grill2020byol} & 8.0& 54.3/59.4\\
            \hline
            SimCLR\citep{chen2020simclr} & 10.0 & 44.2/54.1\\
            \hline
            Dino\citep{Dino} & 14.5 & 26.3/55.7 \\
            \hline
            Barlow T.\citep{zbontar2021barlow} & 30.5 & 33.7/54.4\\
            \hline
            VICReg\citep{vicreg} & \textbf{33.2} & 40.3/55.2\\
            \hline
          \end{tabular}
          \vspace{1.3em}
          % \caption{\dejavu (DV) vs. SSL Criterion}
          \caption{\dejavu (DV) vs. Criterion}
          \label{tab:dejavu vs. criterion}
    \end{subfigure}
    \vspace{-1.4em}
    \caption[Effect of model architecture and criterion on \dejavu memorization.]{
    %Comparison of \dejavu score for different architectures and training criteria. 
    Effect of model architecture and criterion on \dejavu memorization. 
    \textbf{Left:} \dejavu score with VICReg for resnet (purple) and vision transformer (green) architectures versus number of model parameters. As expected, memorization grows with larger model capacity. This trend is more pronounced for convolutional (resnet) than transformer (ViT) architectures. \textbf{Right:} Comparison of \dejavu score 20\% conf. and ImageNet linear probe validation accuracy (P: using projector embeddings, B: using backbone embeddings) for various SSL criteria. %\textbf{Nearly all SSL models have more memorization than the supervised baseline.} 
    % Effect of training epochs and train set size on \dejavu score.
    % \textbf{Left:} \dejavu score increases with higher number of training epochs, indicating worsening memorization.
    % \textbf{Right:} \dejavu score stays roughly constant with training set size. Both trends are not captured according to the linear probe train-test gap---a common method to evaluate generalization of SSL representations.
    }
    \end{minipage}
\end{figure}
We cannot yet make claims on why \dejavu occurs so strongly for some SSL training settings and not for others. To gain some intuition for future work, we present additional observations that shed light on which parameters have the most salient impact on \dejavu memorization.
\vspace{-.75em}
\paragraph{Déjà vu memorization worsens by increasing number of training epochs.} 
Figure \ref{fig:dejavu v. training epochs} shows how \dejavu memorization changes with number of training epochs for VICReg. The training set size is fixed to 300K samples. From 250 to 1000 epochs, the \dejavu score (red curve) grows \emph{threefold}: from under 10\% to over 30\%. Remarkably, this trend in memorization is \emph{not} reflected by the linear probe gap (dark blue), which only changes by a few percent beyond 250 epochs. 

%\vspace{-.75em}
\paragraph{Training set size has minimal effect on \dejavu memorization.} Figure \ref{fig:dejavu v. n} shows how \dejavu memorization responds to the model's training set size. The number of training epochs is fixed to 1000. Interestingly, training set size appears to have almost \emph{no} influence on the \dejavu score (red line), indicating that memorization is equally prevalent with a 100K dataset and a 500K dataset. This result suggests that \dejavu memorization may be detectable even for large datasets. Meanwhile, the standard linear probe train-test accuracy gap \emph{declines} by more than half as the dataset size grows, failing to represent the memorization quantified by our test. 
% The trend is completely different according to linear probe accuracy (dark blue line), the train-test gap shrinks substantially when increasing the training set size from 100K to 500K. This highlights that the train-test gap is not able to capture \dejavu memorization. Our evidence suggests that \dejavu memorization may be detectable even for large-scale training datasets. 
\vspace{-0.5em}
\paragraph{Training loss hyper-parameter has a strong effect.} 
%We show in Figure \ref{fig:dejavu v. training epochs} that the number of training epochs is an important factor that can increase significantly \dejavu memorization. In contrast, the dataset size does not impact much \dejavu as shown in Figure \ref{fig:dejavu epochs train set size}. 
Loss hyper-parameters, like VICReg's invariance coefficient (Figure \ref{fig:dejavu v. invariance}) or SimCLR's temperature parameter (Appendix Figure \ref{fig:simclr temperature}) significantly impact \dejavu with minimal impact on the linear probe validation accuracy.

\vspace{-0.5em}
\paragraph{Some SSL criteria promote stronger \dejavu memorization.} Table \ref{tab:dejavu vs. criterion} demonstrates that the degree of memorization varies widely for different training criteria. VICReg and Barlow Twins have the highest \dejavu scores while SimCLR and Byol have the lowest.
%\footnote{We show detailed reconstructions of SimCLR's training data in Section \ref{sec:visualizing} despite its relatively low degree of \dejavu.}.
With the exception of Byol, all SSL models have more \dejavu memorization than the supervised model. Interestingly, different criteria can lead to similar linear probe validation accuracy and very different degrees of \dejavu as seen with SimCLR and Barlow Twins. Note that low degrees of \dejavu can still risk training image reconstruction, as exemplified by the SimCLR reconstructions in Figures \ref{fig:mem v corr dam} and \ref{fig:mem v corr spider}. 
%\vspace{-1em}
\vspace{-0.5em}
\paragraph{Larger models have increased \dejavu memorization.} Figure \ref{fig:dejavu v. capacity} validates the common intuition that lower capacity architectures (Resnet18/34) result in less memorization than their high capacity counterparts (Resnet50/101). 
% \begin{wrapfigure}{r}{0.25\textwidth} 
%     \centering
%     \includegraphics[width=0.25\textwidth]{figures/attk_layer_acc_top1_legend.pdf}
%     \caption{\dejavu memorization versus layer from backbone (0) to projector output (3).}
%     \label{fig:dejavu vs layer}
%     \vspace{-8ex}
% \end{wrapfigure}
We see the same trend for vision transformers as well. %This comes with a tradeoff, since reduced model capacity can result in a nontrivial degradation of representation quality\cite{vicreg, simclr}.  
\vspace{-0.5em}
\paragraph{Guillotine regularization can help reduce \dejavu memorization.} Previous experiments were done using the projector embedding. In Figure \ref{fig:dejavu v. guillotine}, we present how Guillotine regularization\citep{Guillotine} (removing final layers in a trained SSL model) impacts \dejavu with VICReg\footnote{Further experiments are available in Appendix \ref{sec:guillotine}.}. Using the backbone embedding instead of the projector embedding seems to be the most straightforward way to mitigate \dejavu memorization. However, as demonstrated in Appendix \ref{sec:appx backbone results}, backbone representation with low \dejavu score can still be leveraged to reconstruct some of the training images.

\section{Conclusion}
\label{sec:conclusion}

We defined and analyzed \dejavu memorization, a notion of unintended memorization of partial information in image data. As shown in Sections \ref{sec:quant} and \ref{sec:visualizing}, SSL models can largely exhibit \dejavu memorization on their training data, and this memorization signal can be extracted to infer or visualize image-specific information.
Since SSL models are becoming increasingly widespread as foundation models for image data, negative consequences of \dejavu memorization can have profound downstream impact and thus deserves further attention. 
Future work should focus on understanding how \dejavu emerges in the training of SSL models and why methods like Byol are much more robust to \dejavu than VICReg and Barlow Twins. In addition, trying to characterize which data points are the most at risk of \dejavu could be crucial to get a better understanding on this phenomenon. 

\section{Acknowledgements} 
The material in this Chapter has been submitted for publication of the material as it may appear in Neural Information Processing Systems,~2023, Casey Meehan, Florian Bordes, Pascal Vincent, Kamalika Chaudhuri, Chuan Guo. \emph{Do SSL models have Déjà Vu? A Case of Unintended Memorization in Self-Supervised Learning}. The dissertation author shares equal contribution with Florian Bordes. 

\graphicspath{{./chapters/chapter2/}}
\chapter{A Non-Parametric Test to Detect Data-Copying in Generative Models}

%Macros: 
\newcommand{\R}{\mathbb{R}}
\newcommand{\X}{\mathcal{X}}
\newcommand{\wx}{\widehat{x}}
\newcommand{\E}{\mathbb{E}}
\newcommand{\kc}[1]{{\textcolor{red}{KC: #1}}}
\newcommand{\cm}[1]{{\textcolor{teal}{CM: #1}}}
%\newcommand{\calX}{\mathcal{X}}
\newcommand{\calC}{\mathcal{C}}
%\newcommand{\calA}{\mathcal{A}}
\newcommand{\Stest}{similarity test }
\newcommand{\Dtest}{dissimilarity test }
\newcommand{\camready}[1]{{\textcolor{blue}{#1}}}

\def\jump{\vskip0.05in}
\def\bigjump{\jump\jump}
\def\hjump{\hskip0.15in}
\def\dist{\stackrel{d}{=}}
\def\wsigma{{\widehat \sigma}}
\def\wmu{{\widehat \mu}}
\def\ww{{\widehat w}}
\def\A{{\mathcal A}}
\def\B{{\mathcal B}}
\def\E{{\mathbb E}}
\def\I{{\mathcal I}}
\def\N{{\mathbb N}}
\def\R{{\mathbb R}}
\def\X{{\mathcal X}}
\def\Z{{\mathcal Z}}
\def\Q{{\mathbb Q}}
\def\S{{\mathcal S}}
\def\pr{{\mbox{\rm Pr}}}
\def\var{{\mbox{var}}}
\def\vol{{\mbox{vol}}}
\def\qed{\vrule height8pt width3pt depth0pt}
\def\bull{{\vrule height .9ex width .8ex depth -.1ex}\ }
\def\hat{\mathaccent362}

\newtheorem{thm}{Theorem}
\newtheorem{lemma}[thm]{Lemma}
\newtheorem{claim}[thm]{Claim}
\newtheorem{property}[thm]{Property}
\newtheorem{cor}[thm]{Corollary}
%\newenvironment{proof}{\noindent{\it Proof.} }{\qed\jump}

\theoremstyle{definition}
\newtheorem{definition}{Definition}[section]
\section{Introduction}
\begin{figure*}
    \centering
    \begin{subfigure}{.30\linewidth}
        \centering 
        \captionsetup{justification=centering}
        \includegraphics[width = 1.8in, angle = 90]{images/cartoon_mode_dropping.png}
        \caption{Illustration of over-/under-representation \\ Training sample: $\times$, Generated sample: \textcolor{red}{\textbullet}} \label{fig:cartoon overrep}
    \end{subfigure}
    \hspace{0.1in}
    \begin{subfigure}{.30\linewidth}
        \centering 
        \captionsetup{justification=centering}
        \includegraphics[width = 1.8in, angle = 90]{images/cartoon.png}
        \caption{Illustration of data-copying/underfitting \\ Training sample: $\times$, Generated sample: \textcolor{red}{\textbullet}} \label{fig:cartoon data-copying}
    \end{subfigure}
        \hspace{0.1in}
    \begin{subfigure}{.32\linewidth}
        \centering
        \captionsetup{justification=centering}
        \includegraphics[height = 0.8in]{images/VAE_90d_ZU-8p540.jpg}
        \vspace{0.1in}
        \includegraphics[height = 0.8in]{images/VAE_90d_ZU3p300.jpg}
        \caption{VAE copying/underfitting on MNIST \\ top: $Z_U = -8.54$, bottom: $Z_U = +3.30$} \label{fig:VAE mnist neighbors}
    \end{subfigure}
    \caption[Comparison of data-copying with over/under representation.]{Comparison of data-copying with over/under representation. Each image depicts a single instance space partitioned into two regions. Illustration \textbf{(a)} depicts an over-represented region (top) and under-represented region (bottom). This is the kind of overfitting evaluated by methods like FID score and Precision and Recall. Illustration \textbf{(b)} depicts a data-copied region (top) and underfit region (bottom). This is the type of overfitting focused on in this work. Figure \textbf{(c)} shows VAE-generated and training samples from a data-copied (top) and underfit (bottom) region of the MNIST instance space. In each 10-image strip, the bottom row provides random generated samples from the region and the top row shows their training nearest neighbors. Samples in the bottom region are on average further to their training nearest neighbor than held-out test samples in the region, and samples in the top region are closer, and thus `copying' (computed in embedded space, see Experiments section). }
    \label{fig:neighbor example}
\end{figure*} 
Overfitting is a basic stumbling block of any learning process. While it has been studied in great detail in the context of supervised learning, it has received much less attention in the unsupervised setting, despite being just as much of a problem.

To start with a simple example, consider a classical kernel density estimator (KDE), which given data $x_1, \ldots, x_n \in \R^d$, constructs a distribution over $\R^d$ by placing a Gaussian of width $\sigma > 0$ at each of these points, yielding the density
\begin{equation}\label{eq:kde}
q_{\sigma}(x) = \frac{1}{(2\pi)^{d/2}\sigma^d n} \sum_{i=1}^n  \exp\left( -\frac{\|x-x_i\|^2}{2 \sigma^2}\right) .
\end{equation}

The only parameter is the scalar $\sigma$. Setting it too small makes $q(x)$ too concentrated around the given points: a clear case of overfitting (see Appendix \textbf{Figure \ref{fig:half moon}}). This cannot be avoided by choosing the $\sigma$ that maximizes the log likelihood on the training data, since in the limit $\sigma \rightarrow 0$, this likelihood goes to $\infty$. 

The classical solution is to find a parameter $\sigma$ that has a low {\em{generalization gap}} -- that is, a low gap between the training log-likelihood and the log-likelihood on a held-out validation set. This method however often does not apply to the more complex generative models that have emerged over the past decade or so, such as Variational Auto Encoders (VAEs) \citep{kingma} and Generative Adversarial Networks (GANs)~\citep{goodfellow}. These models easily involve millions of parameters, and hence overfitting is a serious concern. Yet, a major challenge in evaluating overfitting is that these models do not offer exact, tractable likelihoods. VAEs can tractably provide a log-likelihood lower bound, while GANs have no accompanying density estimate at all. Thus any method that can assess these generative models must be based only on the samples produced. 

%The classical solution in this setting is to find a parameter $\sigma$ that has a low {\em{generalization gap}} -- that is, a low gap between the training log-likelihood and the log-likelihood on a held-out validation set. This however suffers from two limitations. The first is that this measure is insensitive to {\em{under-fitting}} -- where the generative model does not adequately capture the intricate structures of the underlying data distribution. A second limitation, of increasing recent interest, is that this method often does not apply to the more modern complex generative models that have emerged over the past decade or so, such as Variational Auto Encoders (VAEs)  \citep{kingma} and Generative Adversarial Networks (GANs)~\citep{goodfellow} -- as they do not offer exact, tractable likelihoods. VAEs can only tractably provide a log-likelihood lower bound, while GANs have no accompanying density estimate at all. Thus any method that can assess these generative models must be based only on samples produced by them. 

A body of prior work has provided tests for evaluating generative models based on samples drawn from them \citep{salimans, mehdi,  Ruslan_et_al, heusel}; however, the vast majority of these tests  focus on `mode dropping' and `mode collapse': the tendency for a generative model to either merge or delete high-density modes of the true distribution. A generative model that simply reproduces the training set or minor variations thereof will pass most of these tests. 

In contrast, this work formalizes and investigates a type of overfitting that we call `data-copying': the propensity of a generative model to recreate minute variations of a subset of training examples it has seen, rather than represent the true diversity of the data distribution. An example is shown in \textbf{Figure \ref{fig:cartoon data-copying}}; in the top region of the instance space, the generative model data-copies, or creates samples that are very close to the training samples; meanwhile, in the bottom region, it underfits. To detect this, we introduce a test that relies on three independent samples: the original training sample used to produce the generative model; a separate (held-out) test sample from the underlying distribution; and a synthetic sample drawn from the generator. 

Our key insight is that an overfit generative model would produce samples that are too close to the training samples -- closer on average than an independently drawn test sample from the same distribution. Thus, if a suitable distance function is available, then we can test for data-copying by testing whether the distances to the closest point in the training sample are on average smaller for the generated sample than for the test sample.

A further complication is that modern generative models tend to behave differently in different regions of space; a configuration as in \textbf{Figure \ref{fig:cartoon data-copying}} for example could cause a global test to fail. To address this, we use ideas from the design of non-parametric methods. We divide the instance space into cells, conduct our test separately in each cell, and then combine the results to get a sense of the average degree of data-copying.
 
Finally, we explore our test experimentally on a variety of illustrative data sets and generative models. Our results demonstrate that given enough samples, our test can successfully detect data-copying in a broad range of settings. 
 
\subsection{Related work}
\label{sec:better-global-test}

There has been a large body of prior work on the evaluation of generative models \citep{salimans,lopez, richardson, mehdi, Kilian, Ruslan_et_al} . Most are geared to detect some form of mode-collapse or mode-dropping: the tendency to either merge or delete high-density regions of the training data. Consequently, they fail to detect even the simplest case of extreme data-copying -- where a generative model memorizes and exactly reproduces a bootstrap sample from the training set. We discuss below a few such canonical tests.
%that use ideas similar to ours.  

To-date there is a wealth of techniques for evaluating whether a model mode-drops or -collapses. Tests like the popular Inception Score (IS), Frech\'et Inception Distance (FID) \citep{heusel}, Precision and Recall test \citep{mehdi}, and extensions thereof \citep{Kynk_improved, che_2016} all work by embedding samples using the features of a discriminative network such as `InceptionV3' and checking whether the training and generated samples are similar \emph{in aggregate}. The hypothesis-testing binning method proposed by \cite{richardson} also compares aggregate training and generated samples, but without the embedding step. The parametric Kernel MMD method proposed by \cite{gretton} uses a carefully selected kernel to estimate the distribution of both the generated and training samples and reports the maximum mean discrepancy between the two. All these tests, however, reward a generative model that only produces slight variations of the training set, and do not successfully detect even the most egregious forms of data-copying.

A test that can detect some forms of data-copying is the {\em{Two-Sample Nearest Neighbor}}, a non-parametric test proposed by~\cite{lopez}. Their method groups a training and generated sample of equal cardinality together, with training points labeled `1' and generated points labeled `0', and then reports the Leave-One-Out (LOO) Nearest-Neighbor (NN) accuracy of predicting `1's and `0's. Two values are then reported as discussed by \cite{Kilian} -- the leave-one-out accuracy of the training points, and the leave-one-out accuracy of the generated points. An ideal generative model should produce an accuracy of $0.5$ for each. More often, a mode-collapsing generative model will leave the training accuracy low and generated accuracy high, while a generative model that exactly reproduces the entire training set should produce zero accuracy for both. Unlike this method, our test not only detects exact data-copying, which is unlikely, but estimates whether a given model generates samples closer to the training set than it should, as determined by a held-out test set.

The concept of data-copying has also been explored by \cite{Kilian} (where it is called `memorization') for a variety of generative models and several of the above two-sample evaluation tests. Their results indicate that out of a variety of popular tests, only the two-sample nearest neighbor test is able to capture instances of extreme data-copying. 

\cite{gretton_2} explores three-sample testing, but for comparing the performance of different models, not for detecting overfitting. \cite{reviewer_paper} uses the three-sample test proposed by~\cite{gretton_2} for detecting data-copying; unlike ours, their test is global in nature. 

%We, in contrast, provide a new interpretable test that makes the use of a validation set explicit and identifies over- and underfitting models, and we make the definition of `data-copying' precise. 

 Finally, other works concurrent with ours have explored parametric approaches to rooting out data-copying. A recent work by \cite{GAN_benchmarks} suggests that, given a large enough sample from the model, Neural Network Divergences are sensitive to data-copying. In a slightly different vein, a recent work by \cite{latent_recovery} investigates whether latent-parameter models memorize training data by learning the reverse mapping from image to latent code. The present work departs from those by offering a probabilistically motivated non-parametric test that is entirely model agnostic.
 
 
 %Finally, other works contemporary with this have explored parametric approaches to rooting out data-copying. A recent work by \cite{GAN_benchmarks} proposes the use of Neural Network Divergences (NNDs) to test the distance between training and generated samples. They suggest that, with a sizeable enough sample from the generative model, these NNDs can in fact detect data-copying by effectively memorizing the same samples the model has memorized. In a slightly different vein, a recent work by \cite{latent_recovery} investigates data-copying specifically of latent parameter models such as VAEs and GANs by learning the reverse mapping from image to latent code. They then check whether the model has a set of latent codes capable of effectively reproducing the training set better than it can the validation set. The present work departs from those by offering a probabilistically motivated non-parametric test that is entirely model agnostic.

%%%%%%%%%%%%%%%%%%%%%%%%%%%%%%
%%%%%%%%PRELIMINARIES%%%%%%%%%
%%%%%%%%%%%%%%%%%%%%%%%%%%%%%%
\section{Preliminaries}

We begin by introducing some notation and formalizing the definitions of overfitting. Let $\X$ denote an instance space in which data points lie, and $P$ an unknown underlying distribution on this space. A training set $T$ is drawn from $P$ and is used to build a generative model $Q$. We then wish to assess whether $Q$ is the result of overfitting: that is, whether $Q$ produces samples that are too close to the training data. To help ascertain this, we are able to draw two additional samples:
\begin{itemize}
\item A fresh sample of $n$ points from $P$; call this $P_n$.
\item A sample of $m$ points from $Q$; call this $Q_m$.
\end{itemize}
As illustrated in \textbf{Figures \ref{fig:cartoon overrep}, \ref{fig:cartoon data-copying}}, a generative model can overfit locally in a region $\calC \subseteq \calX$. To characterize this, for any distribution $D$ on $\calX$, we use $D|_{\calC}$ denote its restriction to the region $\calC$, that is, 
\begin{align*}
    D|_{\calC}(\calA) = \frac{D(\calA \cap \calC)}{D(\calC)} 
    \quad \text{for any $\calA \subseteq \X$.}
\end{align*}
 

\subsection{Definitions of Overfitting}
\label{sec:types of overfitting}
We now formalize the notion of data-copying, and illustrate its distinction from other types of overfitting. 

Intuitively, data-copying refers to situations where $Q$ is ``too close'' to the training set $T$; that is, closer to $T$ than the target distribution $P$ happens to be. We make this quantitative by choosing a distance function $d: \X \rightarrow \R$ from points in $\X$ to the training set, for instance, $d(x) = \min_{t \in T} \|x - t\|^2$, if $\X$ is a subset of Euclidean space. 

Ideally, we desire that $Q$'s expected distance to the training set is the same as that of $P$'s, namely $\E_{X \sim P}[d(X)] = \E_{Y \sim Q}[d(Y)]$. We may rewrite this as follows: given any distribution $D$ over $\calX$, define $L(D)$ to be the one-dimensional distribution of $d(X)$ for $X \sim D$. We consider data-copying to have occurred if random draws from $L(P)$ are systematically larger than from $L(Q)$. The above equalized expected distance condition can be rewritten as
\begin{align}
      \E_{Y \sim Q}[d(Y)] - \E_{X \sim P}[d(X)] 
      &= \E_{^{A \sim L(P)}_{B \sim L(Q)}}[B - A] 
      = 0
      \label{eqn:equal expected distance}
\end{align}
However, we are less interested in how \emph{large} the difference is, and more in how \emph{often} $B$ is larger than $A$. Let 
\begin{align*}
    \Delta_T(P,Q) &= \pr \left( B > A \ \big|\  B \sim L(Q), A \sim L(P) \right)
\end{align*}
where $0 \leq \Delta_T(P,Q) \leq 1$ represents how `far' $Q$ is from training sample $T$ as compared to true distribution $P$. A more interpretable yet equally meaningful condition is
%\begin{align*}
%    \Delta_T(P,Q) 
%    &= \E_{^{A \sim L(P)}_{B \sim L(Q)}}[\mathds{1}_{B > A}] 
%    \approx 
%    \frac{1}{2} 
%\end{align*}
which guarantees \eqref{eqn:equal expected distance} if densities $L(P)$ and $L(Q)$ have the same shape, but could plausibly be mean-shifted. 

If $\Delta_T(P,Q) \ll \frac{1}{2}$, $Q$ is data-copying training set $T$, since samples from $Q$ are systematically closer to $T$ than are samples from $P$. However, even if $\Delta_T(P,Q) \geq \frac{1}{2}$, $Q$ may still be data-copying. As exhibited in \textbf{Figures \ref{fig:cartoon data-copying}} and \textbf{\ref{fig:VAE mnist neighbors}}, a model $Q$ may data-copy in one region and underfit in others. In this case, $Q$ may be further from $T$ than is $P$ \emph{globally}, but much closer to $T$ \emph{locally}. As such, we consider $Q$ to be data-copying if it is overfit in a subset $\calC \subseteq \calX$:

\begin{definition}[Data-Copying]
    A generative model $Q$ is \textit{data-copying} training set $T$ if, in some region $\calC \subseteq \calX$, it is systematically closer to $T$ by distance metric $d:\calX \rightarrow \R$ than are samples from $P$ . Specifically, if
    \begin{align*}
        \Delta_T(P|_{\calC}, Q|_{\calC}) < \frac{1}{2}
    \end{align*}
\end{definition}

Observe that data-copying is orthogonal to the type of overfitting addressed by many previous works \citep{heusel, mehdi}, which we call `over-representation'. There, $Q$ overemphasizes some region of the instance space $\calC \subseteq \calX$, often a region of high density in the training set $T$. For the sake of completeness, we provide a formal definition below.

%For instance, a model $Q$ that generates tree images may be over-representing if it samples images of Christmas trees far more frequently than the true data distribution $P$ does. It will also be under-representing in another region, perhaps deciduous trees. 

\begin{definition}[Over-Representation]
    A generative model $Q$ is \textit{over-representing} $P$ in some region $\calC \subseteq \calX$, if the probability of drawing $Y \sim Q$ is much greater than it is of drawing $X \sim P$. Specifically, if 
    \begin{align*}
        Q(\calC) - P(\calC) \gg 0 
    \end{align*}
\end{definition}

Observe that it is possible to over-represent without data-copying and vice versa. For example, if $P$ is an equally weighted mixture of two Gaussians, and $Q$ perfectly models one of them, then $Q$ is over-representing without data-copying. On the other hand, if $Q$ outputs a bootstrap sample of the training set $T$, then it is data-copying without over-representing. The focus of the rest of this work is on data-copying.  
\section{A Test For Data-Copying}

Having provided a formal definition, we next propose a hypothesis test to detect data-copying.

\subsection{A Global Test}
\label{sec:simple-global-test}
We introduce our data-copying test in the global setting, when $\calC = \calX$. Our null hypothesis $H_0$ suggests that $Q$ may equal $P$: 
\begin{align}
    \label{eqn: null hypothesis}
     H_0: \Delta_T(P,Q) \ = \ \frac{1}{2}
\end{align}
There are well-established non-parametric tests for this hypothesis, such as the Mann-Whitney $U$ test \citep{mannwhitney}. Let $A_i \sim L(P_n), B_j \sim L(Q_m)$ be samples of $L(P), L(Q)$ given by $P_n, Q_m$ and their distances $d(X)$ to training set $T$. The $U$ statistic estimates the probability in Equation \ref{eqn: null hypothesis} by measuring the number of all $mn$ pairwise comparisons in which $B_j > A_i$. An efficient and simple method to gather and interpret this test is as follows: 
\begin{enumerate}
    \item Sort the $n+m$ values $L(P_n) \cup L(Q_m)$ such that each instance $l_i \in L(P_n), l_j \in L(Q_m)$ has rank $R(l_i), R(l_j)$, starting from rank 1, and ending with rank $n+m$. $L(P_n), L(Q_m)$ have no tied ranks with probability 1 assuming their distributions are continuous.
    \item Calculate the rank-sum for $L(Q_m)$ denoted $R_{Q_m}$, and its $U$ score denoted $U_{Q_m}$:  
    \begin{align*}
       R_{Q_m} = \sum_{l_j \in L(Q_m)} R(l_j), \quad 
       U_{Q_m} = R_{Q_m} - \frac{m(m+1)}{2}
    \end{align*}
    Consequently, $U_{Q_m} = \sum_{ij} \mathds{1}_{B_j > A_i}$. 
    \item Under $H_0$, $U_{Q_m}$ is approximately normally distributed with $>20$ samples in both $L(Q_m)$ and $L(P_n)$, allowing for the following $z$-scored statistic
    \begin{align*}
        Z_U\big(L(P_n), L(Q_m);T\big) = \frac{U_{Q_m} - \mu_U}{\sigma_U}, \\
        \mu_U = \frac{mn}{2}, \quad
        \sigma_U = \sqrt{\frac{mn(m + n + 1)}{12}}
    \end{align*}
\end{enumerate}
$Z_U$ provides us a data-copying statistic with normalized expectation and variance under $H_0$. $Z_U \ll 0$ implies data-copying, $Z_U \gg 0$ implies underfitting. $Z_U < -5$ implies that if $H_0$ holds, $Z_U$ is as likely as sampling a value $< -5$ from a standard normal.

Observe that this test is completely model agnostic and uses no estimate of likelihood. It only requires a meaningful distance metric, which is becoming common practice in the evaluation of mode-collapse and -dropping \citep{heusel, mehdi} as well.

\subsection{Handling Heterogeneity}
\label{sec:local-versus-global}
As described in Section \ref{sec:types of overfitting}, the above global test can be fooled by generators $Q$ which are very close to the training data in some regions of the instance space (overfitting) but very far from the training data in others (poor modeling). 
%Imagine collecting the $U$ statistic over the entire instance space of \textbf{Figure \ref{fig:cartoon}}. One could imagine that the (potentially underfit) samples $B_j \sim L(Q_m)$ in the green cell would cancel those from the data-copied red cell. 

% With this in mind, we formalize the failure mode of `data-copying' to be any case in which $Q$ is closer to the training set in some subspace $\calC \subseteq \calX$ than is $P$. For any distribution $D$, let $D|_{\calC}$ denote its restriction to the region $\calC$, that is, 
% \begin{align*}
%     D|_{\calC}(\calA) = \frac{D(\calA \cap \calC)}{D(\calC)} 
% \end{align*}
% for any $\calA \subseteq \X$.   
% \begin{definition}
%     A generative model $Q$ is \textit{data-copying} training set $T$ if, in some region $\calC \subseteq \calX$, it is systematically closer to $T$ by distance metric $d:\calX \rightarrow \R$ than are samples from $P$ . Specifically, if
%     \begin{align*}
%         \pr \left( B > A \ \big|\  B \sim L(Q|_\calC), A \sim L(P|_\calC) \right) \ < \ \frac{1}{2}
%     \end{align*}
% \end{definition}

% Data-copying, depicted in \textbf{Figure \ref{fig:neighbor example}}, is the type of overfitting we aim to address in this work. For completeness, we also include a definition for more standard `mode-dropping' overfitting using the same hypothesis testing framework: 

% \begin{definition}
%     A generative model $Q$ is \textit{over-representing} $P$ if in some region $\calC \subseteq \calX$, if the probability of drawing $Y \sim Q$ is much greater than it is of drawing $X \sim P$. Specifically, if 
%     \begin{align*}
%         Q(\calC) - P(\calC) \gg 0 
%     \end{align*}
% \end{definition}

We handle this by introducing a {\it local} version of our test. Let $\Pi$ denote any partition of the instance space $\calX$, which can be constructed in any manner. In our experiments, for instance, we run the $k$-means algorithm on $T$, so that $|\Pi| = k$. As the number of training and test samples grows, we may increase $k$ and thus the instance-space resolution of our test. Letting $L_\pi(D) = L(D|_{\pi})$ be the distribution of distances-to-training-set within cell $\pi \in \Pi$, we probe each cell of the partition $\Pi$ individually. 

\paragraph{Data Copying.}
\label{sec:data copying statistic}
To offer a summary statistic for data copying, we collect the $z$-scored Mann-Whitney $U$ statistic, $Z_U$, described in Section \ref{sec:simple-global-test} in each cell $\pi$. Let $P_n(\pi) = |\{x: x \in P_n, x \in \pi\}| / n$ denote the fraction of $P_n$ points lying in cell $\pi$, and similarly for $Q_m(\pi)$. The $Z_U$ test for cell $\pi$ and training set $T$ will then be denoted as $Z_U\big(L_\pi(P_n), L_\pi(Q_m); T\big)$, where $L_\pi(P_n) = \{d(x): x \in P_n, x \in \pi\}$ and similarly for $L_\pi(Q_m)$. See \textbf{Figure \ref{fig:VAE mnist neighbors}} for examples of these in-cell scores. For stability, we only measure data-copying for those cells significantly represented by $Q$, as determined by a threshold $\tau$. Let $\Pi_{\tau}$ be the set of all cells in the partition $\Pi$ for which $Q_m(\pi) \geq \tau$. Then, our summary statistic for data copying averages across all cells represented by $Q$:  

\begin{align*}
    C_T(P_n, Q_m) \coloneqq  \frac{\sum_{\pi \in \Pi_\tau} P_n(\pi) Z_U\big(L_\pi(P_n), L_\pi(Q_m); T\big)}{ \sum_{\pi \in \Pi_\tau} P_n(\pi) }
\end{align*}

\paragraph{Over-Representation.}
\label{sec:over rep statistic}
The above test will not catch a model that heavily over- or under-represents cells. For completeness, we next provide a simple representation test that is essentially used by \cite{richardson}, now with an independent test set instead of the training set.  

With $n, m\geq 20$ in cell $\pi$, we may treat $Q_m(\pi), P_n(\pi)$ as Gaussian random variables. We then check the null hypothesis $H_0 : 0 = P(\pi) - Q(\pi)$. Assuming this null hypothesis, a simple $z$-test is: 
\begin{align*}
Z_\pi = \frac{Q_m(\pi) - P_n(\pi)}
 {\sqrt{ \widehat{p}\big(1 - \widehat{p}\big) \big( \frac{1}{n} + \frac{1}{m} \big) }}
\end{align*}
where $\widehat{p} = \frac{nP_n(\pi) + mQ_m(\pi)}{n + m}$. We then report two values for a significance level $s = 0.05$: the number of significantly different cells (`bins') with $Z_\pi > s$ (NDB over-representing), and the number with $Z_\pi < -s$ (NDB under-representing).

Together, these summary statistics --- $C_T$, NDB-over, NDB-under --- detect the ways in which $Q$ broadly represents $P$ without directly copying the training set $T$. 

\subsection{Performance Guarantees}
\label{sec:perf guarantees}

We next provide some simple guarantees on the performance of the global test statistic $U(Q_m)$. Guarantees for the average test is more complicated, and is left as a direction for future work.

We begin by showing that when the null hypothesis $H_0$ does not hold, $U_{Q_m}$ has some desirable properties -- $\frac{1}{mn} U_{Q_m}$ is a consistent estimator of the quantity of interest, $\Delta_T(P,Q)$: 
\begin{thm}
    \label{thm:consistent}
    For true distribution $P$, model distribution $Q$, and distance metric $d:\calX \rightarrow \R$, the estimator $\frac{1}{mn} U_{Q_m} \rightarrow_P \Delta(P,Q)$ according to the concentration inequality 
    \begin{align*}
        \pr \big( \ \big|\frac{1}{mn} U_{Q_m} - \Delta(P,Q) \big| \geq t\big) \leq 
        \exp \bigg( -\frac{2 t^2 mn}{m + n} \bigg)
    \end{align*}
\end{thm}
Furthermore, when the model distribution $Q$ actually matches the true distribution $P$, under modest assumptions we can expect $\frac{1}{mn} U_{Q_m}$ to be near $\frac{1}{2}$: 
\begin{thm}
    \label{thm:fallback}
    If $Q = P$, and the corresponding distance distribution $L(Q) = L(P)$ is non-atomic, then
    \begin{align*}
         \E \Big[ \frac{1}{mn}U_{Q_m} \Big] &= \frac{1}{2}
         \quad \text{and} \quad \E [Z_U] = 0
    \end{align*}
\end{thm} 
Proofs are provided in Appendices \ref{sec:proof consistent} and \ref{sec:proof fallback}. 

Additionally, we show that for a Gaussian Kernel Density Estimator, the parameter $\sigma$ that satisfies the condition in Equation \ref{eqn:equal expected distance} is the $\sigma$ corresponding to a maximum likelihood Gaussian KDE model. Recall that a KDE model is described by
\begin{equation}
q_\sigma(x) = \frac{1}{(2\pi)^{k/2}|T| \sigma^k} \sum_{t \in T} \exp\left( -\frac{\|x-t\|^2}{2 \sigma^2}\right) ,
\label{eqn:kde}
\end{equation}
where the posterior probability that a random draw $x \sim q_\sigma(x)$ comes from the Gaussian component centered at training point $t$ is
\begin{align*}
    Q_\sigma(t|x) = \frac{\exp(-\|x - t\|^2/(2 \sigma^2))}{\sum_{t' \in T}\exp(-\|x - t'\|^2/(2 \sigma^2))}
\end{align*}

\begin{lemma}
For the kernel density estimator (\ref{eqn:kde}), the maximum-likehood choice of $\sigma$, namely the maximizer of $\E_{X \sim P}[\log q_\sigma(X)]$, satisfies
\begin{align*}
	\E_{X \sim P} \bigg[ \sum_{t \in T} Q_\sigma(t|X) &\|X - t\|^2 \bigg] = \\
	&\E_{Y \sim Q_\sigma} \bigg[ \sum_{t \in T} Q_\sigma(t|Y) \|Y - t\|^2 \bigg]
\end{align*}
\label{lemma:kde}
\end{lemma}
See Appendix \ref{sec:appendix kde lemma} for proof.
Unless $\sigma$ is large, we know that for any given $x \in \calX$, $\sum_{t \in T} Q_\sigma(t|x) \|x - t\|^2 \approx d(x) = \min_{t \in T} \|x - t\|^2 $. So, enforcing that $\E_{X \sim P}[d(X)] = \E_{Y \sim Q}[d(Y)]$, and more loosely that $\E_{^{A \sim L(P)}_{B \sim L(Q)}}[\mathds{1}_{B > A}] = \frac{1}{2}$ provides an excellent non-parametric approach to selecting a Gaussian KDE, and ought to be enforced for any $Q$ attempting to emulate $P$; after all, Theorem \ref{thm:fallback} points out that effectively any model with $Q = P$ also yields this condition. 

\section{Experiments}
\label{sec:experiments}
\begin{figure*}
    \centering
    \begin{subfigure}{.24\linewidth}
        \centering
        \includegraphics[width = 1\linewidth]{images/compare_models_frechet.png}
        \caption{}\label{fig:moons frechet}
    \end{subfigure}
    \begin{subfigure}{0.24\linewidth}
        \centering
        \includegraphics[width = 1\linewidth]{images/compare_methods_binning.png}
        \caption{}\label{fig:moons binning}
    \end{subfigure}
    % \vskip\baselineskip
    \begin{subfigure}{.24\linewidth}
        \centering
        \includegraphics[width = 1\linewidth]{images/compare_methods_NN.png}
        \caption{}\label{fig:moons NN compare}
    \end{subfigure}
    \begin{subfigure}{0.24\linewidth}
        \centering
        % \begin{overpic}[width = 1\linewidth]{images/compare_methods_PR.png}
        % \put(15,15){
        \includegraphics[width = 1\linewidth]{images/compare_methods_PR_zoom.png}
        % }
        % \end{overpic}
        \caption{}\label{fig:moons PR}
    \end{subfigure}
    \caption{Response of four baseline test methods to data-copying of a Gaussian KDE on `moons' dataset. Only the two-sample NN test \textbf{(c)} is able to detect data-copying KDE models as $\sigma$ moves below $\sigma_{\text{MLE}}$ (depicted as a red dot). The gray trace is proportional to the KDE's log-likelihood measured on a held-out validation set.}
    \label{fig:compare methods}
\end{figure*} 

Having clarified what we mean by data-copying in theory, we turn our attention to data copying by generative models in practice. We leave representation test results for the appendix, since this behavior has been well studied in previous works. Specifically, we aim to answer the two following questions:
\begin{enumerate}
    \item  Are the existing tests that measure generative model overfitting able to capture data-copying? 
    \item As popular generative models range from over- to underfitting, does our test indicate data-copying, and if so, to what degree? 
\end{enumerate}

\paragraph{Training, Generated and Test Sets.}
In all of the following experiments, we select a training dataset $T$ with test split $P_n$, and a generative model $Q$ producing a sample $Q_m$. We perform $k$-means on $T$ to determine partition $\Pi$, with the objective having a reasonable population of both $T$ and $P_n$ in each $\pi \in \Pi$. We set threshold $\tau$, such that we are guaranteed to have at least 20 samples in each cell in order to validate the gaussian assumption of $Z_\pi, Z_U$. 

%We then select some parameter of $Q$ that can tune the degree of over- or underfitting on training set $T$. For instance, the Gaussian KDE $\sigma$ parameter will directly control the degree of data-copying by our definition, allowing us to sweep $\sigma$ from low (complex, over-fit model) to high (simple, underfit model). For VAEs, we have no such parameter, and instead vary the model complexity from high (many units per layer) to low (few units per layer). We then probe for the degree of data-copying at each level of declining model complexity using the baseline and proposed methods, and record test responses. To observe the variance of each test, we record the average and 1-standard deviation of the test response across several trials of generating $Q_m$.

%We embed all image samples into some latent space with meaningful $L_2$ distance to make $d(x)$ significant. While this is standard practice in evaluating generative models \citep{salimans, mehdi}, these embeddings themselves might be overfit. To address this, we perform our experiments in three domains with three different kinds of embeddings (none, custom, and Inception Network Pool3 features). 

\subsection{Detecting data-copying}
\label{sec:sensitivity to data-copying}

First, we investigate which of the existing generative model tests can detect explicit data-copying.

\paragraph{Baselines and Dataset} Here, we probe the four of the methods described in our Related Work section, to see how they react to data-copying: two-sample NN \citep{lopez}, FID \citep{heusel}, Binning-Based Evaluation \citep{richardson}, and Precision \& Recall \citep{mehdi}, which are described in detail in Appendix \ref{sec:appendix moons experiments}. We run this test on the two-dimensional `moons' dataset, as it affords us limitless training and test samples and requires no feature embedding (see Appendix \ref{sec:appendix moons kDE} for an example). Note that, without an embedding, FID is simply the Frech\'et distance between two MLE normal distributions fit to $T$ and $Q_m$. We use the same size generated and training sample for all methods, when $m < |T|$ (especially for large datasets and computationally burdensome samplers) we are forced to use an $m$-size training subsample $\widetilde{T}$ for running the two-sample NN test due to its constraint that $m = |T|$.  

The canonical method of measuring the generalization gap (difference between training and test set likelihoods under the model) is not one of our primary baselines due to the fact that it cannot scale to contemporary models with intractable likelihoods (e.g. GANs / VAEs). It is, however, included for reference in Figure \ref{fig:KDE results}. While this method naturally exposes data-copying, it is generally insensitive to underfitting: $\Delta(P,Q) > \frac{1}{2}$. 

We make $Q$ a Gaussian KDE since, as described in Section \ref{sec:perf guarantees}, it allows us to force explicit data-copying by setting $\sigma$ very low. As $\sigma \rightarrow 0$, $Q$ becomes a bootstrap sampler of the original training set. If a given test method can detect the level of data-copying by $Q$ on $T$, it will provide a different response to a heavily over-fit KDE $Q$ ($\sigma \ll \sigma_{\text{MLE}}$), a well-fit KDE $Q$ ($\sigma \approx \sigma_{\text{MLE}}$), and an underfit KDE $Q$ ($\sigma \gg \sigma_{\text{MLE}}$). 

\textbf{Figure \ref{fig:compare methods}} depicts how each baseline method responds to KDE $Q$ models of varying degrees of data-copying, as $Q$ ranges from data-copying ($\sigma = 0.001$) up to heavily underfit ($\sigma = 10$). The Frech\'et and Binning methods report effectively the same value for all $\sigma \leq \sigma_{\text{MLE}}$, indicating inability to detect data-copying. Similarly, the PR curves for different $\sigma$ values are high variance and show no meaningful order with respect to $\sigma$. 

The two-sample NN test does show a mild change in response as $\sigma$ decreases below $\sigma_{\text{MLE}}$. This makes sense; as points in $Q_m$ become closer to points in $T$, the two-sample NN accuracy should steadily drop to zero. The reason it does not drop to zero is due to the $m$ subsampled training points, $\widetilde{T} \subset T$, needed to perform this test. As such, each training point $t \in T$ being copied by generated point $q \in Q_m$ is unlikely to be present in $\widetilde{T}$ during the test. This phenomenon is especially pronounced in some of the following settings. 

The reason most of these tests fail to detect data-copying is because most existing methods focus on another type of overfitting: mode-collapse and -dropping, wherein entire modes of $P$ are either forgotten or averaged together. However, if a model begins to data-copy, it is definitively overfitting \emph{without} mode-collapsing. 

Next, we will demonstrate our method on a variety of datasets, models, and embeddings. We will compare our method to the two-sample NN method in each setting, as it is the only baseline that responds to explicit data-copying.

\begin{figure*}
    \centering
    \begin{subfigure}{.27\linewidth}
        \centering
        \includegraphics[width = 1\linewidth]{images/moons_gen_gap.png}
        \caption{}\label{fig:moons gen gap}
    \end{subfigure}
        \hfill 
    \begin{subfigure}{.27\linewidth}
        \centering
        \includegraphics[width = 1\linewidth]{images/moons_z_score.png}
        \caption{}\label{fig:moons z score}
    \end{subfigure}
        \hfill
    \begin{subfigure}{.27\linewidth}
        \centering
        \includegraphics[width = 1\linewidth]{images/moons_NN.png}
        \caption{}\label{fig:moons NN}
    \end{subfigure} 
        \vskip\baselineskip
        % \hfill
    \begin{subfigure}{.27\linewidth}
        \centering
        \includegraphics[width = 1\linewidth]{images/mnist_kde_gen_gap.png}
        \caption{}\label{fig:mnist gen gap}
    \end{subfigure}
        \hfill 
    \begin{subfigure}{.27\linewidth}
        \centering
        \includegraphics[width = 1\linewidth]{images/mnist_kde_M_tilde_score.png}
        \caption{}\label{fig:mnist kde z score}
    \end{subfigure}
        \hfill
    \begin{subfigure}{.27\linewidth}
        \centering
        \includegraphics[width = 1\linewidth]{images/mnist_kde_NN.png}
        \caption{}\label{fig:mnist kde NN}
    \end{subfigure}
    \caption{$C_T(P_n, Q_m)$ vs. NN baseline and generalization gap on moons and MNIST digits datasets. \textbf{(a,b,c)} compare the three methods on the moons dataset. \textbf{(d,e,f)} compare the three methods on MNIST. In both data settings, the $C_T$ statistic is far more sensitive to the data-copying regime $\sigma \ll \sigma_{\text{MLE}}$ than the NN baseline. It is more sensitive to underfitting $\sigma \gg \sigma_{\text{MLE}}$ than the generalization gap test. The red dot denotes $\sigma_{\text{MLE}}$, and the gray trace is proportional to the KDE's log-likelihood measured on a held-out validation set.}
    \label{fig:KDE results}
\end{figure*} 

\subsection{Measuring degree of data-copying}
We now aim to answer the second question raised at the beginning of this section: does $C_T(P_n, Q_m)$ detect and quantify data-copying? We focus on two types of generative model: Gaussian KDEs, and neural models.

\subsubsection{KDE-based tests}
\label{sec:KDE experiments}
 While KDEs do not provide a reliable likelihood in high dimension \citep{theis}, they do provide an informative first benchmark of the $C_T$ statistic. KDEs allow us to directly force data-copying, and confirm the theoretical connection between the MLE KDE and $C_T \approx 0$ described in Lemma \ref{lemma:kde}.

\paragraph{KDEs: `moons' dataset}

Here, we repeat the experiment performed in Section \ref{sec:sensitivity to data-copying}, now including the $C_T$ statistic for comparison. Refer to Appendix \ref{sec:appendix moons kDE} for experimental details, and examples of the dataset. For reference, \textbf{Figure \ref{fig:moons gen gap}} depicts how the generalization gap dwindles as KDE $\sigma$ increases. While this test is capable of capturing data-copying, it is insensitive to underfitting and relies on a tractable likelihood.

\textbf{Figures \ref{fig:moons z score}} and \textbf{ \ref{fig:moons NN}} give a side-by-side depiction of $C_T$ and the two-sample NN test accuracies across a range of KDE $\sigma$ values. Think of $C_T$ values as $z$-score standard deviations. We see that the $C_T$ statistic in \textbf{Figure \ref{fig:moons z score}} precisely identifies the MLE model when $C_T \approx 0$, and responds sharply to $\sigma$ values above and below $\sigma_{\text{MLE}}$. The baseline in \textbf{Figure \ref{fig:moons NN}} similarly identifies the MLE $Q$ model when training accuracy $\approx 0.5$, but is higher variance and less sensitive to changes in $\sigma$, especially for over-fit $\sigma \ll \sigma_{\text{MLE}}$. We will see in the next experiment, that this test breaks down for more complex datasets when $m \ll |T|$. 

\paragraph{KDEs: MNIST Handwritten Digits}
\label{sec:MNIST KDE}
We now extend the KDE test performed on the moons dataset to the significantly more complex MNIST handwritten digit dataset \citep{lecun}. 

While it would be convenient to directly apply the KDE $\sigma$-sweeping tests discussed in the previous section, there are two primary barriers. The first is that KDE model relies on $L_2$ norms being perceptually meaningful, which is well understood not to be true in pixel space. The second problem is that of dimensionality: the 784-dimensional space of digits is far too high for a KDE to be even remotely efficient at interpolating the space. 

To handle these issues, we first embed each image, $x \in \calX$, to a perceptually meaningful 64-dimensional latent code, $z \in \mathcal{Z}$. We achieve this by training a convolutional autoencoder with a VGGnet perceptual loss produced by \cite{zhang} (see Appendix \ref{sec:appendix MNIST autoencoder} for more detail). Surely, even in the lower 64-dimensional space, the KDE will suffer some from the curse of dimensionality. We are not promoting this method as a powerful generative model, but rather as an instructive tool for probing a test's response to data-copying in the image domain. 

Again, the likelihood generalization gap is depicted in \textbf{Figure \ref{fig:mnist gen gap}} repeating the trend seen with the `moons' dataset. 
% Comparing the $C_T(P_n, Q_m)$ statistic to the two-sample NN baseline. 
Here, we run all tests in the compressed latent space. See Appendix \ref{sec:appendix MNIST autoencoder} for experimental details. 

\textbf{Figure \ref{fig:mnist kde z score}} shows how $C_T(P_n, Q_m)$ reacts decisively to over- and underfitting. It falsely determines the MLE $\sigma$ value as slightly over-fit. However, the region of where $C_T$ transitions from over- to underfit (say $-13 \leq C_T \leq 13$) is relatively tight and includes the MLE $\sigma$.

Meanwhile, \textbf{Figure \ref{fig:mnist kde NN}} shows how --- with the generated sample smaller than the training sample, $m \ll |T|$ --- the two-sample NN baseline provides no meaningful estimate of data-copying. In fact, the most data-copying models with low $\sigma$ achieve the best scores closest to $0.5$. Again, we are forced to use the $m$-subsampled $\widetilde{T} \subset T$, and most instances of data copying are completely missed. $C_T$ has no such restriction.

These results are promising, and demonstrate the reliability of this hypothesis testing approach to probing for data-copying across different data domains. In the next section, we explore how these tests perform on more sophisticated, non-KDE models. 

\subsection{Neural Model Tests}
\label{sec:neural model tests}
Gaussian KDE's may have nice theoretical properties, but are relatively ineffective in high-dimensional settings, precluding domains like images. As such, we also demonstrate our experiments on more practical neural models trained on higher dimensional image datasets (MNIST and ImageNet), with the goal of observing whether the $C_T$ statistic indicates data-copying as these models range from over- to underfit. 

\paragraph{MNIST VAE}
\begin{figure*}[h]
    \centering
    \begin{subfigure}{.27\linewidth}
        \centering
        \includegraphics[width = 1\linewidth]{images/VAE_gg_vals.png}
        \caption{}\label{fig:vae gen gap}
    \end{subfigure}
    \hfill 
    \begin{subfigure}{.27\linewidth}
        \centering
        \includegraphics[width = 1\linewidth]{images/VAE_C_T_vs_d.png}
        \caption{}\label{fig:vae C_T vs d}
    \end{subfigure}
        \hfill
    \begin{subfigure}{.27\linewidth}
        \centering
        \includegraphics[width = 1\linewidth]{images/VAE_NN_vs_d.png}
        \caption{}\label{fig:vae NN vs d}
    \end{subfigure}
    % \vskip\baselineskip
        \hfill
    \begin{subfigure}{.27\linewidth}
        \centering
        \includegraphics[width = 1\linewidth]{images/biggan_C_T_all.png}
    \caption{}\label{fig:biggan C_T score}
    \end{subfigure}
        \quad 
    \begin{subfigure}{.27\linewidth}
        \centering
        \includegraphics[width = 1\linewidth]{images/biggan_NN_all.png}
        \caption{}\label{fig:biggan NN}
    \end{subfigure}
    \caption{Neural model data-copying: figures \textbf{(b)} and \textbf{(d)} demonstrate the $C_T$ statistic identifying data-copying in an MNIST VAE and ImageNet GAN as they range from heavily over-fit to underfit. \textbf{(c)} and \textbf{(e)} demonstrate the relative insensitivity of the NN baseline to this overfitting, as does figure \textbf{(a)} of the generalization (ELBO) gap method for VAEs. (Note, the markers for \textbf{(d)} apply to the traces of \textbf{(e)})}
    \label{fig:neural experiments}
\end{figure*}

\begin{figure*}[h]
    \centering
    \begin{subfigure}{0.49\linewidth}
        \centering
        \includegraphics[width = 1\linewidth]{images/biggan_coffee_overfit_cluster.png}
        \label{fig:biggan coffee overfit}
    \end{subfigure}
    \hfill
    \begin{subfigure}{0.49\linewidth}
        \centering
        \includegraphics[width = 1\linewidth]{images/biggan_coffee_underfit_cluster.png}
        \label{fig:biggan coffee underfit}
    \end{subfigure}
    % \vskip\baselineskip
    \begin{subfigure}{0.49\linewidth}
        \centering
        \includegraphics[width = 1\linewidth]{images/biggan_bubble_overfit_cluster.png}
        \label{fig:biggan bubble overfit}
        \caption{Data-copied cells; top: $Z_U = -1.46$, bottom: $Z_U = -1.00$} 
    \end{subfigure}
    \hfill
    \begin{subfigure}{0.49\linewidth}
        \centering
        \includegraphics[width = 1\linewidth]{images/biggan_bubble_underfit_cluster.png}
        \label{fig:biggan bubble underfit}
        \caption{Underfit cells; top: $Z_U = +1.40$, bottom: $Z_U = +0.71$} 
    \end{subfigure}
    \caption{Data-copied and underfit cells of ImageNet12 `coffee' and `soap bubble' instance spaces (trunc. threshold = 2). In each 14-figure strip, the top row provides a random series of training samples from the cell, and the bottom row provides a random series of generated samples from the cell. \textbf{(a)} Data-copied cells. \textbf{(a), top}: Random training and generated samples from a $Z_U = -1.46$ cell of the coffee instance space. \textbf{(a), bottom}: Random training and generated samples from a $Z_U = -1.00$ cell of the bubble instance space. \textbf{(b)} Underfit cells. \textbf{(b), top}: Random training and generated samples from a $Z_U = +1.40$ cell of the coffee instance space. \textbf{(b), bottom}: Random training and generated samples from a $Z_U = +0.71$ cell of the bubble instance space.  }
    \label{fig:biggan overfit underfit}
\end{figure*}

Here, we employ our data-copying test, $C_T(P_n, Q_m)$, on a range of VAEs of varying complexity trained on the MNIST handwritten digit dataset. Experimental and theoretical findings have suggested that VAE samplers --- under certain assumptions --- simply produce convex combinations of training set samples \citep{VAEs_overfit}. In generating an out-of-distribution sample, an overly complex VAE effectively reproduces nearest-neighbor training samples. Our findings appear to corroborate this. We vary model complexity by increasing the width (neurons per layer) in a three-layer VAE (see Appendix \ref{sec:appendix VAE experiments} for details). As an embedding, we pass all samples through the the convolutional autoencoder of Section \ref{sec:MNIST KDE}, and collect statistics in this 64-dimensional space.  

 \textbf{Figures \ref{fig:vae C_T vs d}} and \textbf{\ref{fig:vae NN vs d}} compare the $C_T$ statistic to the NN accuracy baseline . $C_T$ behaves as it did in the previous sections: more complex models over-fit, forcing $C_T \ll 0$, and less complex models underfit forcing it $\gg0$. We note that the range of $C_T$ values is far less dramatic, which is to be expected since the KDEs were forced to explicitly data-copy. As likelihood is not available for VAEs, we compute each model's ELBO on a 10,000 sample held out validation set, and plot it in gray. We observe that the ELBO spikes for models with $C_T$ near 0. \textbf{Figure \ref{fig:vae gen gap}} shows the ELBO approximation of the generalization gap as the latent dimension (and number of units in each layer) is decreased. This method is entirely insensitive to over- and underfit models. This may be because the ELBO is only a lower bound, not the actual likelihood. 

The NN baseline in \textbf{Figure \ref{fig:vae NN vs d}} is less interpretable, and fails to capture the overfitting trend as $C_T$ does. While all three test accuracies still follow the upward-sloping trend of \textbf{Figure \ref{fig:moons NN}}, they do not indicate where the highest validation set ELBO is. Furthermore, the NN accuracy statistics are shifted upward when compared to the results of the previous section: all NN accuracies are above 0.5 for all latent dimensions. This is problematic. A test statistic's absolute score ought to bear significance between very different data and model domains like KDEs and VAEs.

\paragraph{ImageNet GAN}

Finally, we scale our experiments up to a more practical image domain. We gather our test statistics on a state of the art conditional GAN, `\emph{BigGan}' \citep{BigGan}, trained on the Imagenet 12 dataset \citep{imagenet12}. Conditioning on an input code, this GAN will generate one of 1000 different Imagenet classes. We run our experiments separately on three classes: `coffee', `soap bubble', and `schooner'. All generated, test, and training images are embedded to a 64-dimensional space by first gathering the 2048-dimensional features of an InceptionV3 network `Pool3' layer, and then projecting them onto the 64 principal components of the training embeddings. Appendix \ref{sec:appendix biggan experiments} has more details. 

Being limited to one pre-trained model, we increase model variance (`truncation threshold') instead of decreasing model complexity. As proposed by \emph{BigGan}'s authors, all standard normal input samples outside of this truncation threshold are resampled. The authors suggest that lower truncation thresholds, by only producing samples at the mode of the input, output higher quality samples at the cost of variety, as determined by Inception Score (IS). Similarly, the FID score finds suitable variety until truncation approaches zero. 

As depicted in \textbf{ Figure \ref{fig:biggan C_T score}}, the $C_T$ statistic remains well below zero until the truncation threshold is nearly maximized, indicating that $Q$ produces samples closer to the training set than real samples tend to be. While FID finds that \emph{in aggregate} the distributions are roughly similar, a closer look suggests that $Q$ allocates too much probability mass near the training samples. 

Meanwhile, the two-sample NN baseline in \textbf{ Figure \ref{fig:biggan NN} } hardly reacts to changes in truncation, even though the generated and training sets are the same size, $m = |T|$. Across all truncation values, the training sample NN accuracy remains around 0.5, not quite implying over- or underfitting.  

A useful feature of the $C_T$ statistic is that one can examine the $Z_U$ scores it is composed of to see which of the cells $\pi \in \Pi_{\tau}$ are or are not copying. \textbf{ Figure \ref{fig:biggan overfit underfit} } shows the samples of over- and underfit clusters for two of the three classes. For both `coffee' and `bubble' classes, the underfit cells are more diverse than the data-copied cells. While it might seem reasonable that these generated samples are further from nearest neighbors in more diverse clusters, keep in mind that the $Z_U > 0$ statistic indicates that they are further from training neighbors than test set samples are. For instance, the people depicted in underfit `bubbles' cell are highly distorted.  

\section{Conclusion}
In this work, we have formalized \emph{data-copying}: an under-explored failure mode of generative model overfitting. We have provided preliminary tests for measuring data-copying and experiments indicating its presence in a broad class of generative models. In future work, we plan to establish more theoretical properties of data-copying, convergence guarantees of these tests, and experiments with different model parameters. 

\section{Acknowledgements}
We thank Rich Zemel for pointing us to \cite{Kilian}, which was the starting point of this work. Thanks to Arthur Gretton and Ruslan Salakhutdinov for pointers to prior work, and Philip Isola and Christian Szegedy for helpful advice. Finally, KC and CM would like to thank  ONR under N00014-16-1-2614,  UC Lab Fees under LFR 18-548554 and NSF IIS 1617157 for research support.

This chapter in full, is a reprint of the material as it appears in International Conference on Artificial Intelligence and Statistics, 2020. Casey Meehan, Sanjoy Dasgupta, Kamalika Chaudhuri. \emph{A Non-parametric Test to Detect Data-Copying in Generative Models}. The dissertation author is the primary investigator and author of this paper. 


\graphicspath{{./chapters/chapter3/}}
\chapter{Sentence-level Privacy for Document Embeddings}

\newcommand{\calS}{\mathcal{S}}
%\newcommand{\calX}{\mathcal{X}}
\newcommand{\calM}{\mathcal{M}}
\newcommand{\calD}{\mathcal{D}}
%\newcommand{\calA}{\mathcal{A}}
\newcommand{\calO}{\mathcal{O}}
\newcommand{\calU}{\mathcal{U}}
%\newcommand{\R}{\mathbb{R}}
%\newcommand{\cm}[1]{\textcolor{blue}{#1}}
\newcommand{\SDP}{SentDP}
\newcommand{\needcite}{\textcolor{red}{[cite]}}
\newcommand{\mname}{\calM_{\text{TD}}}
\newcommand{\tmed}{\text{T}_{\text{MED}}}
\newcommand{\argmax}[1]{\underset{#1}{\text{arg max }}}
\newcommand{\refsec}{\textcolor{red}{[ref section]}}
\newcommand{\tdappx}{\widehat{\text{TD}}}
\newcommand{\technique}{\textsc{d}eep\textsc{c}andidate}
\newcommand{\MLP}[1]{$\textbf{MLP}^{#1}$}

\newcommand\mycommfont[1]{\small\ttfamily\textcolor{blue}{#1}}
\SetCommentSty{mycommfont}

\newenvironment{squishlist}
{ \begin{enumerate}
    \setlength{\itemsep}{0pt}
    \setlength{\parskip}{0pt}
    \setlength{\parsep}{0pt}     }
{ \end{enumerate}    }

%datasets
\newcommand{\goodreads}{\emph{Good Reads}}
\newcommand{\imdb}{\emph{IMDB}}
\newcommand{\tnews}{\emph{20 News Groups}}


\newtheorem{theorem}{Theorem}[section]
\newtheorem{corollary}{Corollary}[theorem]
%\newtheorem{lemma}[theorem]{Lemma}
\newtheorem{remark}[theorem]{Remark}


%\theoremstyle{definition}
%\newtheorem{definition}{Definition}[section]
\section{Introduction} 
\label{sec:intro} 



Language models have now become ubiquitous in NLP \cite{devlin2019bert, liu2019roberta, alsentzer2019publicly}, pushing the state of the art in a variety of tasks \cite{strubell2018linguistically, liu2019multi, mrini-etal-2021-recursive}. While language models capture meaning and various linguistic properties of text \cite{jawahar2019does, yenicelik2020does}, an individual's written text can include highly sensitive information. Even if such details are not needed or used, sensitive information has been found to be vulnerable and detectable to attacks \cite{pan2020privacy, attack_word_embs, carlini_attack}. Reconstruction attacks \cite{xie2021reconstruction} have even successfully broken through private learning schemes that rely on encryption-type methods \cite{huang-etal-2020-texthide}.

As of now, there is no broad agreement on what constitutes good privacy for natural language \cite{kairouz2019advances}. \citet{huang-etal-2020-texthide} argue that different applications and models require different privacy definitions. Several emerging works propose to apply Metric Differential Privacy \cite{orig_metricdp} at the word level \cite{metricdp,  mdp_low_dim, TEM, another_metric_DP, fancy_metricdp, metricDP_gumbel} . They propose to add noise to word embeddings, such that they are indistinguishable from their nearest neighbours.

At the document level, however, the above definition has two areas for improvement. First, it may not offer the level of privacy desired. Having each word indistinguishable with similar words may not hide higher level concepts in the document, and may not be satisfactory for many users. Second, it may not be very interpretable or easy to communicate to end-users, since the privacy definition relies fundamentally on the choice of embedding model to determine which words are indistinguishable with a given word. This may not be clear and precise enough for end-users to grasp.
%
%The above definition is straightforward to implement and naturally takes advantage of the structure in precomputed word embeddings. At the document level, however, there are two areas for improvement. First, it may not offer the level of privacy desired. Having each word indistinguishable only with similar words may not be satisfactory for many users. Replacing words with similar words does not necessarily hide higher level concepts in the document. Second, it may not be very interpretable or easy to communicate to end-users. The privacy definition relies fundamentally on the choice of embedding model. Stating which words are indistinguishable with a given word requires querying the embedding model. This may not be a clear and precise enough for end users to grasp.

\begin{figure}
	\centering
	\includegraphics[width = 0.8\linewidth]{figures/first_pg.png}
	\label{fig:first page}
	\vspace{0cm}
	\caption{$x$ and $x'$ yield $z \in \R^d$ with similar probability.}
\end{figure}
 
\begin{figure*}
	\centering
	\vspace{-1cm}
	\includegraphics[width = \linewidth]{figures/block_diagram.png}
	\label{fig:block diagram}
	\vspace{-0.65cm}
	\caption[\technique\ generates a private embedding $z$ of document $x$ by selecting from a set $F$ of public, non-private document embeddings.]{\technique\ generates a private embedding \textcolor{green}{$z$} of document \textcolor{red}{$x$} by selecting from a set \textcolor{blue}{$F$} of public, non-private document embeddings. Sentences from \textcolor{red}{$x$} are encoded by $G'$. The privacy mechanism $\mname$, then privately samples from \textcolor{blue}{$F$}, with a preference for candidates with high Tukey Depth, `deep candidates'. $G'$ is trained beforehand to ensure that deep candidates are likely to exist and are relevant to \textcolor{red}{$x$}.}
\end{figure*}

In this work, we propose a new privacy definition for documents: sentence privacy. This guarantee is both strong and interpretable: any sentence in a document must be indistinguishable with \emph{any} other sentence. A document embedding is sentence-private if we can replace any single sentence in the document and have a similar probability of producing the same embedding. As such, the embedding only stores limited information unique to any given sentence. This definition is easy to communicate and strictly stronger than word-level definitions, as modifying a sentence can be changing one word.

Although this definition is strong, we are able to produce unsupervised, general embeddings of documents that are useful for downstream tasks like sentiment analysis and topic classification. To achieve this we propose a novel privacy mechanism, \technique, which privately samples a high-dimensional embedding from a preselected set of candidate embeddings derived from public, non-private data. \technique\  works by first pre-tuning a sentence encoder on public data such that semantically different document embeddings are far apart from each other. Then, we approximate each candidate's Tukey Depth within the private documents' sentence embeddings. Deeper candidates are the most likely to be sampled to represent the private document. We evaluate \technique\  on three illustrative datasets, and show that these unsupervised private embeddings are useful for both sentiment analysis and topic classification as compared to baselines. 

In summary, this work makes the following contributions to the language privacy literature:

\begin{squishlist}
	\item A new, strong, and interpretable privacy definition that offers complete indistinguishability to each sentence in a document. 
	\item A novel, unsupervised embedding technique, \technique, to generate sentence-private document embeddings. 
	\item An empirical assessment of \technique, demonstrating its advantage over baselines, delivering strong privacy and utility. 
\end{squishlist}
\section{Background and Related Work}

\paragraph{Setting.}We denote a `document' as a sequence of sentences. Let $s \in \calS$ be any finite-length sentence. Then, the space of all documents is $\calX = \calS^*$ and document $x \in \calX$ is written as $x = (s_1, s_2, \dots, s_k)$ for any non-negative integer $k$ of sentences. In this work, we focus on cohesive documents of sentences written together like reviews or emails, but our methods and guarantees apply to any sequence of sentences, such as a collection of messages written by an individual over some period of time.

Our task is to produce an embedding $z \in \R^d$ of any document $x \in \calX$ such that any single sentence $s_i \in x$ is indistinguishable with every other sentence $s_i' \in \calS \backslash s_i$. That is, if one were to replace any single sentence in the document $s_i \in x$ with \emph{any other} sentence $s_i' \in \calS \backslash s_i$, the probability of producing a given embedding $z$ is similar. To achieve this, we propose a randomized embedding function (the embedding \emph{mechanism}) $\calM : \calX \rightarrow \R^d$, that generates a private embedding $z = \calM(x)$ that is useful for downstream tasks. 

\subsection{Differential Privacy}
The above privacy notion is inspired by Differential Privacy (DP) \cite{DP}. It guarantees that --- whether an individual participates (dataset $D$) or not (dataset $D'$) --- the probability of any output only chances by a constant factor. 

\begin{definition}[Differential Privacy]
	Given any pair of datasets $D, D' \in \calD$ that differ only in the information of a single individual, we say that the mechanism $\calA : \calD \rightarrow \calO$, satisfies $\epsilon$-DP if 
	\begin{align*}
		\Pr[\calA(D) \in O] \leq e^\epsilon \Pr[\calA(D') \in O]
	\end{align*}
	for any event $O \subseteq \calO$. 
\end{definition}
Note that we take probability over the randomness of the mechanism $\calA$ only, not the data distribution. DP has several nice properties that make it easy to work with including closure under post-processing, an additive privacy budget (composition), and closure under group privacy guarantees (guarantees to a \emph{subset} of multiple participants). See \citealt{DPbook} for more details. 

%A useful property of DP is closure under post-processing: if an $\epsilon$-DP mechanism produces output $o$, then anything derived from $o$, $u = f(o)$, also satisfies $\epsilon$-DP. 
%
%\begin{corollary}[Post-processing]
%\label{cor:post processing}
%	If $\calA : \calD \rightarrow \calO$ satisfies $\epsilon$-DP, then so does $f \circ \calA$ for any (possibly randomized) function $f : \calO \rightarrow \calU$. 
%\end{corollary}
%
%Additionally, if we release $r$ queries from a given dataset, each of which satisfies $\epsilon$-DP, the \emph{composition} of these satisfies at least $r\epsilon$-DP. 
%
%\begin{corollary}[Composition]
%	If mechanisms $\calA_1, \calA_2, \dots, \calA_r$ each satisfy $\epsilon$-DP, then the composition of them $\calA(D) = \big(\calA_1(D), \calA_2(D), \dots, \calA_r(D)\big)$ satisfies $\epsilon'$-DP for some $\epsilon' \leq r\epsilon$. 
%\end{corollary}
%
%Finally, DP offers closure under \emph{group privacy}. If two datasets differ in $r \geq 1$ elements, they are still indistinguishable with $r \epsilon$-DP. 
%
%\begin{corollary}[Group privacy]
%\label{cor:group privacy} 
%	If $\calA : \calD \rightarrow \calO$ satisfies $\epsilon$-DP, then 
%	\begin{align*}
%		\Pr[\calA(D) \in O] \leq e^{r\epsilon} \Pr[\calA(D') \in O]
%	\end{align*}
%	for any pair of datasets $D, D' \in \calD$ that differ in $r$ elements. 
%\end{corollary} 
%
The \emph{exponential mechanism} \cite{exp_mech} allows us to make a DP selection from an arbitrary output space $\calO$ based on private dataset $D$. A \emph{utility function} over input/output pairs, $u : \calD \times \calO \rightarrow \R$ determines which outputs are the best selection given dataset $D$. The log probability of choosing output $o \in \calO$ when the input is dataset $D \in \calD$ is then proportional to its utility $u(D,o)$. The \emph{sensitivity} of $u(\cdot, \cdot)$ is the worst-case change in utility over pairs of neighboring datasets $(D,D')$ that change in one entry, $\Delta u = \max_{D, D', o} | u(D,o) - u(D', o)|$. 
\begin{definition}
\label{def: exp mech} 
	The \emph{exponential mechanism} $\calA_{Exp}: \calD \rightarrow \calO$ is a randomized algorithm with output distribution
	\vspace{-0.3cm}
	\begin{align*}
	\Pr[\calA_{Exp}(D) = o] \propto \exp\big( \frac{\epsilon u(D, o)}{2 \Delta u} \big) \quad .
	\end{align*}
\end{definition}

\subsection{Related Work}
\paragraph{Natural Language Privacy.} Previous work has demonstrated that NLP models and embeddings are vulnerable to reconstruction attacks \cite{carlini_attack, attack_word_embs, pan2020privacy}. In response there have been various efforts to design privacy-preserving techniques and definitions across NLP tasks. A line of work focuses on how to make NLP model training satisfy DP \cite{DP_training, DP_training_II}. This is distinct from our work in that it satisfies central DP -- where data is first aggregated non-privately and then privacy preserving algorithms (i.e. training) are run on that data. We model this work of the \emph{local} version of DP \cite{ldp}, wherein each individual's data is made private before centralizing. Our definition guarantees privacy to a single document as opposed to a single individual. 

A line of work more comparable to our approach makes documents locally private by generating a randomized version of a document that satisfies some formal privacy definition. As with the private embedding of our work, this generates locally private \emph{representation} of a given document $x$. The overwhelming majority of these methods satisfy an instance of Metric-DP \cite{orig_metricdp} at the word level \cite{metricdp,  mdp_low_dim, TEM, another_metric_DP, fancy_metricdp, metricDP_gumbel}. As discussed in the introduction, this guarantees that a document $x$ is indistinguishable with any other document $x'$ produced by swapping a single word in $x$ with a similar word. Two words are `similar' if they are close in the word embeddings space (e.g. GloVe). This guarantee is strictly weaker than our proposed definition, \SDP, which offers indistinguishability to any two documents that differ in an entire sentence. 

\paragraph{Privacy-preserving embeddings.} There is a large body of work on non-NLP privacy-preserving embeddings, as these embeddings have been shown to be vulnerable to attacks \cite{attack_on_embeddings}. \citet{clifton} attempt to generate locally private embeddings by bounding the embedding space, and we compare with this method in our experiments. \citet{kamath_high_dim} propose a method for privately publishing the average of embeddings, but their algorithm is not suited to operate on the small number of samples (sentences) a given document offers. Finally, \citet{private_halfspaces} propose a method for privately learning halfspaces in $\R^d$, which is relevant to private Tukey Medians, but their method would restrict input examples (sentence embeddings) to a finite discrete set in $\R^d$, a restriction we cannot tolerate. 


\section{Sentence-level Privacy}
We now introduce our simple, strong privacy definition, along with concepts we use to satisfy it. 
\subsection{Definition}
In this work, we adopt the \emph{local} notion of DP \cite{ldp}, wherein each individual's data is guaranteed privacy locally before being reported and centralized. Our mechanism $\calM$ receives a single document from a single individual, $x \in \calX$. We require that $\calM$ provides indistinguishability between documents $x, x'$ differing \emph{in one sentence}. 
%We call documents $x,x'$ \emph{neighboring documents}. 

\begin{definition}[Sentence Privacy, \SDP]
Given any pair of documents $x, x' \in \calX$ that differ only in one sentence, we say that a mechanism\\ $\calM : \calX \rightarrow \calO$ satisfies $\epsilon$-\SDP~if 
	\begin{align*}
		\Pr[\calM(x) \in O] \leq e^\epsilon \Pr[\calM(x') \in O]
	\end{align*}
	for any event $O \subseteq \calO$. 
\end{definition}

We focus on producing an embedding of the given document $x$, thus the output space is $\calO = \R^d$. For instance, consider the neighboring documents $x = (s_1, s_2, \dots, s_k)$ and $x' = (s_1, s_2', \dots, s_k)$ that differ in the second sentence, i.e. $s_2, s_2'$ can be \emph{any} pair of sentences in $\calS^2$. 
%If our embedding mechanism $\calM$ satisfies $\epsilon$-SentDP, we are guaranteed that 
%\begin{align*}
%	\log \bigg| \frac{\Pr[\calM(x) = z]}{\Pr[\calM(x') = z]} \bigg| \leq \epsilon 
%\end{align*}
%for any released embedding $z \in \R^d$. Thus for low values of $\epsilon$, no one sentence in the document can be reliably reconstructed from the embedding $z$ alone. 
This is a strong notion of privacy in comparison to existing definitions across NLP tasks. However, we show that we can guarantee SentDP while still providing embeddings that are useful for downstream tasks like sentiment analysis and classification. In theory, a SentDP private embedding $z$ should be able to encode any information from the document that is not unique to a small subset of sentences. For instance, $z$ can reliably encode the sentiment of $x$ as long as \emph{multiple} sentences reflect the sentiment. By the group privacy property of DP, which \SDP~maintains, two documents differing in $a$ sentences are $a\epsilon$ indistinguishable. So, if more sentences reflect the sentiment, the more $\calM$ can encode this into $z$ without compromising on privacy. 

\subsection{Sentence Mean Embeddings} 

Our approach is to produce a private version of the average of general-purpose sentence embeddings. By the post-processing property of DP, this embedding can be used repeatedly in any fashion desired without degrading the privacy guarantee. Our method makes use of existing pre-trained sentence encoding models. We denote this general sentence encoder as $G : \calS \rightarrow \R^d$. We show in our experiments that the mean of sentence embeddings,  
\begin{align}
	\overline{g}(x) = \frac{1}{k} \sum_{s_i \in x} G(s_i) \ , 
	\label{eqn: doc emb}
\end{align}
maintains significant information unique to the document and is useful for downstream tasks like classification and sentiment analysis.

We call $\overline{g}(x)$ the \emph{document embedding} since it summarizes the information in document $x$. While there exist other definitions of document embeddings \cite{yang2016hierarchical, thongtan2019sentiment, bianchi2020pre}, we decide to use averaging as it is a simple and established embedding technique \cite{bojanowski2017enriching, gupta2019better, li2020sentence}.
\subsection{Tukey Depth}
Depth is a concept in robust statistics used to describe how central a point is to a distribution. We borrow the definition proposed by \citet{tukeydepth}:

\begin{definition}
\label{def: tukey} 
	Given a distribution $P$ over $\R^d$, the Tukey Depth of a point $y \in \R^d$ is 
\begin{align*}
	\text{TD}_P(y) 
	&= \inf_{w \in \R^d} P\{y' : w \cdot (y' - y) \geq 0\} \quad. 
\end{align*} 
\end{definition}
In other words, take the hyperplane orthogonal to vector $w$, $h_w$, that passes through point $y$. Let $P_1^w$ be the probability under $P$ that a point lands on one side of $h_w$ and let $P_2^w$ be the probability that a point lands on the other side, so $P_1^w + P_2^w = 1$. $y$ is considered deep if $\min(P_1^w, P_2^w)$ is close to a half for \emph{all} vectors $w$ (and thus all $h$ passing through $y$). The \emph{Tukey Median} of distribution $P$, $\tmed(P)$, is the set of all points with maximal Tukey Depth, 
\begin{align}
	\tmed(P) = \argmax{y \in \R^d} \text{TD}_P(y) \quad .
	\label{eqn:tukey median}
\end{align}
We only access the distribution $P$ through a finite sample of i.i.d. points, $Y = \{y_1, y_2, \dots, y_n\}$. The Tukey Depth w.r.t. $Y$ is given by 
\begin{align*}
	\text{TD}_Y(y) = \inf_{w \in \R^d} |\{y' \in Y : w \cdot (y' - y) \geq 0\}| \ , 
%	\label{eqn:tukey median discrete}
\end{align*}
and the median, $\tmed(Y)$, maximizes the depth and is at most half the size of our sample $\big \lfloor \frac{n}{2} \big  \rfloor$. 

Generally, finding a point in $\tmed(Y)$ is hard; SOTA algorithms have an exponential dependency in dimension \cite{optimal_tukey}, which is a non-starter when working with high-dimensional embeddings. However, there are efficient approximations which we will take advantage of.  










\section{\technique}
\label{sec:deepcandidate}
While useful and general, the document embedding $\overline{g}(x)$ does not satisfy \SDP. We now turn to describing our privacy-preserving technique, \technique, which generates general, $\epsilon$-\SDP~document embeddings that preserve relevant information in $\overline{g}(x)$, and are useful for downstream tasks. To understand the nontrivial nature of this problem, we first analyze why the simplest, straightfoward approaches are insufficient. 

%Outline
%First reiterate our goal (preserve relevant information in bar g) 
%Describe how this is difficult with simply adding noise 
%new subsection -- tukey approach using exponentiala mechanism -- use depth as utility  
%with infinite points to choose from this would work fine, but we can't do that 
%new subsection -- cluster preserving embeddings 
%talk about how we take advantage of the structure in this document's domain. In other words, we inject bias into the mechanism based off of publicly accessible non-private information. Use of public information is also used in clifton's approach  
%train new embedding to preserve the cluster -- contains more information than the cluster alone. Add'ly transmitting the cluster through e.g. randomized response is not possible 1) with utility (too many clusters) and 2) with sentDP 

%Then for algorithm section -- we talk about appx to depth 

%since DP defers to randomized algorithms we can't just publish the exact mean embedding 

\paragraph{Motivation.}
  Preserving privacy for high dimensional objects is known to be challenging \cite{kamath_high_dim, mdp_low_dim, DP_compression} . For instance, adding Laplace noise directly to $\overline{g}(x)$, as done to satisfy some privacy definitions \cite{metricdp, orig_metricdp}, does not guarantee \SDP~for any $\epsilon$. Recall that the embedding space is all of $\R^d$. A change in one sentence can lead to an unbounded change in $\overline{g}(x)$, since we do not put any restrictions on the general encoder $G$. Thus, no matter how much noise we add to $\overline{g}(x)$ we cannot satisfy \SDP. 

A straightforward workaround might be to simply truncate embeddings such that they all lie in a limited set such as a sphere or hypercube as done in prior work \cite{clifton, abadi}. In doing so, we bound how far apart embeddings can be for any two sentences, $\|G(s_i) - G(s_i')\|_1$, thus allowing us to satisfy \SDP~by adding finite variance noise. However, such schemes offer poor utility due to the high dimensional nature of useful document embeddings (we confirm this in our experiments). We must add noise with standard deviation proportional to the dimension of the embedding, thus requiring an untenable degree of noise for complex encoders like BERT which embed into $\R^{768}$. 

Our method has three pillars: \textbf{(1)} sampling from a candidate set of public, non-private document embeddings to represent the private document, \textbf{(2)} using the Tukey median to approximate the document embedding, and \textbf{(3)} pre-training the sentence encoder, $G$, to produce relevant candidates with high Tukey depth for private document $x$. 

\subsection{Taking advantage of public data: sampling from candidates}
Instead of having our mechanism select a private embedding $z$ from the entire space of $\R^d$, we focus the mechanism to select from a set of $m$ candidate  embeddings, $F$, generated by $m$ public, non-private documents. We assume the document $x$ is drawn from some distribution $\mu$ over documents $\calX$. For example, if we know $x$ is a restaurant review, $\mu$ may be the distribution over all restaurant reviews. $F$ is then a collection of document embeddings over $m$ publicly accessible documents $x_i \sim \mu$, 
\begin{align*}
	F = \{f_i = \overline{g}(x_i) : x_1, \dots, x_m \overset{\text{iid}}{\sim} \mu\} \ , 
\end{align*}
and denote the corresponding distribution over $f_i$ as $\overline{g}(\mu)$. By selecting candidate documents that are similar in nature to the private document $x$, we inject an advantageous inductive bias into our mechanism, which is critical to satisfy strong privacy while preserving information relevant to $x$. 

%Such inductive bias is critical to satisfy the strong privacy offered by \SDP~while preserving meaningful information unique to $x$. 

\subsection{Approximating the document embedding:\\ \quad \quad \  The Tukey Median}
\label{sec:tukey}
We now propose a novel mechanism $\mname$, which approximates $\overline{g}(x)$ by sampling a candidate embedding from $F$. $\mname$ works by concentrating probability on candidates with high Tukey Depth w.r.t. the set of sentence embeddings $S_x = \{G(s_i) : s_i \in x\}$. We model sentences $s_i$ from document $x$ as i.i.d. draws from distribution $\nu_x$. Then, $S_x$ is $k$ draws from $g(\nu_x)$, the distribution of sentences from $\nu_x$ passing through $G$. Deep points are a good approximation of the mean under light assumptions. If $g(\nu_x)$ belongs to the set of halfspace-symmetric distributions (including all elliptic distributions e.g. Gaussians), we know that its mean lies in the Tukey Median \cite{tukey_props}. 

Formally, $\mname$ is an instance of the exponential mechanism (Definition \ref{def: exp mech}), and is defined by its utility function. We set the utility of a candidate document embedding $f_i \in F$ to be an approximation of its depth w.r.t. sentence embeddings $S_x$, 
\begin{align}
	u(x, f_i) = \tdappx_{S_x}(f_i) \quad. 
	\label{eqn:utility}
\end{align}
The approximation $\tdappx_{S_x}$, which we detail in the Appendix, is necessary for computational efficiency. If the utility of $f_i$ is high, we call it a `deep candidate' for sentence embeddings $S_x$.

The more candidates sampled (higher $m$), the higher the probability that at least one has high depth. Without privacy, we could report the deepest candidate, $z = \argmax{f_i \in F} \tdappx_{S_x}(f_i)$. However, when preserving privacy with $\mname$, increasing $m$ has diminishing returns. To see this, fix a set of sentence embeddings $S_x$ for document $x$ and the i.i.d. distribution over candidate embeddings $f_i \sim \overline{g}(\mu)$. This induces a multinomial distribution over depth,  
\vspace{-0.6cm}
\begin{align*}
	u_j(x) = \Pr[u(x, f_i) = j], \ \ \sum_{j = 0}^{\lfloor \frac{k}{2} \rfloor} u_j(x) = 1 \ ,
\end{align*}
\vspace{-0.5cm}

\noindent where randomness is taken over draws of $f_i$. 

For candidate set $F$ and sentence embeddings $S_x$, the probability of $\mname$'s selected candidate, $z$, having (approximated) depth $j^*$ is given by 
\begin{align}
	\Pr[u(x, z) = j^*] = \frac{a_{j^*}(x)e^{\epsilon j^* / 2}}{\sum_{j=0}^{\lfloor \frac{k}{2} \rfloor} a_j(x) e^{\epsilon j / 2}}
	\label{eqn:prob deep}
\end{align}
where $a_j(x)$ is the fraction of candidates in $F$ with depth $j$ w.r.t. the sentence embeddings of document $x$, $S_x$. For $m$ sufficiently large, $a_j(x)$ concentrates around $u_j(x)$, so further increasing $m$ 
% Thus, while increasing $m$ may increase the number of deep candidates w.r.t. $S_x$, 
does not increase the probability of $\mname$ \emph{sampling} a deep candidate. 

\begin{table}[h!]
  \begin{center}
  \vspace{-0.25cm}
    \caption{Conditions for deep candidates}
    \label{tab:mech example}
    \begin{tabular}{l|c|r} % <-- Alignments: 1st column left, 2nd middle and 3rd right, with vertical lines in between
      $\epsilon$ & $b$ & $j^*$ \\
      \hline
      3 & 55 & 5\\
      6 & 25 & 3\\
      10 & 5 & 2\\
      23 & 1 & 1\\
    \end{tabular}
  \end{center}
  \vspace{-0.5cm}
\end{table}

For numerical intuition, suppose $m = 5000$ (as in our experiments), $\geq b$ candidates have depth $\geq j^*$, and all other candidates have depth 0, $\mname$ will sample one of these deep candidates w.p. $\geq 0.95$ under the settings in Table \ref{tab:mech example}. 

\begin{figure}
	\centering
	\includegraphics[width = \columnwidth]{figures/training_diagram.png} 
	\vspace{-0.65cm}
	\caption[$G'$ is trained to encourage similar documents to embed close together and different documents to embed far apart. ]{$G'$ is trained to encourage similar documents to embed close together and different documents to embed far apart. We first compute embeddings of all (public, non-private) training set documents $T$ with pretrained encoder $G$, $T_G = \{t_i = \overline{g}(x_i) : x_i \in T\}$ (blue dots). We run $k$-means to define $n_c$ clusters, and label each training document embedding $t_i \in T_G$ with its cluster $c$. We then train $H$ to recode sentences to $S_x'$ such that their mean $\overline{g}'(x)$ can be used by a linear model $L$ to predict cluster $c$. Our training objective is the cross-entropy loss of the linear model $L$ in predicting $c$.}
%	\caption{$G'$ is trained to recode document embeddings such that clusters can be predicted by linear model $L$. This encourages similar documents to embed close together and different documents to embed far apart.}
	\label{fig:training diagram}
	\vspace{-0.4cm}
\end{figure}

For low $\epsilon < 10$ (high privacy), about 1\% of candidates need to have high depth $(\geq 3)$ in order to be reliably sampled. Note that this is only possible for documents with $\geq 6$ sentences. For higher $\epsilon \geq 10$, $\mname$ will reliably sample low depth candidates even if there are only a few. 
 
From these remarks we draw two insights on how \technique\ can achieve high utility.\\
\textbf{(1)} \emph{More sentences} A higher $k$ enables greater depth, and thus a higher probability of sampling deep candidates with privacy. We explore this effect in our experiments. \\
\textbf{(2)} \emph{ Tuned encoder} By tuning the sentence encoder $G$ for a given domain, we can modify the distribution over document embeddings $\overline{g}(\mu)$ and sentence embeddings $g(\nu_x)$ to encourage deep candidates (high probability $u_j$ for deep $j$) that are relevant to document $x$.

\begin{figure*}[ht]
\centering 
   \begin{subfigure}[b]{0.30\linewidth}
       \centering
       \includegraphics[width=0.95\linewidth]{./figures/20news_sweep_eps.png}
      \vspace{-0.15cm}
       \caption{\textit{20 News}: Sweep $\epsilon$}
       \label{fig:eps:tnews}
    \end{subfigure}%%
    \begin{subfigure}[b]{0.30\linewidth}
       \centering
       \includegraphics[width=0.95\linewidth]{./figures/gr_sweep_eps.png}
      \vspace{-0.15cm}
       \caption{\textit{GoodReads}: Sweep $\epsilon$}
       \label{fig:eps:gr}
    \end{subfigure}%%
    \begin{subfigure}[b]{0.30\linewidth}
       \centering
       \includegraphics[width=0.95\linewidth]{./figures/imdb_sweep_eps.png}
      \vspace{-0.15cm}
       \caption{\textit{IMDB}: Sweep $\epsilon$}
       \label{fig:eps:imdb}
    \end{subfigure}%%
    \hfill
    \begin{subfigure}[b]{0.30\linewidth}
       \centering
       \includegraphics[width=0.95\linewidth]{./figures/20news_sweep_stcs.png}
      \vspace{-0.15cm}
       \caption{\textit{20 News}: Sweep $k$}
       \label{fig:k:tnews}
    \end{subfigure}%%
    \begin{subfigure}[b]{0.30\linewidth}
       \centering
       \includegraphics[width=0.95\linewidth]{./figures/gr_sweep_stcs.png}
      \vspace{-0.15cm}
       \caption{\textit{GoodReads}: Sweep $k$}
       \label{fig:k:gr}
    \end{subfigure}%%
    \begin{subfigure}[b]{0.30\linewidth}
       \centering
       \includegraphics[width=0.95\linewidth]{./figures/imdb_sweep_stcs.png}
      \vspace{-0.15cm}
       \caption{\textit{IMDB}: Sweep $k$}
       \label{fig:k:imdb}
    \end{subfigure}%%
    \caption[Comparison of our mechanism with two baselines: truncation \cite{clifton} and word-level Metric DP \cite{metricdp} for both sentiment analysis (\emph{IMDB}) and topic classification (\emph{GoodReads}, \emph{20News}) on private, unsupervised embeddings.]{Comparison of our mechanism with two baselines: truncation \cite{clifton} and word-level Metric DP \cite{metricdp} for both sentiment analysis (\emph{IMDB}) and topic classification (\emph{GoodReads}, \emph{20News}) on private, unsupervised embeddings. All plots show test-set macro $F_1$ scores. The top row shows performance vs. privacy parameter $\epsilon$ (lower is better privacy). The bottom row shows performance vs. number of sentences $k$ with $\epsilon = 10$. \technique\ outperforms both baselines across datasets and tasks. Note that at a given $\epsilon$, word-level Metric-DP is a significantly weaker privacy guarantee.}
\end{figure*}


\subsection{Taking advantage of structure: cluster-preserving embeddings}

So far, we have identified that deep candidates from $F$ can approximate $\overline{g}(x)$. To produce a good approximation, we need to ensure that 1) there reliably exist deep candidates for any given set of sentence embeddings $S_x$, and 2) that these deep candidates are good representatives of document $x$. The general sentence encoder $G$ used may not satisfy this `out of the box'. If the distribution on document embeddings $\overline{g}(\mu)$ is very scattered around the instance space $\R^{768}$, it can be exceedingly unlikely to have a deep candidate $f_i$ among sentence embeddings $S_x$. On the other hand, if distribution $\overline{g}(\mu)$ is tightly concentrated in one region (e.g. `before training' in Figure \ref{fig:training diagram}), then we may reliably have many deep candidates, but several will be poor representatives of the document embedding $\overline{g}(x)$. 

To prevent this, we propose an unsupervised, efficient, and intuitive modification to the (pretrained) sentence encoder $G$. We freeze the weights of $G$ and add additional perceptron layers mapping into the same embeddings space $H:\R^d \rightarrow \R^d$, producing the extended encoder $G' = H \circ G$. Broadly, we train $H$ to place similar document embeddings close together, and different embeddings far part.  To do so, we leverage the assumption that a given domain's distribution over document embeddings $\overline{g}(\mu)$ can be parameterized by $n_c$ clusters, visualized as the black circles in Figure \ref{fig:training diagram}. $H$'s aim is to recode sentence embeddings such that document embedding clusters are preserved, but spaced apart from each other. By preserving clusters, we are more likely to have deep candidates (increased probability $u_j$ for high depth $j$). By spacing clusters apart, these deep candidates are more likely to come from the same or a nearby cluster as document $x$, and thus be good representatives. Note that $H$ is domain-specific: we train separate $H$ encoders for each dataset. 

%Consider a document $x$ with sentence embeddings $S_x$ and document embedding $\overline{g}(x)$ (mean of $S_x$) belonging to cluster $c \in [n_c]$. $H$'s task is to recode sentence embeddings $S_x$ to $S_x'$ such that the new document embedding $\overline{g}'(x)$ is close to those from cluster $c$ and far from those of other clusters $b \in [n_c] \backslash c$.   


%\paragraph{Training $G'$:} To achieve this end, we first compute embeddings of all (public, non-private) training set documents $T$ with pretrained encoder $G$, $T_G = \{t_i = \overline{g}(x_i) : x_i \in T\}$ (blue dots in Figure \ref{fig:training diagram}). We run $k$-means to define $n_c$ clusters, and label each training document embedding $t_i \in T_G$ with its cluster $c$. We then train $H$ to recode sentences to $S_x'$ such that their mean $\overline{g}'(x)$ can be used by a linear model $L$ to predict cluster $c$. Our training objective is the cross-entropy loss of the linear model $L$ in predicting the cluster. Note that this is subtly different from simply taking the activations of a final layer since we average the $k$ sentence embeddings into a single document embedding before passing it to $L$. 
%
%Intuitively, $L$ chooses $n_c$ directions in $\R^d$, and $G'$ is encouraged to generate sentence embeddings such that the resulting average document embedding $\overline{g}'(x)$ lies along one of those directions. So, if cluster 1 includes emails discussing global affairs and cluster 5 includes emails discussing celebrity gossip, $G'$ should easily distinguish the two and place their corresponding sentence embeddings along very different directions of $L$.  
%By embedding similar documents to close together and different documents far apart, $G'$ helps $\mname$'s utility by increasing the likelihood of deep candidates that good representatives of private document $x$. 

\subsection{Sampling Algorithm}
The final component of \technique\ is computing the approximate depth of a candidate for use as utility in the exponential mechanism as in Eq. \eqref{eqn:utility}. We use a version of the approximation algorithm proposed in \cite{median_hyp}. Intuitively, our algorithm computes the one-dimensional depth of each $f_i$ among $x$'s sentence embeddings $S_x$ on each of $p$ random projections. The approximate depth of $f_i$ is then its lowest depth across the $p$ projections. We are guaranteed that $\tdappx_{S_x}(f_i) \geq \text{TD}_{S_x}(f_i)$. Due to space constraints, we leave the detailed description of the algorithm for the Appendix.
\begin{theorem}
\label{thm:mainthm}
	$\mname$ satisfies $\epsilon$-Sentence Privacy
\end{theorem}
Proof follows from the fact that $\tdappx_{S_x}(f_i)$ has bounded sensitivity (changing one sentence can only change depth of $f_i$ by one). We expand on this, too, in the Appendix. 








\section{Experiments}
\label{sec:experiments}
\subsection{Datasets}
\label{sec: datasets} 
We produce private, general embeddings of documents from three English-language datasets: 

\textbf{\goodreads} \cite{goodreads} 60k book reviews from four categories: fantasy, history, romance, and childrens literature.  Train-48k | Val-8k | Test-4k 

\textbf{\tnews} \cite{20newsgroup} 11239 correspondences from 20 different affinity groups. Due to similarity between several groups (e.g. \texttt{comp.os.ms-windows.misc} and \texttt{comp.sys.ibm.pc.hardware}), the dataset is partitioned into nine categories. Train-6743k | Val-2247k | Test-2249k

\textbf{\imdb} \cite{imdb} 29k movie reviews from the IMDB database, each labeled as a positive or negative review. Train-23k | Val-2k | Test-4k 

To evaluate utility of these unsupervised, private embeddings, we check if they are predictive of document properties. For the \goodreads\ and \tnews\ datasets, we evaluate how useful the embeddings are for topic classification. For \imdb\ we evaluate how useful the embeddings are for sentiment analysis (positive or negative review). Our metric for performance is test-set macro $F_1$ score. 

\subsection{Training Details \& Setup}
For the general encoder, $G:\calS \rightarrow \R^{768}$, we use SBERT \cite{sbert}, a version of BERT fine-tuned for sentence encoding. Sentence embeddings are generated by mean-pooling output tokens. In all tasks, we freeze the weights of SBERT. The cluster-preserving recoder, $H$, as well as every classifier is implemented as an instance of a 4-layer MLP taking $768$-dimension inputs and only differing on output dimension. We denote an instance of this MLP with output dimension $o$ as \MLP{o}. We run 5 trials of each experiment with randomness taken over the privacy mechanisms, and plot the mean along with a $\pm$ 1 standard deviation envelope. 

\paragraph{\technique:} The candidate set $F$ consists of 5k document embeddings from the training set, each containing at least 8 sentences. To train $G'$, we find $n_c = 50$ clusters with $k$-means. We train a classifier $C_{\text{dc}} = $ \MLP{r} on document embeddings $g'(x)$ to predict class, where $r$ is the number of classes (topics or sentiments). 

\subsection{Baselines}
We compare the performance of \technique\ with 4 baselines: \textbf{Non-private}, \textbf{Truncation}, \textbf{Word-level Metric-DP}, and \textbf{Random Guesser}. 

\textbf{Non-private:} This demonstrates the usefulness of non-private sentence-mean document embeddings $\overline{g}(x)$. We generate $\overline{g}(x)$ for every document using SBERT, and then train a classifier $C_{\text{nonpriv}} = $ \MLP{r} to predict $x$'s label from $\overline{g}(x)$. 

\textbf{Truncation:} We adopt the method from \citealt{clifton} to truncate (clip) sentence embeddings within a box in $\R^{768}$, thereby bounding sensitivity as described at the beginning of Section \ref{sec:deepcandidate}. Laplace noise is then added to each dimension. Documents with more sentences have proportionally less noise added due to the averaging operation reducing sensitivity. 

%\paragraph{Truncation:} The truncation baseline \needcite\ requires first constraining the embedding instance space. We do so by computing the 75\% median interval on each of the 768 dimensions of training document embeddings $T_G$. Sentence embeddings are truncated at each dimension to lie in this box. In order to account for this distribution shift, a new classifier $C_{\text{trunc}} = $ \MLP{r} is trained on truncated mean embeddings to predict class. The number of epochs is determined with the validation set. At test time, a document's sentence embeddings $S_x$ are truncated and averaged. We then add Laplace noise to each dimension with scale factor $\frac{768 w}{k \epsilon}$, where $w$ is the width of the box on that dimension (\emph{sensitivity} in DP terms). Note that the standard deviation of noise added is inversely proportional to the number of sentences in the document, due to the averaging operation reducing sensitivity. 

\textbf{Word Metric-DP (MDP):} The method from \citealt{metricdp} satisfies $\epsilon$-word-level metric DP by randomizing words. We implement MDP to produce a randomized document $x'$, compute $\overline{g}(x')$ with SBERT, and predict class using $C_{\text{nonpriv}}$. 

\textbf{Random Guess:} To set a bottom-line, we show the theoretical performance of a random guesser only knowing the distribution of labels. 

\subsection{Results \& Discussion} 
\textbf{How does performance change with privacy parameter $\epsilon$?}\\ 
This is addressed in Figures \ref{fig:eps:tnews} to \ref{fig:eps:imdb}. Here, we observe how the test set macro $F_1$ score changes with privacy parameter $\epsilon$ (a lower $\epsilon$ offers stronger privacy). Generally speaking, for local differential privacy, $\epsilon < 10$ is taken to be a strong privacy regime, $10 \leq \epsilon < 20$ is moderate privacy, and $\epsilon \geq 25$ is weak privacy. The \textbf{truncation} baseline mechanism does increase accuracy with increasing $\epsilon$, but never performs much better than the random guesser. This is to be expected with high dimension embeddings, since the standard deviation of noise added increases linearly with dimension. 

The word-level \textbf{MDP} mechanism performs significantly better than \textbf{truncation}, achieving relatively good performance for $\epsilon \geq 30$. There are two significant caveats, however. First, is the privacy definition: as discussed in the Introduction, for the same $\epsilon$, word-level MDP is strictly weaker than \SDP. 
%The MDP definition only guarantees that changing a single word with a similar word is $\epsilon$-indistinguishable to an adversary trying to recover the original document. Sentence-DP guarantees that adding, removing, or modifying \emph{any number of words in any way} in a given sentence is $\epsilon$-indistinguishable -- a significantly stronger guarantee. 
The second caveat is the level of $\epsilon$ at which privacy is achieved. Despite a weaker privacy definition, the MDP mechanism does not achieve competitive performance until the weak-privacy regime of $\epsilon$. We suspect this is due to two reasons. First, is the fact that the MDP mechanism does not take advantage of contextual information in each sentence as our technique does; randomizing each word independently does not use higher level linguistic information. Second, is the fact that the MDP mechanism does not use domain-specific knowledge as our mechanism does with use of relevant candidates and domain specific sentence encodings. 

In comparison, \technique\ offers strong utility across tasks and datasets for relatively low values of $\epsilon$, even into the strong privacy regime. Beyond $\epsilon = 25$, the performance of \technique\ tends to max out, approximately 10-15\% below the non-private approach. This is due to the fact that \technique\ offers a noisy version of an \emph{approximation} of the document embedding $\overline{g}(x)$ -- it cannot perform any better than deterministically selecting the deepest candidate, and even this candidate may be a poor representative of $x$. We consider this room for improvement, since there are potentially many other ways to tune $G'$ and select the candidate pool $F$ such that deep candidates are nearly always good representatives of a given document $x$. 

\noindent\textbf{How does performance change with the number of sentences $k$?}\\
This is addressed in Figures \ref{fig:k:tnews} to \ref{fig:k:imdb}. We limit the test set to those documents with $k$ in the listed range on the x-axis. We set $\epsilon = 10$, the limit of the strong privacy regime. Neither baseline offers performance above that of the random guesser at this value of $\epsilon$.  \technique\ produces precisely the performance we expect to see: documents with more sentences result in sampling higher quality candidates, confirming the insights of Section \ref{sec:tukey}. Across datasets and tasks, documents with more than 10-15 sentences tend to have high quality embeddings. 

\section{Conclusions and Future Work}
\vspace{-0.5em}
We introduce a strong and interpretable local privacy guarantee for documents, \SDP, along with \technique, a technique that combines principles from NLP and robust statistics to generate general $\epsilon$-\SDP\ embeddings. Our experiments confirm that such methods can outperform existing approaches even with with more relaxed privacy guarantees. Previous methods have argued that it is ``virtually impossible'' to satisfy pure local DP \cite{metricdp, mdp_low_dim} at the word level while capturing linguistic semantics. Our work appears to refute this notion at least at the document level. 

To follow up, we plan to explore other approaches (apart from $k$-means) of capturing the structure of the embedding distribution $\overline{g}(\mu)$ to encourage better candidate selection. We also plan to experiment with decoding private embeddings back to documents by using novel candidates produced by a generative model trained on $F$. 

% \clearpage

\section*{Acknowledgements} 
KC and CM would like to thank ONR under N00014-20-1-2334. KM gratefully acknowledges funding from an Amazon Research Award and Adobe Unrestricted Research Gifts. We would would also like to thank our reviewers for their insightful feedback.
%Questions are 
%\begin{enumerate}
%	\item How does performance compare w/ baselines as we change privacy parameter $\epsilon$.  
%	\item How does performance compare w/ baselines as we change number of sentences $k$ in the private document. 
%\end{enumerate}
%
%\cm{First subsection is datasets, maybe a table. Tran/val/test splits and explain what the dataset is (sentiment / classification). } \\
%\cm{Second subsection: training details \& setup. Talk about SBERT and MLP layer shit and dimension and train with this kind of loss for this many epochs with this particular optimizer. Don't mention SBERT earlier just here. Number of parameters in MLP.} \\
%\cm{In second subsection: setup: e.g. ``for all experiments we get the frozen embeddings'' we select the best MLP classifier based on da da da. We use Macro F1 for metric. } \\
%\cm{Results \& Discussion: show questions here. talk about how the figures answer the questions. } \\


\graphicspath{{./chapters/chapter4/}}
\chapter{Privacy Implications of Shuffling}

\newcommand{\arc}[1]{{\color{blue} \emph{[[ARC: #1]]}}}

%Macros: 
%\newcommand{\ba}{\mathbf{a}}
\newcommand{\bA}{\mathbf{A}}
%\newcommand{\bx}{\mathbf{x}}
\newcommand{\by}{\mathbf{y}}
\newcommand{\bk}{\mathbf{k}}
\newcommand{\bp}{\mathbf{p}}
\newcommand{\bw}{\mathbf{w}}
\newcommand{\bmm}{\mathbf{m}}
\newcommand{\bz}{\mathbf{z}}
%\newcommand{\bb}{\mathbf{b}}
\newcommand{\bv}{\mathbf{v}}
%\newcommand{\bx}{\mathbf{x}}
\newcommand{\bt}{\mathbf{t}}
%\newcommand{\by}{\mathbf{y}}
%\newcommand{\bz}{\mathbf{z}}
\newcommand{\bX}{\mathbf{X}}
\newcommand{\bH}{\mathbf{H}}
\newcommand{\bS}{\mathbf{S}}
\newcommand{\bs}{\mathbf{s}}
\newcommand{\bU}{\mathbf{U}}
\newcommand{\bu}{\mathbf{u}}
\newcommand{\bW}{\mathbf{W}}
\newcommand{\bY}{\mathbf{Y}}
\newcommand{\bZ}{\mathbf{Z}}
\newcommand{\bmu}{\pmb{\mu}}
\newcommand{\bSigma}{\pmb{\Sigma}}
\newcommand{\bLambda}{\pmb{\Lambda}}
\newcommand{\btheta}{\mathbf{\theta}}
\newcommand{\bzro}{\mathbf{0}}
%\newcommand{\trace}{\textbf{tr}}
\newcommand{\hess}{\textbf{H}}
% \newcommand{\R}{\mathbb{R}}
% \newcommand{\E}{\mathbb{E}}
%\newcommand{\N}{\mathcal{N}}
%\newcommand{\var}{\textbf{var}}
\newcommand{\mse}{\textbf{MSE}}
\newcommand{\so}{\text{, so }}
% \newcommand{\KL}{\textbf{KL}}
%\newcommand{\I}{\textbf{I}}

\newcommand{\calZ}{\mathcal{Z}}
%\newcommand{\calS}{\mathcal{S}}
\newcommand{\calL}{\mathcal{L}}
%\newcommand{\calX}{\mathcal{X}}
\newcommand{\calT}{\mathcal{T}}
\newcommand{\calE}{\mathcal{E}}
\newcommand{\calP}{\mathcal{P}}
%\newcommand{\calA}{\mathcal{A}}
\newcommand{\calG}{\mathcal{G}}
\newcommand{\calI}{\mathcal{I}}
\newcommand{\calY}{\mathcal{Y}}

%Theorems and stuff 
\theoremstyle{plain}
%\newtheorem{thm}{Theorem} 
%\newtheorem{lemma}{Lemma} 
\newtheorem{prope}{Property}
%\newtheorem{corollary}{Corollary}
%\newtheorem{prop}{Proposition} 

\theoremstyle{definition}
%\newtheorem{defn}{Definition}
%\newtheorem{exmp}{Example} 
\newtheorem{sett}{Setting} 

\newtheorem{defn}{Definition}[section]
\newtheorem{conj}{Conjecture}[section]
\newtheorem{exmp}{Example}[section]



%% Amrita
\newcommand{\DP}{\textsf{DP}~}
\newcommand{\ldp}{\textsf{LDP}~}
\newcommand{\DO}{\textsf{DO}}
\newcommand{\name}{$d_\sigma$}

\newcommand{\squishlistfour}{
	\begin{list}{$\bullet$}
		{
			\setlength{\itemsep}{0pt}
			\setlength{\parsep}{3pt}
			\setlength{\topsep}{3pt}
			\setlength{\partopsep}{0pt}
			\setlength{\leftmargin}{1em}
			\setlength{\labelwidth}{0.5em}
			\setlength{\labelsep}{0.5em} } }
	
\newcommand{\squishendfour}{
\end{list}  }
\newcommand{\squishlistnum}{
	\begin{enumerate}
		{
			\setlength{\itemsep}{0pt}
			\setlength{\parsep}{3pt}
			\setlength{\topsep}{3pt}
			\setlength{\partopsep}{0pt}
			\setlength{\leftmargin}{1.5em}
			\setlength{\labelwidth}{1em}
			\setlength{\labelsep}{0.5em} } }
	
\newcommand{\squishendnum}{
\end{enumerate}  }

\renewenvironment{quote}{%
   \list{}{%
     \leftmargin0.2cm   % this is the adjusting screw
     \rightmargin\leftmargin
   }
   \item\relax
}
{\endlist}

%\newcommand{\al}{\mathcal{M}_{\mathbb{X}}
%\newcommand{\indep}{$\perp \!\!\! \perp$}
% \newcommand{\squishlist}{
% 	\begin{list}{$\bullet$}
% 		{
% 			\setlength{\itemsep}{0pt}
% 			\setlength{\parsep}{0.5pt}
% 			\setlength{\topsep}{0.5pt}
% 			\setlength{\partopsep}{0pt}
% 			\setlength{\leftmargin}{0.1em}
% 			\setlength{\labelwidth}{0.1em}
% 			\setlength{\labelsep}{0.2em} } }
	
% \newcommand{\squishend}{
% \end{list}  }\newcommand{\squishlistnum}{
% 	\begin{enumerate}
% 		{
% 			\setlength{\itemsep}{0pt}
% 			\setlength{\parsep}{3pt}
% 			\setlength{\topsep}{1pt}
% 			\setlength{\partopsep}{0pt}
% 			\setlength{\leftmargin}{1.25em}
% 			\setlength{\labelwidth}{1em}
% 			\setlength{\labelsep}{0.5em} } }
	
% \newcommand{\squishendnum}{
% \end{enumerate}  }

\section{Introduction}
\label{sec:intro}
Differential Privacy (\textsf{DP}) and its local variant (\textsf{LDP}) are the most commonly accepted notions of data privacy. \ldp has the significant advantage of not requiring a trusted centralized aggregator, and has become a popular model for commercial deployments, such as those of Microsoft~\citep{Microsoft}, Apple~\citep{Apple}, and Google~\citep{Rappor1,Rappor2,Prochlo}. Its formal guarantee asserts that an adversary cannot infer the value of an individual's private input by observing the noisy output. However in practice, a vast amount of \textit{public auxiliary information}, such as address, social media connections, %(in terms of friends/followers), 
court records, property records, %\citep{home},
income and birth dates \citep{birth}, is available for every individual. An adversary, with access to such auxiliary information, \emph{can} learn about an individual's private data from several \emph{other} participants' noisy responses. We illustrate this as follows.%\vspace{-0.1cm}Its formal guarantee asserts that an adversary does not learn much about a participant's private input after observing the mechanism's noisy representation of it. However, an adversary \emph{can} draw reliable inferences about a participant's private input after observing several \emph{other} participants' noisy responses. Practically speaking, this is due to the fact that \ldp does not formally hide the identity of data owners: e.g. an \ldp response from a participant's browser can still be linked to that participant's device or account and thereby to their identity. With the preponderance of public auxiliary information available today (e.g. address, social media connections, court records, property records \nocite{home}, income and birth rates \citep{birth}), an adversary can track down other participants who (positively or negatively) correlate with them and leverage those responses to make a reliable inference of their underlying private input. We illustrate this as follows: 

% However in practice, a vast amount of \textit{public auxiliary information}, such as address, social media connections, %(in terms of friends/followers), 
% court records, property records \cite{home}, income and birth dates \citep{birth}, is available for every individual. An adversary with access to such auxiliary information \emph{can} learn about an individual's private data from several \emph{other} participants' noisy responses. We illustrate this as follows.%\vspace{-0.1cm}

\begin{tcolorbox}\vspace{-0.25cm}
\textbf{Problem.} An analyst runs a medical survey in Alice's community to investigate how the prevalence of a highly contagious disease changes from neighborhood to neighborhood. Community members report a binary value indicating whether they have the disease.  \vspace{-0.2cm}
\end{tcolorbox}\vspace{-0.1cm}
Next, consider the following two data reporting strategies. \vspace{-0.1cm}
\begin{tcolorbox}\vspace{-0.2cm} \textbf{Strategy $\mathbf{1}$.} Each data owner passes their data through an appropriate randomizer (that flips the input bit with some probability) in their local devices and reports the noisy output to the untrusted data analyst. 
\vspace{-0.2cm}
\end{tcolorbox}\vspace{-0.1cm}
\begin{tcolorbox}\vspace{-0.2cm} \textbf{Strategy $\mathbf{2}$.}  The noisy responses from the local devices of each of the data owners %\footnote{The individual  responses are appropriately anonymized by removing all identifiable information such as IP address of packets.} 
are collected by an intermediary trusted shuffler which dissociates the device IDs (metadata) from the responses and  uniformly randomly shuffles them before sending them to the analyst.\vspace{-0.2cm}
\end{tcolorbox}\vspace{-0.1cm}
\textbf{Strategy $\mathbf{1}$} corresponds to the standard \ldp deployment model (for example, Apple and Microsoft's deployments). Here \textit{the order of the noisy responses is informative of the identity of the data owners} -- the noisy response at index $1$ corresponds to the first data owner and so on. Thus, the noisy responses can be directly linked with its associated device/account ID and subsequently, auxiliary information. This puts Alice's data under the threat of inference attacks.
For instance, an adversary\footnote{The analyst and the adversary could be same, we refer to them separately for the ease of understanding.} may know the home addresses of the participants and use this to identify the responses of all the individuals from Alice's household.  
Being highly infectious, all or most of them 
 %\vspace{-0.4cm} 
\begin{figure*}[t]
\begin{minipage}{0.2\linewidth}
\begin{subfigure}[b]{\linewidth}
\centering
    \includegraphics[width=0.9\columnwidth]{./figures/data_radii.png}
        \caption{Original Data}
        \label{fig:data}
    \end{subfigure}
    \end{minipage}
    \begin{minipage}{0.8\linewidth}
    \begin{subfigure}[b]{0.25\linewidth}\centering
    \includegraphics[width=0.67\columnwidth]{./figures/LDP.png}\vspace{-0.25cm}
        \caption{\ldp}
        \label{fig:ldp}
    \end{subfigure}%%
    \begin{subfigure}[b]{0.25\linewidth}\centering
    \includegraphics[width=0.67\columnwidth]{./figures/Shuffle_r1.png}\vspace{-0.25cm}
        \caption{Our scheme: $r_a$}
        \label{fig:r}\end{subfigure}%%
    \begin{subfigure}[b]{0.25\linewidth}\centering
    \includegraphics[width=0.67\columnwidth]{./figures/shuffle_r2.png}\vspace{-0.25cm}
        \caption{Our scheme: $r_b$}
        \label{fig:R}\end{subfigure}%%
      \begin{subfigure}[b]{0.25\linewidth}\centering
    \includegraphics[width=0.67\columnwidth]{./figures/Uniform.png}\vspace{-0.25cm}
        \caption{Uniform shuffle}
        \label{fig:uniform}
    \end{subfigure}\\ \vspace{-0.05cm}
    \begin{subfigure}[b]{0.25\linewidth}\centering
    \includegraphics[width=0.67\columnwidth]{./figures/LDP_data_1.png}\vspace{-0.25cm}
        \caption{Attack: LDP}
        \label{fig:ldp:attack}
    \end{subfigure}%%
    \begin{subfigure}[b]{0.25\linewidth}\centering
  \includegraphics[width=0.67\columnwidth]{./figures/r1_attack.png}\vspace{-0.25cm}
       \caption{Attack: $r_a$}
        \label{fig:r:attack}\end{subfigure}%%
    \begin{subfigure}[b]{0.25\linewidth}\centering
   \includegraphics[width=0.67\columnwidth]{./figures/r2_attack.png}\vspace{-0.25cm}
        \caption{Attack: $r_b$}
        \label{fig:R:attack}\end{subfigure}%%
      \begin{subfigure}[b]{0.25\linewidth}\centering
   \includegraphics[width=0.67\columnwidth]{./figures/attack_uniform.png}\vspace{-0.25cm}
        \caption{Attack: unif. shuff.}
        \label{fig:uniform:attack}
    \end{subfigure}
 \end{minipage}\vspace{-0.15cm}
    \caption{Demonstration of how our proposed scheme thwarts inference attacks at different granularities. Fig. \ref{fig:data}  depicts the original sensitive data (such as income bracket) with eight color-coded labels. The position of the points represents public information (such as home address) used to correlate them. There are three levels of granularity: warm vs. cool clusters, blue vs. green and red vs. orange crescents, and light vs. dark within each crescent. Fig. \ref{fig:ldp} depicts \scalebox{0.9}{$\epsilon = 2.55$} \textsf{LDP}. Fig. \ref{fig:r} and \ref{fig:R} correspond to our scheme, each with $\alpha = 1$ (privacy parameter, Def. \ref{def:privacy}). The former uses a smaller  distance threshold ($r_1$, used to delineate the granularity of grouping -- see Sec. \ref{sec:privacy:def}) that mostly shuffles in each crescent. The latter uses a larger distance threshold ($r_2$) that shuffles within each cluster. Figures in the bottom row  demonstrate an inference attack (uses Gaussian process correlation) %with a fixed length scale) 
    on all four cases. We see that  \ldp  reveals almost the entire dataset (Fig. \ref{fig:ldp:attack}) while uniform shuffling prevents all classification (\ref{fig:uniform:attack}). However, the granularity can be controlled with our scheme (Figs. \ref{fig:r:attack}, \ref{fig:R:attack}). \vspace{-2em}}
   \label{fig:demonstration}
%   \vspace{-0.15cm}
    \end{figure*}
  will have the same true value ($0$ or $1$). Hence, the adversary can reliably infer Alice's value by taking a simple majority vote of her and her household's noisy responses. Note that this does not violate the \ldp guarantee since the inputs are appropriately randomized when observed in isolation. Additionally, on account of being public, the auxiliary information is known to the adversary (and analyst) \emph{a priori} -- no mechanism can prevent their disclosure. For instance, any attempts to include Alice's address as an additional feature of the data and then report via \ldp is \emph{futile} --   the adversary would simply discard the reported noisy address and use the auxiliary information about the exact addresses to identify the responses of her household members. We call such threats \emph{inference attacks} -- recovering an individual's private input using all or a subset of other participants' noisy responses. It is well known that protecting against inference attacks that rely on underlying data correlations is beyond the purview of \DP  \citep{Pufferfish, sok}. 
  
\textbf{Strategy 2} corresponds to the recently introduced shuffle \DP model, such as Google's Prochlo \citep{Prochlo}.  Here, the noisy responses are completely anonymized -- the adversary cannot identify which \ldp responses correspond to Alice and her household.  Under such a model, only information that is completely order agnostic (i.e., symmetric functions that can be computed over just the \textit{bag} of values, such as aggregate statistics) can be extracted. Consequently, the analyst also fails to accomplish their original goal as all the underlying data correlation is destroyed.
 
Thus, we see that the two models of deployment for \ldp present a trade-off between vulnerability to inference attacks and scope of data learnability. In fact, as demonstrated in \cite{Kifer}, it is impossible to defend against \emph{all} inference attacks while simultaneously maintaining utility for learning. In the extreme case that the adversary knows \emph{everyone} in Alice's community has the same true value (but not which one), no mechanism can prevent revelation of Alice's datapoint short of destroying all utility of the dataset. This then begs the question: \textbf{\emph{Can we formally suppress \underline{specific} inference attacks targeting each data owner while maintaining some meaningful learnability of the private data?}} Referring back to our example, can we thwart attacks inferring Alice's data using specifically her households' responses and still allow the medical analyst to learn its target trends? Can we offer this to every data owner participating?
 
%In this paper, we strike a balance and we propose a generalized shuffle framework for deployment that can interpolate between the two extremes.
In this paper, we strike a balance and propose a generalized shuffle framework that meets the utility requirements of the above analyst while formally protecting data owners against inference attacks. 
Our solution is based on the key insight: \textit{the order of the data acts as the proxy for the identity of data owners} as illustrated above. The granularity at which the ordering is maintained formalizes resistance to inference attacks while retaining some meaningful learnability of the private data. Specifically, we guarantee each data owner that their data is shuffled together with a carefully chosen group of other data owners. Revisiting our example, consider uniformly shuffling the responses from Alice's household and her immediate neighbors. Now an adversary cannot use her household's responses to predict her value any better than they could with a random sample of responses from this group. 
In the same way that \ldp prevents reconstruction of her datapoint using specifically \emph{her} noisy response, this scheme prevents reconstruction of her datapoint using specifically \emph{her households'} responses. The real challenge is offering such guarantees \textit{equally} to \textit{every} data owner. Bob, Alice's neighbor, needs his households' responses shuffled in with his neighbors, as does Luis who is a neighbor of Bob but \textit{not} of Alice. Thus, we have $n$ data owners with $n$ distinct, overlapping groups. Our scheme supports arbitrary groupings (overlapping or not), introducing a diverse and tunable class of privacy/utility trade-offs which is not attainable with either \ldp or uniform shuffling alone.
%This disallows the trivial strategy of shuffling the noisy responses of each group uniformly. To this end, we propose shuffling the responses in a systematic manner that tunes the privacy guarantee, trading it off with data learnability. 
For the above example, our scheme can formally protect each data owner from inference attacks using specifically their household, while still learning how disease prevalence changes across the neighborhoods of Alice's community.\\
This work offers two key contributions to the machine learning privacy literature: 
 \vspace{-0.2cm}
\squishlistfour
    \item \textbf{Novel privacy guarantee.} We propose a novel privacy definition, \name-privacy that captures the privacy of the \textit{order} of a data sequence (Sec. \ref{sec:privacy:def}) and formalizes the degree of resistance against inference attacks (Sec. \ref{sec:privacy:implications}). \name-privacy allows assigning an arbitrary group, $G_i$, to each data owner, $\DO_i, i \in [n]$. For instance, the groups can represent individuals in the same age bracket, `friends' on social media, or individuals living in each other's vicinity (as in case of Alice in our example).  Recall that the order is informative of the data owner's identity. Intuitively,  \name-privacy protects $\DO_i$ from inference attacks that arise from knowing the \textit{identity} of the members of their group $G_i$ (Sec. \ref{sec:privacy:implications}). %\textcolor{blue}{By shuffling within each group, \name-privacy protects $\DO_i$ against inference attacks that utilize the data of any subset of the members of $G_i$, while still revealing the statistics of each group to the analyst.} 
    %The group assignment is based on a public auxiliary information  -- individuals of a single group are `similar' w.r.t the auxiliary information. 
    Additionally, this grouping determines a threshold of learnability -- any learning that is order agnostic within a group (disease prevalence in a neighborhood -- the data analyst's goal in our example) is utilitarian and allowed; whereas analysis that involves identifying the values of individuals within a group (disease prevalence within specific households -- the adversary's goal) is regarded as a privacy threat and protected against.
 See Fig. \ref{fig:demonstration} for a toy demonstration of how our guarantee allows \textit{tuning the granularity at which trends can be learned}. 
    
    \item \textbf{Novel shuffle framework.} We  propose a novel mechanism that shuffles the data systematically and achieves \name-privacy. This 
   provides a generalized shuffle framework that interpolates between no shuffling (\textsf{LDP}) and uniform random shuffling (shuffle model) in terms of protection against inference attacks and data learnability.
    %\textcolor{blue}{allows analysis of subsets of the reported data (such as within neighborhoods across a city) while 
    %Our experimental results (Sec. \ref{sec:eval}) demonstrates its efficacy against realistic inference attacks.  
\squishendfour 
  \vspace{-0.4cm} 
%   \arc{TO-DOs \\1)Include discussion about viewing a sequence as <bag of val, order> 2) discussion about side information - why is it immutable 3) highlight privacy guarantee - users choose groups which should be overlapping for generality  }
\section{Related Work}\label{sec:related_work} \vspace{-0.3cm} 
The shuffle model of \DP \citep{Bittau2017,shuffle2,shuffling1} differs from our scheme as follows. These works $(1)$ study \DP benefits of shuffling whereas we study the inferential privacy benefits, and $(2)$ only study uniformly random shuffling where ours generalizes this to tunable, non-uniform shuffling (see App. \ref{app:related}). 
% Where these works study the differential private benefits of shuffling (lower $\epsilon$), our work studies its inferential privacy benefits. Furthermore, those works only consider uniform random shuffling of the dataset, where our approach  allows configurable, non-uniform shuffling distributions that allow us to tune the tradeoff between inferential risk and the granularity of learnable trends.
% These works differ from our approach in two ways. First, they only study shuffling as a means to amplify local differentially private guarantees, effectively offering a lower $\epsilon$ when viewed from the alternative, central \DP model. Our results cater to local \emph{inferential} privacy (Sec. \ref{sec:ip}), limiting what can be inferred about one individual from other individuals' responses. Second, the shuffle model only studies uniform random shuffling of the dataset. In contrast, our approach allows configurable, non-uniform shuffling distributions that allow us to tune the tradeoff between inferential risk and the granularity of learnable trends. 
\\A steady line of work has studied inferential privacy  \citep{semantics, Kifer,  IP, Dalenius:1977, dwork2010on, sok}. %Kifer et al. \cite{Kifer} formally studied privacy degradation in the face of data correlations and later proposed a  privacy framework, Pufferfish \cite{Pufferfish, Song,Blowfish}, for analyzing inferential privacy. 
%Subsequently, several other privacy definitions have also been proposed for the inferential privacy setting \cite{DDP,BDP,correlated,correlated2,CWP}. 
Our work departs from those in that we focus on \emph{local} inferential privacy and do so via the new angle of shuffling. 
\\Older works such as $k$-anonymity \citep{kanon},  $l$-diversity \cite{ldiv}, Anatomy \citep{anatomy} and others \citep{older1, older2, older3, older4, older5} have studied the privacy risk of non-sensitive auxiliary information or `quasi identifiers'. These works $(1)$ focus on the setting of dataset release, whereas we focus on dataset collection, and $(2)$ do not offer each data owner formal inferential guarantees, whereas we do. %As such, QIs can be manipulated and controlled, whereas we place no restriction on the amount or type of auxiliary information accessible to the adversary, nor do we control it. 
%Additionally, our work offers each individual formal inferential guarantees against informed adversaries, whereas those works do not. 
The De Finetti attack \citep{definetti} shows how shuffling schemes are vulnerable to inference attacks that correlate records to recover the original permutation of sensitive attributes. A strict instance of our privacy guarantee can thwart such attacks (at the cost of no utility, App. \ref{app:de finetti}). 
\section{Background}\label{sec:background} 
\textbf{Notations.} \textbf{Boldface} (such as $\bx=\langle x_1, 
 \cdots, x_n\rangle$) denotes a data sequence (ordered list); normal font (such as $x_1$) denotes individual values and $\{\cdot\}$ represents a multiset or bag of values.\vspace{-0.3cm}
\subsection{Local Differential Privacy}\label{sec:ldp}
 \vspace{-0.2cm}The local model consists of a set of data owners and an untrusted data aggregator (analyst); each individual perturbs their data using a \ldp algorithm (randomizers) and sends it to the analyst. % which uses these noisy responses to glean information about the entire dataset.
The \ldp guarantee is formally defined as
\begin{defn}\vspace{-0.1cm}[Local Differential Privacy, \ldp \cite{Warner,Evfimievski:2003:LPB:773153.773174,Kasivi}]
 A randomized algorithm $\mathcal{M} : \mathcal{X} \rightarrow \mathcal{Y}$ is $\epsilon$-locally differentially private (or $\epsilon$-\ldp), if for any pair of private values $x, x' \in \mathcal{X}$ and any subset of output, %$\mathcal{W} \subseteq \mathcal{Y}$ we have $\mathrm{Pr}\big[\mathcal{M}(x) \in \mathcal{W}\big] \leq e^{\epsilon} \cdot \mathrm{Pr}\big[\mathcal{M}(x') \in \mathcal{W}  \big]$.
 %\end{defn}
 \begin{gather}
 \mathrm{Pr}\big[\mathcal{M}(x) \in \mathcal{W}\big] \leq e^{\epsilon} \cdot \mathrm{Pr}\big[\mathcal{M}(x') \in \mathcal{W}  \big]
  \end{gather}
  \end{defn}
The shuffle model is an extension of the local model where the data owners first randomize their inputs. Additionally, an intermediate trusted shuffler applies a \textit{uniformly random permutation} to all the noisy responses before the analyst can view them. The anonymity provided by the shuffler requires less noise than the local model for achieving the same
privacy.  
%%%%%%%%%%%%%%%%%%%%%%%%%%%%%
%INFERENTIAL PRIVACY BACKGROUND 
%%%%%%%%%%%%%%%%%%%%%%%%%%%%%

% \subsection{Local Inferential Privacy}  \vspace{-0.2cm}
% %introduce pufferfish inferntial log loss 
% %In this section, we introduce some context for inferential privacy in the \ldp setting. 
% %Inferential privacy captures the privacy loss in the face of an informed adversary in a Bayesian framework. 
% Local inferential privacy captures what information a Bayesian adversary \cite{Pufferfish}, with some prior, can learn in the \ldp setting. 
% Specifically, it measures the largest possible ratio between the adversary's posterior and prior beliefs about an individual’s data after observing a mechanism's output .%\footnote{This quantity is identical to the \ldp parameter of the mechanism when\textit{individuals’ data are independent}\cite{sok,}.}.
% \begin{defn}(Local Inferential Privacy Loss \cite{Pufferfish}) Let $\bx=\langle x_1, \cdots, x_n\rangle$ and let $\by=\langle y_1, \cdots, y_n \rangle$ denote the input (private) and output sequences (observable to the adversary) in the \ldp setting. Additionally, the adversary's auxiliary knowledge is modeled by a prior distribution $\mathcal{P}$ on $\mathbf{x}$. The inferential privacy loss for the input sequence $\mathbf{x}$ is given by
% % \vspace{0cm} 
% \begin{equation}
% % \vspace{-0.1cm}
% \small \mathbb{L}_{\calP}(\mathbf{x})=\max_{\substack{i\in [n]\\ a,b \in \calX}}\Bigg(\log\frac{\mathrm{Pr}_{\calP}[\mathbf{y}|x_i=a]}{\mathrm{Pr}_{\calP}[\mathbf{y}|x_i=b]}\Bigg)
% =\max_{\substack{i\in [n]\\ a,b \in \calX}}\Bigg (	\bigg| \log \frac{\mathrm{Pr}_{\calP}[x_i = a | \bf{y} ]}{\mathrm{Pr}_{\calP}[x_i = b | \bf{y}]}
% 	- \log \frac{\mathrm{Pr}_{\calP}[x_i = a]}{\mathrm{Pr}_{\calP}[x_i = b]} \bigg|\Bigg)
% \end{equation}
% \label{def:ip}
% \vspace{-1em}
% \end{defn}
% % Using Bayes' theorem, we have\vspace{-0.2cm}
% % \begin{gather*}\small\vspace{-0.4cm} \mathbb{L}_{\calP}(\mathbf{x})=\max_{\substack{i\in [n]\\ a,b \in \calX}}\Bigg (	\bigg| \log \frac{\mathrm{Pr}_{\calP}[x_i = a | \bf{y} ]}{\mathrm{Pr}_{\calP}[x_i = b | \bf{y}]}
% % 	- \log \frac{\mathrm{Pr}_{\calP}[x_i = a]}{\mathrm{Pr}_{\calP}[x_i = b]} \bigg|\Bigg)\vspace{-0.7cm}\end{gather*}
% Bounding  $\mathbb{L}_{\calP}(\mathbf{x})$  would imply
%  that the adversary's belief about the value of any $x_i$ does not change by much even after observing the output sequence $\bf{y}$. This means that an informed adversary does not learn much about the individual $i$'s private input upon observation of the entire private dataset $\by$.

%%%%%%%%%%%%%%%%%%%%%%%%%%%%%
%END INFERENTIAL PRIVACY BACKGROUND 
%%%%%%%%%%%%%%%%%%%%%%%%%%%%%

\subsection{Mallows Model} \label{sec:background:MM}
A permutation of a set $S$ is a bijection $S\mapsto S$. The set of permutations of $[n], n \in \mathbb{N}$ forms a symmetric group $\mathrm{S}_n$. As a shorthand, we use $\sigma(\bx)$ to denote applying permutation $\sigma \in \mathrm{S}_n$ to a data sequence $\bx$ of length $n$. Additionally, $\sigma(i), i\in [n], \sigma \in \mathrm{S}_n$ denotes the value at index $i$ in $\sigma$ and $\sigma^{-1}$ denotes its inverse. For example, if $\sigma=( 1 \: 3 \: 5\: 4\: 2)$ and $\bx=\langle 21, 33, 45, 65 , 67\rangle$, then $\sigma(\bx)=\langle 21, 45, 67, 65, 33\rangle$, $\sigma(2)=3, \sigma(3)=5$ and $\sigma^{-1}=(1 \: 5 \: 2 \: 4 \: 3)$.
\\Mallows model is a popular probabilistic model for permutations \citep{MM}.  %It is an exponential location model and is usually referred to as the Gaussian distribution for permutations. 
The mode of the distribution is given
by the reference permutation $\sigma_0$ -- the probability of a permutation increases as we move `closer' to $\sigma_0$ as measured by rank distance metrics, such as the Kendall's tau distance (Def. \ref{def:kendall}). The dispersion parameter $\theta$ controls how fast this increase happens.  %The formal definition of the Mallows model is as follows.

%\begin{wrapfigure}{r}{0.4\textwidth}
%%    \vspace{-1.5cm}
%    % \hspace{-1cm}
%    \begin{center}
%    \includegraphics[width=0.38\textwidth]{figures/shuffle_image.png} 
%    \end{center}
%%    \vspace{-1em}
%    \caption{\small{Trusted shuffler mediates on $\by$}} 
%%    \vspace{-2.5em}
%    \label{fig:problemsetting}
%\end{wrapfigure}

\begin{defn}
\label{def: mallows}
For a dispersion parameter $\theta$, a reference permutation $\sigma_o \in \mathrm{S}_n$, and a rank distance measure $\textswab{d}: \mathrm{S}_n \times \mathrm{S}_n \mapsto \R $,   
$
\mathbb{P}_{\Theta,\textswab{d}}(\sigma:\sigma_0)=\frac{1}{\psi(\theta,\textswab{d})} e^{-\theta  \textswab{d}(\sigma, \sigma_0)}$
is the Mallows model where $\psi(\theta,\textswab{d})=\sum_{\sigma \in \mathrm{S}_n} e^{-\theta \textswab{d}(\sigma,\sigma_0)}$ is a normalization term and  $\sigma \in \mathrm{S}_n$.
\end{defn}






%\vspace{-0.5cm}
\section{Data Privacy and Shuffling}\vspace{-0.2cm}

\begin{wrapfigure}{r}{0.44\textwidth}
	\vspace{-3em} 
%     \hspace{-1cm}
%    \begin{center}
    \includegraphics[width=0.43\textwidth]{figures/shuffle_image.png} 
%    \end{center}
    \caption{\small{Trusted shuffler mediates on $\by$}} 
    \label{fig:problemsetting}
    \vspace{-3em} 
\end{wrapfigure}

In this section, we present \name-privacy and a shuffling mechanism capable of achieving the \name-privacy guarantee. 
%First, we describe the problem setting. Next, we present our novel privacy definition,  \name-privacy, followed by a semantic understanding of its privacy implications. A utility metric for shuffling mechanisms is presented next. Finally, we introduce a novel shuffling  mechanism capable of achieving the \name-privacy guarantee. 
\vspace{-0.3cm}
\subsection{Problem Setting} \vspace{-0.1cm}

In our problem setting, we have $n$ data owners $\DO_i, i \in [n]$ each with a private input $x_i \in \mathcal{X}$ (Fig. \ref{fig:problemsetting}).  
 The data owners first randomize their inputs via a $\epsilon$-\ldp mechanism to generate $y_i=\mathcal{M}(x_i)$. 
 %I think we can cut out anything about the adv having side information until the experiments? It's just distracting reviewers? 
 %We consider an informed adversary with public auxiliary information $\mathbf{t}=\langle t_1, \cdots, t_n \rangle, t_i \in \mathcal{T}$ about each individual. 
 Additionally, just like in the shuffle model, we have a trusted shuffler. It mediates upon the noisy responses $\mathbf{y}=\langle y_1,\cdots,y_n \rangle$ 
 %and systematically shuffles them based on $\bt$ (since $\bt$ is public, it is also accessible to the shuffler) 
 to obtain the final output sequence $\bf{z}=\mathcal{A}(\bf{y})$ ($\mathcal{A}$ corresponds to Alg. 1) 
 which is sent to the untrusted data analyst. The shuffler can be implemented via trusted execution environments (TEE) just like Google's Prochlo. 
 %The underlying data correlations in $\mathbf{t}$ is modeled as a prior distribution $\calP$ on $\bf{x}$. 
 Next, we formally discuss the notion of order and its implications. 
\begin{defn}(Order) The order of a sequence $\mathbf{x}=\langle x_1,\cdots, x_n\rangle$ refers to the indices of its set of values $\{x_i\}$ and is represented by permutations from $\mathrm{S}_n$.\vspace{-0.2cm}\end{defn} 
% \begin{figure}
%     \centering
%     \begin{subfigure}[b]{0.49\textwidth}
%          \centering
%          \includegraphics[height = 3.5cm]{shuffle_image.png}
%          \caption{Trust model (similar to shuffle model)}
%          \label{fig:problemsetting}
%      \end{subfigure}
%      \begin{subfigure}[b]{0.49\textwidth}
%          \centering
%          \includegraphics[height = 3.5cm]{graph.png}
%          \caption{An example social media connectivity graph $\bt_{e.g}$}% that acts as the public auxiliary information.}
%          \label{fig:example}
%      \end{subfigure}
%      \caption{how data is privately collected (a) in the face of auxiliary information (b) that can be leveraged to correlate \ldp responses $\langle y_1, \dots, y_n \rangle$}
% \end{figure}

When the noisy response sequence $\mathbf{y}=\langle y_1, \cdots, y_n\rangle$ is represented by the identity permutation $\sigma_{I}=(1 \: 2 \: \cdots \: n)$, the value at index $1$ corresponds to $\DO_1$ and so on. Standard \ldp releases the identity permutation w.p. 1. The output of the shuffler, $\bf{z}$, is some permutation of the sequence $\bf{y}$, i.e.,
% \vspace{-0.5em}
\begin{align*}
\mathbf{z}=\sigma(\by)=
\langle y_{\sigma(1)},\cdots,y_{\sigma(n)}\rangle
%\vspace{-1em}
\end{align*}
where $\sigma$ is determined via $\calA(\cdot)$. For example, for $\sigma=(4 \: 5\: 2 \:3 \: 1)$, we have $\mathbf{z}=\langle y_4, y_5, y_2, y_3, y_1\rangle$ which means that the value at index $1$ ($\DO_1$) now corresponds to that of $\DO_4$ and so on.
%\arc{define analyst ' goal concretely here} \vspace{-0.3cm}

%\textcolor{blue}{The shuffler's distribution over permutations $\sigma$ satisfies \name-privacy by thoroughly shuffling within each data owner's group $G_i$ (formalized in the following section). The analyst, meanwhile, wishes to indefinitely query the statistics of \ldp values at the level of a group or union of groups. To enhance their utility, we optimize the mechanism to preserve statistics within groups (i.e. minimize the extent to which \ldp values shuffle between groups). More on this in Section \ref{sec:mechanism}. In this way, every choice of grouping results in a unique privacy/utility tradeoff. }
   \vspace{-0.1cm}
\subsection{Definition of \name-privacy}\label{sec:privacy:def}
 \vspace{-0.3cm}
 
 \begin{wrapfigure}{r}{0.3\linewidth}
    \centering
    \vspace{-1.5em}
    \includegraphics[height=2cm]{./figures/graph.png}
    \vspace{-1.25em}
    \caption{An example social media connectivity graph $\bt_{e.g}$}
    \vspace{-2em}
    \label{fig:example}
\end{wrapfigure}
 
% \arc{Make groups more general}
Inferential risk captures the threat of an adversary who infers $\DO_i$'s private $x_i$ using all or a subset of other data owners' released $y_j$'s. Since we cannot prevent all such attacks and maintain utility, our aim is to formally limit \emph{which data owners} can be leveraged in inferring $\DO_i$'s private $x_i$. To make this precise, each $\DO_i$ may choose a corresponding group, $G_i \subseteq [n]$, of data owners.
 \name-privacy guarantees that $y_j$ values originating from a data owner's group $G_i$ are shuffled together. In doing so, the \ldp values corresponding to subsets of $\DO_i$'s group $I \subset G_i$ cannot be reliably identified, and thus cannot be singled out to make inferences about $\DO_i$'s $x_i$. If Alice's group includes her whole neighborhood, \ldp data originating from her household cannot be singled out to recover her private $x_i$. %We formalize this guarantee in Sec. \ref{sec:privacy:implications}. 
\\Any choice of grouping $\calG = \{G_1, G_2, \dots, G_n\}$ can be accommodated under \name-privacy. Each data owner may choose a group large enough to hide anyone they feel sufficient risk from.  We outline two systematic approaches to assigning groups as follows: %If one feels inferential risk from their close friends, perhaps their group ought to include all of their first-order social media connections. If one feels inferential risk from their close colleagues, perhaps their group should include their entire company. In turn, the analyst may still access aggregate statistics roughly at the group level. 
   \vspace{-0.1cm}\squishlistfour    \vspace{-0.1cm}
\item Let $\mathbf{t}=\langle t_1, \cdots, t_n \rangle, t_i \in \mathcal{T}$ denote some public auxiliary information about each individual. $\DO_i$'s group, $G_i$, could consist of all those $\DO_j$'s who are similar to $\DO_i$ w.r.t. the public auxiliary information $t_i, t_j$ according to some distance measure $d:\calT \times \calT \rightarrow \R$. Here, we define `similar' as being under a threshold\footnote{We could also have different thresholds, $r_i$, for every data owner, $\DO_i$.} $r \in \R$ such that $G_i = \{j \in [n] \big| d(t_i,t_j) \leq r\},     \forall i \in [n]$.
For example, $d(\cdot)$ can be Euclidean distance if $\calT$ corresponds to geographical locations, thwarting inference attacks leveraging one's household or immediate neighbors.
If $\calT$ represents a social media connectivity graph, $d(\cdot)$ can measure the path length between two nodes, thwarting inference attacks using specifically one's close friends. For the example social media connectivity graph depicted in Fig. \ref{fig:example}, assuming distance metric path length and $r=2$, the groups are defined as  $G_1=\{1,7,8,2,5,6\}, G_2=\{2,1,7,5,6,3\}$ and so on. 

\item  Alternatively, the data owners might opt for a group of a specific size $r < n$. Collecting private data from a social media network, we may set $r = 50$, where each $G_i$ is encouraged to include the $50$ data owners $\DO_i$ interacts with most frequently. 
\squishendfour 
   \vspace{-0.3cm}
Intuitively, \name-privacy protects $\DO_i$ against inference attacks that leverages correlations at a finer granularity than $G_i$. In other words, under \name-privacy, one subset of $k$ data owners $\subset G_i$ (e.g. household) is no more useful for targeting $x_i$ than any other subset of $k$ data owners $\subset G_i$ (e.g. some combination of neighbors). 
This leads to the following key insight for the formal privacy definition. 

\textbf{Key Insight.} Formally, our privacy goal is to prevent the leakage of ordinal information from within a group. We achieve this by  systematically \textit{bounding the dependence of the mechanism's output on the relative ordering (of data values corresponding to the data owners) within each group}. \\First, we introduce the notion of neighboring permutations. 

%By (non-uniformly) shuffling within each group $G_i \in \calG$, we prevent an adversary from learning whether a set of $k$ \ldp values from $G_i$ correspond to one subset within $G_i$ or another. An adversary cannot distinguish whether these $k$ values came from $\DO_i$'s close friends vs. their distant relatives or from their close neighbors vs. residents on the other side of town. However, an analyst can still observe how the distribution of \ldp values changes across neighborhoods or social circles. 
\begin{defn}   (Neighboring Permutations) Given a group assignment $\mathcal{G}$,  two permutations \scalebox{0.9}{$\sigma, \sigma' \in \mathrm{S}_n$}  are defined to be neighboring w.r.t. a group $G_i \in \calG$ (denoted as \scalebox{0.9}{$\sigma\hspace{-0.1cm} \approx_{G_i}\hspace{-0.1cm} \sigma'$}) if  $\sigma(j) = \sigma'(j) \ \forall j \notin G_i$.
\end{defn} \vspace{-0.25cm}
% \begin{gather} 
% \vspace{-0.2cm}
% \sigma(j) = \sigma'(j) \ \forall j \notin G_i 
% \vspace{-0.4cm}
% \end{gather} 
% \end{defn}
 %\vspace{-0.2cm}
Neighboring permutations differ only in the indices of its corresponding group $G_i$.
 For example, \scalebox{0.9}{$\sigma=(\underline{1} \:\underline{ 2} \: 4 \: 5 \: \underline{7} \: \underline{6} \: \underline{10} \: \underline{3} \: 8 \:9)$} and \scalebox{0.9}{$\sigma'=(\underline{7} \:\underline{3} \: 4 \: 5 \: \underline{6} \: \underline{2} \: \underline{1} \: \underline{10} \: 8 \: 9 )$} are neighboring w.r.t \scalebox{0.9}{$G_1$} (Fig. \ref{fig:example}) since they differ only in \scalebox{0.9}{$\sigma(1), \sigma(2), \sigma(5), \sigma(6), \sigma(7)$} and \scalebox{0.9}{$\sigma(8)$}. We denote the set of all neighboring permutations as %\vspace{-0.2cm}
    \vspace{0.1cm}\begin{gather}      \mathrm{N}_{\calG}=\{(\sigma,\sigma')|\sigma \approx_{G_i} \sigma', \exists G_i \in \calG \}     \vspace{0.2cm}\end{gather}    \vspace{0.2cm}
Now, we formally define \name-privacy as follows.
 \vspace{-0.1cm}\begin{defn}[\name-privacy] For a given group assignment $\calG$ on a set of $n$ entities and a privacy parameter $\alpha \in \R_{\geq0}$, a randomized  mechanism $\calA: \mathcal{Y}^n \mapsto \mathcal{V} $ is $(\alpha,\mathcal{G})$-\name~private if for all $\mathbf{y} \in \mathcal{Y}^n$ and neighboring permutations $\sigma, \sigma' \in \mathrm{N}_\calG$ and any subset of output $O\subseteq \mathcal{V}$, we have
\vspace{0.1cm} 
\begin{equation} 
    %\vspace{-0.5cm}
    \mathrm{Pr}[\calA\big(\sigma(\mathbf{y})\big) \in O] \leq e^\alpha \cdot \mathrm{Pr}\big[\calA\big(\sigma'(\mathbf{y})\big) \in O \big] \label{eq:privacy} 
\end{equation}   
%where $\bz=\pi(\by)=\langle y_{\pi(1)},\cdots, y_{\pi(n)}\rangle, \pi \in \mathrm{S}_n$
 $\sigma(\mathbf{y})$ and $\sigma'(\mathbf{y})$  are defined to be \textit{neighboring sequences}. 
\label{def:privacy}\end{defn} \vspace{-0.2cm}
 \name-privacy states that, for any group $G_i$,  the mechanism is (almost) agnostic of the order of the data within the group.  Even after observing the output, an adversary cannot learn about the relative ordering of the data within any group. Thus, two neighboring sequences are indistinguishable to an adversary. %For example, for any data sequence $\mathbf{y}\in \mathcal{Y}^{10}$, $\sigma=(1 \: 2 \: 4 \: 5 \: 7 \: 6 \: 10 \: 3 \: 8 \:9)$ and $\sigma'=(7 \:3 \: 4 \: 5 \: 6 \: 2 \: 1 \: 10 \: 8 \: 9 )$, $\sigma(\mathbf{y})$ and $\sigma'(\mathbf{y})$ are indistinguishable to an adversary ($\sigma\approx_{G_1}\sigma'$ for Fig. \ref{fig:example}). 
 %In other words, a \name-private mechanism is (almost) order-agnostic for any group in $\mathcal{G}$. 
An important property of \name-privacy is that post-processing computations does not degrade privacy. Additionally, when applied multiple times, the privacy guarantee degrades gracefully. Both the properties are analogous to \DP and are presented in App. \ref{app:post-processing}. %Interestingly, \ldp mechanisms achieve a weak degree of \name-privacy. 
 \begin{comment}
    
\begin{lem} 
    An $\epsilon$-\ldp mechanism is $(k\epsilon, \calG)$-\name~ private for any group assignment $\calG$ such that $
        k \geq \max_{G_i \in \calG} |G_i|
$ (proof in App. \ref{app:post-processing}).\label{lemma:LDP} \vspace{-0.2cm}
\end{lem} 
 \end{comment}

\textbf{Note.} Any data sequence $\mathbf{x}=\langle x_1,\cdots, x
_n\rangle$ can be viewed as a two-tuple,  $\big(\{x\}, \sigma\big)$, where $\{x\}$ denotes the \textit{bag} of values and $\sigma \in S_n$ denotes the corresponding indices of the values which represents the \textit{order} of the data.
 The $\epsilon$-LDP protects the bag of data values, $\{x\}$, while $d_\sigma$-privacy protects the order, $\sigma$. Thus, the two privacy guarantees cater to orthogonal parts of a data sequence (see Thm. \ref{thm: decision theoretic} ). Also, $\alpha=\infty~(0), r = 0~(n)$ represents the standard $\ldp$(shuffle \textsf{DP}) setting.
 

%The proof of the above lemma is presented in App. \ref{app}. 
%So, if a mechanism satisfies $(\epsilon = 2)$-\ldp, then it also satisfies $(10,\calG)$ \name-privacy for any group assignment $\calG$ whose largest group contains $5$ individuals.


 \vspace{-1em}
\subsection{Privacy Implications}
\label{sec:privacy:implications}
\vspace{-0.2cm} 
% In this section, we describe the implications of the \name-privacy guarantee in our setting. As discussed above, \name-privacy  delineates the \textit{granularity at which the underlying data correlation can be leveraged} by the adversary. Specifically, the group assignment $\mathcal{G}$ delineates a threshold of learnability as follows
%We now turn to \name-privacy's semantic guarantees: what can/cannot be learned from the released sequence $\bz$? 
The group assignment $\mathcal{G}$ delineates a threshold of learnability which determines the privacy/utility tradeoff as follows.
% As with \DP, the \name-privacy definition alone does not communicate a meaningful notion of privacy. For this, we introduce two semantic guarantees --- limiting what an adversary may do and learn --- that result from satisfying \name-privacy. Each of these is a precise way of stating the same concept: \name-privacy prevents adversaries from reconstructing any $x_i$ using the any specific subset of sanitized $y_j$ values in $\DO_i$'s group, $G_i$. 

%\name-privacy offers a form of local inferential privacy: informed Bayesian adversaries learn very little about $\DO_i$'s private value $x_i$ from the shuffled \sequence $\bz$. 

\vspace{-0.3cm}
\squishlistfour
    \item \textbf{Learning allowed (Analyst's goal)}. 
    % Anything that can be learned about $\DO_i$ from (1) the correlations at the granularity of $G_i$, and (2) individuals outside $\DO_i$ is allowed -- this encodes the utility of the data from the analyst's perspective. The former allows learning of information which can be extracted from just the \textit{bag} of the corresponding (noisy) data values, denoted as  $\{y_{G_i}\}$.  The latter allows learning of information from $\by_{\overline{G}_i}$, the (ordered and noisy) data sequence for all data owners outside $G_i$. The rationale for allowing this is that individuals outside $G_i$ are not similar to $\DO_i$ (w.r.t auxiliary information $\bt$) and hence, not very informative about $\DO_i$ (not a potential privacy threat).
  \name-privacy can answer queries that are order agnostic within groups, such as aggregate statistics of a group. In Alice's case, the analyst can estimate the disease prevalence in her neighborhood. 
    \item  \textbf{Learning disallowed (Adversary's goal)}. 
    %Utilizing correlation within the group $G_i$ (i.e., for a given set of values $\{y_{G_i}\}$, additional information extractable from the order of $\by_{G_i}$) to learn about $\DO_i$ is disallowed. This encodes the privacy threat from the adversary's perspective.
    Adversaries cannot identify (noisy) values of individuals  within any group. While they may learn the disease prevalence in Alice's neighborhood, they cannot determine the prevalence within her household and use that to recover her value $x_i$.
\squishendfour  
\vspace{-0.2cm}

%%%%%%%%%%%%%%%%%%%%%%%%%%%%
%BAYESIAN BACKGROUND
%%%%%%%%%%%%%%%%%%%%%%%%%%%%

To make this precise, we first formalize the privacy implications of the \name~guarantee in the standard Bayesian framework, typically used for studying inferential privacy. Next, we formalize the privacy provided by the combination of \ldp and \name~guarantees by way of a decision theoretic adversary. %against $(1)$ a Bayesian adversary trying to infer $\DO_i$'s true value, $x_i$ and $(2)$ a decision theoretic adversary who wants to identify the $z_i$ values corresponding to a given subset of $k$ data owners, such as the $k$ members of Alice's household. \\
\\\textbf{Bayesian Adversary.} Consider a Bayesian adversary with any prior $\calP$ on the joint distribution of noisy responses, $\Pr_\calP[\by]$, which models their beliefs on the correlation between the participants (such as the correlation between Alice and her households' disease status). Their goal is to infer $\DO_i$'s private input $x_i$. As with early \DP works \citep{dwork_early}, we consider an \emph{informed} adversary. Here, the adversary %With \textsf{DP}, informed adversaries know the private input $x_j$ of every data owner $\DO_j$ except $x_i$. 
 knows %$(1)$ the (unordered) bag of noisy values $\{y_{G_i}\}$ in $i$'s group, and $(2)$ the (ordered) sequence of noisy values $\mathbf{y}_{\overline{G}_i}$ outside $i$'s group. 
 $(1)$ the sequence (assignment) of noisy values outside $G_i$, $\by_{\overline{G}_i}$, and $(2)$ the (unordered) bag of noisy values in $G_i$, $\{y_{G_i}\}$. \name-privacy bounds the prior-posterior odds gap on $x_i$ for such as informed adversary as follows:  
\begin{thm}
\label{thm: semantic guarantee}
For a given group assignment $\calG$ on a set of $n$ data owners, if a shuffling mechanism $\calA:\calY^n\mapsto \calY^n$ is $(\alpha,\calG)$-\name private, then for each data owner $\DO_i, i \in [n]$, %\vspace{-0.1cm}
\begin{align*}
   \max_{\substack{i\in [n]\\ a,b \in \calX}} \bigg|\log \frac{\Pr_\calP [x_i = a | \bz, \{y_{G_i}\},\by_{\overline{G}_i}]}{\Pr_\calP [x_i = b | \bz, \{y_{G_i}\},\by_{\overline{G}_i}]} - \log \frac{\Pr_\calP [x_i = a | \{y_{G_i}\},\by_{\overline{G}_i}]}{\Pr_\calP [x_i = b | \{y_{G_i}\},\by_{\overline{G}_i}]} \bigg| \leq \alpha  %\vspace{-0.5cm}
\end{align*}
for a prior distribution $\calP$, where \scalebox{0.9}{$\bz=\calA(\by)$} and \scalebox{0.9}{$\by_{\overline{G}_i}$} is the noisy sequence for data owners outside \scalebox{0.9}{$G_i$}. %(proof in App. \ref{app:thm:semantic}). 
\end{thm}
\vspace{-1em}
% The above privacy loss variable differs slightly from that of Def. \ref{def:ip}, since the informed adversary already knows $\{y_{G_i}\}$ and $\by_{\overline{G}_i}$. %, much like a \DP informed adversary knowing every other datapoint $\bx_{-i}$. 
 %Equivalently, this bounds the prior-posterior odds gap on $x_i$: 
%\vspace{-0.2em}
See App \ref{app:bayesian proof} for the proof and further discussion on the semantic meaning of the above guarantee. 
 %\vspace{-1em}
%The above Bayesian analysis measures what can be learned about a data owner $\DO_i$'s private data $x_i$ from the released output $\bz$ relative to some conditioned information. %With \ldp alone, we condition on every other data owner's private value $x_j$. This implies that releasing the private sequence $\by$ cannot provide much more information about $x_i$ than releasing every other $\DO_j$'s $x_j$ would. So, only modest information unique to $x_i$ can be garnered by any Bayesian adversary. For Alice, this may be a concern, since making inferences on her disease state from those of her household is indeed a privacy violation. 
%Under \name-privacy, we condition on the bag of \ldp values in Alice's group $\{y_{G_i}\}$ as well as the sequence of $\ldp$ values outside her group $\by_{\overline{G_i}}$. 
%Eq. \ref{eq:Bayesian} implies that releasing the shuffled sequence $\bz$ cannot provide more information about Alice's private data $x_i$ than releasing the output sequence (data value and order)  of members outside her neighborhood (her group $G_i$) and the bag of (noisy) values inside her neighborhood. This disallows the adversary from leveraging any specific subset of Alice's group (such as her household) to infer her private disease state $x_i$ since they fail to link the data values in $\{y_{G_{i}}\}$ to the members of $G_i$. We re-identfication attack in the following section.  %Thus Thm. \ref{thm: semantic guarantee} formalizes the privacy afforded by protecting the ordering information within her group -- the exact target of \name-privacy. 
%Contrast this with plain \ldp where the the order (analogously identity) of the data responses of Alice's group is also known to the adversary thereby allowing them to use the data of Alice's household to violate her privacy.  See App \ref{app:} for more illustrations. %The adversary can learn about $x_i$ from the disease prevalence outside her neighborhood and, on average, inside her neighborhood but not much beyond that. 

% We illustrate this with the following example on \scalebox{0.9}{$\bt_{e.g}$} (Fig. \ref{fig:example}). The adversary knows (i.e. prior \scalebox{0.9}{$\mathrm{Pr}_\calP[\by | \bt_{e.g.}]$}) that data owner \scalebox{0.9}{$\DO_1$} is strongly correlated with their close friends \scalebox{0.9}{$\DO_2$} and \scalebox{0.9}{$\DO_7$}. Let \scalebox{0.9}{$\by=\langle 1, 1, y_3, y_4, 0,0, 1, 0, y_9, y_{10} \rangle$} and \scalebox{0.9}{$\by'=\langle  0, 0, y_3, y_4, 1,1, 0, 1, y_9, y_{10}\rangle$} represent two neighboring sequences w.r.t \scalebox{0.9}{$G_1=\{1,7,8,2,6,5\}$}, for any \scalebox{0.9}{$y_3,y_4,y_9,y_{10}\in \{0,1\}^4$}. Under \name-privacy, the adversary cannot distinguish between  $\by$  (data sequence where \scalebox{0.9}{$\DO_1$}, \scalebox{0.9}{$\DO_2$} and \scalebox{0.9}{$\DO_7$} have value $1$) and $\by'$ (data sequence where \scalebox{0.9}{$\DO_1$}, \scalebox{0.9}{$\DO_2$} and \scalebox{0.9}{$\DO_7$} have value $0$).  Hence, after seeing the shuffled sequence, the adversary can only know the `bag' of values \scalebox{0.9}{$\{y_{G_1}\}=\{0,1,0,1,0,1\}$} and cannot specifically leverage \scalebox{0.9}{$\DO_1$}'s immediate friends' responses \scalebox{0.9}{$\{y_2,y_7\}$} to target \scalebox{0.9}{$x_1$}. However, analysts may still answer queries that are order-agnostic in \scalebox{0.9}{$G_1$}, which could not be achieved with uniform shuffling. 

% \textbf{Note.} 
% By the post-processing \cite{Dwork} property of \textsf{LDP}, the shuffled sequence $\bz$ retains the $\epsilon$-\ldp guarantee. The granularity of the group assignment determined by distance threshold $r$ and the privacy degree $\alpha$ act as control knobs of the privacy spectrum. For instance w.r.t. $\bt_{e.g}$ (Fig. \ref{fig:example}), for \scalebox{0.9}{$r=0$}, we have \scalebox{0.9}{$G_i=\{i\}$} and the problem reduces to the pure \ldp setting. For \scalebox{0.9}{$r=\infty, \alpha=0$}, we get \scalebox{0.9}{$G_i=[n]$} which corresponds to the case of uniform random shuffling (standard shuffle model). All other pairs of \scalebox{0.9}{$(r, \alpha)$} represent intermediate points in the privacy spectrum which are achievable via \name-privacy. 

%%%%%%%%%%%%%%%%%%%%%%%%%%%%
%END BAYESIAN BACKGROUND
%%%%%%%%%%%%%%%%%%%%%%%%%%%%

%%%%%%%%%%%%%%%%%%%%%%%%%%%%
%DECISION THEORETIC EXPLANATION
%%%%%%%%%%%%%%%%%%%%%%%%%%%%

\newcommand{\Aunif}{\calA_{\text{unif}}}
\newcommand{\Ashuff}{\calA_{\text{shuff}}}
\newcommand{\Pzo}{P_{0 \rightarrow 1}}
\newcommand{\Poz}{P_{1 \rightarrow 0}}
\textbf{Decision Theoretic Adversary.} Here, we analyse the privacy provided by the combination of \ldp and \name~guarantees.
%The guarantee above formalizes this by comparing what a Bayesian adversary learns about $x_i$ from $\bz$ to what they learn from the bag of $\ldp$ values inside $\DO_i$'s group and the sequence of \ldp values outside it. 
%We make this concrete by showing that no adversary can reliably find which released %For instance, in combination with a low $\epsilon$, \name-privacy guarantees that no adversary can reliably pick out which $z_i$ values originated from Alice's household, and thus cannot specifically leverage those to make inferences on her. 
Consider a decision theoretic adversary who aims to identify the noisy responses, $\{z_I\}$, that originated from a specific subset of data owners, $I \subset G_i$ (such as the members of Alice's household). %This re-identification attack is a key step in carrying out the inference attack  %observes the output sequence $\bz$ and chooses which indices in $\bz$ correspond to a subset of data owners $I \subset G_i$. 
We denote the adversary by a (possibly randomized) function mapping from the output $\bz$ sequence to a set of $k$ indices, $\mathcal{D}_{Adv}: \calY^n \rightarrow [n]^k$, where $k = |I|$. These $k$ indices, $H \in [n]^k$, represent the elements of $\bz$ that $\mathcal{D}_{Adv}$ believes originated from the data owners in $I$. $\mathcal{D}_{Adv}$ wins if $> k/2$ of the chosen indices indeed originated from $I$, i.e,  $|\sigma(H) \cap I| > k/2 $, where $z_i = y_{\sigma(i)}$ and $\sigma(H) = \{\sigma(i) : i \in H\}$. $\mathcal{D}_{Adv}$ loses if most of $H$ did not originate from $I$, i.e.,  $|\sigma(H) \cap I| \leq k/2 $.
% \footnote{$k \ll r$; the adversary must recover the \ldp values of the entire subgroup $I \subset G_i$ to make inferences on $x_i$.} 
We choose the above adversary because this re-identification is a key step in carrying out inference attacks -- in failing to reliably re-identify the noisy values originating from $I$, one cannot make inferences on $x_i$ specifically from the subset $I \subset G_i$. 

%We assume that $k < \frac{r}{2}$, where $r$ is the size of Alice's group, $r = |G_i|$. 

% $A_I$ wins if at least $l$ of the $k$ indices it selects in $\bz$, $H$, indeed originated from $l$ of the $k$ data owners in $I$. $A_I$ loses if less than $l$ of the $k$ indices it selects originated from data owners in $I$. We assume that $l < \frac{k}{2}$ and that $k < \frac{r}{2}$, where $r$ is the size of the group, $r = |G_i|$. 

% This notion that the \ldp $y_i$ values of a specific subset of $G_i$ cannot be leveraged is made even more concrete by considering 

% Consider the original notation where the mechanism is not split into bag of values and order. We can also provide a guarantee against any adversary trying to identify the indices of $\bz$ originating from data owners $I \subset G_i$ (non differentially). 

% Formally, consider an adversary who tries to identify the \ldp datapoints originating from subset $I \subset G_i$, $A_I:\calZ^{n} \rightarrow [n]^{k}$, where $|I| = k$ and $|G_i| = r > k$. Upon observing any partially-shuffled sequence $\bz$, $A_I$ returns $H\subset [n]$, a set of $k$ indices in $\bz$ which it believes originated from data owners $I$. We lower bound the error rate of $A_I$: 

\begin{thm}
\label{thm: decision theoretic}
   For $\mathcal{A}(\mathcal{M}(\bx))=\bz$ where $\mathcal{M}(\cdot)$ is $\epsilon$-\ldp and $\mathcal{A}(\cdot)$ is $\alpha$ - \name private, we have  
 \begin{align*}
     \Pr[\mathcal{D}_{Adv} \text{ loses}] \geq \big\lfloor \frac{r-k}{k} \big\rfloor e^{-(2k\epsilon+\alpha)} \cdot \Pr[\mathcal{D}_{Adv} \text{ wins}]
 \end{align*}
 for any input subgroup $I \subset G_i, r = |G_i|$ and  $k < r/2$. 
\end{thm}
\vspace{-0.3cm}
The adversary's ability to re-identify the $\{z_I\}$ values comes partially from the \textit{bag of values } (quantified by $\epsilon$) and partially from the \textit{order} (quantified by $\alpha$). We highlight two implications of this fact. 
\vspace{-0.4cm}
\squishlistfour
    \item When $\epsilon$ is small ($\ll 1$), an adversary's ability to re-identify the noisy values $\{z_I\}$ originating from $I$ may very well be dominated by $\alpha$. For instance, if $\epsilon = 0.2$ and $k = 5$, the adversary's advantage is dominated by $\alpha$ for any $\alpha > 2$. When using $\ldp$ alone (no shuffling), $\alpha = \infty$ and the adversary can exactly recover which values came from Alice's household. As such, even a moderate $\alpha$ value (obtained via \name-privacy) significantly reduces the ability to re-identify the values. 
    \item When the loss is dominated by $\epsilon$ ($2k \epsilon \gg \alpha$), the above expression allows us to disentangle the \textit{source of privacy loss}. In this regime, adversaries get most of their advantage from the bag of values released, not from the order of the release. That is, even if $\alpha = 0$ (uniform random shuffling), participants still suffer a large risk of re-identification simply due to the noisy values being reported. Thus, no shuffling mechanism can prevent re-identification in this regime. %One could imagine that if the group size $r$ is small and $\epsilon$ is large, no shuffling mechanism can prevent re-identification if the adversary has a strong prior on the user's values. 
\squishendfour \vspace{-0.2cm}
\textbf{Discussion.} In spirit, \DP does not guarantee protection against recovering $\DO_i$'s private $x_i$ value. It guarantees that -- had a user not participated (or equivalently submitted a false value $x_i'$) -- the adversary would have about the same ability to learn their true value, potentially from the responses of other data owners. In other words, the choice to participate is unlikely to be responsible for the disclosure of $x_i$. Similarly, \name-privacy does not prevent disclosure of $x_i$. By requiring indistinguishability of neighboring permutations, it guarantees that -- had the data owners of any group $G_i$ completely swapped identities -- the adversary would have about the same ability to learn $x_i$. So most likely, Alice's household is not uniquely responsible for a disclosure of her $x_i$: had her household swapped identities with any of her neighbors, the adversary would probably draw the same conclusion on $x_i$. 
Or, as detailed in Thm.\ref{thm: decision theoretic}, an adversary cannot reliably resolve which $\{z\}$ values originated from Alice's household, so they cannot draw conclusions based on her household's responses. 
In a nutshell,    \squishlistfour   
    \item Inference attacks can recover a data owner $\DO_i$'s private data $x_i$ from the responses of other data owners. The order of the data acts as the proxy for the data owner's identity which can aid an adversary in corralling the subset of other data owners who correlate with $\DO_i$ (required to make a reliable inference of $x_i$). 
    \item  \DP alleviates concerns that \underline{$\DO_i$'s choice to share data} ($y_i$) will result in disclosure of $x_i$, and \name-privacy alleviates concerns that $\DO_i$'s \underline{group's ($G_i$) choice to share their identity} will result in disclosure of $x_i$.
    \vspace{-0.2cm}
\squishendfour
   \vspace{-0.3cm}%As such, \DP alleviates concerns that $\DO_i$'s \emph{choice to share data} ($y_i$) will result in disclosure of $x_i$, and \name-privacy alleviates concerns that $\DO_i$'s group's \emph{choice to share their identity} will result in disclosure of $x_i$.


% The signficance of the above results are that 1) a significant amount of re-identification privacy loss derives from transmitting the values alone, 2) that the privacy loss from the act of transmitting data that can be identified along with an individual (e.g. from ones device or from their ISP hub) is limited by $\alpha$ and 3) that groups much larger than the subset size $r \gg k$ can offer significant resistance to reidentification. 

% To formalize this, we rewrite our shuffling mechanism $\mathcal{A}$ as the composition of two mechanisms: $\Aunif$ which outputs $\bv$, a uniformly shuffled sequence of the LDP values $\by$ and $\Ashuff$ which outputs the random permutation $\gamma$, which partially reorders $\bv$ to produce a sequence with the same distribution as $\bz$, above. Specifically, $\Aunif$ samples a uniformly random permutation $\sigma^* \sim \mathrm{S}_n$ and releases $\bv = \sigma^*(\by)$. Then, $\Ashuff(\sigma^*)$ samples a permutation $\sigma$ from a \name-private shuffling mechanism as before and releases $\gamma = \gamma^* \sigma$ where $\gamma^*$ is the inverse of $\sigma^*$. 

% By dissecting the released sequence into a bag of values $\bv$ and an order $\gamma$, we are able to show how \name-privacy formally prevents an adversary from matching \ldp releases with their data owners. For $\DO_i$, an adversary cannot reliably distinguish which subset of $k$ data owners in $G_i$ are responsible for a given set of $k$ values in $\bv$. To make this precise, take any two subsets of $k$ data owners in group $G_i$, $I$ and $J$. If $G_i$ represents $\DO_i$'s neighbors, then $I$ could represent the $k$ members of their household, and $J$ could represent $k$ members of their neighbors' household. The indices $\sigma^*(I)$ then indicate which $\bv$ values originated in $\DO_i$'s household and $\sigma^*(J)$ indicate which values originated in the neighbor's. 

% The adversary, $A$, receives the bag of values $\bv$, the partial reordering $\gamma$, and a set of $k$ indices $H$. Under the null hypothesis, $H = \sigma^*(J)$, i.e. $H$ tells which values of $\bv$ belong to the neighbors, and $A$ must return a 0 to win. Under the alternative hypothesis, $H = \sigma^*(I)$, indicating which $\bv$ values belong to $\DO_i$'s household, and $A$ must return a 1 to win. \name-privacy formally limits how much an adversary can leverage the order, $\gamma$, to determine whether values $\bv_H$ belong to individuals $I$ or $J$. We let $\Pzo$ ($\Poz$) be the probability that, under the null (alternative) hypothesis $A$ errantly chooses the alternative (null) hypothesis. 

% \begin{thm}
% A $\alpha$-\name-private pair of mechanisms $\Aunif, \Ashuff$, operating on an $\epsilon$-\ldp sequence $\by$ guarantees that 
% \begin{align*}
%     \Pzo + e^{2k \epsilon + \alpha} \Poz &\geq 1, \text{ and, } \\
%     e^{2k \epsilon + \alpha} \Pzo +  \Poz &\geq 1
% \end{align*}
% for any adversary $A(\bv, \gamma, H)$, any pair of subsets $I, J \subset G_i$ of size $k$, and any group $G_i \in \calG$. 
% \end{thm}

% The above theorem formally bounds how successfully any adversary can distinguish which \ldp values in $\bv$ belong to which subsets of any group $G_i$. This provides two key insights:
% \begin{enumerate}
%     \item For any shuffling mechanism, there is an inherent limit to how well we can prevent re-identification. Even with uniform random shuffling ($\alpha = 0$), we experience a $2k \epsilon$ loss in privacy simply as a result of the \emph{values} of the data we are transmitting $\bv$. The class of adversaries with a prior on users' true values can use this re-identify their anonymized $\ldp$ releases. 
%     \item The privacy loss contributed by the shuffling mechanism itself is isolated to $\alpha$. This has an intuitive interpretation. With a low $\alpha$, a data owner is assured that data can only be reliably identified at the level of their group. As such, they need not worry that their data and neighbors' data can be identified by any side information e.g. their device IDs, or their IP addresses. 
% \end{enumerate}



%%%%%%%%%%%%%%%%%%%%%%%%%%%%
%END DECISION THEORETIC EXPLANATION
%%%%%%%%%%%%%%%%%%%%%%%%%%%%



%\vspace{-0.2cm}

% \subsection{Utility of a Shuffling Mechanism}
% \label{sec:utility}
% \vspace{-0.2cm}
% We now introduce a novel metric, $(\eta,\delta)$-preservation, for assessing the utility of any shuffling mechanism. Let $S\subseteq [n]$ correspond to a set of indices in $\by$. The metric is defined as follows.
% %representing data owners in Alice's neighborhood for instance.  
% %$(\eta,\delta)$-preservation measures how well the shuffling mechanism preserves the original indices in $S$ after shuffling, i.e. the fraction of data owners in Alice's neighborhood that still correspond to datapoints from the neighborhood after shuffling:
% \begin{defn}($(\eta,\delta)$-preservation) A shuffling mechanism $\calA:\calY^n\mapsto\calY^n$ is defined to be $(\eta,\delta)$-preserving $(\eta, \delta 
% \in [0,1])$ w.r.t to a given subset $S\subseteq [n]$, if \begin{gather}\Pr\big[|S_{\sigma}\cap S|\geq \eta\cdot|S|\big]\geq 1-\delta,  \sigma \in \mathrm{S}_n\end{gather} where $\bz=\calA(\by)=\sigma(\by)$ and $S_{\sigma}=\{\sigma(i)|i \in S\}$. \label{def:utility} 
% % \vspace{-0.2cm}
% \end{defn}
% For example, consider \scalebox{0.9}{$S=\{1,4,5,7,8\}$}. If \scalebox{0.9}{$\calA(\cdot)$} permutes the output according to  \scalebox{0.9}{$\sigma=(\underline{5}\:3\:2\:\underline{6}\:\underline{7}\:9\:\underline{8}\:\underline{1}\:4\:10)$}, then  \scalebox{0.9}{$S_{\sigma}=\{5,6,7,8,1\}$}  which preserves \scalebox{0.9}{$4$} or \scalebox{0.9}{$80\%$} of its original indices.  This means that for any data sequence $\by$, at least \scalebox{0.9}{$\eta$} fraction of its data values corresponding to the subset \scalebox{0.9}{$S$} overlaps with that of shuffled sequence $\bz$ with high probability \scalebox{0.9}{$(1-\delta)$}. Assuming, \scalebox{0.9}{$\{y_S\}=\{y_{i}|i
% \in S\}$} and \scalebox{0.9}{$\{z_S\}=\{z_i|i \in S\}=\{y_{\sigma(i)}| i \in S\}$} denotes the set of data values corresponding to $S$ in data sequences $\by$ and $\bz$ respectively, we  have \scalebox{0.9}{$\Pr\big[|\{y_S\}\cap \{z_S\}|\geq \eta \cdot |S|\big]\geq 1-\delta, \: \forall \by $}.
% % \vspace{-0.2cm}
% % \begin{gather}
% % \vspace{-0.2cm} 
% % \Pr\big[|\{y_S\}\cap \{z_S\}|\geq \eta \cdot |S|\big]\geq 1-\delta, \: \forall \by 
% % \vspace{-0.2cm}
% % \end{gather} 

% For example, let $S$ be the set of individuals from Nevada. Then, for a shuffling mechanism that provides \scalebox{0.9}{$(\eta =0.8, \delta=0.1)$}-preservation to $S$, with probability \scalebox{0.9}{$\geq 0.9$}, \scalebox{0.9}{$\geq 80\%$} of the values that are reported to be from Nevada in $\bz$ are genuinely from Nevada. The rationale behind this metric is that it captures the utility of the learning allowed by \name-privacy -- if $S$ is equal to some group \scalebox{0.9}{$G \in \calG$}, \scalebox{0.9}{$(\eta, \delta)$} preservation allows overall statistics of \scalebox{0.9}{$G$} to be captured. Note that this utility metric is \textit{agnostic of both the data distribution and the analyst's query}. Hence, it is a conservative analysis of utility which serves as a lower bound for learning from $\{z_S\}$. %We suspect that with the knowledge of the data distribution and/or the query, a tighter utility analysis is possible. 
% % \vspace{-1em}
\subsection{\name-private Shuffling Mechanism}
\label{sec:mechanism}\vspace{-0.2cm}
%\vspace{2cm}

We now describe our novel shuffling mechanism that can achieve \name-privacy. In a nutshell, our mechanism samples a permutation from a suitable Mallows model and shuffles the data sequence accordingly. We can characterize the \name-privacy guarantee of our mechanism in the same way as that of the \DP guarantee of classic mechanisms \citep{Dwork} -- with variance and sensitivity. Intuitively, a larger dispersion parameter $\theta \in \R$ (Def. \ref{def: mallows}) reduces randomness over permutations, increasing utility and increasing (worsening) the privacy parameter $\alpha$. The maximum value of $\theta$ for a given $\alpha$ guarantee depends on the sensitivity of the rank distance measure $\textswab{d}(\cdot)$ over all neighboring permutations $N_\calG$. Formally, we define the sensitivity as 
%\resizebox{0.95\linewidth}{!}{
%\begin{minipage}{\linewidth}
%\begin{align*}
%    \Delta(\sigma_0 : \textswab{d}, \calG) =
%    \max_{(\sigma, \sigma') \in N_\calG} |\textswab{d}(\sigma_0 \sigma, \sigma_0) - \textswab{d}(\sigma_0 \sigma', \sigma_0)|~, 
%\end{align*}
%\end{minipage}
%}
\begin{align*}
    \Delta(\sigma_0 : \textswab{d}, \calG) =
    \max_{(\sigma, \sigma') \in N_\calG} |\textswab{d}(\sigma_0 \sigma, \sigma_0) - \textswab{d}(\sigma_0 \sigma', \sigma_0)|~, 
\end{align*}
%\vspace{-0.2cm}
% \begin{prope}
% For group assignment $\calG$, a  mechanism $\calA(\cdot)$ that shuffles according to a permutation sampled from the Mallows model $\mathbb{P}_{\theta,\textswab{d}}(\cdot)$, satisfies $(\alpha, \calG)$-\name privacy where
% \vspace{-0.1cm}
% \begin{gather}
% \vspace{-0.2cm}
%  \hspace{-0.5cm}\Delta(\sigma_0 : \textswab{d}, \calG) = \max_{(\sigma, \sigma') \in N_\calG} |\textswab{d}(\sigma, \sigma_0) - \textswab{d}(\sigma', \sigma_0)|\\
%     \alpha 
%     = \theta \cdot \Delta(\sigma_0 : \textswab{d}, \calG)
% \vspace{-0.2cm} 
% \end{gather} 
% We refer to $\Delta(\sigma_0 : \textswab{d}, \calG) $ as the sensitivity of the rank-distance measure $\textswab{d}(\cdot)$ (details in App. \ref{app:prop}).
% \label{prop:1}
% \end{prope}
% \vspace{-0.2cm}
  the maximum change in distance $\textswab{d}(\cdot)$ from the reference permutation $\sigma_0$ for any pair of neighboring permutations $(\sigma,\sigma') \in N_\calG$ permuted by $\sigma_0$. The privacy parameter of the mechanism is then proportional to its sensitivity \scalebox{1}{$\alpha = \theta \cdot \Delta(\sigma_0 : \textswab{d}, \calG)$}. 
  
  Given $\mathcal{G}$ and a reference permutation $\sigma_0$, the sensitivity of a rank distance measure $\textswab{d}(\cdot)$ depends on the \emph{width}, $\omega_{\calG}^{\sigma}$, which measures how `spread apart' the members of any group of $\mathcal{G}$ are in $\sigma_0$:\vspace{-0.2cm}
 \begin{align*}
     \omega_{G_i}^{\sigma}= \max_{(j,k) \in G_i \times G_i} \Big| \sigma^{-1}(j) - \sigma^{-1}(k) \Big|, i \in [n];  \hspace{0.5cm}
    \omega_{\calG}^{\sigma} = \max_{G_i \in \calG} \omega_{G_i}^{\sigma}
     \vspace{-1em}
 \end{align*}
% \begin{defn}(Width) 
% For a permutation $\sigma$, the width of a group assignment, $\omega_{\calG}^{\sigma}$, is defined as the maximum separation in $\sigma$ between any two members of a group in $\calG$. 
%  \begin{gather*}
%  %\vspace{-0.3cm}
%  \hspace{-0.5cm}\omega_{G_i}^{\sigma}= \max_{(j,k) \in G_i \times G_i} \Big| \sigma^{-1}(j) - \sigma^{-1}(k) \Big|, i \in [n];  \hspace{0.5cm}
%     \omega_{\calG}^{\sigma} = \max_{G_i \in \calG} \omega_{G_i}^{\sigma}%\vspace{-0.3cm}
% \end{gather*}
% %\vspace{-0.3cm}
% \label{def:width}
% \end{defn}
% \vspace{-0.4cm}
% $\omega_{G_i}^{\sigma}$ measures how `spread apart' the members of $G_i$ in permutation $\sigma$ are. 
For example, for \scalebox{0.9}{$\sigma=(1\:3\:7\:8\:6\:4\:5\:2\:9\:10)$} and \scalebox{0.9}{$G_1=\{1,7,8,2,5,6\}$}, \scalebox{0.9}{$\omega_{G_1}^{\sigma}=|\sigma^{-1}(1)-\sigma^{-1}(2)|=7$}. The sensitivity is an increasing function of the width. For instance, for Kendall's \scalebox{0.9}{$\tau$} distance \scalebox{0.9}{$\textswab{d}_\tau(\cdot )$} we have \scalebox{0.9}{$\Delta(\sigma_0 : \textswab{d}_\tau, \calG)
    =\omega_{\calG}^{\sigma_0}(\omega_{\calG}^{\sigma_0} + 1)/2$}. \\If a reference permutation clusters the members of each group closely together (low width), then the groups are more likely to permute within themselves. This has two benefits. First, for the same $\theta$ ($\theta$ is an indicator of utility as it determines the dispersion of the sampled permutation), a lower value of width gives lower $\alpha$ (better privacy).  Second, if a group is likely to shuffle within itself, it will have better \scalebox{0.9}{$(\eta, \delta)$}-preservation -- a novel utility metric, we propose, for a shuffling mechanism. Intuitively, a mechanism is $(\eta,\delta)$-preserving w.r.t a subset of indices \scalebox{0.9}{$S \subset [n]$} if at least \scalebox{0.9}{$\eta\%$}  of its indices are shuffled within itself with probability \scalebox{0.9}{$(1-\delta)$}. The rationale behind this metric is that it captures the utility of the learning allowed by \name-privacy -- if \scalebox{0.9}{$S$} is equal to some group \scalebox{0.9}{$G \in \calG$}, high \scalebox{0.9}{$(\eta, \delta)$}-preservation allows overall statistics of \scalebox{0.9}{$G$} to be captured since $\eta\%$ of the correct data values remain preserved.   We present the formal discussion in App. \ref{app:utility}. 
    
\begin{wrapfigure}{R}{0.47\textwidth}
    \vspace{-3.5em}
    \IncMargin{1em}
    \setlength{\textfloatsep}{-1pt}
    \begin{algorithm}[H]
    \caption{\name-private Shuffling Mech.}
    \setstretch{1.25}
    \KwIn{\ldp sequence \scalebox{0.9}{$\by=\langle y_1, \cdots, y_n \rangle$}\;
    \hspace{3em}Public aux. info. $\bt=\langle t_1, \cdots t_n \rangle$\;
    \hspace{3em}Dist. threshold $r$; Priv. param. $\alpha$\;
    }
    %\hspace{1cm} 
    \KwOut{$\bz$ - Shuffled output sequence\;}
    \vspace{0.1em}
    \nl $\calG=$ \scalebox{0.9}{$ComputeGroupAssignment$} $(\bt,r)$\;
    % \hspace{0.7cm}\textcolor{blue}{$\rhd$} Assign groups for each data owner using Eq.~\eqref{eq:group2}\vspace{0.05cm}
    \nl Construct graph $\mathbb{G}$ with \\
     \hspace{1em} a) vertices \scalebox{0.9}{$V=\{1,2,\cdots,n\}$}\\ 
     \hspace{1em} b) edges \scalebox{0.9}{$E = \{(i,j): j \in G_i, G_i \in \calG\}$} \\
    %  \Statex\hfill \textcolor{blue}{$\rhd$} Translate the group assignment $\calG$ to a graph $\mathbb{G}$ 
    \nl \scalebox{0.95}{$root=\arg \max_{i \in [n]} |G_i|$}\;
    %  \Statex\hfill \textcolor{blue}{$\rhd$} \scalebox{0.95}{$root$} corresponds to the node with the largest group \vspace{0.05cm}
    \nl \scalebox{0.95}{$\sigma_0=\textsf{BFS}(\mathbb{G},root)$}\;
    %  \Statex\hfill \textcolor{blue}{$\rhd$} Compute reference permutation via a BFS traversal on $\mathbb{G}$ 
    \nl $\Delta$= \scalebox{0.9}{$ComputeSensitivity$}$(\sigma_0,\calG)\;$
      
    % \hfill\textcolor{blue}{$\rhd$} Using Prop. \ref{prop:1} 
    \nl \scalebox{0.9}{$\theta=\alpha/\Delta$}\; 
    % \hfill\textcolor{blue}{$\rhd$} Compute the dispersion parameter 
    \nl \scalebox{0.9}{$\hat{\sigma} \sim \mathbb{P}_{\theta,\textswab{d}}(\sigma_0) $ }\;
    %  \hfill\textcolor{blue}{$\rhd$} Sampling from the Mallows model
    \nl $\sigma^*=\sigma_0^{-1}\hat{\sigma}$\; 
    \nl $\bz=\langle y_{\sigma^*(1)}, \cdots y_{\sigma^*(n)}\rangle$\; 
    % \hfill\textcolor{blue}{$\rhd$} Shuffle the output   
    \nl Return $\bz$\;
    \label{algo:main}
    \end{algorithm}
    \vspace{-2em}
\end{wrapfigure}

Unfortunately, minimizing $\omega_\calG^\sigma$ is an NP-hard problem (Thm. \ref{thm:NP} in App. \ref{app:NP}). Instead, we estimate the optimal $\sigma_0$ using the following heuristic\footnote{The heuristics only affect $\sigma_0$ (and utility). Once $\sigma_0$ is fixed, $\Delta$ is computed exactly as discussed above.} approach based on a graph breadth first search. 

\textbf{Algorithm Description.}
Alg. 1 above proceeds as follows. We first compute the group assignment, $\calG$, based on the public auxiliary information and desired threshold $r$ following discussion in Sec. \ref{sec:privacy:def} (Step 1). Then we construct $\sigma_0$ with a breadth first search (BFS) graph traversal. 
\\We translate $\calG$ into an undirected graph \scalebox{0.9}{$(V,E)$}, where the vertices are indices \scalebox{0.9}{$V = [n]$} and two indices \scalebox{0.9}{$i,j$} are connected by an edge if they are both in some group (Step 2). Next, \scalebox{0.9}{$\sigma_0$} is computed via a breadth first search traversal (Step 4) --  if the \scalebox{0.9}{$k$}-th node in the traversal is \scalebox{0.9}{$i$}, then \scalebox{0.9}{$\sigma_0(k) = i$}. The rationale is that neighbors of \scalebox{0.9}{$i$} (members of \scalebox{0.9}{$G_i$}) would be traversed in close succession. Hence, a neighboring node \scalebox{0.9}{$j$} is likely to be traversed at some step \scalebox{0.9}{$h$} near \scalebox{0.9}{$k$} which means \scalebox{0.9}{$|\sigma_0^{-1}(i) - \sigma_0^{-1}(j)| = |h - k|$} would be small (resulting in low width). Additionally, starting from the node with the highest degree (Steps 3-4) which corresponds to the largest group in $\calG$ (lower bound for $\omega_{\calG}^{\sigma}$ for any  $\sigma$) helps to curtail the maximum width in $\sigma_0$. See App. \ref{apx:heuristic eval} for evaluations of this heuristic's approximation.  
% \vspace{-0.25cm}
% \squishlist 
% \item We translate $\calG$ into an undirected graph $(V,E)$, where the vertices are indices $V = [n]$ and two indices $i,j$ are connected by an edge if they are both in some group (Step 2). Next, $\sigma_0$ is computed via a breadth first search traversal (Step 4) --  if the $k$-th node in the traversal is $i$, then $\sigma_0(k) = i$. The rationale is that neighbors of $i$ (members of $G_i$) would be traversed in close succession. Hence, a neighboring node $j$ is likely to be traversed at some step $h$ near $k$ which means $|\sigma_0^{-1}(i) - \sigma_0^{-1}(j)| = |h - k|$ would be small (resulting in low width).  \vspace{-0.03cm} \item  We start from the node with the highest degree (Steps 3-4) which corresponds to the largest group in $\calG$ (lower bound for $\omega_{\calG}^{\sigma}$ for any  $\sigma$). This is a good heuristic since it curtails the spread out of largest group.  \vspace{-0.25cm}
% \squishend

This is followed by the computation of the dispersion parameter, \scalebox{0.9}{$\theta$}, for our Mallows model (Steps 5-6). 
Next, we sample a permutation from the Mallows model (Step 7) \scalebox{0.9}{$\hat{\sigma} \sim  \mathbb{P}_{\theta}(\sigma:\sigma_0) $} and we apply the inverse reference permutation to it, \scalebox{0.9}{$\sigma^* = \sigma_0^{-1} \hat{\sigma}$} to obtain the desired permutation for shuffling. Recall that $\hat{\sigma}$ is (most likely) close to $\sigma_0$, which is unrelated to the original order of the data. \scalebox{0.9}{$\sigma_0^{-1}$} therefore brings \scalebox{0.9}{$\sigma^*$} back to a shuffled version of the original sequence (identity permutation $\sigma_I$). Note that since Alg. 1 is publicly known, the adversary/analyst knows $\sigma_0$. Hence, even in the absence of this step from our algorithm, the adversary/analyst could perform this anyway. Finally, we permute $\by$ according to $\sigma^*$ and output the result \scalebox{0.9}{$\bz = \hat{\sigma}(\by)$} (Steps 9-10).  
%\vspace{-0.03cm} 
\begin{thm} Alg. 1 is $(\alpha,\calG)$-\name~private where $\alpha = \theta \cdot \Delta(\sigma_0 : \textswab{d}, \calG)$.
% \vspace{-0.25cm} 
\label{thm:privacy} 
\end{thm} 
The proof is in App. \ref{app:thm:privacy}.
Note that Alg.  1 provides the same level of privacy \scalebox{0.9}{$(\alpha)$} for any two group assignment \scalebox{0.9}{$\calG, \calG'$} as long as they have the same sensitivity, i.e, \scalebox{0.9}{$\Delta(\sigma_0 : \textswab{d}_\tau, \calG)=\Delta(\sigma_0 : \textswab{d}_\tau, \calG')$}. This leads to the following theorem which generalizes the privacy guarantee for any group assignment. 

\begin{thm} Alg. 1 satisfies 
$(\alpha',\calG')$-\name privacy for any group assignment $\calG'$ with $ \alpha'=\alpha\frac{\Delta(\sigma_0 : \textswab{d}, \calG')}{\Delta(\sigma_0 : \textswab{d}, \calG)}$ (proof in App. \ref{app:thm:generalized}.) %For instance, for Kendall $\tau$'s distance we have $\alpha'=\alpha\frac{\omega^{\sigma_0}_{\calG'}(\omega^{\sigma_0}_{\calG'}-1)}{\omega^{\sigma_0}_{\calG}(\omega^{\sigma_0}_{\calG}-1)}$. 
 \vspace{-0.2cm}
\label{thm:generalized:privacy}
\end{thm} 
% Next, we present a utility theorem for Alg. 1 that formalizes the $(\eta,\delta)$-preservation for Hamming distance $\textswab{d}_{H}(\cdot)$  (we chose $\textswab{d}_{H}(\cdot)$ for the ease of numerical computation).
% \begin{thm} For a given set $S \subset [n]$ and Hamming distance metric,  $\textswab{d}_H(\cdot)$,   Alg. \ref{algo:main} is $(\eta,\delta)$-preserving for $\delta=\frac{1}{\psi(\theta, \textswab{d}_H)}\sum_{h=2k+1}^{n} (e^{-\theta\cdot h} \cdot c_h)$ where \scalebox{0.9}{$k=\lceil(1-\eta)\cdot |S|\rceil$} and $c_h$ is the number of permutations with hamming distance $h$ from the reference permutation that do not preserve \scalebox{0.9}{$\eta\%$} of $S$ (exact formula and proof in App. \ref{app:utility}).
% \label{thm:utility}
% \end{thm}
% \vspace{-0.25cm}

% The time complexity of Alg. 1 is dominated by \scalebox{0.9}{$\calG$}'s computation. Regardless of the method for selecting the reference permutation, the time complexity for the shuffling mechanism is at least \scalebox{0.9}{$\Omega(\sum_i|G_i|)$} (minimum computation for listing \scalebox{0.9}{$\calG$}). Note that Alg. \ref{algo:main}'s computation of the reference permutation takes \scalebox{0.9}{$O(|V|+|E|)=O(\sum_i|G_i|)$}. Another observation is that 


%The proof is in  App. \ref{app:thm:generalized}. %A utility theorem for Alg. 1 that formalizes the $(\eta,\delta)$- preservation for Hamming distance $\textswab{d}_{H}(\cdot)$  (we chose $\textswab{d}_{H}(\cdot)$ for the ease of numerical computation) is in App. \ref{app:utility:formal}. 

\textbf{Note.} Producing $\sigma^*$ is completely data ($\by$) independent. It only requires access to the public auxiliary information $\bt$. Hence, Steps $1-6$ can be performed in a pre-processing phase and do not contribute to the actual running time. See App. \ref{app:alg:illustration} for an illustration of Alg. 1 and runtime analysis. \nocite{RIM}
\newcommand{\calib}{\texttt{Cal}}
\section{Evaluation}\label{sec:eval}
\vspace{-1.5em}
\begin{figure*}[ht]
   \begin{subfigure}[b]{0.25\linewidth}
       \centering
       \includegraphics[width=0.95\linewidth]{./figures/Texas_attack.png}
    %  \vspace{-0.15cm}
       \caption{\textit{PUDF}: Attack}
       \label{fig:Texas:attack}
    \end{subfigure}%%
    % \begin{subfigure}[b]{0.24\linewidth}
    %     \centering
    %     \includegraphics[width=\linewidth]{./figures/Adult_attack_new.png}
    %     %\vspace{-0.15cm}
    %     \caption{\textit{Adult}: Attack }%($r$)}
    %     \label{fig:Adult:attack}
    % \end{subfigure}%%
    \begin{subfigure}[b]{0.25\linewidth}
        \centering
        \includegraphics[width=0.95\linewidth]{./figures/Twitch_attack.png}
        \caption{\textit{Twitch}: Attack}
        \label{fig:Twitch:attack}
    \end{subfigure}%%
    %   \begin{subfigure}[b]{0.24\linewidth}
    %     \centering
    %     \includegraphics[width=\linewidth]{./figures/Adult_alpha.png}
    %     %   \vspace{-0.15cm} 
    %     \caption{\textit{Adult}: Attack ($\alpha$)}
    %     \label{fig:Adult:alpha}
    % \end{subfigure}\\
    \begin{subfigure}[b]{0.25\linewidth}
        \centering
        \includegraphics[width=0.95\linewidth]{./figures/Texas_utility_1.png}
        %\vspace{-0.15cm}
        \caption{\textit{PUDF}: Learnability}
        \label{fig:Texas:utility}
    \end{subfigure}%%
    % \begin{subfigure}[b]{0.24\linewidth}
    %     \centering
    %     \includegraphics[width=\linewidth]{./figures/Adult_utility_1.png}
    %     %\vspace{-0.15cm}
    %     \caption{\textit{Adult}: Learnability}
    %     \label{fig:Adult:utility}
    % \end{subfigure}
    \begin{subfigure}[b]{0.25\linewidth}
        \centering
        \includegraphics[width=0.95\linewidth]{./figures/Twitch_utility.png}
        \caption{\textit{Twitch}: Learnability}
        \label{fig:Twitch:utility}
    \end{subfigure}
    % \begin{subfigure}[b]{0.24\linewidth}
    %     \centering
    %     \includegraphics[width = \linewidth]{./figures/Syn_utility.png}
    %     %\vspace{-0.15cm}
    %     \caption{\textit{Syn}: Learnability}% that acts as the public auxiliary information.}
    %     \label{fig:Syn:utility}
    % \end{subfigure}
    \vspace{-0.2cm}
   \caption{
   Our scheme interpolates between standard LDP (orange line) and uniform shuffling (blue line) in both privacy and data learnability. All plots increase group size along x-axis (except (d)). 
   (a) $\rightarrow$ (b): The fraction of participants vulnerable to an inferential attack.  
   %(d): Attack success with varying $\alpha$ for a fixed $r$. 
   (c) $\rightarrow$ (d): The accuracy of a calibration model trained on $\bz$ predicting the distribution of \ldp outputs at any point $t \in \calT$, such as the distribution of medical insurance types used specifically in the Houston area (not possible when uniformly shuffling across Texas). 
   %(h): Test accuracy of a classifier trained on $\bz$ for the synthetic dataset in Fig. \ref{fig:demonstration}. 
   }
   \label{fig:results}
   \vspace{-0.3cm}
\end{figure*}


The previous sections describe how our shuffling framework interpolates between standard \ldp and uniform random shuffling. We now experimentally evaluate this asking the following two questions -- 

\textbf{Q1.} Does the Alg. 1 mechanism protect against realistic inference attacks? \\\textbf{Q2.} How well can Alg. 1 tune a model's ability to learn trends within the shuffled data, i.e., tune \emph{data learnability}?

We evaluate on four datasets.  We are not aware of any prior work that provides comparable local inferential privacy. Hence, we baseline our mechanism with the two extremes: standard \textsf{LDP} and uniform random shuffling.  For concreteness, we detail our procedure with the \textit{PUDF} dataset~\citep{PUDF} \href{https://www.dshs.state.tx.us/THCIC/Hospitals/Download.shtm}{(license)}, which comprises $n \approx 29$k psychiatric patient records from Texas. Each data owner's sensitive value $x_i$ is their medical payment method, which is reflective of socioeconomic class (such as medicaid or charity). Public auxiliary information $t \in \calT$ is the hospital's geolocation. Such information is used for understanding how payment methods (and payment amounts) vary from town to town for insurances in practice \citep{insurance}. Uniform shuffling across Texas precludes such analyses. Standard \ldp risks inference attacks, since patients attending hospitals in the same neighborhood have similar socioeconomic standing and use similar payment methods, allowing an adversary to correlate their noisy $y_i$'s. To trade these off, we apply Alg. 1 with $d(\cdot)$ being distance (km) between hospitals, $\alpha = 4$ and Kendall's $\tau$ rank distance measure for permutations. %For each $r$, we recompute the grouping $\calG$, and produce a new shuffled $\bz$. 

Our inference attack predicts $\DO_i$'s $x_i$ by taking a majority vote of the $z_j$ values of the $25$ data owners within $r^*$ of $t_i$ and who are most similar to $\DO_i$ w.r.t some additional privileged auxiliary information $t^p_j \in \calT_p$. For PUDF, this includes the $25$ data owners who attended hospitals that are within $r^*$ km of $\DO_i$'s hospital, and are most similar in payment amount $t^p_j$. Using an \scalebox{1}{$\epsilon = 2.5$} randomized response mechanism, we resample the \ldp sequence $\by$ 50 times, and apply Alg. 1's chosen permutation to each, producing 50 $\bz$'s. We then mount the majority vote attack on each $x_i$ for each $\bz$. If the attack on a given $x_i$ is successful across $\geq 90\%$ of these \ldp trials, we mark that data owner as vulnerable -- although they randomize with \textsf{LDP}, there is a $\geq 90\%$ chance that a simple inference attack can recover their true value. We record the fraction of vulnerable data owners as $\rho$. %(an adversary could very feasibly know which data owners are vulnerable a priori). 
We report 1-standard deviation error bars over 10 trials.   

Additionally, we evaluate \emph{data learnability} --  how well the underlying statistics of the dataset are preserved across $\calT$. %how well a model trained on $\bz$ can infer the distribution of $x_i$'s at any point $t \in \calT$. 
For \textit{PUDF}, this means training a model on the shuffled $\bz$ to predict the distribution of payment methods used near, for instance, $t_i=$ Houston for $\DO_i$. For this, we train a calibrated model, \scalebox{1}{$\calib:\calT \rightarrow \mathcal{D}_x$}, on the shuffled outputs where \scalebox{1}{$\mathcal{D}_x$} is the set of all distributions on the domain of sensitive attributes \scalebox{1}{$\calX$}. We implement $\calib$ as a gradient boosted decision tree (GBDT) model \citep{gradientboosting} calibrated with Platt scaling \citep{calibration}. For each location $t_i$, we treat the empirical distribution of $x_i$ values within $r^*$ as the ground truth distribution at $t_i$, denoted by \scalebox{1}{$\mathcal{E}(t_i) \in \mathcal{D}_x$}. Then, for each $t_i$, we measure the Total Variation error between the predicted and ground truth distributions \scalebox{1}{$\text{TV}\big( \mathcal{E}(t_i), \calib_r(t_i) \big)$}. We then report $\lambda(r)$ -- the average TV error for distributions predicted at each $t_i \in \bt$ normalized by the TV error of naively guessing the uniform distribution at each $t_i$. With standard \textsf{LDP}, this task can be performed relatively well at the risk of inference attacks. With uniformly shuffled data, it is impossible to make geographically localized predictions unless the distribution of payment methods is identical in every Texas locale. 
 



% \textbf{Datasets.} We use the following datasets: 
%  \vspace{-0.3cm}
% \squishlist 
% \item \textit{PUDF}~\cite{PUDF}. This is a medical dataset of $\approx29K$ psychiatric patient discharge records from Texas. %Each entry has 31 attributes.
% \item \textit{Adult}~\cite{UCI}. This dataset is derived from the 1994 Census and has $\approx33$K records. 
% \item \textit{Twitch}~\cite{twitch}. This dataset, gathered from the \emph{Twitch} social media platform, includes a graph of 9,498 edges (mutual friendships) along with node features. 
% \item \textit{Syn}~\cite{Syn}. A synthetic dataset of size $20K$ which can be classified at three granularities -- 8-way, 4-way and 2-way as represented by Fig. \ref{fig:data} (which shows a scaled down version of the dataset). 
% \squishend
% \vspace{-0.3cm}
% Due to space constraints, we present the results for Twitch in App. \ref{app:extraresults}.\\
% \textbf{Setup.} In our experiments, the adversary $\mathbb{A}$ has access to (1) public auxiliary information $\bt\in\mathcal{T}^n$  (2) privileged auxiliary information $\bt_P\in{\mathcal{T}_{P}}^n$ which is \textit{not used} by Alg. \ref{algo:main}. For each $\DO_i$, the adversary (1) constructs a group $G_i^{\mathbb{A}}$ based on $\calT$ and threshold $r^*$, (2) finds a subset of $k=25$ members from $G_i^{\mathbb{A}}$ based on $\calT_P$ (denoted as $K_i^{\mathbb{A}}$),  (3)  predicts $\DO_i$'s sensitive attribute by a majority vote over the noisy (\textsf{LDP}) values corresponding to $K_i^{\mathbb{A}}$ ($k$-NN classifier). The attack is considered a success if it predicts correctly with probability $\geq 0.9$ for our trials. We report the efficacy of the attack in terms of the \% of total data owners inferred, $\rho$. 
% \newcommand{\calib}{\texttt{Cal}}
% %To demonstrate the practical privacy/utility tradeoff of our method, we compare the risk of a $k$ Nearest Neighbor ($k$NN) correlation attack with the utility of a model calibration task. 
% \\We measure the learnability of the data (w.r.t $\bt$) by computing how well the underlying statistics of the dataset are preserved across $\calT$. For this, we train a calibrated model, \scalebox{1}{$\calib:\calT \rightarrow \mathcal{D}_x$}, on the shuffled outputs where \scalebox{1}{$\mathcal{D}_x$} is the set of all distributions on the domain of sensitive attributes \scalebox{1}{$\calX$}. We implement $\calib$ as a gradient boosted decision tree (GBDT) model \cite{gradientboosting} calibrated with Platt scaling \cite{calibration}. We treat the empirical distribution of \scalebox{1}{$\bx_{G_i}$}(data corresponding to $G_i$), denoted by \scalebox{1}{$\mathcal{E}(t_i) \in \mathcal{D}_x$}, as the ground truth distribution at $t_i$, where $G_i$ is define w.r.t threshold $r^*$. Then, for each $t_i$, we measure the Total Variation (TV) distance between the predicted and ground truth distributions. For each privacy radius $r$, we generate $\bz$, train \scalebox{1}{$\calib_r$} on \scalebox{1}{$(\bz, \bt)$}, and report  \scalebox{0.87}{$\lambda(r) \hspace{-0.1cm}= \hspace{-0.01cm} \sum_{i=1}^n  \hspace{-0.1cm}\text{TV}\big( \mathcal{E}(t_i), \calib_r(t_i) \big) \Big/ n \lambda_o$}  \hspace{-0.1cm}-- the mean TV error normalized by the TV error of uniform distribution, \scalebox{1}{$\lambda_o$}.
% \squishlist \vspace{-0.25cm} \item \textit{PUDF}.  The payment method (which can be reflective of social class) is considered private $\calX = \{$ medicare, medicaid, charity, private$\}$. $\calT$ is the hospital zipcode and $\calT_P$ is the charge amount. In fact, this correlation is used for medical insurances in practice \cite{insurance}. \vspace{-0.05cm}
% \item \textit{Adult}. Whether $\DO_i$'s annual income is $\geq 50$k is considered private, $\calX = \{\geq 50k, <50k\}$. $\calT = [17, 90]$ is age and $\calT_P$ is the  individual's marriage status.\vspace{-0.05cm}
% %\item \textit{Twitch}. The user's history of explicit language is private $\calX = \{0,1\}$. $\calT$ is a user's mutual friendships as represented by a public social media graph, i.e. $t_i$ is the $i$'th row of the graph's adjacency matrix. We do not have any $\calT_P$ here, and select the 25 nearest neighbors randomly. 
% \item \textit{Syn}. The eight color labels are private $\calX = [8]$; the 2D-positions are public $\calT = \R^2$. For learnability, we measure the accuracy of $8$-way, $4$-way and $2$-way GBDT models trained on $\bz$ on an equal sized test set at each $r$.\vspace{-0.1cm}
% \squishend  \vspace{-0.2cm}
% All experiments have been implemented in Python and we report the mean over $10$ trials. Additionally, we use $\epsilon=2.5$, $\alpha=4$, and Kendall's $\tau$ rank distance measure. We are not aware of any prior works with mechanisms that provide comparable locally private guarantees. As such we baseline our mechanism with the two extremes of shuffling: no shuffling at all (standard \ldp), and uniform random shuffling as in \cite{shuffle2, shuffling1}. Our mechanism offers interpolation between these two baselines. 

We additionally perform the above experiments on the following three datasets 
\squishlistfour \vspace{-0.25cm} 
% \item \textit{PUDF}.  The payment method (which can be reflective of social class) is considered private $\calX = \{$ medicare, medicaid, charity, private$\}$. $\calT$ is the hospital zipcode and $\calT_P$ is the charge amount. In fact, this correlation is used for medical insurances in practice \cite{insurance}. \vspace{-0.05cm}
\item \textit{\href{http://snap.stanford.edu/data/twitch-social-networks.html}{Twitch}}~\citep{twitch}. This dataset, gathered from the \emph{Twitch} social media platform, includes a graph of $\approx 9K$ edges (mutual friendships) along with node features. The user's history of explicit language is private $\calX = \{0,1\}$. $\calT$ is a user's mutual friendships, i.e. $t_i$ is the $i$'th row of the graph's adjacency matrix. We do not have any $\calT_P$ here and select the 25 neighbors randomly. 
\item \textit{Syn}. This is a synthetic dataset of size $20K$ which can be classified at three granularities -- 8-way, 4-way and 2-way  (Fig. \ref{fig:data} shows a scaled down version of the dataset). The eight color labels are private $\calX = [8]$; the 2D-positions are public $\calT = \R^2$. For learnability, we measure the accuracy of $8$-way, $4$-way and $2$-way GBDT models trained on $\bz$ on an equal sized test set at each $r$.
\item \textit{\href{https://archive.ics.uci.edu/ml/datasets/Adult}{Adult}}~\citep{adult}. This dataset is derived from the 1994 Census and has $\approx33$K records.  Whether $\DO_i$'s annual income is $\geq 50$k is considered private, $\calX = \{\geq 50k, <50k\}$. $\calT = [17, 90]$ is age and $\calT_P$ is the  individual's marriage status. Due to lack of space figures are in App. \ref{app:adult experiments}. %\textcolor{blue}{Trends observed on Adult reinforce those of the three datasets included here, see  for figures.}
\vspace{-0.05cm}
\vspace{-0.1cm}
\squishendfour 

%\vspace{-0.3cm}
\textbf{Experimental Results.}
%The plots of Figure \ref{fig:results} indicate that our scheme successfully interpolates between standard \ldp and uniform shuffling. We return to our two primary questions: 
\\\textbf{Q1.} Our formal guarantee on the inferential privacy loss (Thm. \ref{thm: semantic guarantee}) is described w.r.t to a `strong' adversary (with access to  \scalebox{1}{$\{y_{G_i}\},\by_{\overline{G}_i}$}). Here, we test how well does our proposed scheme (Alg. 1) protect against inference attacks on real-world datasets without any such assumptions. Additionally, to make our attack more realistic, the adversary has access to extra privileged auxiliary information $\calT_P$ which is \textit{not used} by Alg. \ref{algo:main}.  Fig. \ref{fig:Texas:attack}$\rightarrow$ \ref{fig:Twitch:attack} show  that our scheme significantly reduces the attack efficacy. For instance, $\rho$ is reduced by \scalebox{1}{$2.7X$} at the attack distance threshold $r^*$ for \textit{PUDF}.  \begin{wrapfigure}{r}{0.3\linewidth}
     \vspace{-0.4cm}
    \hspace{-0.8cm}
    \centering
    \includegraphics[height=3cm]{./figures/Syn_utility.png} 
    \vspace{-0.3cm}
    \caption{\textit{Syn}: Learnability}% that acts as the public auxiliary information.}
    \label{fig:Syn:utility}
    \vspace{-1.5em}
\end{wrapfigure} Additionally, $\rho$ for our scheme varies from that of \textsf{LDP}\footnote{Our scheme gives lower $\rho$ than \ldp at \scalebox{1}{$r=0$} because the resulting groups are non-singletons. For instance, for PUDF, $G_i$ includes all individuals with the same zipcode as $\DO_i$.} (minimum privacy)   to uniform shuffle (maximum privacy) with increasing $r$ (equivalently group size as in Fig. \ref{fig:Twitch:attack}) thereby spanning the entire privacy spectrum. As expected, $\rho$ decreases with decreasing privacy parameter $\alpha$ (Fig. \ref{fig:Adult:alpha}).
\\\textbf{Q2.} Fig.\ref{fig:Texas:utility} \scalebox{1}{$\rightarrow$} \ref{fig:Twitch:utility} show that $\lambda$ varies from that of \ldp (maximum learnability) to that of uniform shuffle (minimum learnability) with increasing $r$ (equivalently, group size), thereby providing tunability.  Interestingly, for \textit{Adult} our scheme reduces $\rho$ by \scalebox{1}{$1.7X$} at the same $\lambda$ as that of \ldp for \scalebox{1}{$r=1$} (Fig. \ref{fig:Adult:utility}). Fig. \ref{fig:Syn:utility} shows that the distance threshold $r$ defines the granularity at which the data can be classified. \ldp allows 8-way classification while uniform shuffling allows none. The granularity of classification can be tuned by our scheme -- $r_8$, $r_4$ and $r_2$ mark the thresholds for $8$-way, $4$-way and $2$-way classifications, respectively. %Experiments on evaluation of \scalebox{1}{$(\eta,\delta)$}-preservation are in  App. \ref{app:extraresults}.

\section{Conclusion}
We have proposed a new privacy definition, \name-privacy that casts new light on the inferential privacy benefits of shuffling and a novel shuffling mechanism to achieve the same. %Additionally, we propose a generalized shuffle framework that interpolates between \ldp and uniform shuffling in terms of the protection afforeded against inference attacks and data learnability.




\graphicspath{{./chapters/chapter5/}}
\chapter{Location Trace Privacy}

\newcommand{\inputTikZ}[2]{%  
     \scalebox{#1}{\input{#2}}  
}

\newcommand{\bG}{\mathbf{G}}
\newcommand{\bT}{\mathbf{T}}
\newcommand{\trace}{\mathbf{tr}}
\newcommand{\calF}{\mathcal{F}}
\newcommand{\calN}{\mathcal{N}}
\newcommand{\calQ}{\mathcal{Q}}
\newcommand{\calR}{\mathcal{R}}
\newcommand{\Zu}{{Z_{\mathbb{I}_U}}}
\newcommand{\Zs}{{Z_{\mathbb{I}_S}}}
\newcommand{\Zx}{{Z_{\mathbb{I}_x}}}
\newcommand{\Zy}{{Z_{\mathbb{I}_y}}}
\newcommand{\Xu}{{X_{\mathbb{I}_U}}}
\newcommand{\Xs}{{X_{\mathbb{I}_S}}}
\newcommand{\Xx}{{X_{\mathbb{I}_x}}}
\newcommand{\Xy}{{X_{\mathbb{I}_y}}}
\newcommand{\Gu}{{G_{\mathbb{I}_U}}}
\newcommand{\Gs}{{G_{\mathbb{I}_S}}}
\newcommand{\Gx}{{G_{\mathbb{I}_x}}}
\newcommand{\Gy}{{G_{\mathbb{I}_y}}}
\newcommand{\Spairs}{\calS_{\text{pairs}}}
\newcommand{\Sigmag}{\Sigma^{(g)}}
\newcommand{\Sigmaeff}{\Sigma_{\text{eff}}}
\newcommand{\maxeig}{{\max \text{eig}}}
\newcommand{\leff}{l_{\text{eff}}}
\newcommand{\Is}{{\mathbb{I}_S}}
\newcommand{\Iu}{\mathbb{I}_U}
\newcommand{\Ix}{\mathbb{I}_x}
\newcommand{\Iy}{\mathbb{I}_y}


\newtheorem{prop}[theorem]{Proposition}

\theoremstyle{definition}

%For moving around footnote at end 
\newcommand\blfootnote[1]{%
  \begingroup
  \renewcommand\thefootnote{}\footnote{#1}%
  \addtocounter{footnote}{-1}%
  \endgroup
}
\section{Introduction}
\label{sec:introduction}

%motivation + specification 
Location data is acutely sensitive information, detailing where we live, work, eat, shop, worship, and often when, too. Yet increasingly, location data is being uploaded for smartphone services such as ride hailing and weather forecasting and then being brokered in a thriving user location aftermarket to advertisers and even investors \citep{nyt}. Users share location `traces' when they release a sequence of locations, often across a short period of time. These traces are then used by central servers to monitor traffic trends, track individual fitness, target marketing, and even to study the effectiveness of social-distancing ordinances \citep{wash_post}. Here, we aim to provide a \emph{local} privacy guarantee, wherein traces are sanitized at the user level before being transmitted to a centralized service. Note that this requires different guarantees and mechanisms than in \emph{aggregate} applications making queries on large location trace databases. 

Specifically, we guarantee a radius $r$ of privacy at any sensitive time point or combination of time points within a given trace. This is challenging due to the fact that the locations within traces are highly inter-dependent. Informally, traces tend to follow relatively smooth trajectories in time. If not sanitized carefully, that knowledge alone may be exploited to infer actual locations from the released version of the trace. This work centers on designing meaningful privacy definitions and corresponding mechanisms that takes this dependence into account. 


Broadly speaking, the vast majority of prior work on rigorous data privacy can be divided into two classes that differ by the kind of guarantee offered: differential and inferential privacy. Differential privacy (DP) guarantees that the participation of a single person in a dataset does not change the probability of any outcome by much. In contrast, inferential privacy guarantees that an adversary who has a certain degree of prior knowledge cannot make certain sensitive inferences.

DP for releasing aggregate statistics of a spatio-temporal dataset has been well studied \citep{traffic_monitoring, quantifying_dp_cao, bayesian_DP, dependent_dp}. There, the idea is to add enough noise to released statistics such that the effect of any user's participation is obscured, even if their locations are highly correlated to each other or to those of other users. Here, such a guarantee does not apply since we aim to release a sanitized version of a single user's trace.

In this local case we cannot rule out the possibility that the data curator knows who each individual is and who participated. Instead, we want to guarantee that event level information \emph{about} each trace remains private. In this work, at any sensitive time $t$ we mask whether the user visited location A or location B for any A,B less than $r$ apart. Without \emph{ad hoc} modifications, standard DP tools are insufficient for achieving this for the primary reasons that 1) the domain of location is virtually unbounded and 2) locations are highly dependent across a short period of time. To see this, consider the following instinctual approaches to achieving location trace privacy. 

\paragraph{Approach A:} apply Local Differential Privacy (LDP) to each trace. Imagine a dataset of traces, each from a separate individual. Applying LDP implies that every trace has nearly the same probability of releasing the same sanitized version. This would be robust to arbitrary side information about dependence between locations in any one trace. Unfortunately, the amount of additive noise needed to achieve this would destroy nearly all utility: sanitized traces from California would have almost the same probability of showing up in Connecticut as do those from New York. Even if we constrained the domain to just Manhattan, this definition would not permit enough utility to perform e.g. traffic monitoring. 

\paragraph{Approach B:} apply LDP to each location within a trace. To preserve some utility, imagine a single trace as a dataset of $n$ locations, each of which enjoys $\varepsilon$-LDP guarantees. This alone is not robust to arbitrary dependence between locations. By the logic of group LDP, it does satisfy $k \varepsilon$-LDP regardless of the dependence between any $k$ locations. This approach has two setbacks. First, how to set $k$ is unclear. Technically, all points in the trace are correlated, so to ward off worst-case correlations one might set it to the length of the trace, which is identical to Approach A. Second, even if location is bounded to a single city or county, satisfying this definition would still destroy nearly all utility. We cannot use sanitized traces for traffic monitoring if locations from either side of town have about same probability of being sanitized to the same value. 

\paragraph{Approach C:} apply LDP guarantees to each location within a trace, but only within any region less than width $r$. This definition is known as Geo-Indistinguishability (GI) \citep{GI}. GI provides a substitute for restricting the domain of location allowing us to salvage some utility. Here, only locations within $r$ of each other are required to have $\varepsilon$-LDP guarantees. In DP parlance, we might say that `neighboring traces' have one location altered by $\leq r$ and are identical everywhere else. This gives us the guarantee we want for a trace with one location, but not with more than one location. To see why, compare with Approach B. Analogously, $(\varepsilon, r)$-GI along a trace provides $(k \varepsilon, r)$-GI to any subset of $k$ locations. Like Approach B, setting $k$ is unclear. Yet unlike Approach B, GI is not resistant to arbitrary dependence between any $k$ locations. Any dependence where a change in one or more location(s) by $r$ implies a change in some other location(s) by $\geq r$ breaks the GI guarantee. Even with the simplest models of dependence (e.g. if we know the true trace ought to move in a straight line) this is a problem. 

To reiterate, applying LDP to traces or to locations within traces (Approaches A \& B) does not provide a principled method for meaningful privacy with reasonable utility. GI adapts LDP by giving guarantees only within a radius $r$. But in relaxing LDP, GI compromises the standard DP tools for handling obvious dependences between data-points like group DP. In our eyes, this warrants an \emph{inferentially private} approach. Here, we continue to provide privacy within a radius $r$, thus allowing for utility. Yet instead of providing resistance to arbitrary dependence across any $k$ locations, we aim to provide resistance to natural models of dependence between all locations. One may view such models as an adversary's prior beliefs about what traces are likely, like the straight-line prior mentioned earlier. 


In contrast with differential privacy, providing inferential privacy guarantees is more complex, and has been less studied. It is however appropriate for applications such as ours, where information must be released based on a single person's data, the features of which are private and dependent. \cite{pufferfish} provide a formal inferential privacy framework called Pufferfish, and design mechanisms for specific Pufferfish instances. As these instances do not apply to our setting, we adapt the Pufferfish framework to location privacy and more broadly to releasing any sequence of real-valued private information. 

%GRAPHICAL MODEL
\begin{figure*}[h]
	\centering
	\begin{subfigure}[b]{.45\textwidth}
		\centering
		\inputTikZ{0.8}{chapters/chapter5/graphical_model.tex}
		\caption{}
		\label{fig:full model}
	\end{subfigure}
	\begin{subfigure}[b]{.45\textwidth}
		\centering
		\inputTikZ{0.8}{chapters/chapter5/graphical_model-1.tex}
		\caption{}
		\label{fig:condensed model}
	\end{subfigure}
	\caption{(a) An example graphical model of a four point trace $X$. (b) The more general grouped version of the model in (a), with the secret set $\Xs = \{X_1, X_2\}$ and the remaining set $\Xu = \{X_3, X_4\}$. 
	%$Z_u$, $\Zs$ are shown.
	}
	\label{fig:graphical models}
\end{figure*}

\paragraph{Contributions:}In this work, we propose an inferentially private approach to guaranteeing a radius $r$ of privacy for sensitive points in location traces in three parts: 
\begin{itemize}
	\item First, we propose an adaptable privacy framework tailored to sequences of highly dependent datapoints that adapts Pufferfish privacy \citep{Pufferfish} to use R\'enyi Differential Privacy (RDP) \citep{renyi}. Given a model of dependence between points, this framework more appropriately estimates the risk of inference within radius $r$ on points of interest than do vanilla LDP approaches. 
	\item We then demonstrate how to implement our framework for the highly flexible and expressive setting of Gaussian process (GP) priors. These nonparametric models capture the spatiotemporal aspect of location data \citep{PCS_GP, ATM_GP, Traffic_GP}. GPs have a natural synergy with R\'enyi privacy enabling an interpretable upper bound on privacy loss for additive Gaussian privacy mechanisms (that add Gaussian noise to each point). Using this, we design a semidefinite program (SDP) that optimizes the correlation of such mechanisms to minimize privacy loss without destroying utility, efficiently thwarting the inference of sensitive locations. 
	\item Finally, we provide experiments on both location trace and home temperature data to demonstrate the advantage of these techniques over Approach C mechanisms like GI. We find that our mechanisms successfully obscure sensitive locations while respecting utility constraints, even when the prior model is misspecified. 
\end{itemize}
 
Ultimately, by resisting only reasonable kinds of dependence in the data we are able to offer both meaningful privacy and utility. We show that our framework is robust to misspecification of this reasonable dependence and offers a privacy loss that is both tractable and interpretable. 



\section{Preliminaries and Problem Setting}
\label{sec:preliminaries}
%notation / definitions
A user transmits a sequence of $N$ 2-dimensional locations along with their corresponding timestamps, collectively forming a `trace'. We `unroll' the trace into $n$ real-valued random variables $X = \{X_1, X_2, \dots, X_n\}$. A trace of 10 2d locations has $n = 2 \times 10 = 20$ random variables $X_i$. Instead of releasing the raw trace $X$, the user releases a private version $Z = \{Z_1, Z_2, \dots, Z_n\}$, by way of an additive noise mechanism $Z = X + G$, where $G = \{G_1,G_2, \dots, G_n\}$ is random noise produced by a privacy mechanism.

%notation / definitions 
An adversary, receiving the obscured trace $Z$, then reasons about the true locations at some sensitive time(s). To reference the sensitive times, we use index set $\Is$. If the sensitive indices are $\Is = \{1,2\}$, the corresponding location values are $\Xs = \{X_{1}, X_{2}\}$ (e.g. referring to the two coordinates of one location). When inferring the true value of $\Xs$, the adversary makes use of the remaining points in the trace at indices $\mathbb{I}_U = [n] \backslash \Is$, denoted $\Xu$, with obscured values $\Zu$. This separation of points into $\Xs$ and $\Xu$ is represented in \textbf{Figure \ref{fig:graphical models}}. 

We use location as a guiding example, but such inter-dependent traces $X$ could take the form of home temperature time series data or spatial data like 3D facial maps used for identification. Going forward, we will continue to denote $X = \{X_1, X_2, \dots, X_n\}$ with the understanding that \emph{any} subsequence of $d$ points e.g. $\Xs = \{X_2, X_6, \dots\}$ could represent a $d$-dimensional sensitive value, or $Nd$ points could represent $N$ $d$-dimensional sensitive values. 

For the real-valued distributions considered here, $P_{\times}( \bullet )$ refers to a density of distribution $\times$ on r.v. $\bullet$ and $P_{\times}(\bullet | *)$ is its regular conditional density given $*$. 

\subsection{Background}
GI limits what can be inferred about the sensitive $\Xs$ from its corresponding $\Zs$, but not from the remaining locations $\Zu$. To do so we need a privacy definition that specifies what events of random variable $\Xs$ we wish to obscure, which realistic priors of inter-dependence to protect against, and a privacy loss. 

\subsection{Basic and Compound Secrets}

We borrow heavily from the Pufferfish framework \citep{pufferfish}, and specialize it for the setting of location traces. We define our own set of \emph{secrets} --- the collection of events we wish to obscure --- and \emph{discriminative pairs}, the pairs of secret events we do not want an adversary to tell between. 
%To give privacy within a radius $r$, we do not want an adversary to be able to infer whether one visited any two locations A vs. B within $r$ of each other. To formalize this, we define two classes of secrets: 

\paragraph{Basic Secrets \& Pairs} 
After releasing $Z$, we do not want an adversary with a reasonable prior on $X$, $\calP \in \Theta$, to have sharp posterior beliefs about the user's location at some sensitive time (e.g. one of the sensitive times in \textbf{Figure \ref{fig:nyc_example}} of Appendix \ref{apx: Illustrations}). As such, the adversary cannot distinguish whether the user visited location A or some nearby location B at that time. Let $x_s\in \R^2$ represent a possible assignments to $\Xs$, hypothesizing the true sensitive location. Any such assignment is secret, $\calS = \{ \Xs = x_s  : x_s \in \R^2\}$. Specifically, we want the posterior probability of any two assignments to $\Xs$ within a radius $r$ to be close: $\Spairs = \{(x_s, x_s'):\|x_s - x_s'\|_2 \leq r\}$. This protects a single time within a trace of locations. More generally, in the context of spatiotemporal data of any dimension, we call this a \emph{basic secret}. 

\paragraph{Compound Secrets \& Pairs} 
%\textbf{Figure \ref{fig:nyc_trace}} has not one but three time points that have sensitive locations in the trace. Let their index sets be ${\Is}_1$, ${\Is}_2$, and ${\Is}_3$. 
Suppose we have three sensitive times (again as in \textbf{Figure \ref{fig:nyc_example}}). A mechanism that blocks inference on each of these separately does not prevent inference on the combination of them simultaneously. To obscure hypotheses on \emph{all three} of these, we modify our set of secrets to any combination of assignments to each secret location: 
\begin{align*}
	\calS = \big\{ \{\Xs_1 = x_{s1}\} \cap \{\Xs_2 = x_{s2}\} \cap \{\Xs_3 = x_{s3} &\} \\
	: x_{si} \in \R^2, i \in [3] &\big\} \ .
\end{align*} 
Now, the set of discriminative pairs is any two assignments to all three secret locations: 
\begin{align*}
	\Spairs = \Big\{\big( \{x_{s1}, x_{s2}, x_{s3}\} &, \{x_{s1}', x_{s2}', x_{s3}'\}\big) \\
	&: \| x_{si} - x_{si}' \|_2 \leq r, \ i \in [3] \Big\}
\end{align*}
This protects against compound hypotheses: if daycare and work are within $r$ of each other, this keeps an adversary from inferring $\Xs_1 = $ `daycare' \emph{and} $\Xs_2 = $ `work' versus $\Xs_1 = $ `work' \emph{and} $\Xs_2 = $ `daycare'. More generally, in the context of spatiotemporal data of any dimension, we call this a \emph{compound secret}. Intuitively, a mechanism that protects a compound secret of locations close together in time prevents a Bayesian adversary from leveraging the remainder of the trace to infer direction of motion at those sensitive times. Note that bounding the privacy loss of a compound secret does not bound the privacy loss of its constituent basic secrets.

Going forward, we refer to $\Is$ as the `secret set'. 

\subsubsection{Gaussian Processes}
For the purpose of location privacy, it is important to choose a prior class $\Theta$ such that the conditional distribution $P_\calP(\Xu | \Xs)$ is simple to compute for any secret set $\Is$ and any prior $\calP \in \Theta$. Of course, it is also critical that the prior class naturally models the data, and thus consists of `reasonable assumptions' for adversaries. GPs satisfy both these requirements. We model a full $d$-dimensional trace sampled at $N$ times by `unrolling' it into a $n = dN$ dimensional GP. 
\begin{definition}\emph{Gaussian process} 
	A trace $X$ is a Gaussian process if $X_{\mathbb{I}_M}$ has a multivariate normal distribution for any set of indices $\mathbb{I}_M \subset [n]$. If $X$ is a gaussian process, then the function $i \rightarrow \E[X_i]$ is called the mean function and the function $(i,j) \rightarrow \text{Cov}(X_i, X_j)$ is called the kernel function. 
\end{definition}
In this work, the kernel uses locations' time stamps to compute their covariance $(t_i, t_j) \rightarrow \text{Cov}(X_i, X_j)$, but generally could use any side information provided with each location. 
%Let $(X_t^x)_{t \geq 0}$ be the continuous representation of one dimension of trace $X$ in time $t$. 
%\begin{definition}\emph{Gaussian process} 
%	A stochastic process $(X_t^x)_{t \geq 0}$ is considered a Gaussian process if $[X_{t_1}^x, X_{t_1}^x, \dots, X_{t_n}^x]$ has a multivariate normal distribution for any set of times $t_1, t_2, \dots t_n \geq 0$. If $(X_t^x)_{t \geq 0}$ is a gaussian process, then the function $t \rightarrow \E[X_t]$ is called the mean function and the function $(s,t) \rightarrow \text{Cov}(X_s, X_t)$ is called the kernel function. 
%\end{definition}

%This is particularly nice since, if different dimensions are independent, we may treat them as separate traces. If not, correlation across dimensions can be modeled into the covariance matrix. We will describe this in greater detail in the following sections. 

GPs have simple, closed form conditional distributions. Let $X \sim \calN(\mu, \Sigma)$, where $\mu \in \R^{n}$ and $\Sigma \in \R^{{n} \times {n}}$. Then, the random variable $\Xu | \{\Xs = x_s\} \sim \calN(\mu_{u|s}, \Sigma_{u|s})$, where $\mu_{u|s} = \mu_u + \Sigma_{us} \Sigma_{ss}^{-1} (x_s - \mu_s)$ and $\Sigma_{u|s} = \Sigma_{uu} - \Sigma_{us}\Sigma_{ss}^{-1} \Sigma_{su}$. Here, $\mu_s$ denotes the mean vector $\mu$ accessed at indices $\Is$ and $\Sigma_{su}$ denotes the covariance matrix $\Sigma$ accessed at rows $\Is$ and columns $\mathbb{I}_U$. 

For GP priors, we will use additive noise $G \sim \calN(\mathbf{0}, \Sigmag)$. Thus $Z = X + G$, too, is multivariate normal. Furthermore, the distribution of any set of variables conditioned on any other set of variables in \textbf{Figure \ref{fig:graphical models}} belongs to some multivariate normal distribution.

GPs have been shown to successfully model mobility \citep{Traffic_GP, PCS_GP, ATM_GP}, even in the domain of surveillance video \citep{surveillance_GP}.  Furthermore, although these non-parametric models are characterized by second order statistics, GPs are capable of complexity rivaling that of deep neural networks \citep{Deep_NN_GP}, allowing for scalability to more complex models and domains. Our proposed results and algorithms may be applied regardless of the complexity of the chosen GP. 


\subsubsection{R\'enyi Differential Privacy}
\label{sec:renyi_dp}
In the following section, we propose a privacy definition that adapts R\'enyi Differential Privacy (RDP) \citep{renyi} to the Pufferfish framework. RDP resembles Differential Privacy \citep{DP}, except instead of bounding the maximum probability ratio or \emph{max divergence} of the distribution on outputs for two neighboring databases, it bounds the \emph{R\'enyi divergence} of order $\lambda$, defined in Equation \eqref{eqn: renyi} for distributions $\calP_1$ and $\calP_2$. The R\'enyi divergence bears a nice synergy with Gaussian processes. If $\calP_1 = \calN(\mu_1, \Sigma)$ and $\calP_2 = \calN(\mu_2, \Sigma)$ --- two mean-shifted normal distributions --- the R\'enyi divergence takes on a simple closed form shown in Equation \eqref{eqn: normal renyi}. 
%For distributions $\calP_1$ and $\calP_2$ on random variable $X$, recall that the R\'enyi Divergence is defined as in Equation --- (left). Additionally, R\'enyi divergence bears nice synergy with Gaussian processes, since the R\'enyi divergence of any two mean-shifted multivariate normal distributions has a simple closed form. If  as seen below in Equation --- (right). 
\begin{align}
	\label{eqn: renyi}
	D_\lambda \binom{\calP_1}{\calP_2} 
	&= \frac{1}{\lambda - 1} \log \E_{x \sim \calP_2} \Big( \frac{P_{\calP_1}(X = x)}{P_{\calP_2}(X = x)} \Big)^\lambda \\
	\label{eqn: normal renyi}
	&= \frac{\lambda}{2} (\mu_1 - \mu_2)^\intercal \Sigma^{-1} (\mu_1 - \mu_2)
\end{align}
%As shown in \citep{subsampled_renyi}, $(\lambda, \varepsilon)$-RDP can be directly translated into the language of $(\varepsilon, \delta)$-DP, if the DP interpretation is preferred. 
%\begin{align*}
%	D_\lambda \binom{\calN(\mu_1, \Sigma)}{\calN(\mu_1, \Sigma)}
%	&= \frac{\lambda}{2} (\mu_1 - \mu_2)^\intercal \Sigma^{-1} (\mu_1 - \mu_2)
%\end{align*}
We will make use of this in defining and bounding privacy loss in the next section. 
\section{Conditional Inferential Privacy}
We now propose a privacy framework that is tailored to sequences of correlated data, Conditional Inferential Privacy (CIP). CIP guarantees a radius $r$ of indistinguishability for the basic or compound secrets associated with any secret set $\Is$. Specifically, CIP protects against any adversary with a specific prior on \emph{the shape} of the trace, and is agnostic to their prior on the absolute location of the trace. We call the set of such prior distributions a Conditional Prior Class.

\begin{definition} \emph{Conditional Prior Class}
\label{def: conditional prior class}
%	For $X = \{X_1, \dots, X_n\}$, prior distributions $\calP_i, \calP_j$ on $X$ are said to belong to the same conditional prior class $\Theta$ if the distribution of any subset of $X$ conditioned on the rest of $X$ is identical up to a mean shift. Formally, if conditional distributions $P_{\calP_i}(\Xu | \Xs = x_s) = P_{\calP_j}(c_{ij\Is} + \Xu | \Xs = x_s)$ for all $x_s$. Constant $c_{ij\Is}$ may change for different $\calP_i,\calP_j$ and partitions $\Is, \Iu$.
	For $X = \{X_1, \dots, X_n\}$, prior distributions $\calP_i, \calP_j$ on $X$ are said to belong to the same conditional prior class $\Theta$ if a constant shift in the conditioned $x_s$ results in a constant shift on the distribution of $\Xu$. Formally, if conditional distributions $P_{\calP_i}(\Xu | \Xs = x_s) = P_{\calP_j}(\Xu + c_{ij\Is}^u  | \Xs = x_s + c_{ij\Is}^s )$ for all $x_s$.
\end{definition}

For instance, prior $P_{\calP_i}$ may concentrate probability on traces passing through Los Angeles, while $P_{\calP_j}$ concentrates on traces passing through London. Conditioning on each secret in the pair $(x_s, x_s')$ in L.A. is analogous to conditioning on each secret in the pair $(x_s + c_{ij\Is}^s, x_s' + c_{ij\Is}^s)$ in London. The corresponding pair of conditional distributions on $\Xu$ in London ($P_{\calP_j}$) are copies of those in L.A. ($P_{\calP_i}$) shifted by $c_{ij\Is}^u$. What matters is that the set of all pairs of conditional distributions under $P_{\calP_i}$ induced by secret pairs $(x_s, x_s')$ is identical to those under $P_{\calP_j}$ up to a mean shift. See Appendix \ref{apx: GP prior class} for a more detailed discussion of conditional prior classes.  
 
% Two prior distributions $\calP_1$ and $\calP_2$ may have completely different marginal distributions on any $X_i$ but still belong to the same conditional prior class. For instance, $\calP_1$ could concentrate probability on the event that the middle point in the trace passes through Manhattan, and $\calP_2$ could have high uncertainty over which state in the U.S. the trace is in. As long as they have the same prior on the dependence between points, the trace is protected by the same CIP mechanism. 

\begin{definition} \emph{ $(\varepsilon, \lambda)$-Conditional Inferential Privacy $(\Spairs, r, \Theta)$}
	Given compound or basic discriminative pairs $\Spairs$ associated with $\Is$, a radius of privacy $r$, a conditional prior class, $\Theta$, and a privacy parameter, $\varepsilon > 0$, a privacy mechanism $Z = \calA(X)$ satisfies $(\varepsilon, \lambda)$-CIP$(\Spairs, r, \Theta)$ if for all $(s_i, s_j) \in \Spairs$, 
%	where 
%	\begin{align*}
%	\Spairs = \Big\{\big( \{x_{s1}, x_{s2}, \dots\}&, \{x_{s1}', x_{s2}', \dots\}\big) \\
%	 &: \| x_{sk} - x_{sk}' \|_2 \leq 2r, \forall \ k\Big\} 
%	\end{align*} 
	and all prior distributions $\calP \in \Theta$, where $P_\calP(s_i), P_\calP(s_j) > 0$, 
	\begin{align}
		\label{eqn:CIP loss}
		D_\lambda \binom{P_{\calA, \calP} (Z | \Xs = s_i)}{P_{\calA, \calP} (Z | \Xs = s_j)} &\leq \varepsilon
	\end{align}
\end{definition}

%Comparison with DP and PF
CIP departs from DP type notions of privacy like Approaches A$\rightarrow$C primarily by resisting only a restricted class of inter-dependence --- the conditional prior class --- as opposed to arbitrary dependence of any $k$ locations. Unlike approaches A and B, we are able to preserve utility for tasks like traffic monitoring. Unlike approach C, CIP is still resistant to realistic models of location inter-dependence. 

While this definition borrows heavily from the Pufferfish framework, it has a few key modifications. Pufferfish is generally described from a central, not local model. We specialize the kinds of secrets and discriminative pairs for the case of local location trace privacy. Additionally, we specialize the type of prior distribution class needed for this local setting: the conditional prior class. Finally, we relax the strict max divergence (max log odds) criterion of the Pufferfish definition to a R\'enyi divergence. This guarantees that --- with high probability on draws of \emph{realistic} traces $Z|\Xs$ --- the log odds will be bounded by $\varepsilon$. As $\lambda \rightarrow \infty$, the log odds are bounded for all traces, i.e. the max divergence is bounded. We formalize this in Theorem \ref{thm: prior-posterior}. 

The R\'enyi criterion of CIP greatly improves its flexibility. Unlike the standard DP Approaches A$\rightarrow$C which only take probabilities over the mechanism, we do not have full control over the randomness at play: it is partially from $\calA$ defined by us and from $\calP$ intrinsic to the data. Unlike max divergence, R\'enyi divergence is available in closed form for many distributions, allowing for a more flexible privacy framework. The $\lambda$ parameter helps us tune how strict a CIP definition is and how much noise we need to add. This allows us to design mechanisms that are resistant to natural models of dependence while preserving utility. 

%Second, providing a reasonable bound on the max divergence -- even for a restricted conditional prior class -- would require adding an unreasonable amount of noise and destroy utility.

\subsection{Properties}

We now identify key properties that make the CIP guarantee interpretable and robust. 

\paragraph{Interpretability:} CIP guarantees that a Bayesian adversary with any prior distribution on traces $\calP$ in the conditional prior class $\Theta$ does not learn much about basic or compound secrets from the released trace $Z$. For basic secrets, this means that the adversary's posterior beliefs regarding sensitive location $\Xs$ are not much sharper than their prior beliefs before witnessing $Z$.  
\begin{theorem} \emph{Prior-Posterior Gap:} 
\label{thm: prior-posterior}
	An $(\varepsilon, \lambda)$-CIP mechanism with conditional prior class $\Theta$ guarantees that for any event $O$ on sanitized trace $Z$
	\begin{align*}
		\bigg| \log \frac{P_{\calP, \calA}(s_i | Z \in O)}{P_{\calP, \calA}(s_j | Z \in O)} - \log \frac{P_{\calP}(s_i)}{P_{\calP}(s_j)} \bigg| \leq \varepsilon'
	\end{align*}
	for any $\calP \in \Theta$ with probability $\geq 1 - \delta$ over draws of $Z|\Xs=s_i$ or $Z|\Xs=s_j$, where $\varepsilon'$ and $\delta$ are related by
	\begin{align*}
		\varepsilon' = \varepsilon + \frac{\log \nicefrac{1}{\delta}}{\lambda - 1} \ .
	\end{align*}
	This holds under the condition that $Z|\Xs = s_i$ and $Z|\Xs = s_j$ have identical support. 
\end{theorem}
A CIP mechanism depends only on the conditional prior describing the data, not the data itself. Suppose an adversary's prior beliefs on $\Xs$ are uniform over some region. For $\lambda = 5$ and $\varepsilon = 0.1$, there is only a $\approx 1\%$ chance that their posterior odds on $s_i,s_j$ will be more than 3.5, and a $\approx 10\%$ chance that they will be more than 2. This `chance' is over draws of likely remaining locations $\Xu$ and the additive noise $G$.  Proofs of all results are in Appendix \ref{apx: proofs}.

For additive noise mechanisms like $\calA(X) = X + G = Z$, the CIP loss can be split into two terms: one accounting for the direct privacy loss of $\Zs$ on $\Xs$ and a second accounting for the inferential privacy loss of $\Zu$ on $\Xs$ via $\Xu$.

\begin{lemma}\emph{Conditional Independence}
	\label{lem: renyi additive loss}
	For an additive noise mechanism, a fully dependent trace as in \textbf{Figure \ref{fig:full model}}, and any prior $\calP$ on $X$ the CIP loss may be expressed as
	\begin{align}
	\label{eqn:two terms}
		&D_\lambda \binom{P_{\calA, \calP}(Z | \Xs = s_i)}{P_{\calA, \calP}(Z | \Xs = s_j)}  \\ 
		\vspace{2em}
		&= \sum_{i \in \Is} \bigg[ D_\lambda \binom{P_\calA(Z_i | X_i = s_i)}{P_\calA(Z_i | X_i = s_j)} \bigg]
		+ D_\lambda \binom{P_{\calA, \calP}(\Zu | \Xs = s_i)}{P_{\calA, \calP}(\Zu | \Xs = s_j)} \notag
	\end{align}
\end{lemma} 
One interpretation of GI is that it assumes all locations $X_i$ are independent. In this case, the second term vanishes and the privacy loss only depends on randomness of the mechanism, not the prior. 

\paragraph{Robustness:}
\cite{no_free_lunch} show that it is impossible to achieve both utility and privacy resistant to all priors. CIP provides resistance to a reasonable class of priors $\calP \in \Theta$, but it is possible that the true distribution $\calQ \notin \Theta$. In this case, the privacy guarantees degrade gracefully as the divergence between $\calQ$ and $\calP \in \Theta$ grows. 
\begin{theorem}\emph{Robustness to Prior Misspecification}
\label{thm: prior misspecification}
	Mechanism $\calA$ satisfies $\varepsilon(\lambda)$-CIP for prior class $\Theta$. Suppose the finite mean true distribution $\calQ$ is not in $\Theta$. The CIP loss of $\calA$ against prior $\calQ$ is bounded by 
	\begin{align*}
		D_\lambda \binom{P_{\calA, \calQ}(Z | \Xs = s_i)}{P_{\calA, \calQ}(Z | \Xs = s_j)} \leq \varepsilon'(\lambda)
	\end{align*}
	where
	\begin{align*}
		\varepsilon'(\lambda) 
		&= \frac{\lambda - \frac{1}{2}}{\lambda - 1} \ \Delta(2\lambda) + 
		\Delta(4\lambda - 3) +
		\frac{2\lambda - \frac{3}{2}}{2\lambda - 2} \ \varepsilon(4 \lambda -2)
	\end{align*}
	and where $\Delta(\lambda)$ is
	\begin{align*}
		\inf_{\calP \in \Theta} \sup_{s_i \in \calS} \max \bigg\{ 
		D_\lambda \binom{P_{ \calP}(\Xu | \Xs = s_i)}{P_{ \calQ}(\Xu | \Xs = s_i)}, 
		D_\lambda \binom{P_{ \calQ}(\Xu | \Xs = s_i)}{P_{ \calP}(\Xu | \Xs = s_i)}
		\bigg\}
	\end{align*}
\end{theorem}

As long as the conditional distribution on $\Xu|\Xs = s_i$ of prior $\calQ$ is close to that of some $\calP \in \Theta$, the privacy guarantees should change only marginally. This bound is tightest when $\varepsilon(\lambda)$ does not grow quickly with order $\lambda$.



%
%\begin{center}
%\begin{figure*}
%\begin{minipage}[t]{0.26\textwidth}
%	\begin{algorithm}[H]
%		\SetAlgoLined
%		\KwInput{$\Is, \Sigma, o_t$}
%		\KwOutput{$\Sigmag$}
%			\vskip 2mm
%			$\argmax_{\Sigmag \succeq 0} \  \beta^*$\;
%			\vskip 1mm
%			$ \text{s.t. } \tilde{A}^{-1} \tilde{B}^{-1} \tilde{A}^{-\intercal} \succeq \beta^* \mathbf{I}$\;
%			$ \quad \ \ \ \trace(\Sigmag) \leq n o_t$\;
%			\vskip 2mm
%		\Return $\Sigmag$\;
%		\caption{SIG OPT}
%	\label{alg: sig opt}
%	\end{algorithm}
%\end{minipage}%
%\hspace{0.04\textwidth}
%\begin{minipage}[t]{0.3\textwidth}
%	\begin{algorithm}[H]
%		\SetAlgoLined
%		\KwInput{$\calF$}
%		\KwOutput{$\Sigmag$}
%			\vskip 2mm
%			$\argmin_{\Sigmag } \  \trace(\Sigmag)$\;
%			\vskip 1mm
%			$ \text{s.t. } \Sigmag \succeq \Sigmag_i , \ \forall \Sigmag_i \in \calF$\;
%			\vskip 2mm
%		\Return $\Sigmag$\;
%		\caption{ALL SECRETS}
%	\label{alg: all secrets}
%	\end{algorithm}
%\end{minipage}
%\hspace{0.04\textwidth}
%\begin{minipage}[t]{0.34\textwidth}
%	\begin{algorithm}[H]
%		\SetAlgoLined
%		\KwInput{$\mathbb{I}_{\calS_b}, \Sigma, o_t$}
%		\KwOutput{$\Sigmag$}
%		\vskip 2mm
%		$\calF = \emptyset$\;
%		
%		\For{$\mathbb{I}_{S_i} \in \mathbb{I}_{\calS_b}$}
%			{
%				$\Sigmag_i =$ SIG OPT$(\mathbb{I}_{S_i}, \Sigma, o_t)$\; 
%				
%				$\calF = \calF \cup \Sigmag_i$\;
%			}
%		\vskip 2mm
%			\Return ALL SECRETS$(\calF)$\;
%	
%	\caption{Basic Mechanism}
%	\label{alg: basic mechanism}
%	\end{algorithm}
%\end{minipage}%	
%\end{figure*}
%\end{center}
%

\subsection{CIP for Gaussian Process Priors}
\label{sec: CIP for GP} 
A \emph{GP conditional prior class} is the set of all GP prior distributions with the same kernel function $(i,j) \rightarrow \text{Cov}(X_i, X_j)$ and any mean function $i \rightarrow \E[X_i]$. With an additive Gaussian mechanism $G \sim \calN(\mathbf{0}, \Sigmag)$, the CIP loss of Equation \eqref{eqn:two terms} can be bounded for any GP conditional prior class. See Appendix \ref{apx: GP prior class} for further discussion of the GP conditional prior class. 

\begin{theorem}\emph{CIP loss bound for GP conditional priors:}
\label{thm:GP bound}
	Let $\Theta$ be a GP conditional prior class. Let $\Sigma$ be the covariance matrix for $X$ produced by its kernel function. Let $\calS$ be the basic or compound secret associated with $\Is$, and $S$ be the number of unique times in $\Is$. The mechanism $\calA(X) = X + G = Z$, where $G \sim \calN(\mathbf{0}, \Sigmag)$, then satisfies $(\varepsilon, \lambda)$-Conditional Inferential Privacy $(\Spairs, r, \Theta)$, where 
	\begin{align}
		\varepsilon
		&\leq \frac{\lambda}{2} S r^2 \Big(  \frac{1 }{\sigma_s^2} + \alpha^*  \Big) 
		\label{eqn: priv bound}
	\end{align}
	\text{ }\vspace{1mm}\\
	where $\sigma_s^2$ is the variance of each $G_i \in \Gs$ (diagonal entries of $\Sigmag_{ss}$) and $\alpha^*$ is the maximum eigenvalue of $\Sigmaeff = \big(\Sigma_{us} \Sigma_{ss}^{-1}\big)^\intercal \big( \Sigma_{u | s} + \Sigma_{uu}^{(g)} \big)^{-1} \big(\Sigma_{us} \Sigma_{ss}^{-1}\big)$. 
\end{theorem}

The above bound is tight for basic secrets ($S = 1$). The two terms of Equation \eqref{eqn: priv bound} represent the direct $(\frac{1}{\sigma_s^2})$ and inferential $(\alpha^*)$ loss terms of Equation \eqref{eqn:two terms}. We assume that each diagonal entry of $\Sigmag_{ss}$ equals some $\sigma_s^2$, so that each $X_i \in \Xs$ experiences identical direct privacy loss, which is optimal under utility constraints. 

The above bound composes gracefully when multiple traces of an individual are released. 

\begin{corollary}\emph{Graceful Composition in Time}
\label{cor: composition}
	Suppose a user releases two traces $X$ and $\hat{X}$ with additive noise $G \sim \calN(\mathbf{0}, \Sigmag)$ and $\hat{G} \sim \calN(\mathbf{0}, \hat{\Sigma}^{(g)})$, respectively. Then basic or compound secret $\Xs$ of $X$ enjoys $(\bar{\varepsilon}, \lambda)$-CIP, where 
	\begin{align*}
		\bar{\varepsilon} \leq \frac{\lambda}{2} S r^2 \Big(  \frac{1 }{\sigma_s^2} + \bar{\alpha}^*  \Big) 
	\end{align*}
	and where $\bar{\alpha}^*$ is the maximum eigenvalue of $\bar{\Sigma}_{\text{eff}} = \big(\Sigma_{us} \Sigma_{ss}^{-1}\big)^\intercal \big( \Sigma_{u | s} + \bar{\Sigma}_{uu}^{(g)} \big)^{-1} \big(\Sigma_{us} \Sigma_{ss}^{-1}\big)$. $\Sigma$ is the covariance matrix of the joint distribution on $X, \hat{X}$ and 
	\begin{align*}
	\bar{\Sigma}^{(g)} =
		\begin{bmatrix}
			 \Sigmag & 0 \\
			 0 &  \hat{\Sigma}^{(g)} \ .
		\end{bmatrix}
	\end{align*}
\end{corollary}
\text{ } \vspace{2mm} \\
This bound is identical to that of Theorem \ref{thm:GP bound}, only using the joint distribution over $X$, $\hat{X}$ and $G, \hat{G}$. This provides some insight to the fact that, unlike DP, even parallel composition guarantees are not automatic. Composition depends on the conditional prior. In the GP setting, if the chosen kernel function decays over time, we can expect composition to have minimal effects on privacy for traces separated by long durations. 

To reduce the upper bound of Theorem \ref{thm:GP bound}, we optimize the correlation (off-diagonal) of $\Sigmag$ to minimize $\alpha^*$, and optimize its variance (diagonal) to balance a noise budget between lowering inferential ($\alpha^*$) and direct ($\frac{1}{\sigma_s^2}$) loss.
\section{Optimized Privacy Mechanisms}
\label{sec: algorithms}

Theorem \ref{thm:GP bound} characterizes the privacy loss for GP conditional priors. We next show how to use this Theorem to design mechanisms that can strategically reduce CIP loss given a utility constraint. We measure `utility loss' as the total mean squared error (MSE) between the released ($Z$) and true ($X$) traces: $\text{MSE}(\Sigmag) = \sum_{i=1}^n \E[Z_i - X_i] = \trace(\Sigmag)$. We bound the utility loss by $\trace(\Sigmag) \leq n o_t$, where $o_t$ is the average per-point utility loss.

It can be shown that optimizing the privacy loss under this utility constraint can be described by a semidefinite program (SDP) (formalization/derivation of SDPs in Appendix \ref{apx: algorithmns}). For a given trace $X$, define its covariance matrix $\Sigma$ using the the kernel of the GP conditional prior $\Sigma_{ij} = k(i,j)$. Then pass $\Sigma$, the secret set $\Is$, and the utility constraint $o_t$ to our first program, $\text{SDP}_\text{A}$, which returns noise covariance $\Sigmag$. This defines an additive noise mechanism $G \sim \calN(0, \Sigmag)$ that minimizes CIP loss to $\Is$. 
\begin{align*}
	\Sigmag = \text{SDP}_\text{A}(\Sigma, \Is, o_t)
\end{align*} 
We can thus use a SDP to minimize the CIP loss to any single compound or basic secret. However, a trace may contain multiple locations or combinations thereof that one wishes to protect. It remains to produce a single mechanism $\Sigmag$ that bounds the CIP loss to multiple basic and/or compound secrets in a single trace. 

For this we propose $\text{SDP}_\text{B}$, which uses the fact that if ${\Sigmag}' \succ \Sigmag$ it will have lower CIP loss (see Appendix \ref{apx: SDP B}). $\text{SDP}_\text{B}$ takes in a set of covariance matrices $\calF = \{\Sigmag_1, \dots, \Sigmag_m\}$, each designed to minimize CIP loss for a single compound or basic secret $\Is_i$. It then returns a single covariance matrix $\Sigmag \succeq \Sigmag_i, i \in [m]$ that maintains the privacy guarantee each $\Sigmag_i$ offered its corresponding $\Is_i$, while minimizing utility loss. 

\begin{figure*}[h]
    \centering
    \begin{subfigure}{.24\linewidth}
        \centering 
        \captionsetup{justification=centering}
        \includegraphics[width = 1\linewidth]{./images/figure_2a.png}
        \caption{}
        \label{fig: RBF basic}
    \end{subfigure}
    \begin{subfigure}{.24\linewidth}
        \centering 
        \captionsetup{justification=centering}
        \includegraphics[width = 0.95\linewidth]{./images/figure_2b.png}
        \caption{}
        \label{fig: RBF compound}
    \end{subfigure}
    \begin{subfigure}{.24\linewidth}
        \centering 
        \captionsetup{justification=centering}
        \includegraphics[width = 0.95\linewidth]{./images/figure_2c.png}
        \caption{}
        \label{fig: RBF all}
    \end{subfigure}
    \begin{subfigure}{.24\linewidth}
        \centering 
        \captionsetup{justification=centering}
        \includegraphics[width = 0.95\linewidth]{./images/figure_2d.png}
        \caption{}
        \label{fig: RBF misspec}
    \end{subfigure}
    \begin{subfigure}{.24\linewidth}
        \centering 
        \captionsetup{justification=centering}
        \includegraphics[width = 1\linewidth]{./images/figure_2e.png}
        \caption{}
        \label{fig: PER basic}
    \end{subfigure}
    \begin{subfigure}{.24\linewidth}
        \centering 
        \captionsetup{justification=centering}
        \includegraphics[width = 0.95\linewidth]{./images/figure_2f.png}
        \caption{}
        \label{fig: PER compound}
    \end{subfigure}
    \begin{subfigure}{.24\linewidth}
        \centering 
        \captionsetup{justification=centering}
        \includegraphics[width = 0.95\linewidth]{./images/figure_2g.png}
        \caption{}
        \label{fig: PER all}
    \end{subfigure}
    \begin{subfigure}{.24\linewidth}
        \centering 
        \captionsetup{justification=centering}
        \includegraphics[width = 0.95\linewidth]{./images/figure_2h.png}
        \caption{}
        \label{fig: PER misspec}
    \end{subfigure}
    \caption[Posterior uncertainty interval (higher$=$better privacy) on $\Xs$ of a GP Bayesian adversary.]{$^1$Posterior uncertainty interval (higher$=$better privacy) on $\Xs$ of a GP Bayesian adversary. A larger $\leff$ corresponds to greater inter-dependence and reduces posterior uncertainty. The gray interval depicts the middle 50\% of the MLE $\leff$ among traces in each dataset, and the black dotted line the median $\leff$. \textbf{(a)}$\rightarrow$\textbf{(c)}, \textbf{(e)}$\rightarrow$\textbf{(g)} show SDP mechanisms (blue) maintaining relatively high uncertainty compared to two GI (Approach C) baselines of equal utility (MSE). \textbf{(d)}, \textbf{(h)} show the (minor) change in posterior uncertainty when the prior covariance $\Sigma$  used in $\text{SDP}_{\text{A}}$ is misspecified: when it is identical to the true covariance $\Sigma^*$ known to the adversary (blue), is more correlated (orange), or is less correlated (green).
    }
    \label{fig: experiments}
\end{figure*} 

\SetKwInput{KwInput}{Input}                % Set the Input
\SetKwInput{KwOutput}{Output}              % set the Output

\begin{algorithm}
		\SetAlgoLined
		\KwInput{$\Is_1, \dots, \Is_m, o_t, \Sigma$}
		\KwOutput{$\Sigmag$}
			\vskip 1mm
			$\calF = \emptyset$\;
			
			\For{$i \in [m]$}
			{
				$\Sigmag_i =$ $\text{SDP}_\text{A}(\Sigma, \Is_i, o_t)$\; 
				
				$\calF = \calF \cup \Sigmag_i$\;
			}
			\vskip 1mm
			$\Sigmag= \text{SDP}_{\text{B}}(\calF)$\;
			
			\Return $\Sigmag$\;
		\caption{Multiple Secrets}
\label{alg: Multiple Secrets}
\end{algorithm}
In our experiments, we use Algorithm \ref{alg: Multiple Secrets} to design a single mechanism that protects all locations in the trace --- all basic secrets --- while minimizing utility loss.





\section{Experiments}
\label{sec: experiments}
%\textcolor{blue}{
Here, we aim to empirically answer: \textbf{1)} 
Do our SDP mechanisms maintain high posterior uncertainty of sensitive locations? How do they compare to Approach C baselines of equal MSE? 
%What privacy improvements do our SDP-based mechanisms offer over Approach C baselines of identical MSE? 
\textbf{2)} How robust is the $\text{SDP}_\text{A}$ mechanism when the prior covariance $\Sigma$ is misspecified? 
%}

\paragraph{Methods} To answer these questions, we look at the range of conditional prior classes that fit real-world data. For location trace data, we use the GeoLife GPS Trajectories dataset \citep{geolife} containing 10k human mobility traces after preprocessing (see Appendix \ref{apx: experiments} for details). We also consider the privacy risk of room temperature data \citep{home_monitoring}, using the SML2010 dataset \citep{sml2010}, which contains approximately 40 days of room temperature data sampled every 15 minutes. 

For the location data, having observed that the correlation between latitude and longitude is low ($ \approx 0.06$) we treat each dimension as independent. By way of Corollary \ref{cor: independence}, this allows us to bound privacy loss and design mechanisms for each dimension separately. Furthermore, having observed that each dimension fits nearly the same conditional prior, we treat our dataset of 10k 2-dimensional traces as a dataset of 20k 1-dimensional traces, where each trace represents one dimension of a 2d location trajectory.

We model the location trace data with a Radial Basis Function (RBF) kernel GP and the temperature series data with a periodic kernel GP:
\begin{align*}
	k_{\text{RBF}}(t_i, t_j) 
	&=  \sigma_x^2 \exp \Big( -\frac{(t_i - t_j)^2}{2 l^2} \Big) \\
	k_{\text{PER}}(t_i, t_j) 
	&=  \sigma_x^2 \exp \Big(  \frac{-2 \sin^2(\pi |t_i - t_j| / p)}{l^2} \Big)
\end{align*}
In both kernels, the intrinsic degree of dependence between points is captured by the lengthscale $l$. However, the fact that sampling rates vary significantly between traces means that traces with equal length scales can have very different degrees of correlation. To encapsulate both of these effects, we study the empirical distribution of \emph{effective} length scale of each trace
\begin{align*}
	l_{\text{eff},x} = \frac{l_x}{P}
	\quad
	l_{\text{eff},y} = \frac{l_y}{P}
\end{align*}
where $P$ is the trace's sampling period and $l_x,l_y$ are the its optimal length scales for each dimension. 

$l_{\text{eff},x},l_{\text{eff},y}$ tell us the average number of neighboring locations that are highly correlated, instead of time period. For instance, a given trace with an optimal $l_{\text{eff},x} = 8$ tells us that every eight neighboring location samples in the $x$ dimension have correlation $> 0.8$. The empirical distribution of effective length scales across all traces describes -- over a range of logging devices (sampling rates), users, and movement patterns -- how many neighboring points are highly correlated in location trace data. After this preprocessing, we are able to use the kernels that take indices (not time) as arguments: 
\begin{align*}
	k_{\text{RBF}}(i, j) 
	&=  \exp \Big( -\frac{(i - j)^2}{2\leff^2} \Big) \\
	k_{\text{PER}}(i, j) 
	&=  \exp \Big(  \frac{-2 \sin^2(\pi |i - j| / p)}{\leff^2} \Big)
\end{align*}
See Appendix \ref{apx: experiments} for a more detailed discussion of how the empirical distribution of $\leff$ across traces is measured. 

To impart the range of realistic conditional priors the gray interval of each plot depicts the middle 50\% of the empirical $\leff$ among traces in each dataset. The dashed vertical line reports the median $\leff$. 



%Details of preprocessing and definition of $\leff$ are presented in Appendix \ref{apx: experiments}. 

%\begin{align}
%	\label{eqn: kernels}
%	k_{\text{RBF}}(i, j) 
%	&=  \exp \Big( -\frac{(i - j)^2}{2\leff^2} \Big), \notag \\
%	k_{\text{PER}}(i, j) 
%	&=  \exp \Big(  \frac{-2 \sin^2(\pi |i - j| / p)}{\leff^2} \Big)
%\end{align}
Each figure increases the degree of dependence, $\leff$, used by the kernel to compute the prior covariance $\Sigma(\leff)$. $\Sigma(\leff)$ is then used in one of the SDP routines of Section \ref{sec: algorithms} to produce a mechanism $\Sigmag(\leff)$ that protects a basic secret ($\text{SDP}_\text{A}$), a compound secret ($\text{SDP}_\text{A}$), or the union of all basic secrets (Multiple Secrets). We then observe the 68\% confidence interval of the Gaussian posterior on sensitive points $\Xs$ (blue line). This is the $2\sigma$ uncertainty of a Bayesian adversary with a GP prior represented by $\Sigma(\leff)$ (see Appendix \ref{apx: experiments} for how this is computed). As $\leff$ increases, their posterior uncertainty will reduce. Our aim is to mitigate this as much as possible with the given utility constraint. For scale, recall that prior variance $\textbf{diag}(\Sigma)$ is normalized to one. In the case of all basic secrets, we report the average posterior uncertainty over locations. 

We compare the SDP mechanisms with two mechanisms using the logic of Approach C (all three of equal MSE utility loss): \emph{independent/uniform} and \emph{independent/concentrated}. The uniform approach adds independent Gaussian noise evenly along the whole trace regardless of $\Is$, $\Sigmag = o_tI$. The concentrated approach allocates the entire noise budget to the sensitive set $\Is$. 
\paragraph{Results}
For our first question, see \textbf{Figures \ref{fig: RBF basic}$\rightarrow$\ref{fig: RBF all}, \ref{fig: PER basic}$\rightarrow$\ref{fig: PER all}}. For both location and temperature data, our SDP mechanisms maintain higher posterior uncertainty than the baselines with identical utility cost for a single basic secret, a compound secret, and all basic secrets. By actively considering the conditional prior class parametrized by $\Sigma$, the SDP mechanisms can strategize to both correlate noise samples and concentrate noise power such that posterior inference is thwarted at the sensitive set $\Is$. For an intuitive illustration of the chosen $\Sigmag$'s, see Appendix \ref{apx: juxtaposition}. 

To answer our second question, see \textbf{Figures \ref{fig: RBF misspec}} and \textbf{\ref{fig: PER misspec}}. When the prior covariance $\Sigma$ does not represent the true data distribution known to the adversary, a smaller posterior uncertainty may be achieved. The orange line indicates the uncertainty interval of an adversary who knows the data is \emph{less} correlated than we believe i.e. the true $\Sigma^* = \Sigma(0.5 \leff)$. The blue line represents an adversary who knows the data is \emph{more} correlated than we believe i.e. the true $\Sigma^* = \Sigma(1.5 \leff)$. Both plots confirm the robustness of our privacy guarantees stated by Theorem \ref{thm: prior misspecification}. Particularly around the median $\leff$ we see that the change in posterior uncertainty with this change in prior is indeed marginal. 

\section{Discussion}
\paragraph{Related Work}
Few works have proposed solutions to the \emph{local} guarantee when releasing individual traces. A mechanism offered in \cite{synthesizing_plausible_deniability} releases synthesized traces satisfying the notion of \emph{plausible deniability} \citep{plausible_deniability}, but this is distinctly different from providing a radius of privacy to sensitive locations. Meanwhile, the frameworks proposed in \cite{temporal} and \cite{priste} nicely characterize the risk of inference in location traces, but use only first-order Markov models of correlation between points, do not offer a radius of indistinguishability as in this work, and are not suited to continuous-valued spatiotemporal traces.

Perhaps more technically similar to this work, \cite{song_pufferfish_2017} provide a general mechanism that applies to any Pufferfish framework, as well as a more computationally efficient mechanism that applies when the joint distribution of an individual's features can be described by a graphical model. The first is too computationally intensive. The second is for discrete settings, and cannot accommodate spatiotemporal effects.
%The first mechanism is too computationally intensive for our setting, and the second only looks at discrete or categorical functions of data, and cannot (at least directly) accommodate spatiotemporal effects. 

\paragraph{Conclusion}
This work proposes a framework for both identifying and quantifying the \emph{inferential} privacy risk for highly dependent sequences of spatiotemporal data. As a starting point, we have provided a simple bound on the privacy loss for Gaussian process priors, and an SDP-based privacy mechanism for minimizing this bound without destroying utility. We hope to extend this work to other data domains with different conditional priors, and different sets of secrets.

\subsubsection*{Acknowledgements}
This chapter, in full, is a reprint of the material as it appears in International Conference on Artificial Intelligence and Statistics, 2021. Casey Meehan, Kamalika Chaudhuri. \emph{Location Trace Privacy Under Conditional Priors}. The dissertation author is the primary investigator and author of this paper. 
KC and CM would like to thank ONR under N00014-20-1-2334 and UC Lab Fees under LFR 18-548554  for research support. We would also like to thank our reviewers for their insightful feedback. 
\chapter*{Concluding Remarks}
The above chapters provide a diverse set of examples of how privacy risks can be measured or mitigated in different scenarios. While highly different from each other, these examples all highlight the following three guiding principles for data privacy.

\subsection*{No privacy definition is a `gold standard'}
Differential privacy (DP) is often touted as the `gold standard' of data privacy. The cases studied in Chapters 3-5 challenge this by proposing entirely different privacy definitions for entirely different settings and risks. Chapter 3 proposes sentence privacy, which is DP-like but uses a different neighboring notion. Chapter 4, on the other hand, proposes a shuffling-based privacy definition that is almost orthogonal to DP. That is because the correlation adversary considered in that setting cannot be thwarted by DP alone. However, the broad population trends that we wish to learn are still accessible under our semi-random shuffling approach. Similarly, Chapter 5 analyzes the threat of correlation adversaries in the domain of location traces. Here, we see that a DP-based definition has to add \emph{more noise} in order to thwart these adversaries.

Taken together, we see that sometimes DP definitions are effective, and other times they require one to choose between meaningful privacy and utility. If one `gold standard' definition were effective in all of these settings, we would not need to propose so many contrasting privacy definitions and methods. 

\subsection*{No Free Lunch}
The goal of data privacy is to allow the release of high-level information (\emph{e.g.} data distribution) while obscuring low-level information (\emph{e.g.} individuals' data features). It is natural to wonder whether it is possible to design a privacy definition under which we can release highly level information and defend against \emph{any} adversary. The answer is unequivocally, \emph{no}. This fact is known as the No-Free-Lunch theorem, made precise in \cite{Kifer}. The Theorem shows that releasing any information about a dataset that is useful to one person can be leveraged by an adversary to learn fine-grained information. The No-Free-Lunch theorem is instructive, because it shifts our attention from the question of whether we can provide air-tight privacy (impossible) to whether the adversaries our definition allows are \emph{realistic} in our setting. 

The No-Free-Lunch principle is fundamental to the approaches of Chapters 4 and 5 in particular. Here, we propose novel privacy definitions that are adversary-focused. Note that in both of these papers, we consider limited classes of adversaries. As stated above, it is impossible to block the inferences of \emph{all adversaries} while still sharing useful information derived from the sensitive data. By practically evaluating what prior knowledge an adversary might have, like a correlation prior, we can formalize a privacy definition that gives strong guarantees in realistic settings. 

\subsection*{Perfect is the enemy of good}
Chapters 1 and 2 offer no formal privacy definition or provably private mechanism. Instead, they offer statistical tests to empirically evaluate a model's memorization of its training data, and thereby risk of exposing that data. In both chapters, we examine how model selection can effect the degree of memorization as detected by our tests. While our proposed test statistics do not confer any formal privacy guarantees, they guide practitioners towards models that memorize less. In many cases, our tests showed that it is possible to find models which have significantly less memorization at little to no cost in utility. 

While formal privacy definitions are a valuable goal they make up only a small part of an ML practitioners privacy toolkit. To preserve privacy, we as researchers ought to put equal effort into methodical empirical privacy tests as we do formally private algorithms. These tests tend to be far more accessible to practitioners and allow them to significantly improve model privacy. Although empirical privacy tests are imperfect, the practical benefits to be gained by proposing them are undoubtedly a positive good. Do not let perfect privacy be the enemy of good privacy. 


\appendix
%\Blinddocument
\graphicspath{{./chapters/chapter1/}}
\chapter{ }
\subsection{Limitations and societal impact}
\paragraph{Limitations. }
Our work sets out to define, quantify, and visualize data memorization in SSL. Our tests guide us towards potential mitigation strategies. However, note that these strategies are distinct from provable privacy (e.g. DP), and do not guarantee that data is not memorized. It is possible that --- even if our tests detect no memorization --- data is being memorized in some other fashion, and could be detected with a different test. Furthermore, we focus on detecting image memorization with a curated, de-duplicated dataset (ImageNet-1k), which may over- or underestimate data memorization in practice. We chose this in order to claim the learning algorithm as the cause for memorization as opposed to the dataset itself. It is possible that models exhibit different memorization behavior on larger, less curated datasets. With orders of magnitude more data it is possible that memorization is reduced, but with more data duplication it also may be exacerbated. 

\paragraph{Societal Impact. }
Our work's findings have a critical societal impact from a privacy perspective. We show that it is possible for SSL---an increasingly popular learning paradigm---to memorize training images, which could have significant privacy implications. This direction of research is important if we want to understand how we can train such models without exposing user data. Additionally, our proposed mitigation strategies point to the possibility of having strong privacy without significant loss in utility. Ultimately, we open a promising direction towards making SSL vision models more secure.  

\subsection{Experimental details}

\subsubsection{Details on dataset splits}
\label{sec:appx splits}
Imagenet1k provides bounding box annotations of foreground objects to a subset of examples in each class. 
Private sets $\calA$ and $\calB$ contain shared examples, $\calA \cap \calB$, without bounding box annotations, and unique examples with bounding box annotations. Denote the unique examples in each set as $\Abox = \calA \backslash (\calA \cap \calB)$ and $\Bbox = \calB \backslash (\calA \cap \calB)$. To identify memorization, our tests only attempt to infer the labels of the unique examples $\Abox$ and $\Bbox$ that differentiate the two private sets. The periphery crop, $\crop{A_i}$, is computed as the largest possible crop that does not intersect with the foreground object bounding box. In some instances the largest periphery crop is small, and not high enough resolution to get a meaningful embedding. To circumvent this, we only run the test on bounding box examples where the periphery crop is at least $100 \times 100$ pixels.

Each size of training set, 100k to 500k, includes an equal number of examples per class in both sets $\calA$ and $\calB$. The total bounding box annotated examples of each class are evenly divided between $\Abox$ and $\Bbox$. The remaining examples in each class are the shared examples $\calA \cap \calB$. Shared examples are necessary due to a limit number of bounding box examples and a limited number of total images. However, we reiterate that the bounding box examples in set $\calA$ are \emph{unique} to set $\calA$, and thus can only be memorized by $\SSL_A$. 

The disjoint public set, $X$, contains ground truth labels but no bounding-box annotations. The size and content of $X$ remains fixed for all tests. 

\subsubsection{Details on the training setup}
\paragraph{Model Training:} We use PyTorch~\citep{pytorch} with FFCV-SSL~\citep{demo}. All models are trained for 1000 epochs with model checkpoints taken at 50, 100, 250, 500, 750, and 1000 epochs. We note that 1000 epochs is used in the original papers of both VICReg and SimCLR. All sweeps of epochs use the 300k dataset. All sweeps of datasets use the final, 1000 epoch checkpoint. We use a batch size of 1024, and LARS optimizer \citep{LARS} for all SSL models. All models use Resnet101 for the backbone. As seen in Appendix \ref{sec:appx rn50}, a Resnet50 backbone results in \dejavu consistent with that of Resnet101. 

\paragraph{VICReg Training:} VICReg is trained with the 3-layer fully connected projector used in the original paper with layer dimensions 8192-8192-8192. The invariance, variance, and covariance parameters are set to $\lambda = 25, \mu = 25, \nu = 1$, respectively, which are used in the original paper \citep{vicreg}. The LARS base learning rate is set to 0.2, and weight decay is set to 1e-6. 

\paragraph{SimCLR Training:} SimCLR is trained with the 2-layer fully connected projector used in the original paper with layer dimensions 2048-256. The temperature parameter is set to $\tau=0.15$. The LARS base learning rate is set to 0.3, and weight decay is set to 1e-6. 

\paragraph{Supervised Training:} Unlike the SSL models, the supervised model is trained with label access using cross-entropy loss. To keep architectures as similar as possible, the supervised model also uses a Resnet101 backbone and the same projector as VICReg. A final batchnorm, ReLU, and linear layer is added to bring the 8192 dimension projector output to 1000-way classification activations. We use these activations as the supervised model's projector embedding. The supervised model uses the LARS optimizer with learning rate 0.2. 

\subsubsection{Details on the evaluation setup}
\paragraph{KNN:} For each test, we build two KNN's: one using the target model, $\SSL_A$ (or $\CLF_A$), and one using the reference model $\SSL_B$ (or $\CLF_B$). As depicted in Figure \ref{fig:split_and_pipeline_cartoon}, each KNN is built  using the projector embeddings of all images in the public set $\calX$ as the neighbor set. When testing for memorization on an image $A_i \in \calA$, we first embed $\crop{A_i}$ using $\SSL_A$, and find its $K = 100$ $L_2$ nearest neighbors within the $\SSL_A$ embeddings of $\calX$. See section \ref{sec:appx KNN} for a discussion on selection of $K$. We then take the majority vote of the neighbors' labels to determine the class of $A_i$. This entire pipleline is repeated using reference model $\SSL_B$ and its KNN to compute reference model accuracy. 

In practice, all of our quantitative tests are repeated once with $\SSL_A$ as the target model (recovering labels of images in set $\calA$) and again with $\SSL_B$ as the target model (recovering labels of images in set $\calB$). All results shown are the average of these two tests. Throughout the paper, we describe $\SSL_A$ as the target model and $\SSL_B$ as the reference model for ease of exposition. 

\paragraph{RCDM:} The RCDM is trained on a face-blurred version of ImageNet~\citep{imagenet} and is used to decode the SSL backbone embedding of an image back into an approximation of the original image. All RCDMs are trained on the public set of images $\calX$ used for the KNN. A separate RCDM must be trained for each SSL model, since each model has a unique mapping from image space to embedding space. 

At inference time, the RCDM is used to reconstruct the foreground object given only the periphery cropping. To produce this reconstruciton, the RCDM needs an approximation of the backbone embedding of the original image. The backbone of image $A_i$ is approximated by \textbf{1)} computing crop embedding $\SSLpj_A(\crop{A_i})$, \textbf{2)} finding the five public set nearest neighbors of the crop embedding, and \textbf{3)} averaging the five nearest neighbors' backbone embeddings. In practice, these public set nearest neighbors are often a very good approximation of the original image, capturing aspects like object class, position, subspecies, etc..  

\clearpage

\subsection{Additional quantitative experiments}
\label{sec:appx quantitative}

\subsubsection{Sample-level memorization}
\begin{figure}[ht]
%%%
%VICREG
%%%
     \centering
     \begin{subfigure}[b]{0.49\textwidth}
         \centering
         \includegraphics[width=\textwidth]{figures/sample_level_training_epochs.pdf}
         \caption{Categories of training samples vs. number of epochs}
         \label{fig:sample level epochs}
     \end{subfigure}
     \hfill
     \begin{subfigure}[b]{0.49\textwidth}
         \centering
         \includegraphics[width=\textwidth]{figures/sample_level_training_set_size.pdf}
         \caption{Categories of training samples vs. training set size}
         \label{fig:sample level training size}
     \end{subfigure}
\caption{
\definecolor{part_blue}{rgb}{0.2824, 0.4706, .8157}
\definecolor{part_red}{rgb}{0.8392, 0.3725, 0.3725}
\definecolor{part_orange}{rgb}{0.9333, 0.5216, 0.2902}
Partition of samples $A_i \in \calA$ into the four categories: {\color{gray}unassociated} (not shown), {\color{part_orange}memorized}, {\color{part_red}misrepresented} and {\color{part_blue}correlated}. The {\color{part_orange}memorized} samples---ones whose labels are predicted by $\KNN_A$ but not by $\KNN_B$---occupy a significantly larger share for VICReg compared to the supervised model, indicating that sample-level \dejavu memorization is more prevalent in VICReg. %The trends across number of training epochs and training set sizes are consistent with those observed in Figures \ref{fig:dejavu epochs and dataset} and \ref{fig:dejavu criteria and architecture}.
}
\label{fig:partition attack main appendix}
\end{figure}
\begin{figure*}[h]
%%%
%VICREG
%%%
     % \centering
     \begin{subfigure}[b]{0.49\textwidth}
         \centering
         \includegraphics[width=0.49\textwidth]{figures/sample_level_simclr_param.pdf}
         \includegraphics[width=0.49\textwidth]{figures/linear_probe_simclr_param.pdf}
         \caption{SimCLR}
         \label{fig:simclr temperature}
     \end{subfigure}
     \hfill
     \begin{subfigure}[b]{0.49\textwidth}
         \centering
         \includegraphics[width=0.49\textwidth]{figures/sample_level_vicreg_param.pdf}
         \includegraphics[width=0.49\textwidth]{figures/linear_probe_vicreg_param.pdf}
         \caption{VICReg}
         \label{fig:vicreg temperature}
     \end{subfigure}
\caption[Effect of SSL hyperparameter on \dejavu memorization.]{
Effect of SSL hyperparameter on \dejavu memorization. The left plot of Figures \ref{fig:simclr temperature} and \ref{fig:vicreg temperature} show the size of the memorized set as a function of the temperature parameter for SimCLR and invariance parameter for VICReg, respectively. \Dejavu memorization is the highest within a narrow band of hyperparameters, and one can mitigate against \dejavu memorization by selecting hyperparameters outside of this band. Doing so has negligible effect on the quality of SSL embeddings as indicated by the linear probe accuracy on ImageNet validation set.
 }
\label{fig:sweep params}
% \vspace{-1.5em} 
\end{figure*} 
Many SSL algorithms contain hyperparameters that control how similar the embeddings of different views should be in the training objective. 
We show that these hyperparameters directly affect \dejavu memorization. Figure \ref{fig:sweep params} shows the size of the memorized set for SimCLR (left) and VICReg (right) as a function of their respective hyperparameters, $\tau$ and $\lambda$. We observe that the memorized set is largest within a relatively narrow band of hyperparameter values, indicating strong \dejavu memorization. By selecting hyperparameters outside this band, \dejavu memorization sharply decreases while the linear probe validation accuracy on ImageNet remains roughly the same.

%\clearpage

\subsubsection{Selection of $K$ for KNN}
\label{sec:appx KNN} 
In this section, we describe the impact of $K$ on the KNN label inference accuracy. 

\begin{figure}[ht]
     \centering
     \begin{subfigure}[b]{0.49\textwidth}
         \centering
         \includegraphics[width=\textwidth]{figures/attk_knn_acc_top1_legend.pdf}
         \caption{Vicreg, Accuracy}
         \label{fig:vicreg v. epoch}
     \end{subfigure}
     \hfill
     \begin{subfigure}[b]{0.49\textwidth}
         \centering
         \includegraphics[width=\textwidth]{figures/attk_knn_partition_top1_legend.pdf}
         \caption{Vicreg, Share of memorized examples }
         \label{fig:vicreg lp v. epoch}
     \end{subfigure}
     \hfill
\caption{
Impact of $K$ on label inference accuracy for target and reference models. \textbf{Left:} the population-level label inference accuracy experiment of Section \ref{sec:label inference accuracy} on VICReg vs. $K$. \textbf{Right:} the individualized memorization test of Section \ref{sec:dissection} on VICReg vs. $K$. In both cases, we see that our tests are relatively robust to choice of $K$ beyond $K=50$.  
}
\label{fig:attack v K}
\end{figure}

Figure \ref{fig:attack v K} above shows how the tests of Section \ref{sec:quant} change with number of public set nearest neighbors $K$ used to make label inferences. Both tests are relatively robust to any choice of $K$. Results are shown on VICReg trained for 1k epochs on the 300k dataset. We see that any choice of $K$ greater than 50 and less than the number of examples per class (300, in this case) appears to have good performance. Since our smallest dataset has 100 images per class, we chose to set $K = 100$ for all experiments. 

\clearpage

% \subsection{Additional Quantitative Test Results}
\subsubsection{Effect of SSL criteria}
\label{sec:appx simclr results} 
We repeat the quantitative memorization tests of Section \ref{sec:quant} on different models: VICReg\citep{vicreg}, Barlow-Twins\citep{zbontar2021barlow}, Dino\citep{Dino}, Byol\citep{grill2020byol}, SimCLR\citep{simclr} and a supervised model in \cref{fig:all_models_quantitative}. We observe differences between SSL training criteria with respect to Dejavu memorization. The easy ones to attack are VICReg and Barlow Twins whereas SimCLR and Byol are more robust to these attacks. While the degree of memorization appears to be reduced for SimCLR compared with VICReg, it is still stronger than the supervised baseline.

\begin{figure}[ht]
\captionsetup[subfigure]{font=scriptsize,labelfont=scriptsize}
     \centering
     \includegraphics[width=\textwidth]{figures/dejavu_ssl_criterions.pdf}
\caption{
Comparison of \dejavu memorization for VICReg, Barlow Twins, Dino, Byol, SimCLR, and a supervised model. All tests are described in Section \ref{sec:quant}. We are showing \dejavu vs. number of training epochs. We see that SimCLR (center row) shows less \dejavu than VICReg, yet marginally more than the supervised model. Even with this reduced degree of memorization, we are able to produce detailed reconstructions of training set images, as shown in Figures \ref{fig:mem v corr dam} and \ref{fig:mem v corr spider}. 
}
\label{fig:all_models_quantitative}
\end{figure}

\clearpage 

\subsubsection{Effect of Model Architecture and Complexity}
\label{sec:appx rn50}
Results shown in the main paper use Resnet101 for the model backbone. To understand the relationship between \dejavu and overparameterization, we compare with the smaller Resnet50 and Resnet18 in Figure \ref{fig:rn101 v. rn50}. Overall, we find that increasing the number of parameters of the model leads to higher degree of \dejavu memorization. The same trend holds when using Vision Transformers (VIT-Tiny, -Small, -Base, and -Large with patch size of 16) of various sizes as the SSL backbone, instead of a Resnet. This highlights that \dejavu memorization is not unique to convolution architectures. 

\begin{figure}[ht]
\captionsetup[subfigure]{font=scriptsize,labelfont=scriptsize}
     \centering
     \includegraphics[width=\textwidth]{figures/plot_arch_comp.pdf}
     \caption{Comparison of VICReg \dejavu memorization for different architectures and model sizes. On the left, we present deja vu memorization using VIT architectures (from vit-tiny in the first row to vit-base in the last row). On the right, we use Resnet based architectures (from resnet18 in the first row to resnet101 in the last row). All tests are described in Section \ref{sec:quant}, with the plots showing \dejavu vs. number of training epochs. Reducing model complexity from Resnet101 to Resnet18 or from Vit-Large to Vit-tiny has a significant impact on the degree of memorization.}
\label{fig:rn101 v. rn50}
\end{figure}

\clearpage

\subsubsection{The impact of Guillotine Regularization on Deja Vu}
\label{sec:guillotine}
In our experiments, we show \dejavu using the projector representation. The SSL loss directly incentivizes the projector representation to be invariant to random crops of a particular image. As such, we expect the projector to be the \emph{most} overfit and produce the strongest \dejavu. Here, we study whether earlier representations between the projector and backbone exhibit less \dejavu memorization. This phenomenon -- `guillotine regularization' -- has recently been studied from the perspective of generalization in \citet{Guillotine}. Here, we study it from the perspective of \emph{memorization}. 

To show how guillotine regularization impacts \dejavu, we repeat the tests of Section \ref{sec:quant} on each layer of the VICReg projector: the 2048-dimension backbone (layer 0) up to the projector output (layer 3). We evaluate whether memorization is indeed reduced for the more \emph{regularized} layers between the projector output and the backbone. 

\begin{figure*}[h]
\captionsetup[subfigure]{font=scriptsize,labelfont=scriptsize}
     \centering
     \begin{subfigure}[b]{0.49\textwidth}
         \centering
         \includegraphics[width=\textwidth]{figures/attk_layer_acc_top1_legend.pdf}
         \caption{population level accuracy}
     \end{subfigure}
     \begin{subfigure}[b]{0.49\textwidth}
         \centering
         \includegraphics[width=\textwidth]{figures/attk_layer_partition_top1_legend.pdf}
         \caption{share of memorized examples}
     \end{subfigure}
     \hfill
\caption{
\dejavu memorization versus layer from backbone (0) to projector output (3). The memorization tests of Section \ref{sec:quant} are evaluated at each level of the VICReg projector. We see that \dejavu is significantly stronger closer to the projector output and nearly zero near the backbone. Interestingly, most memorization appears to occur in the final two layers of VICReg.
}
\label{fig:guillotine}
\end{figure*}

Figure \ref{fig:guillotine} shows how guillotine regularization significantly reduces the degree of memorization in VICReg. The vast majority of VICReg's \dejavu appears to occur in the final two layers of the projector (2,3): in earlier layers (0,1), the label inference accuracy of the target model and reference model are comparable. This suggests that -- like the hyperparameter selection of Section \ref{sec:conclusion} -- guillotine regularization can also significantly mitigate \dejavu memorization. In the following, we extend this result to SimCLR and supervised models by measuring the degree of \dejavu in the backbone (layer 0) versus training epochs and dataset size. 

\newpage

\paragraph{Comparison of \dejavu in projector and backbone vs. epochs and dataset size}
Since the backbone is mostly used at inference time, we now evaluate how much \dejavu exists in the backbone representation for VICReg and SimCLR. We repeat the tests of Section \ref{sec:quant} versus training epochs and train set size. 

\begin{figure}[h]
     \centering
     \begin{subfigure}[b]{\textwidth}
         \centering
         \includegraphics[width=\textwidth]{figures/plot_vicreg_projback.pdf}
         \caption{VICReg}
         \label{fig:vicreg v. epoch}
     \end{subfigure}
     \begin{subfigure}[b]{\textwidth}
         \centering
         \includegraphics[width=\textwidth]{figures/plot_simclr_projback.pdf}
         \caption{SimCLR}
         \label{fig:vicreg lp v. epoch}
     \end{subfigure}
     \hfill
\caption{
Accuracy of label inference on VICReg and SimCLR using projector and backbone representations. \textbf{First two columns:} Effect of training epochs on memorization for each representation. \textbf{Last two columns:} Effect of training set size on memorization for each representation. In contrast with VICReg, the \dejavu memorization detected in SimCLR's projector and backbone representations is quite similar. While SimCLR's projector memorization appears weaker than that of VICReg, its backbone memorization is markedly stronger. This kind be easily explained as a byproduct of Guillotine Regularization~\citep{Guillotine}, i.e. removing layers close to the objective reduce the bias of the network with respect to the training task. Since SimCLR's projector has fewer layers than VICReg's, the impact of Guillotine Regularization is less salient.
}
\label{fig:proj v backbone}
\end{figure}

Figure \ref{fig:proj v backbone} shows that, indeed, \dejavu is significantly reduced in the backbone representation. For SimCLR, however, we see that backbone memorization is comparable with projector memorization. In light of the Guillotine regularization results above, this makes some sense since SimCLR uses fewer layers in its projector. Given that we were able to generate accurate reconstructions with the SimCLR projector (see Figures \ref{fig:mem v corr spider} and \ref{fig:mem v corr dam}), we now evaluate whether we can produce accurate reconstructions of training examples using the SimCLR backbone alone. 

\clearpage

\subsection{Additional reconstruction examples}
\label{sec:appx visualization}
The two reconstruction experiments of Section \ref{sec:visualizing} are each exemplified within one class. However, we see strong reconstructions using $\SSL_A$ in several classes, and similar experimental results. To demonstrate this, we repeat the experiments \ref{sec:mem v corr} using the \emph{yellow garden spider} class and the and the \emph{aircraft carrier} class. 
\begin{figure*}[h]
%%%
%VICREG
%%%
     % \centering
     \begin{subfigure}[b]{0.49\textwidth}
         \centering
         \includegraphics[width=0.49\textwidth]{figures/sample_level_simclr_param.pdf}
         \includegraphics[width=0.49\textwidth]{figures/linear_probe_simclr_param.pdf}
         \caption{SimCLR}
         \label{fig:simclr temperature}
     \end{subfigure}
     \hfill
     \begin{subfigure}[b]{0.49\textwidth}
         \centering
         \includegraphics[width=0.49\textwidth]{figures/sample_level_vicreg_param.pdf}
         \includegraphics[width=0.49\textwidth]{figures/linear_probe_vicreg_param.pdf}
         \caption{VICReg}
         \label{fig:vicreg temperature}
     \end{subfigure}
\caption[Effect of SSL hyperparameter on \dejavu memorization.]{
Effect of SSL hyperparameter on \dejavu memorization. The left plot of Figures \ref{fig:simclr temperature} and \ref{fig:vicreg temperature} show the size of the memorized set as a function of the temperature parameter for SimCLR and invariance parameter for VICReg, respectively. \Dejavu memorization is the highest within a narrow band of hyperparameters, and one can mitigate against \dejavu memorization by selecting hyperparameters outside of this band. Doing so has negligible effect on the quality of SSL embeddings as indicated by the linear probe accuracy on ImageNet validation set.
 }
\label{fig:sweep params}
% \vspace{-1.5em} 
\end{figure*}
\paragraph{Selection of Memorized and Correlated Images:} The images of Figure \ref{fig:mem v corr dam} and \ref{fig:mem v corr spider} were chosen methodically as follows. 

\paragraph{Image selection:} The 20 images of Figures \ref{fig:mem v corr dam} and \ref{fig:mem v corr spider} are  selected deterministically using label inference accuracy and KNN confidence score. The 10 most correlated images are those images in the correlated set (both models infer label correctly) of $\calA$ with the highest confidence agreement between models $\SSL_A$ and $\SSL_B$. To measure confidence agreement we take the minimum confidence of the two models. The 10 most memorized images are those images in the memorized set (only target model infers the label correctly) of $\calA$ with the highest confidence difference between models $\SSL_A$ and $\SSL_B$.

\paragraph{Class selection:} To find classes with a high degree of \dejavu, classes were sorted by the label inference accuracy gap between the target and reference model. We selected the class based on a handful of criteria. First, we prioritized classes without images of human faces, thereby removing classes like `basketball', `bobsled', `train station', and even `tench' which is a fish often depicted in the hands of a fisherman. Second, we prioritized classes that include at least ten images with a high confidence difference between the target and reference models (`most memorized' images described above) and at least ten images with high confidence agreement (`most correlated' images described above). This led us to the \emph{dam} and \emph{yellow garden spider} classes. 
\clearpage

\input{chapters/chapter1/ship_examples.tex}

\paragraph{Selection of Beyond-Label-Inference Images:} The images of Figure \ref{fig:in class badger} and \ref{fig:in class ship} were chosen methodically as follows. 

\paragraph{Image selection:} The four images of Figures \ref{fig:in class badger} 
 and \ref{fig:in class ship} 
are selected using KNN confidence score, and, necessarily, hand picked selection for unlabeled features. Within a given class, we look at the top 40 images with highest target model KNN confidence scores. We then filter through these images to identify a distinguishable feature like different species within the same class or different object positions within the same class. This step is necessary because we are looking for features that are not labeled by ImageNet. We then choose two of these top 40 with one feature (e.g. American badger) and two with the alternative feature (e.g. European badger). 

\paragraph{Class selection:} To find classes with a high degree of \dejavu, classes were sorted by the target model's top-40 KNN confidence values within each class. As in the memorization vs. correlation experiment, we prioritized classes without images of human faces.

\subsubsection{Reconstructions using SimCLR Backbone} 
\label{sec:appx backbone results} 
The label inference results in Appendix \ref{sec:guillotine} show that the SimCLR backbone exhibits a similar degree of \dejavu memorization as the projector does. To evaluate the risk of such memorization, we repeat the reconstruction experiment of Section \ref{sec:visualizing} on the \emph{dam} class using the SimCLR backbone instead of its projector.\\

\begin{figure}[h]
     \centering
     \begin{subfigure}[b]{\textwidth}
         \centering
         \includegraphics[width=\textwidth]{figures/dam_simclr_backbone_1.png}
         \caption{First {\color{part_orange}memorized} dam example}
     \end{subfigure}
     \begin{subfigure}[b]{\textwidth}
         \centering
         \includegraphics[width=\textwidth]{figures/dam_simclr_backbone_2.png}
         \caption{Second {\color{part_orange}memorized} dam example}
     \end{subfigure}
     \hfill
     \caption{Instances of \dejavu memorization by the SimCLR backbone representation. Here, the backbone embedding of the crop is used instead of the projector embedding on the same training images used in Figure \ref{fig:mem v corr dam}. Interestingly, we see that \dejavu memorization is still present in the SimCLR backbone representation. Here, the nearest neighbor set recovers dam given an uninformative crop of still or running water. Even without projector access, we are able to reconstruct images in set $\calA$ using $\SSL_A$, and are unable using $\SSL_B$. }
     \label{fig:simclr backbone dejavu}
\end{figure}


Figure \ref{fig:simclr backbone dejavu} demonstrates that we are able to reconstruct training set images using the SimCLR backbone alone. This indicates that \dejavu memorization can be leveraged to make detailed inferences about training set images without \emph{any} access to the projector. As such, withholding the projector for model release may not be a strong enough mitigation against \dejavu memorization.  

\clearpage

\subsection{Detecting \Dejavu without Bounding Box Annotations}
\label{sec:appx corner crop} 
The memorization tests presented critically depend on bounding box annotations in order to separate the foreground object from the periphery crop. Since such annotations are often not available, we propose a heuristic test that simply uses the lower left corner of an image as a surrogate for the periphery crop. Since foreground objects tend to be near the center of the image, the corner crop usually excludes the foreground object and does not require a bounding box annotation. 

\cref{corner crop} demonstrates that this heuristic test can successfully capture the trends of the original tests (seen in Figure \ref{fig:all_models_quantitative}) \emph{without} access to bounding box annotations. However, as compared to Figure \ref{fig:all_models_quantitative}, the heuristic tends to slightly underestimate the degree of memorization. This is likely due to the fact that some corner crops partially include the foreground object, thus enabling the $\KNN$ to successfully recover the label with the reference model where it would have failed with a proper periphery crop that excludes the foreground object. 

\begin{figure}[ht]
\captionsetup[subfigure]{font=scriptsize,labelfont=scriptsize}
     \centering
     \includegraphics[width=\textwidth]{figures/plot_models_corner_crop.pdf}
\caption{
Deja Vu Memorization using a simple corner crop instead of the periphery crop extracted using bounding box annotations. While the heuristic overall underestimates the degree of \dejavu, it roughly follows the same trends versus dataset size and training epochs. This is crucial, since it allows us to estimate \dejavu without access to bounding box annotations. 
}
\label{corner crop}
\end{figure}
 
\graphicspath{{./chapters/chapter2/}}
\chapter{ }
\subsection{Proof of Theorem \ref{thm:consistent}}
\label{sec:proof consistent}
A restatement of the theorem: 

\textit{
For true distribution $P$, model distribution $Q$, and distance metric $d:\calX \rightarrow \R$, the estimator $\frac{1}{mn} U_{Q_m} \rightarrow_P \Delta(P,Q)$ according to the concentration inequality 
\begin{align*}
    \pr \big( \ \big|\frac{1}{mn} U_{Q_m} - \Delta(P,Q) \big| \geq t\big) \leq 
    \exp \bigg( -\frac{2 t^2 mn}{m + n} \bigg)
\end{align*}
}
\begin{proof}
We establish consistency using the following nifty lemma 
\begin{lemma}{(Bounded Differences Inequality)}
\label{lem:bounded diff}
Suppose $X_1, \dots, X_n \in \calX$	are independent, and $f:\calX^n \rightarrow \R$. Let $c_1, \dots, c_n$ satisfy 
\begin{align*}
	\sup_{x_1, \dots, x_n, x_i'} \big| &f(x_1, \dots, x_i, \dots, x_n) - f(x_1, \dots, x_i', \dots, x_n) \big| \\
	&\leq c_i
\end{align*}
for $i = 1, \dots, n$. Then we have for any $t > 0$
\begin{align}
	\Pr\big(\big| f - \E[f]\big| \geq t\big) \leq \exp\bigg( \frac{-2t^2}{\sum_{i=1}^n c_i^2} \bigg) \label{eqn:bounded diff}
\end{align}
\end{lemma}

This directly equips us to prove the Theorem. 

It is relatively straightforward to apply Lemma \ref{lem:bounded diff} to the normalized $\overline{U} = \frac{1}{mn} U_{Q_m}$. First, think of it as a function of $m$ independent samples of $X \sim Q$ and $n$ independent samples of $Y \sim P$, $\overline{U}(X_1, \dots, X_m, Y_1, \dots, Y_n) = \frac{1}{mn} \sum_{ij} \mathds{1}_{d(X_i) > d(Y_j)}$
\begin{align*}
	\overline{U}: (\R^{d})^{mn} \rightarrow \R 
\end{align*}
Let $b_i$ bound the change in $\overline{U}$ after substituting any $X_i$ with $X_i'$, and $c_j$ bound the change in $\overline{U}$ after substituting any $Y_j$ with $Y_j'$. Specifically 
\begin{align*}
	\sup_{x_1, \dots, x_m, y_1, \dots, y_n,\  x_i'} \big| &\overline{U}(x_1, \dots, x_i, \dots, x_m, y_1, \dots, y_n) \\
	- &\overline{U}(x_1, \dots, x_i', \dots, x_m, y_1, \dots, y_n)\big| \\
	&\leq b_i \\
	\sup_{x_1, \dots, x_m, y_1, \dots, y_n,\  y_j'} \big| &\overline{U}(x_1, \dots, x_m, y_1, \dots, y_j, \dots  y_n) \\
	- &\overline{U}(x_1, \dots, x_m, y_1, \dots, y_j', \dots  y_n)\big| \\
	&\leq c_i
\end{align*}
We then know that $b_i = \frac{n}{mn} = \frac{1}{m}$ for all $i$, with equality when $d(x_i') < d(y_j) < d(x_i)$ for all $j \in [n]$. In this case, substituting $x_i$ with $x_i'$ flips $n$ of the indicator comparisons in $\overline{U}$ from 1 to 0, and is then normalized by $mn$. By a similar argument, $c_j = \frac{m}{nm} = \frac{1}{n}$ for all $j$. 

Equipped with $b_i$ and $c_j$, we may simply substitute into Equation \ref{eqn:bounded diff} of the Bounded Differences Inequality, giving us 
\begin{align*}
	\Pr\big(\big|  \overline{U}- \E[\overline{U}]\big| \geq t\big) 
	&=  \Pr\big(\big|  \frac{1}{mn} U_{Q_m} - \Delta(\mu_p, \mu_q)\big| \geq t\big)\\
	&\leq \exp\bigg( \frac{-2t^2}{ \sum_{i=1}^m b_i^2 + \sum_{j=1}^n c_j^2} \bigg) \\
	&= \exp\bigg( \frac{-2t^2}{ \sum_{i=1}^m \frac{1}{m^2} + \sum_{j=1}^n \frac{1}{n^2} } \bigg) \\
	&= \exp\bigg( \frac{-2t^2}{ \frac{1}{m} + \frac{1}{n} } \bigg) 
	= \exp\bigg( \frac{-2t^2mn}{ m+n } \bigg)
\end{align*}
\end{proof}

\subsection{Proof of Theorem \ref{thm:fallback}}
\label{sec:proof fallback}
A restatement of the theorem: 

\textit{
When $Q = P$, and the corresponding distance distribution $L(Q) = L(P)$ is non-atomic, 
    \begin{align*}
         \E \Big[ \frac{1}{mn}\overline{U} \Big] &= \frac{1}{2}
    \end{align*}
}
\begin{proof}
	For random variables $A \sim L(P)$ and $B \sim L(P)$, we can partition the event space of $A \times B$ into three disjoint events: 
	\begin{align*}
		\Pr(A > B) &+ \Pr(A < B) \\
		&+ \Pr(A = B) = 1
	\end{align*}
	Since $Q = P$, the first two events have equal probability, $\Pr(A > B) = \Pr(A < B)$, so 
	\begin{align*}
		2\Pr(A > B) + \Pr(A = B) = 1
	\end{align*}
	And since the distributions of $A$ and $B$ are non-atomic (i.e. $\Pr\big(B = b\big) = 0, \quad \forall \ b \in \R$) we have that $\Pr(A = B) = 0$, and thus 
	\begin{align*}
		2\Pr(A > B) &= 1 \\
		\Pr(A > B) &= \Delta(P, Q) =  \frac{1}{2}
	\end{align*}
\end{proof}

\subsection{Proof of Lemma \ref{lemma:kde}} 
\label{sec:appendix kde lemma}
\textbf{Lemma \ref{lemma:kde}}
\textit{
For the kernel density estimator (\ref{eq:kde}), the maximum-likehood choice of $\sigma$, namely the maximizer of $\E_{X \sim P}[\log q_\sigma(X)]$, satisfies
\begin{align*}
	\E_{X \sim P} \bigg[ \sum_{t \in T} Q_\sigma(t|X) &\|X - t\|^2 \bigg] = \\
	&\E_{Y \sim Q_\sigma} \bigg[ \sum_{t \in T} Q_\sigma(t|Y) \|Y - t\|^2 \bigg]
\end{align*}
}
\begin{proof}
We have
\begin{align*}
&\E_{X \sim P} \left[ \ln q_\sigma(X) \right]  \\
&= \E_{X \sim P} \left[ - \ln ((2\pi)^{k/2}|T| \sigma^k) + \ln \sum_{t \in T} \exp\left( -\frac{\|x-t\|^2}{2 \sigma^2}\right)  \right] \\
&=
\mbox{constant} - k \ln \sigma + \E_{X \sim P} \left[\ln \sum_{t \in T} \exp\left( -\frac{\|x-t\|^2}{2 \sigma^2}\right) \right] 
\end{align*}
Setting the derivative of this to zero and simplifying, we find that the maximum-likelihood $\sigma$ satisfies
\begin{equation}
\sigma^2 = \frac{1}{k} \, \E_{X \sim P} \left[ \sum_{t \in T} Q_\sigma(t|X) \|X - t\|^2 \right] .
\label{eq:ml-choice}
\end{equation}
Now, interpreting $Q_\sigma$ as a mixture of $|T|$ Gaussians, and using the notation $t \in_R T$ to mean that $t$ is chosen uniformly at random from $T$, we have
\begin{align*}
	&\E_{Y \sim Q_\sigma} \left[ \sum_{t \in T} Q_\sigma(t|Y) \|Y - t\|^2 \right] \\
	&=\E_{t \in_R T} \E_{Y \sim N(t, \sigma^2 I_k)} \left[ \|Y - t\|^2 \right] 
	= k \sigma^2 .
\end{align*}
Combining this with (\ref{eq:ml-choice}) yields the lemma.
\end{proof}

\subsection{Procedural Details of Experiments}
\subsubsection{Moons Dataset, and Gaussian KDE}
\label{sec:appendix moons kDE}
\begin{figure*}[h]
    \centering
    \begin{subfigure}{.245\linewidth}
        \centering
        \includegraphics[width = 1\linewidth]{images/half_moon.png}
        \caption{}\label{fig:moons T}
    \end{subfigure}
        \hfill
    \begin{subfigure}{.245\linewidth}
        \centering
        \includegraphics[width = 1\linewidth]{images/half_moon_p01.png}
        \caption{}\label{fig:moons Q .01}
    \end{subfigure}
        \hfill
    \begin{subfigure}{.245\linewidth}
        \centering
        \includegraphics[width = 1\linewidth]{images/half_moon_p13.png}
        \caption{}\label{fig:moons Q .13}
    \end{subfigure}
       \hfill
    \begin{subfigure}{.245\linewidth}
        \centering
        \includegraphics[width = 1\linewidth]{images/half_moon_p5.png}
        \caption{}\label{fig:moons Q .5}
    \end{subfigure}
    \caption{Contour plots of KDE fit on $T$: a) training $T$ sample, b) over-fit `data copying' KDE, c) max likelihood KDE, d) underfit KDE}
    \label{fig:half moon}
\end{figure*} 

\paragraph{moons dataset}
`Moons' is a synthetic dataset consisting of two curved interlocking manifolds with added configurable noise. We chose to use this dataset as a proof of concept because it is low dimensional, and thus KDE friendly and easy to visualize, and we may have unlimited train, test, and validation samples. 

\paragraph{Gaussian KDE}
We use a Gaussian KDE as our preliminary generative model $Q$ because its likelihood is theoretically related to our non-parametric test. Perhaps more importantly, it is trivial to control the degree of data-copying with the bandwidth parameter $\sigma$. $\textbf{Figures \ref{fig:moons Q .01}, \ref{fig:moons Q .13}, \ref{fig:moons Q .5}}$ provide contour plots of of a Gaussian KDE $Q$ trained on the moons dataset with progressively larger $\sigma$. With $\sigma = 0.01$, $Q$ will effectively resample the training set. $\sigma = 0.13$ is nearly the MLE model. With $\sigma = 0.5$, the KDE struggles to capture the unique definition of $T$. 

\subsubsection{Moons Experiments}
\label{sec:appendix moons experiments}
Our experiments that examined whether several baseline tests could detect data-copying (Section \ref{sec:sensitivity to data-copying}), and our first test of our own metric (Section \ref{sec:KDE experiments}) use the moons dataset. In both of these, we fix a training sample, $T$ of 2000 points, a test sample $P_n$ of 1000 points, and a generated sample $Q_m$ of 1000 points. We regenerate $Q_m$ 10 times, and report the average statistic across these trials along with a single standard deviation. If the standard deviation buffer along the line is not visible, it is because the standard deviation is relatively small. We artificially set the constraint that $m,n \ll |T|$, as is true for big natural datasets, and more elaborate models that are computationally burdensome to sample from. 

\paragraph{Section \ref{sec:sensitivity to data-copying} Methods} Here are the routines we used for the four baseline tests: 
\begin{itemize}
    \item \textbf{Frech\'et Inception Distance (FID)} \citep{heusel}: Normally, this test is run on two samples of images ($T$ and $Q_m$) that are first embedded into a perceptually meaningful latent space using a discriminative neural net, like the Inception Network. By `meaningful' we mean points that are closer together are more perceptually alike to the human eye. Unlike images in pixel space, the samples of the moons dataset require no embedding, so we run the Frech\'et test directly on the samples. 
    
    First, we fit two MLE Gaussians: $\mathcal{N}(\mu_T, \Sigma_T)$ to $T$, and $\mathcal{N}(\mu_Q, \Sigma_Q)$ to $Q_m$, by collecting their respective MLE mean and covariance parameters. The statistic reported is the Frech\'et distance between these two Gaussians, denoted $\text{Fr}(\bullet, \bullet)$, which for Gaussians has a closed form: 
    \begin{align*}
        \text{Fr}&\big(\mathcal{N}(\mu_T, \Sigma_T), \mathcal{N}(\mu_Q, \Sigma_Q)\big) = \\
         &\|\mu_T - \mu_Q\| + \textbf{Tr}\big( \Sigma_T - \Sigma_Q - 2(\Sigma_T \Sigma_Q)^{\nicefrac{1}{2}} \big)
    \end{align*}
    Naturally, if $Q$ is data-copying $T$, its MLE mean and covariance will be nearly identical, rendering this test ineffective for capturing this kind of overfitting. 
    
    \item \textbf{Binning Based Evaluation} \citep{richardson}: This test, takes a hypothesis testing approach for evaluating mode collapse and deletion. The test bears much similarity to the test described in Section \ref{sec:local-versus-global}. The basic idea is as follows. Split the training set into partition $\Pi$ using $k$-means; the number of samples falling into each bin is approximately normally distributed if it has >20 samples. Check the null hypothesis that the normal distribution of the fraction of the training set in bin $\pi$, $T(\pi)$, equals the normal distribution of the fraction of the generated set in bin $\pi$, $Q_m(\pi)$. Specifically: 
    \begin{align*}
        Z_\pi = \frac{Q_m(\pi) - T(\pi)}
         {\sqrt{ \widehat{p}\big(1 - \widehat{p}\big) \big( \frac{1}{|T|} + \frac{1}{m} \big) }}
    \end{align*}
    where $\widehat{p} = \frac{|T|T(\pi) + mQ_m(\pi)}{|T| + m}$. We then perform a one-sided hypothesis test, and compute the number of positive $Z_\pi$ values that are greater than the significance level of 0.05. We call this the number of statistically different bins or NDB. The NDB/$k$ ought to equal the significance level if $P = Q$. 
    \item \textbf{Two-Sample Nearest-Neighbor} \citep{lopez}: In this test --- our primary baseline --- we report the three LOO NN values discussed in \cite{Kilian}. The generated sample $Q_m$ and training sample (subsampled to have equal size, $m$), $\widetilde{T} \subseteq T$, are joined together create sample $S = \widetilde{T} \cup Q_m$ of size $2m$, with training samples labeled `1' and test samples labeled `0'. One then fits a 1-Nearest-Neighbor classifier to $S$, and reports the accuracy in predicting the training samples (`1's), the accuracy in predicting the generated samples (`0's), and the average. 
    
    One can expect that --- when $Q$ collapses to a few mode centers of $T$ --- the training accuracy is low, and the generated accuracy is high, thus indicating over-representation. Additionally, one could imagine that when the training and generated accuracies are near 0, we have extreme data-copying. However, as explained in Experiments section, when we are forced to subsample $T$, it is unlikely that a given copied training point $t \in T$ is used in the test, thus making the test result unclear. 
    \item \textbf{Precision and Recall} \citep{mehdi}: This method offers a clever technique for scaling classical precision and recall statistics to high dimensional, complex spaces. First, all samples are embedded to Inception Network Pool3 features. Then, the author's use the following insight: for distribution's $Q$ and $P$, the precision and recall curve is approximately given by the set of points: 
    \begin{align*}
        \widehat{\text{PRD}}(Q,P) &= \{(\alpha(\lambda), \beta(\lambda) | \lambda \in \Lambda \}
    \end{align*}
    where
    \begin{align*}
        \Lambda &= \{ \tan\big( \frac{i}{r + 1} \frac{\pi}{2} \big) | i \in [r] \} \\
        \alpha(\lambda) &= \sum_{\pi \in \Pi} \min \big( \lambda P(\pi), Q(\pi) \big) \\
        \beta(\lambda) &= \sum_{\pi \in \Pi} \min \big( P(\pi), \frac{Q(\pi)}{\lambda} \big) 
    \end{align*}
    and where $r$ is the `resolution' of the curve, the set $\Pi$ is a partition of the instance space and $P(\pi), Q(\pi)$ are the fraction of samples falling in cell $\pi$. $\Pi$ is determined by running $k$-means on the combination of the training and generated sets. In our tests here, we set $k = 5$, and report the average PRD curve measured over 10 $k$-means clusterings (and then re-run 10 times for 10 separate trials of $Q_m$). 
    
    \end{itemize}

\subsubsection{MNIST Experiments}
\label{sec:appendix MNIST autoencoder}
\begin{figure*}[h]
	\centering
	\includegraphics[width = \linewidth]{images/latent_interp}
	\caption{Interpolating between two points in the latent space to demonstrate $L_2$ perceptual significance}
	\label{fig:latent_interp}
\end{figure*}

\begin{figure*}[h]
    \centering
    \begin{subfigure}{.31\linewidth}
        \centering
        \includegraphics[width = 1\linewidth]{images/moons_KDE_rep.png}
        \caption{}\label{fig:moons KDE rep}
    \end{subfigure}
        \hfill
    \begin{subfigure}{.31\linewidth}
        \centering
        \includegraphics[width = 1\linewidth]{images/mnist_kde_rep.png}
        \caption{}\label{fig:mnist kde rep}
    \end{subfigure}
        \hfill
    \begin{subfigure}{.31\linewidth}
        \centering
        \includegraphics[width = 1\linewidth]{images/VAE_rep.jpg}
        \caption{}\label{fig:vae rep}
    \end{subfigure}
    \caption[Number of statistically different bins, both those over and under the significance level of 0.05.]{Number of statistically different bins, both those over and under the significance level of 0.05. The black dotted line indicates the total number of cells or `bins'. \textbf{(a,b)} KDEs tend to start misrepresenting with $\sigma \gg \sigma_{\text{MLE}}$, which makes sense as they become less and less dependent on training set. \textbf{(c)} it makes sense that the VAE over- and under-represents across all latent dimensions due to its reverse KL loss. There is slightly worse over- and under-representation for simple models with low latent dimension. }
    \label{fig:rep tests 1}
\end{figure*} 

\begin{figure*}[h]
    \centering
    \begin{subfigure}{.31\linewidth}
        \centering
        \includegraphics[width = 1\linewidth]{images/biggan_coffee_rep.png}
        \caption{}\label{fig:moons KDE rep}
    \end{subfigure}
        \hfill
    \begin{subfigure}{.31\linewidth}
        \centering
        \includegraphics[width = 1\linewidth]{images/biggan_bubble_rep.png}
        \caption{}\label{fig:mnist kde rep}
    \end{subfigure}
        \hfill
    \begin{subfigure}{.31\linewidth}
        \centering
        \includegraphics[width = 1\linewidth]{images/biggan_schooner_rep.png}
        \caption{}\label{fig:vae rep}
    \end{subfigure}
    \caption[This GAN model produces relatively equal representation according to our clustering for all three classes.]{This GAN model produces relatively equal representation according to our clustering for all three classes. It makes sense that a low truncation level tends to over-represent for one class, as a lower truncation threshold causes less variance. Even though it places samples into all cells, some cells are data-copying much more aggressively than others.}
    \label{fig:biggan rep tests }
\end{figure*} 
The experiments of Sections \ref{sec:MNIST KDE} and \ref{sec:neural model tests} use the MNIST digit dataset \citep{lecun}. We use a training sample, $T$, of size $|T| = 50,000$, a test sample $P_n$ of size $n = 10,000$, a validation sample $V_l$ of $l = 10,000$, and create generated samples of size $m = 10,000$. 

Here, for a meaningful distance metric, we create a custom embedding using a convolutional autoencoder trained using a VGG perceptual loss proposed by \cite{zhang}. The encoder and decoder each have four convolutional layers using batch normalization, two linear layers using dropout, and two max pool layers. The autoencoder is trained for 100 epochs with a batch size of 128 and Adam optimizer with learning rate 0.001. For each training sample $t \in \R^{784}$, the encoder compresses to $z \in \R^{64}$, and decoder expands back up to $\widehat{t} \in \R^{784}$. Our loss is then 
\begin{align*}
    L(t, \widehat{t}) = \gamma(t, \widehat{t}) +  \lambda  \max \{\| z \|_2^2 - 1, 0  \}
\end{align*}
where $\gamma(\bullet, \bullet)$ is the VGG perceptual loss, and $ \lambda  \max \{\| z \|_2^2 - 1, 0  \}$ provides a linear hinge loss outside of a unit $L_2$ ball. The hinge loss encourages the encoder to learn a latent representation within a bounded domain, hopefully augmenting its ability to interpolate between samples. It is worth noting that the perceptual loss is not trained on MNIST, and hopefully uses agnostic features that help keep us from overfitting. We opt to use a standard autoencoder instead of a stochastic autoencoder like a VAE, because we want to be able to exactly data-copy the training set $T$. Thus, we want the encoder to create a near-exact encoding and decoding of the training samples specifically. \textbf{Figure \ref{fig:latent_interp}} provides an example of linearly spaced steps between two training samples. While not perfect, we observe that half-way between the `2' and the `0' is a sample that appears perceptually to be almost almost a `2' and almost a `0'. As such, we consider the distance metric $d(x)$ on this space used in our experiments to be meaningful. 

\begin{figure*}[h]
    \centering
    \begin{subfigure}{0.49\linewidth}
        \centering
        \includegraphics[width = 1\linewidth]{images/biggan_schooner_overfit_cluster.png}
        \label{fig:biggan schooner overfit}
        \caption{$Z_U = -1.05$}
    \end{subfigure}
    \hfill
    \begin{subfigure}{0.49\linewidth}
        \centering
        \includegraphics[width = 1\linewidth]{images/biggan_schooner_underfit_cluster.png}
        \label{fig:biggan schooner underfit}
        \caption{$Z_U = +1.32$}
    \end{subfigure}
    \caption[Example of data-copied and underfit cell of ImageNet `schooner' instance space, from `BigGan' with trunc.]{Example of data-copied and underfit cell of ImageNet `schooner' instance space, from `BigGan' with trunc. threshold = 2. We note here, that --- limited to only 50 training samples --- the insufficient $k = 3$ clustering is perhaps not fine grain enough for this class. Notice that the generated samples falling into the underfit cell (mostly training images of either masts or fronts of boats) are hardly any different from those of the over-fit cell.  They are likely on the boundary of the two cells. With that said, the samples of the data-copied cell \textbf{(a)} are certainly close to the training samples in this region.}
    \label{fig:biggan schooner examples}
\end{figure*}

\paragraph{\textbf{KDE tests}:}
In the MNIST KDE experiments, we fit each KDE $Q$ on the 64-d latent representations of the training set $T$ for several values of $\sigma$; we gather all statistical tests in this space, and effectively only decode to visaully inspect samples. We gather the average and standard deviation of each data point across 5 trials of generating $Q_m$. For the Two-Sample Nearest-Neighbor test, it is computationally intense to compute the nearesnt neighbor in a 64-dimensional dataset of 20,000 points $\widetilde{T} \cup Q_m$ 20,000 times. To limit this, we average each of the training and generated NN accuracy over 500 training and generated samples. We find this acceptable, since the test results depicted in \textbf{Figure \ref{fig:mnist kde NN}} are relatively low variance. 

\paragraph{VAE experiments:}
\label{sec:appendix VAE experiments}
In the MNIST VAE experiments, we only use the 64-d autoencoder latent representation in computing the $C_T$ and 1-NN test scores, and not at all in training. Here, we experiment with twenty standard, fully connected, VAEs using binary cross entropy reconstruction loss. The twenty models have three hidden layers and latent dimensions ranging from $d = 5$ to $d = 100$ in steps of 5. The number of neurons in intermediate layers is approximately twice the number of the layer beneath it, so for a latent space of 50-d, the encoder architecture is $784 \rightarrow 400 \rightarrow 200 \rightarrow 100 \rightarrow 50$, and the decoder architecture is the opposite. 

To sample from a trained VAE, we sample from a standard normal with dimensionality equivalent to the VAEs latent dimension, and pass them through the VAE decoder to the 784-d image space. We then encode these generated images to the agnostic 64-d latent space of the perceptual autoencoder described at the beginning of the section, where $L_2$ distance is meaningful. We also encode the training sample $T$ and test sample $P_n$ to this space, and then run the $C_T$ and two-sample NN tests. We again compute the nearest neighbor accuracies for 500 of the training and generated samples (the 1-NN classifier is fit on the 20,000 sample set $\widetilde{T} \cup Q_m$), which appears to be acceptable due to low test variance. 

\subsubsection{ImageNet Experiments}
\label{sec:appendix biggan experiments}
Here, we have chosen three of the one thousand ImageNet12 classes that `BigGan' produces. To reiterate, a conditional GAN can output samples from a specific class by conditioning on a class code input. We acknowledge that conditional GANs combine features from many classes in ways not yet well understood, but treat the GAN of each class as a uniquely different generative model trained on the training samples from that class. So, for the `coffee' class, we treat the GAN as a coffee generator $Q$, trained on the 1300 `coffee' class samples. For each class, we have 1300 training samples $|T|$, 2000 generated samples $m$, and 50 test samples $n$. Being atypically training sample starved ($m > |T|$), we subsample $Q_m$ (not $T$!), to produce equal size samples for the two-sample NN test. As such, all training samples used are in the combined set $S$. We also note that the 50 test samples provided in each class is highly limiting, only allowing us to split the instance space into about three cells and keep a reasonable number of test samples in each cell. As the number of test samples grows, so can the number of cells and the resolution of the partition. \textbf{Figure \ref{fig:biggan schooner examples}} provides an example of where this clustering might be limited; the generated samples of the underfit cell seem hardly any different from those of the over-fit cell. A finer-grain partition is likely needed here. However, the data-copied cell to the left does appear to be very close to the training set, potentially too close according to $Z_U$. 

In performing these experiments, we gather the $C_T(P_n, Q_m)$ statistic for a given class of images. In an attempt to embed the images into a lower dimensional latent space with $L_2$ significance, we pass each image through an InceptionV3 network and gather the 2048-dimension feature embeddings after the final average pooling layer (Pool3). We then project all inception-space images ($T, P_n, Q_m$) onto the 64 principal components of the training set embeddings. Finally, we use $k$-means to partition the points of each sample into one of $k = 3$ cells. The number of cells is limited by the 50 test images available per class. Any more cells would strain the Central Limit Theorem assumption in computing $Z_U$. Finally, we gather the $C_T$ and two-sample NN baseline statistics on this 64-d space. 

\begin{figure*}[h]
    \centering
    \begin{subfigure}{0.49\linewidth}
        \centering
        \includegraphics[width = 1\linewidth]{images/VAE_C_T_vs_d.png}
        \caption{$C_T(P_n, Q_m)$}
        \label{fig:kMMD comparison vae}
    \end{subfigure}
    \hfill
    \begin{subfigure}{0.49\linewidth}
        \centering
        \includegraphics[width = 1\linewidth]{images/VAE_kMMD_XZ_vs_d.png}
        \caption{kernel MMD}
        \label{fig:kMMD comparison kMMD}
    \end{subfigure}
    \caption{Comparison of the $C_T(P_n, Q_m)$ test presented in this paper alongside the three sample kMMD test. }
    \label{fig:kMMD comparison}
\end{figure*}

\subsection{Comparison with three-sample Kernel-MMD:}
Another three-sample test not shown in the main body of this work is the three-sample kernel MMD test introduced by \cite{gretton_2} intended more for model comparison than for checking model overfitting. For samples $X \sim P$ and $Y \sim Q$, we can estimate the squared kernel MMD between $P$ and $Q$ under kernel $k$ by empirically estimating 

\begin{align*}
    &\text{MMD}^2(P,Q) \\
    &= \E_{x,x' \sim P}[k(x,x')] - 2\E_{x\sim P, y \sim Q}[k(x,y)] +  \E_{y,y' \sim Q}[k(y,y')]
\end{align*}

More recent works such as \cite{reviewer_paper} have repurposed this test for measuring generative model overfitting. Intuitively, if the model is overfitting its training set, the empirical kMMD between training and generated data may be smaller than that between training and test sets. This may be triggered by the data-copying variety of overfitting. 

This test provides an interesting benchmark to consider in addition to those in the main body. \textbf{Figure \ref{fig:kMMD comparison}} demonstrates some preliminary experimental results repeating the MNIST VAE experiment \textbf{Figure \ref{fig:vae C_T vs d}}. To implement the kMMD, we used code posted by \cite{gretton_2} https://github.com/eugenium/MMD, and ran the three sample RBF-kMMD test on twenty MNIST VAEs with decreasing complexity (latent dimension). \textbf{Figure \ref{fig:kMMD comparison kMMD}} attached plots the kMMD distance to training set for both the generated (orange) and test samples (blue). We observe that this test does not appear sensitive to over-parametrized VAEs ($d > 50$) in the same way our proposed test (Figure 1a attached) is. It is sensitive to underfitting ($d << 50$), however. The $p$-values of that work's corresponding hypothesis test (figure not shown) similarly did not respond to data-copying.
We suspect that this insensitivity to data-copying is due to the global nature of this test, incapable of capturing data-copying at smaller scales. 

\graphicspath{{./chapters/chapter3/}}
\chapter{ }

\label{sec:stc priv appendix} 

\subsection{Privacy Mechanism}
We now describe in detail our instance of the exponential mechanism $\mname$. Recall from Definition \ref{def: exp mech} that the exponential mechanism samples candidate $f_i \in F$ with probability
\begin{align*}
	\Pr[\calM(x) = f_i] \propto \exp\big( \frac{\epsilon u(x, f_i)}{2 \Delta u} \big) \ .
\end{align*}
Thus, $\mname$ is fully defined by its utility function, which, as listed in Equation \eqref{eqn:utility}, is approximate Tukey Depth, 
\begin{align*}
u(x, f_i) = \tdappx_{S_x}(f_i) \quad.
\end{align*}
We now describe our approximation algorithm of Tukey Depth $\tdappx_{S_x}(f_i)$, which is an adaptation of the general median hypothesis algorithm proposed by \citet{median_hyp}. 

\begin{algorithm}
    \SetKwFunction{isOddNumber}{isOddNumber}
    % \SetKwInput{Input}{Input}
    % \SetKwInput{Output}{Output}
    \SetKwInOut{KwIn}{Input}
    \SetKwInOut{KwOut}{Output}

    \KwIn{$m$ candidates $F$, \\sentence embs. $S_x = (s_1, \dots, s_k)$,\\ number of projections $p$}
    \KwOut{probability of sampling each candidate $P_F = [P_{f_1}, \dots, P_{f_m}]$}
    
    $v_1, \dots, v_p \gets $ random vecs. on unit sphere 
    
    \tcp{Project all embeddings}
  
    \For{$i \in [k]$}{
    \For{$j \in [p]$}{
    $s_i^j \gets s_i^\intercal v_j$
    }
    }
    
    \For{$i \in [m]$}{
    \For{$j \in [p]$}{
    $f_i^j \gets f_i^\intercal v_j$
    
    \tcc{Compute depth of $f_i$ on projection $v_j$}
    
    $h_j(x,f_i) \gets \#\{s_l^j : s_l^j \geq  f_i^j, l \in [k]\}$
    
    $u_j(x,f_i) \gets -\big| h_j(x,f_i) - \frac{k}{2} \big|$ 
    }
    $u(x,f_i) \gets \max_{j \in [p]} u_j(x,f_i)$
    $\hat{P}_{f_i} \gets \exp(\epsilon u(x,f_i) / 2)$
    }
    
    $\Psi \gets \sum_{i=1}^{m} \hat{P}_{f_i}$
    
    \For{$i \in [m]$}{
    $P_{f_i} \gets \frac{1}{\Psi} \hat{P}_{f_i}$
    }
    
    \KwRet{$P_F$}
    \caption{$\mname$ compute probabilities}
    \label{alg:main alg}
\end{algorithm}

Note that we can precompute the projections on line 10. The runtime is $O(mkp)$: for each of $m$ candidates and on each of $p$ projections, we need to compute the scalar difference with $k$ sentence embeddings. Sampling from the multinomial distribution defined by $P_F$ then takes $O(m)$ time. 

Additionally note from lines 13 and 15 that utility has a maximum of 0 and a minimum of $-\frac{k}{2}$, which is a semantic change from the main paper where maximum utility is $\frac{k}{2}$ and minimum is 0. 

\subsection{Proof of Privacy}

\textbf{Theorem \ref{thm:mainthm}} \emph{
	$\mname$ satisfies $\epsilon$-Sentence Privacy
}
\begin{proof}
%\begin{lemma}
%	The maximum change in the minimum $h_j$ for neighboring documents is 1. 
%	\begin{proof}
%		\max_{x, x', f_i} \big| 
%	\end{proof}
%\end{lemma}

	It is sufficient to show that the sensitivity, 
	\begin{align*}
		\Delta u = \max_{x, x', f_i} | u(x,f_i) - u(x', f_i)| \leq 1 \quad . 
	\end{align*} 
	Let us expand the above expression using the terms in Algorithm \ref{alg:main alg}. 
	\begin{align*}
		\Delta u &= \max_{x, x', f_i} | \max_{j \in [p]} u_j(x,f_i)  - \max_{j' \in [p]} u_{j'}(x',f_i)| \\ 
		&= \max_{x, x', f_i} | \min_{j \in [p]} \big| h_j(x,f_i) - \frac{k}{2} \big|  \\
		&- \min_{j' \in [p]} \big| h_{j'}(x',f_i) - \frac{k}{2} \big|| \\
		&\leq \max_{ f_i} | \min_{j \in [p]} \big| h_j(x,f_i) - \frac{k}{2} \big|  \\
		&- \big( \min_{j' \in [p]} \big| h_{j'}(x,f_i) - \frac{k}{2} \big|-1\big) | \\
		&\leq 1
	\end{align*}
	The last step follows from the fact that $|h_j(x, f_i) - h_j(x', f_i)| \leq 1$ for all $j \in [p]$. In other words, by modifying a single sentence embedding, we can only change the number of embeddings greater than $f_i^j$ on projection $j$ by 1. So, the distance of $h_j(x, f_i)$ from $\frac{k}{2}$ can only change by 1 on each projection. In the `worst case', the distance $\big| h_j(x,f_i) - \frac{k}{2} \big|$ reduces by 1 on every projection $v_j$. Even then, the minimum distance from $\frac{k}{2}$ across projections (the worst case depth) can only change by 1, giving us a sensitivity of 1. 
\end{proof}





%$\Delta u = \max_{D, D', o} | u(D,o) - u(D', o)|$. 

\subsection{Experimental Details}

Here, we provide an extended, detailed version of section \ref{sec:experiments}. 

For the general encoder, $G:\calS \rightarrow \R^{768}$, we use SBERT \cite{sbert}, a version of BERT fine-tuned for sentence encoding. Sentence embeddings are generated by mean-pooling output tokens. In all tasks, we freeze the weights of SBERT. The cluster-preserving recoder, $H$, as well as every classifier is implemented as an instance of a 4-layer MLP taking $768$-dimension inputs and only differing on output dimension. We denote an instance of this MLP with output dimension $o$ as \MLP{o}. We run 5 trials of each experiment with randomness taken over the privacy mechanisms, and plot the mean along with a $\pm$ 1 standard deviation envelope. 

\paragraph{Non-private:} For our non-private baseline, we demonstrate the usefulness of sentence-mean document embeddings. First, we generate the document embeddings $\overline{g}(x_i)$ for each training, validation, and test set document using SBERT, $G$. We then train a classifier $C_{\text{nonpriv}} = $ \MLP{r} to predict each document's topic or sentiment, where $r$ is the number of classes. The number of training epochs is determined with the validation set. 

\paragraph{\technique :} We first collect the candidate set $F$ by sampling 5k document embeddings from the subset of the training set containing at least 8 sentences. We run $k$-means with $n_c = 50$ cluster centers, and label each training set document embedding $t_i \in T_G$ with its cluster. The sentence recoder, $H = $ \MLP{768} is trained on the training set along with the linear model $L$ with the Adam optimizer and cross-entropy loss. For a given document $x$, its  sentence embeddings $S_x$ are passed through $H$, averaged together, and then passed to $L$ to predict $x$'s cluster. $L$'s loss is then back-propagated through $H$. A classifier $C_{\text{dc}} = $ \MLP{r} is trained in parallel using a separate instance of the Adam optimizer to predict class from the recoded embeddings, where $r$ is the number of classes (topics or sentiments). The number of training epochs is determined using the validation set. At test time, (generating private embeddings using $\mname$), the optimal number of projections $p$ is empirically chosen for each $\epsilon$ using the validation set. 

\paragraph{Truncation:} The truncation baseline \cite{clifton} requires first constraining the embedding instance space. We do so by computing the 75\% median interval on each of the 768 dimensions of training document embeddings $T_G$. Sentence embeddings are truncated at each dimension to lie in this box. In order to account for this distribution shift, a new classifier $C_{\text{trunc}} = $ \MLP{r} is trained on truncated mean embeddings to predict class. The number of epochs is determined with the validation set. At test time, a document's sentence embeddings $S_x$ are truncated and averaged. We then add Laplace noise to each dimension with scale factor $\frac{768 w}{k \epsilon}$, where $w$ is the width of the box on that dimension (\emph{sensitivity} in DP terms). Note that the standard deviation of noise added is inversely proportional to the number of sentences in the document, due to the averaging operation reducing sensitivity. 

\paragraph{Word Metric-DP:} Our next baseline satisfies $\epsilon$-word-level metric DP and is adopted from \cite{metricdp}. The corresponding mechanism $\text{MDP}: \calX \rightarrow \calX$ takes as input a document $x$ and returns a private version, $x'$, by randomizing each word individually. For comparison, we generate document embeddings by first randomizing the document $x' = \text{MDP}(x)$ as prescribed by \cite{metricdp}, and then computing its document embedding $\overline{g}(x')$ using SBERT. At test time, we classify the word-private document embedding using $C_{\text{nonpriv}}$. 

\paragraph{Random Guess:} To set a bottom-line, we show the theoretical performance of a random guesser. The guesser chooses class $i$ with probability $q_i$ equal to the fraction of $i$ labels in the training set. The performance is then given by $\sum_{i = 1}^{r} q_i^2$. 



\subsection{Reproducability Details}
We plan to publish a repo of code used to generate the exact figures in this paper (random seeds have been set) with the final version. Since we do not train the BERT base model $G$, our algorithms and training require relatively little computational resouces. Our system includes a single Nvidia GeForce RTX 2080 GPU and a single Intel i9 core. All of our models complete an epoch training on all datasets in less than one minute. We never do more than 20 epochs of training. All of our classifier models train (including linear model) have less than 11 million parameters. The relatively low amount of parameters is due to the fact that we freeze the underlying language model. The primary hyperparameter tuned is the number of projections $p$. We take the argmax value on the validation set between 10 and 100 projections. We repeat this for each value of $\epsilon$. 

\paragraph{Dataset preprocessing:} For all datasets, we limit ourselves to documents with at least 2 sentences. 

\imdb: This dataset has pre-defined train/test splits. We use the entire training set and form the test set by randomly sampling 4,000 from the test set provided. We do this for efficiency in computing the Metric-DP baseline, which is the slowest of all algorithms performed. Since the Metric-DP baseline randomizes first, we cannot precompute the sentence embeddings $G(s_i)$ -- we need to compute the sentence embeddings every single time we randomize. Since we randomize for each sentence of each document at each $\epsilon$ and each $k$ over 5 trials -- this takes a considerable amount of time. 

\goodreads: This dataset as provided is quite large. We randomly sample 15000 documents from each of 4 classes, and split them into 12K training examples, 2K validation examples, and 1K test examples per class. 

\tnews: We preprocess this dataset to remove all header information, which may more directly tell information about document class, and only provide the model with the sentences from the main body. We use the entire dataset, and form the Train/Val/Test splits by random sampling. 

















% \twocolumn
% [\textbf{Local Inferential Privacy through Data Shuffling -- Supplementary Material }]

\graphicspath{{./chapters/chapter4/}}
\chapter{ }


\section{Appendix}\label{app}
\subsection{Background Cntd.}\label{app:background}

\subsection{Local Inferential Privacy}  \vspace{-0.2cm}
%introduce pufferfish inferntial log loss 
%In this section, we introduce some context for inferential privacy in the \ldp setting. 
%Inferential privacy captures the privacy loss in the face of an informed adversary in a Bayesian framework. 
Local inferential privacy captures what information a Bayesian adversary \cite{Pufferfish}, with some prior, can learn in the \ldp setting. 
Specifically, it measures the largest possible ratio between the adversary's posterior and prior beliefs about an individual’s data after observing a mechanism's output .%\footnote{This quantity is identical to the \ldp parameter of the mechanism when\textit{individuals’ data are independent}\cite{sok,}.}.
\begin{defn}(Local Inferential Privacy Loss \cite{Pufferfish}) Let $\bx=\langle x_1, \cdots, x_n\rangle$ and let $\by=\langle y_1, \cdots, y_n \rangle$ denote the input (private) and output sequences (observable to the adversary) in the \ldp setting. Additionally, the adversary's auxiliary knowledge is modeled by a prior distribution $\mathcal{P}$ on $\mathbf{x}$. The inferential privacy loss for the input sequence $\mathbf{x}$ is given by
% \vspace{0cm} 
\begin{equation}
% \vspace{-0.1cm}
\small \mathbb{L}_{\calP}(\mathbf{x})=\max_{\substack{i\in [n]\\ a,b \in \calX}}\Bigg(\log\frac{\mathrm{Pr}_{\calP}[\mathbf{y}|x_i=a]}{\mathrm{Pr}_{\calP}[\mathbf{y}|x_i=b]}\Bigg)
=\max_{\substack{i\in [n]\\ a,b \in \calX}}\Bigg (	\bigg| \log \frac{\mathrm{Pr}_{\calP}[x_i = a | \bf{y} ]}{\mathrm{Pr}_{\calP}[x_i = b | \bf{y}]}
	- \log \frac{\mathrm{Pr}_{\calP}[x_i = a]}{\mathrm{Pr}_{\calP}[x_i = b]} \bigg|\Bigg)
\end{equation}
\label{def:ip}
\vspace{-1em}
\end{defn}
% Using Bayes' theorem, we have\vspace{-0.2cm}
% \begin{gather*}\small\vspace{-0.4cm} \mathbb{L}_{\calP}(\mathbf{x})=\max_{\substack{i\in [n]\\ a,b \in \calX}}\Bigg (	\bigg| \log \frac{\mathrm{Pr}_{\calP}[x_i = a | \bf{y} ]}{\mathrm{Pr}_{\calP}[x_i = b | \bf{y}]}
% 	- \log \frac{\mathrm{Pr}_{\calP}[x_i = a]}{\mathrm{Pr}_{\calP}[x_i = b]} \bigg|\Bigg)\vspace{-0.7cm}\end{gather*}
Bounding  $\mathbb{L}_{\calP}(\mathbf{x})$  would imply
 that the adversary's belief about the value of any $x_i$ does not change by much even after observing the output sequence $\bf{y}$. This means that an informed adversary does not learn much about the individual $i$'s private input upon observation of the entire private dataset $\by$.

Here we define two rank distance measures \begin{defn}[Kendall's $\tau$ Distance] For any two permutations, $\sigma, \pi \in \mathrm{S}_n$, the Kendall's $\tau$
distance $\textswab{d}_{\tau}(\sigma, \pi)$ counts the number of pairwise disagreements between $\sigma$ and $\pi$, i.e., the
number of item pairs that have a relative order in one permutation and a different order in
the other. Formally, 
\begin{equation}
\textswab{d}_{\tau}(\sigma,\pi)=\Big| \ \big\{(i,j) : i < j,  \big[\sigma(i) > \sigma(j) \wedge \pi(i) < \pi(j) \big]\\\hspace{2cm} \\
		\vee \big[\sigma(i) < \sigma(j) \wedge \pi(i) > \pi(j)\big] \big\} \ \Big|\label{eq:kendalltau}
\end{equation}  
\label{def:kendall} 
\end{defn}
%Equivalently, $d_{\tau}(\sigma, \pi)$ is defined as the number of adjacent swaps to convert$\sigma^{-1}$into $\pi^{-1}$.
For example, if $\sigma=(1 \:\: 2 \: \: 3 \: \: 4 \:  \: 5 \: \: 6 \: \: 7 \: \: 8 \: 9 \: 10)$ and  $\pi=(1\:2\:3\:\underline{6} \: 5 \: \underline{4}\:7\:8\:9\:10)$, then $\textswab{d}_{\tau}(\sigma,\pi)=3$.

 Next, Hamming distance measure is defined as follows.
 
\begin{defn}[Hamming Distance] 
For any two permutations, $\sigma, \pi \in \mathrm{S}_n$, the Hamming distance $\textswab{d}_{H}(\sigma, \pi)$ counts the number of positions in which the two permutations disagree. Formally, 
\begin{align*}
    \textswab{d}_H(\sigma, \pi)
    &= \Big| \big\{ i \in [n] : \sigma(i) \neq \pi(i) \big\} \Big| 
\end{align*}
Repeating the above example, if $\sigma=(1 \:\: 2 \: \: 3 \: \: 4 \:  \: 5 \: \: 6 \: \: 7 \: \: 8 \: 9 \: 10)$ and  $\pi=(1 \: 2 \: 3 \: \underline{6} \: 5 \: \underline{4} \: 7 \: 8 \: 9 \: 10)$, then $\textswab{d}_{H}(\sigma,\pi)=2$.
\end{defn}

\subsection{\name-privacy and the De Finetti attack}
\label{app:de finetti}
We now show that a strict instance of \name privacy is sufficient for thwarting any de Finetti attack \cite{definetti} on individuals. The de Finetti attack involves a Bayesian adversary, who, assuming some degree of correlation between data owners, attempts to recover the true permutation from the shuffled data. As written, the de Finetti attack assumes the sequence of sensitive attributes and side information $(x_1, t_1), \dots, (x_n, t_n)$ are \emph{exchangeable}: any ordering of them is equally likely. By the de Finetti theorem, this implies that they are i.i.d. conditioned on some latent measure $\theta$. To balance privacy with utility, the $\bx$ sequence is non-uniformly randomly shuffled w.r.t. the $\bt$ sequence producing a shuffled sequence $\bz$, which the adversary observes. Conditioning on $\bz$ the adversary updates their posterior on $\theta$ (i.e. posterior on a model predicting $x_i | t_i$), and thereby their posterior predictive on the true $\bx$. The definition of privacy in \cite{definetti} holds that the adversary's posterior beliefs are close to their prior beliefs by some metric on distributions in $\calX$, $\delta(\cdot, \cdot)$: 
\begin{align*}
    \delta\Big( \Pr[x_i], \Pr[x_i | \bz] \Big) \leq \alpha 
\end{align*}

We now translate the de Finetti attack to our setting. First, to align notation with the rest of the paper we provide privacy to the sequence of $\ldp$ values $\by$ since we shuffle those instead of the $\bx$ values as in \cite{definetti}. We use max divergence (multiplicative bound on events used in \DP) for $\delta$: 
\begin{align*}
    \Pr[y_i \in O] &\leq e^\alpha \Pr[y_i \in O | \bz] \\
    \Pr[y_i \in O | \bz] &\leq e^\alpha \Pr[y_i \in O]
\end{align*}
which, for compactness, we write as 
\begin{align}
    \Pr[y_i \in O] \approx_\alpha \Pr[y_i \in O | \bz] \quad. 
    \label{eq:definetti privacy}
\end{align}
We restrict ourselves to shuffling mechanisms, where we only randomize the order of sensitive values. By learning the unordered values $\{y\}$ alone, an adversary may have arbitrarily large updates to its posterior (e.g. if all values are identical), breaking the privacy requirement above. With this in mind, we assume the adversary already knows the unordered sequence of values $\{y\}$ (which they will learn anyway), and has a prior on permutations $\sigma$ allocating values from that sequence to individuals. We then generalize the de Finetti problem to an adversary with an \emph{arbitrary} prior on the true permutation $\sigma$, and observes a randomize permutation $\sigma'$ from the shuffling mechanism. We require that the adversary's prior belief that $\sigma(i) = j$ is close to their posterior belief for all $i,j \in [n]$: 
\begin{align}
    \Pr[\sigma \in \Sigma_{i,j} ] \approx_\alpha \Pr[\sigma \in \Sigma_{i,j} | \sigma'] \quad \forall i,j \in [n], \forall \sigma' \in S_n \quad ,
    \label{eq:definetti privacy II}
\end{align}
where $\Sigma_{i,j} = \{\sigma \in S_n : \sigma(i) = j\}$, the set of permutations assigning element $j$ to $\DO_i$. Conditioning on any unordered sequence $\{y\}$ with all unique values, the above condition is necessary to satisfy Eq. \eqref{eq:definetti privacy} for events of the form $O = \{y_i = a\}$, since $\{y_i = a\} = \{\Sigma_{i,j}\}$ for some $j \in [n]$. For any $\{y\}$ with repeat values, it is sufficient since $\Pr[y_i = a]$ is the sum of probabilities of disjoint events of the form $\Pr[\sigma \in \Sigma_{i,k}]$ for various $k \in [n]$ values. 

We now show that a strict instance of \name-privacy satisfies Eq. \eqref{eq:definetti privacy II}. Let $\widehat{\calG}$ be any group assignment such that at least one $G_i \in \widehat{\calG}$ includes all data owners, $G_i = \{1, 2, \dots, n\}$. 

\begin{prope}
A $(\widehat{\calG}, \alpha)$-\name-private shuffling mechanism $\sigma' \sim \calA$ satisfies  
\begin{align*}
    \Pr[\sigma \in \Sigma_{i,j} ] \approx_\alpha \Pr[\sigma \in \Sigma_{i,j} | \sigma']
\end{align*}
for all $i,j \in [n]$ and all priors on permutations $\Pr[\sigma]$. 
\end{prope}

\begin{proof}

\begin{lemma}
\label{lem:definetti equivalent}
    For any prior $\Pr[\sigma]$, Eq. \eqref{eq:definetti privacy II} is equivalent to the condition
    \begin{align}
        \frac{\sum_{\hat{\sigma} \in \overline{\Sigma}_{i,j}} \Pr[\hat{\sigma}] \Pr[\sigma' | \hat{\sigma}] }{
        \sum_{\hat{\sigma} \in {\Sigma_{i,j}}} \Pr[\hat{\sigma}] \Pr[\sigma' | \hat{\sigma}] } 
        \approx_\alpha 
        \frac{\sum_{\hat{\sigma} \in \overline{\Sigma}_{i,j}} \Pr[\hat{\sigma}] }{
        \sum_{\hat{\sigma} \in {\Sigma_{i,j}}} \Pr[\hat{\sigma}] }
        \label{eq:definetti privacy III}
    \end{align}
    where the set $\overline{\Sigma}_{i,j}$ is the complement of ${\Sigma}_{i,j}$. 
\end{lemma}
Under grouping $\hat{\calG}$, every permutation $\sigma_a \in {\Sigma}_{i,j}$ neighbors every permutation $\sigma_b \in \overline{\Sigma}_{i,j}$, $\sigma_a \approx_{\hat{\calG}} \sigma_b$, for any $i,j$. By the definition of \name-privacy, we have that for any observed permutation $\sigma'$ output by the mechanism: 
\begin{align*}
    \Pr[\sigma' | \sigma = \sigma_a] \approx_\alpha \Pr[\sigma' | \sigma = \sigma_b]
    \quad \forall \sigma_a \in {\Sigma}_{i,j}, \sigma_b \in \overline{\Sigma}_{i,j}, \sigma' \in S_n 
    \quad .
\end{align*}
This implies Eq. \ref{eq:definetti privacy III}. Thus, $(\widehat{\calG}, \alpha)$-\name-privacy implies Eq. \ref{eq:definetti privacy III}, which implies Eq. \ref{eq:definetti privacy II}, thus proving the property. 
\end{proof}

Using Lemma \ref{lem:definetti equivalent}, we may also show that this strict instance of \name-privacy is \emph{necessary} to block all de Finetti attacks: 

\begin{prope}
A $(\widehat{\calG}, \alpha)$-\name-private shuffling mechanism $\sigma' \sim \calA$ is necessary to satisfy 
\begin{align*}
    \Pr[\sigma \in \Sigma_{i,j} ] \approx_\alpha \Pr[\sigma \in \Sigma_{i,j} | \sigma']
\end{align*}
for all $i,j \in [n]$ and all priors on permutations $\Pr[\sigma]$. 
\end{prope}

\begin{proof}
If our mechanism $\calA$ is not $(\widehat{\calG}, \alpha)$-\name-private, then for some pair of true (input) permutations $\sigma_a \neq \sigma_b$ and some released permutation $\sigma' \sim \calA$, we have that 
\begin{align*}
    \Pr[\sigma' | \sigma_b] \geq e^\alpha \Pr[\sigma' | \sigma_a]\quad. 
\end{align*}
Under $\hat{\calG}$, all permutations neighbor each other, so $\sigma_a \approx_{\hat{\calG}} \sigma_b$. Since $\sigma_a \neq \sigma_b$, then for some $i,j \in [n]$, $\sigma_a \in \Sigma_{i,j}$ and $\sigma_b \in \overline{\Sigma}_{i,j}$: one of the two permutations assigns some $j$ to some $\DO_i$ and the other does not. Given this, we may construct a bimodal prior on the true $\sigma$ that assigns half its probability mass to $\sigma_a$ and the rest to $\sigma_b$, 
\begin{align*}
    \Pr[\sigma_a] = \Pr[\sigma_b] = \frac{1}{2} \quad .
\end{align*}
Therefore, for released permutation $\sigma'$, the RHS of Eq. \ref{eq:definetti privacy III} is 1, and the LHS is 
\begin{align*}
    \frac{\sum_{\hat{\sigma} \in \overline{\Sigma}_{i,j}} \Pr[\hat{\sigma}] \Pr[\sigma' | \hat{\sigma}] }{
        \sum_{\hat{\sigma} \in {\Sigma_{i,j}}} \Pr[\hat{\sigma}] \Pr[\sigma' | \hat{\sigma}] }
        &= \frac{\nicefrac{1}{2} \Pr[\sigma' | \sigma_b]}{\nicefrac{1}{2} \Pr[\sigma' | \sigma_a]} \\
        &\geq e^\alpha \quad , 
\end{align*}
violating Eq. \ref{eq:definetti privacy III}, thus violating Eq. \ref{eq:definetti privacy II}, and failing to prevent de Finetti attacks against this bimodal prior. 
\end{proof}

Ultimately, unless we satisfy \name-privacy shuffling the entire dataset, there exists some prior on the true permutation $\Pr[\sigma]$ such that after observing the shuffled $\bz$ permuted by $\sigma'$, the adversary's posterior belief on one permutation is larger than their prior belief by a factor $\geq e^\alpha$. If we suppose that the set of values $\{y\}$ are all distinct, this means that for some $a \in \{y\}$, the adversary's belief that $y_i = a$ is signficantly larger after observing $\bz$ than it was before. 

Now to prove Lemma \ref{lem:definetti equivalent}: 
\begin{proof}
\begin{align*}
    \Pr[\sigma \in \Sigma_{i,j} ] 
    &\approx_\alpha \Pr[\sigma \in \Sigma_{i,j} | \sigma'] \\
    \Pr[\sigma \in \Sigma_{i,j} ]
    &\approx_\alpha \frac{\Pr[\sigma' | \sigma \in \Sigma_{i,j}] \Pr[\sigma \in \Sigma_{i,j} ]}{\sum_{\hat{\sigma} \in S_n} \Pr[\hat{\sigma}] \Pr[\sigma' | \hat{\sigma}]} \\
    \sum_{\hat{\sigma} \in S_n} \Pr[\hat{\sigma}] \Pr[\sigma' | \hat{\sigma}] 
    &\approx_\alpha \Pr[\sigma' | \sigma \in \Sigma_{i,j}] \\
    \sum_{\hat{\sigma} \in S_n} \Pr[\hat{\sigma}] \Pr[\sigma' | \hat{\sigma}] 
    &\approx_\alpha \Pr[\sigma \in \Sigma_{i,j}]^{-1} \sum_{\hat{\sigma} \in \Sigma_{i,j}} \Pr[\hat{\sigma}] \Pr[\sigma' | \hat{\sigma}] \\
    \sum_{\hat{\sigma} \in \Sigma_{i,j}} \Pr[\hat{\sigma}] \Pr[\sigma' | \hat{\sigma}] +
    \sum_{\hat{\sigma} \in \overline{\Sigma}_{i,j}} \Pr[\hat{\sigma}] \Pr[\sigma' | \hat{\sigma}] 
    &\approx_\alpha \Pr[\sigma \in \Sigma_{i,j}]^{-1} \sum_{\hat{\sigma} \in \Sigma_{i,j}} \Pr[\hat{\sigma}] \Pr[\sigma' | \hat{\sigma}] \\
    \sum_{\hat{\sigma} \in \overline{\Sigma}_{i,j}} \Pr[\hat{\sigma}] \Pr[\sigma' | \hat{\sigma}] 
    &\approx_\alpha \sum_{\hat{\sigma} \in \Sigma_{i,j}} \Pr[\hat{\sigma}] \Pr[\sigma' | \hat{\sigma}] 
    \Big( \frac{1}{\Pr[\sigma \in \Sigma_{i,j}]} - 1 \Big)  \\
    \frac{\sum_{\hat{\sigma} \in \overline{\Sigma}_{i,j}} \Pr[\hat{\sigma}] \Pr[\sigma' | \hat{\sigma}] }{
    \sum_{\hat{\sigma} \in {\Sigma_{i,j}}} \Pr[\hat{\sigma}] \Pr[\sigma' | \hat{\sigma}] } 
    &\approx_\alpha 
    \frac{\sum_{\hat{\sigma} \in \overline{\Sigma}_{i,j}} \Pr[\hat{\sigma}] }{
    \sum_{\hat{\sigma} \in {\Sigma_{i,j}}} \Pr[\hat{\sigma}] }
\end{align*}
\end{proof}

As such, a strict instance of \name-privacy can defend against any de Finetti attack (i.e. for any prior $\Pr[\sigma]$ on permutations), wherein at least one group $G_i \in \calG$ includes all data owners. Furthermore, it is necessary. This makes sense. In order to defend against any prior, we need to significantly shuffle the entire dataset. Without a restriction of priors as in Pufferfish \cite{Pufferfish}, the de Finetti attack (i.e. uninformed Bayesian adversaries) is an indelicate metric for evaluating the privacy of shuffling mechanisms: to achieve significant privacy, we must sacrifice all utility. This in many regards is reminiscent of the no free lunch for privacy theorem established in \cite{Kifer}. As such, there is a need for more flexible privacy definitions for shuffling mechanisms.  

% Consider that an adversary with a prior on $\bx$ can similarly be written as a prior on the unordered set of values $\{x\}$ and on the permutation allocating values from that set to individuals $\sigma | \{x\}$. Any event may be written as 
% \begin{align*}
%     \Pr[\bx \in O] = \Pr[\sigma \in \Sigma | \{x\}] \Pr[\{x\} \in X] 
% \end{align*}
% for any event $O \subseteq \calX^n$ and corresponding events $\Sigma \subseteq S_n$ and $X \subseteq \calE$, where $\calE$ is the sigma field generated by all sets of $n$ elements in $\calX$. For shuffling mechanisms, we assume the adversary already knows the set of values $\{x\}$ since they will learn them anyway. Formally, the posterior $\Pr[\{x\} \in X | \bz]$ concentrates on the single outcome of $\{x\} = \{z\}$

% De Finetti's privacy definition asks that the prior $\Pr[y_i]$ is close to the posterior in some distance (I use ldp values here which is analagous to the $x_i$ values used in the paper, since they shuffled the $x_i$ directly): 
% \begin{align*}
%     \delta( \Pr[y_i], \Pr[y_i | \bz] ) \leq b
% \end{align*}

% if we use max divergence as this distance, we get that 
% \begin{align*}
%     \Pr[y_i = a] \leq \Pr[y_i = a | \bz]
% \end{align*}
% for all $i$, all $a \in \calX$, all $\bz \in \calZ^n$. 

% I'll make this next part brief, but I need to write out a longer bit to make it clear. The point is that we can equivalently have an arbitrary prior/posterior on permutations as the De Finetti paper has a prior posterior on models. I have a more mathematical way of writing this but it take a little effort. De Finetti assumes the sequence of underlying $(y_i, t_i)$ are exchangeable. There is some latent model $f_\theta: \calT \rightarrow \calY$ and a prior on $t_i$'s. Conditioning on knowing $f_\theta$ and the prior on $t_i$'s all the data is iid. Otherwise, by observing a few partially shuffled entries, we can update our posterior belief on the model $\theta$, which then updates our posterior predictive on the true permutation of the remaining points. 

% For shuffling mechanisms, we may assume the adversary knows the orderless set of $\by$, $\{y\}$, since they will see it anyway, and then make our proof for all possible $\{y\}$. Our adversary has a prior on true permutations $\sigma$ and they observe a noisy permutation $\sigma'$. We want their prior to be close to their posterior. 

% \begin{align*}
%     \Pr[\sigma] \leq e^\epsilon \Pr[\sigma | \sigma']
% \end{align*}
% and
% \begin{align*}
%     e^\epsilon \Pr[\sigma] \geq \Pr[\sigma | \sigma']
% \end{align*}
% for all true permutations $\sigma$ and observed permutations $\sigma'$. Note that this is equivalent to the original posterior/prior requirement on $y_i$: a prior on permutations implies that the belief on $y_i$ is a mixture over the believed permutuations. If we have a large divergence on permutations, we can construct a $\{y\}$ to make the divergence on $y_i$ equally large. Can write a more formal version. 
% From here the rest is relatively simple: 

% \begin{align*}
%     \Pr[\sigma] &\approx_{\epsilon} \Pr[\sigma | \sigma'] \\
%     \Pr[\sigma] &\approx_{\epsilon} \Pr[\sigma' | \sigma] \Pr[\sigma] \big/ \Pr[\sigma'] \\
%     \Pr[\sigma'] &\approx_{\epsilon} \Pr[\sigma' | \sigma] \\
%     \sum_{\sigma^*} \Pr[\sigma^*] \Pr[\sigma' | \sigma^*] &\approx_{\epsilon} \Pr[\sigma' | \sigma] \\
% \end{align*}

% So, if we don't have $\Pr[\sigma' | \sigma] \leq e^\epsilon \Pr[\sigma' | \sigma^*]$ for all $\sigma', \sigma, \sigma^*$, then we can construct a prior $\Pr[\sigma^*]$ that will break the above $\approx_\epsilon$. But also, if we have $\Pr[\sigma' | \sigma] \leq e^\epsilon \Pr[\sigma' | \sigma^*]$, then we \emph{can't} construct a prior $\Pr[\sigma^*]$ to break the above $\approx_\epsilon$. 

% The requirement $\Pr[\sigma' | \sigma] \leq e^\epsilon \Pr[\sigma' | \sigma^*]$ is just \name-privacy for $\alpha = \epsilon$ and with at least one group $G_i \in \calG$ being all data owners. 

% The main part that I need to write out more here is explaining how you need a low divergence on permutations to have a low divergence on beliefs on $y_i$. The proof is by showing that one's beliefs on $y_i$ is just a mixture over these permutations. If I can have a high divergence on permutations, I can construct some $\{y\}$ and some pair of distributions on permutations to make the divergence on $y_i$ equally large. 
 
\subsection{ Additional Properties of \name-privacy} \label{app:post-processing}

\begin{lemma}[Convexity] \label{lem:convexity}
Let $\calA_1, \dots \calA_k: \mathcal{Y}^n \mapsto \mathcal{V}$ be a collection of $k$ $(\alpha,\calG)$-\name private mechanisms for a given group assignment $\calG$ on a set of $n$ entities. Let $\calA: \mathcal{Y}^n \mapsto \mathcal{V}$ be a convex combination of these $k$ mechanisms, where the probability of releasing the output of mechanism $\calA_i$ is $p_i$, and $\sum_{i=1}^k p_i = 1$. $\calA$ is also $(\alpha,\calG)$-\name private w.r.t. $\calG$. 
\end{lemma}
\begin{proof}
For any $(\sigma, \sigma') \in \mathrm{N}_\calG$ and $\by \in \calY$: 
\begin{align*}
    \mathrm{Pr} [\calA \big( \sigma( \by) \big) \in O]
    &= \sum_{i=1}^k p_i \mathrm{Pr} [\calA_i \big( \sigma( \by) \big) \in O] \\
    &\leq e^\alpha \sum_{i=1}^k p_i \mathrm{Pr} [\calA_i \big( \sigma'( \by) \big) \in O] \\
    &=  \mathrm{Pr} [\calA \big( \sigma'( \by) \big) \in O]
\end{align*}
\end{proof}

% For a given group assignment $\calG$ on a set of $n$ entities and a privacy parameter $\alpha \in \R_{\geq0}$, a randomized  mechanism $\calA: \mathcal{Y}^n \mapsto \mathcal{V} $ is $(\alpha,\mathcal{G})$-\name~private if for all $\mathbf{y} \in \mathcal{Y}^n$ and neighboring permutations $\sigma, \sigma' \in \mathrm{N}_\calG$ and any subset of output $O\subseteq \mathcal{V}$, we have\vspace{-0.2cm} 
% \begin{gather*} \vspace{-0.5cm}
%     \mathrm{Pr}[\calA\big(\sigma(\mathbf{y})\big) \in O] \leq e^\alpha \cdot \mathrm{Pr}\big[\calA\big(\sigma'(\mathbf{y})\big) \in O \big] \numberthis \label{eq:privacy} \vspace{-0.2cm}
% \end{gather*}
% %where $\bz=\pi(\by)=\langle y_{\pi(1)},\cdots, y_{\pi(n)}\rangle, \pi \in \mathrm{S}_n$
%  $\sigma(\mathbf{y})$ and $\sigma'(\mathbf{y})$  are defined to be \textit{neighboring sequences}. 

\begin{thm}[Post-processing]\label{theorem:post}
Let \scalebox{0.9}{$\mathcal{A}: \mathcal{Y}^n \mapsto \mathcal{V}$} be  $(\alpha,\calG)$-\name private for a given group assignment $\calG$ on a set of $n$ entities. Let \scalebox{0.9}{$f : \mathcal{V} \mapsto \mathcal{V}'$} be an
arbitrary randomized mapping. Then \scalebox{0.9}{$f \circ \mathcal{A} : \mathcal{Y}^n \mapsto \mathcal{V}'$} is also $(\alpha,\calG)$-\name private. \vspace{-0.2cm}\end{thm}

\begin{proof}
Let $g: \mathcal{V}\rightarrow \mathcal{V}'$ be a deterministic, measurable function. For any output event $\mathcal{Z}\subset \mathcal{V}'$, let $\mathcal{W}$ be its preimage: \newline $\mathcal{W}=\{v \in \mathcal{V}| g(v) \in \mathcal{Z}\}$. Then, for any $(\sigma, \sigma') \in \mathrm{N}_\calG$,
\begin{align*}
    \mathrm{Pr}\Big[g\Big(\calA\big(\sigma(\by)\big)\Big)\in \mathcal{Z}\Big] 
    &= \mathrm{Pr}\Big[\calA\big(\sigma(\by)\big)\in \mathcal{W}\Big] \\ 
    &\leq e^{\alpha} \cdot \mathrm{Pr}\Big[\calA\big(\sigma'(\by)\big)\in \mathcal{W}\Big]\\ 
    &=e^{\alpha}\cdot \mathrm{Pr}\Big[g\Big(\calA\big(\sigma'(\by)\big)\Big)\in \mathcal{Z}\Big] 
\end{align*}
This concludes our proof because any randomized mapping
can be decomposed into a convex combination of measurable, deterministic functions \cite{Dwork}, and as Lemma \ref{lem:convexity} shows, a convex combination of $(\alpha,\calG)$-\name private mechanisms is also $(\alpha,\calG)$-\name private. 
\end{proof}

\begin{thm}[Sequential Composition] \label{theorem:seq}
If $\calA_1$ and $\calA_2$ are $(\alpha_1, \calG)$- and $(\alpha_2, \calG)$-\name private mechanisms, respectively, that use independent randomness, then releasing the outputs $\big( \calA_1(\by), \calA_2(\by) \big)$ satisfies $(\alpha_1+\alpha_2, \calG)$-\name privacy. 
\end{thm}
\begin{proof}
We have that $\calA_1: \calY^n \rightarrow \mathcal{V}'$ and $\calA_1: \calY^n \rightarrow \mathcal{V}''$ each satisfy \name-privacy for different $\alpha$ values. Let $\calA:\calY^n \rightarrow (\mathcal{V}' \times \mathcal{V}'')$ output $\big(\calA_1(\by), \calA_2(\by)\big)$. Then, we may write any event $\mathcal{Z} \in (\mathcal{V}' \times \mathcal{V}'')$ as $\mathcal{Z}' \times \mathcal{Z}''$, where $\mathcal{Z}' \in \mathcal{V}'$ and $\mathcal{Z}'' \in \mathcal{V}''$. We have for any $(\sigma, \sigma') \in \mathrm{N}_\calG$, 
\begin{align*}
    \mathrm{Pr} \big[ \calA\big( \sigma(\by) \big)  &\in \mathcal{Z} \big]  
    = \mathrm{Pr} \big[ \big(\calA_1\big( \sigma(\by) \big) , \calA_2\big( \sigma(\by) \big) \big) \in \mathcal{Z} \big] \\
    &= \mathrm{Pr} \big[ \{\calA_1\big( \sigma(\by) \big)  \in \mathcal{Z}'\} \cap \{\calA_2\big( \sigma(\by) \big)   \in \mathcal{Z}'' \} \big]  \\
    &= \mathrm{Pr} \big[ \{\calA_1\big( \sigma(\by) \big)  \in \mathcal{Z}'\} \big] 
    \mathrm{Pr} \big[ \{\calA_2\big( \sigma(\by) \big)   \in \mathcal{Z}'' \} \big] \\
    &\leq e^{\alpha_1 + \alpha_2}
    \mathrm{Pr} \big[ \{\calA_1\big( \sigma'(\by) \big)  \in \mathcal{Z}'\} \big] 
    \mathrm{Pr} \big[ \{\calA_2\big( \sigma'(\by) \big)   \in \mathcal{Z}'' \} \big] \\
    &= e^{\alpha_1 + \alpha_2} \cdot 
    \mathrm{Pr} \big[ \calA\big( \sigma'(\by) \big)  \in \mathcal{Z} \big] 
\end{align*}
\end{proof}

% Proof of Lemma \ref{lemma:LDP} \\

% \textbf{Lemma \ref{lemma:LDP}} \emph{
% An $\epsilon$-\ldp mechanism is $(k\epsilon, \calG)$-\name~ private for any group assignment $\calG$ such that $
%         k \geq \max_{G_i \in \calG} |G_i|
% $
% }
% \begin{proof}
% This follows from $k$-group privacy \cite{Dwork}. $\by$ are $\varepsilon$-LDP outputs $\calA_{\text{LDP}}(\bx)$ from input sequence $\bx$. For any $\sigma \approx_{G_i} \sigma'$, we know by definition that $\sigma(j) = \sigma'(j)$ for all $j \notin G_i$. As such, the permuted sequences $\sigma(\bx)_j = \sigma'(\bx)_j$ for all $j \notin G_i$, and differ in at most $|G_i|$ entries. In other words, 
% \begin{align*}
%     \textswab{d}_H \big(\sigma(\bx), \sigma'(\bx)\big) \leq |G_i| 
% \end{align*}
% Using this fact, we have from the $k$-group property of LDP that 
% \begin{align*}
%     \mathrm{Pr}\big[ \calA_{\text{LDP}} \big( \sigma(\bx) \big) \in O \big] 
%     \leq e^{|G_i| \epsilon} \mathrm{Pr}\big[ \calA_{\text{LDP}} \big( \sigma'(\bx) \big) \in O \big] 
% \end{align*}
% and thus if $k \geq \max_{G_i \in \calG} |G_i|$, 
% \begin{align*}
%     \mathrm{Pr}\big[ \calA_{\text{LDP}} \big( \sigma(\bx) \big) \in O \big] 
%     \leq e^{k \epsilon} \mathrm{Pr}\big[ \calA_{\text{LDP}} \big( \sigma'(\bx) \big) \in O \big] 
% \end{align*}
% for all $(\sigma, \sigma') \in \mathrm{N}_\calG$. 
% \end{proof}

\subsection{Proof for Thm. \ref{thm: semantic guarantee}}\label{app:thm:semantic}
\label{app:bayesian proof} 
\textbf{Theorem \ref{thm: semantic guarantee}} 
\emph{
For a given group assignment $\calG$ on a set of $n$ data owners, if a shuffling mechanism $\calA:\calY^n\mapsto \calY^n$ is $(\alpha,\calG)$-\name private, then for each data owner $\DO_i, i \in [n]$, %\vspace{-0.1cm}
\begin{align*}
   \max_{\substack{i\in [n]\\ a,b \in \calX}} \bigg|\log \frac{\Pr_\calP [x_i = a | \bz, \{y_{G_i}\},\by_{\overline{G}_i}]}{\Pr_\calP [x_i = b | \bz, \{y_{G_i}\},\by_{\overline{G}_i}]} - \log \frac{\Pr_\calP [x_i = a | \{y_{G_i}\},\by_{\overline{G}_i}]}{\Pr_\calP [x_i = b | \{y_{G_i}\},\by_{\overline{G}_i}]} \bigg| \leq \alpha  %\vspace{-0.5cm}
\end{align*}
for a prior distribution $\calP$, where \scalebox{0.9}{$\bz=\calA(\by)$} and \scalebox{0.9}{$\by_{\overline{G}_i}$} is the noisy sequence for data owners outside \scalebox{0.9}{$G_i$}.
}
\begin{proof}
We prove the above by bounding the following equivalent expression for any $i \in [n]$ and $a, b \in \calX$. 
\begin{align*}
     &\frac{\Pr_\calP[\bz | x_i=a, \{y_{G_i}\}, \by_{\overline{G}_i}]}{\Pr_\calP [\bz | x_i=b, \{y_{G_i}\}, \by_{\overline{G}_i}]}\\
     &= \frac{\int \Pr_\calP [\by | x_i = a, \{y_{G_i}\}, \by_{\overline{G}_i}] \Pr_\calA[\bz | \by] d\by}{\int \Pr_\calP [\by | x_i = b, \{y_{G_i}\}, \by_{\overline{G}_i}] \Pr_\calA[\bz | \by] d\by} \\
     &= \frac{\sum_{\sigma \in \mathrm{S}_r} \Pr_\calP[ \sigma(\by_{G_i}^*) | x_i = a, \by_{\overline{G}_i}] \Pr_\calA [\bz | \sigma(\by_{G_i}^*), \by_{\overline{G}_i}] }
     {\sum_{\sigma \in \mathrm{S}_r} \Pr_\calP[ \sigma(\by_{G_i}^*) | x_i = b, \by_{\overline{G}_i}] \Pr_\calA [\bz | \sigma(\by_{G_i}^*), \by_{\overline{G}_i}]} \\
     &\leq \max_{\{ \sigma, \sigma' \in \mathrm{S}_r\}}
     \frac{ \Pr_\calA [\bz | \sigma(\by_{G_i}^*), \by_{\overline{G}_i}]}{ \Pr_\calA [\bz | \sigma'(\by_{G_i}^*), \by_{\overline{G}_i}]}  \\
     &\leq \max_{\{ \sigma, \sigma' \in \mathrm{N}_{G_i}\}}
     \frac{\Pr_\calA[\bz | \sigma(\by)]}{\Pr_\calA[\bz | \sigma'(\by)]}  \\
     &\leq e^\alpha 
\end{align*}
The second line simply marginalizes out the full noisy sequence $\by$. The third line reduces this to a sum over permutations of of $\by_{G_i}$, where $r = |G_i|$ and $\by^*_{G_i}$ is any fixed permutation of values $\{y_{G_i}\}$. This is possible since we are given the values outside the group, $\by_{\overline{G}_i}$, and the unordered set of values inside the group, $\{y_{G_i}\}$. Note that the permutations $\sigma$ written here are possible permutations of the \ldp input, not permutations output by the mechanism applied to the input as sometimes written in other parts of this document. 

The fourth line uses the fact that the numerator and denominator are both convex combinations of $\Pr_\calA [\bz | \sigma(\by_{G_i}^*), \by_{\overline{G}_i}]$ over all $\sigma \in \mathrm{S}_r$. 

The fifth line uses the fact that for any $\by_{\overline{G}_i}$, $$(\sigma(\by_{G_i}^*), \by_{\overline{G}_i}) \approx_{G_i} (\sigma'(\by_{G_i}^*), \by_{\overline{G}_i}) \ . $$
This allows a further upper bound over all neighboring sequences w.r.t. $G_i$, and thus over any permutation of $\by_{\overline{G}_i}$, as long as it is the same in the numerator and denominator. 
\end{proof}

\paragraph{Discussion}
The above Bayesian analysis measures what can be learned about $\DO_i$'s $x_i$ from observing the private release $\bz$ relative to some other known information (the conditioned information). 
%With \ldp alone, we condition on every other data owner's private value $x_j$. This implies that releasing the private sequence $\by$ cannot provide much more information about $x_i$ than releasing every other $\DO_j$'s $x_j$ would. So, only modest information unique to $x_i$ can be garnered by any Bayesian adversary. For Alice, this may be a concern, since making inferences on her disease state from those of her household is indeed a privacy violation. 
Under \name-privacy, we condition on the bag of \ldp values in Alice's group $\{y_{G_i}\}$ as well as the sequence (order and value) of $\ldp$ values outside her group $\by_{\overline{G_i}}$. This implies that releasing the shuffled sequence $\bz$ cannot provide much more information about Alice's $x_i$ than would releasing the \ldp values outside her neighborhood (her group) and the unordered bag of \ldp values inside her neighborhood, regardless of the adversary's prior knowledge $\calP$. This is a communicable guarantee: if Alice feels comfortable with the data collection knowing that her entire neighborhood's responses will be uniformly shuffled together (including those of her household), then she ought to be comfortable with \name-privacy. Now, we have to provide this guarantee to Bob, a neighbor of Alice, as well as Luis, a neighbor of Bob but \textit{not} of Alice. Thus, Bob, Alice and Luis have \textit{distinct} and \textit{overlapping} groups (neighborhoods). Hence, the trivial solution of uniformly shuffling the noisy responses of every group separately does not work in this case. % Of course, we could instead simply shuffle Alice's neighborhood uniformly, but we could not do this for each data owner's group while still maintaining utility -- the analyst can still estimate disease prevalence within neighborhoods and districts. With overlapping groups this may require shuffling the entire dataset uniformly. 
\name-privacy, however, offers the above guarantee to each user (knowing that their entire neighborhood is \textit{nearly} uniformly shuffled) while still maintaining utility (estimate disease prevalence within neighborhoods). Semantically, this is very powerful, since it implies that the noisy responses specific to one's household cannot be leveraged to infer one's disease state $x_i$. %The adversary can learn about $x_i$ from the disease prevalence outside Alice's neighborhood and, on average, inside her neighborhood, but not much beyond that. 

% For a given group assignment $\calG$ on a set of $n$ data owners, if a shuffling mechanism $\calA:\calY^n\mapsto \calY^n$ is $(\alpha,\calG)$-\name private, then for each data owner $\DO_i, i \in [n]$, we have 
% \begin{align}
%     \mathbb{L}^{\sigma}_{\mathcal{P}}(\bx)
%     =\max_{\substack{i\in [n]\\ a,b \in \calX}}\Big|\log \frac{\Pr_\calP[\bz | x_i=a, \{y_{G_i}\}]}{\Pr_\calP [\bz | x_i=b, \{y_{G_i}\}]} \Big| 
%     \leq \alpha  
% \end{align}
% for any prior distribution $\calP$ such that $(x_{i},\{y_{G_i}\}) \perp \!\!\! \perp \by_{\overline{G}_i}|\calP$ where $\bz=\calA(\by)$ and $\by_{\overline{G}_i}$ is the (ordered and noisy) data sequence for all data owners outside $G_i$. 

% \begin{align*}
%      &\frac{\Pr_\calP[\bz | x_i=a, \{y_{G_i}\}]}{\Pr_\calP [\bz | x_i=b, \{y_{G_i}\}]}\\
%      &= \frac{\int \Pr_\calP [\by | x_i = a, \{y_{G_i}\}] \Pr_\calA[\bz | \by] d\by}{\int \Pr_\calP [\by | x_i = b, \{y_{G_i}\}] \Pr_\calA[\bz | \by] d\by} \\
%      &= \frac{\int \textcolor{red}{\Pr_\calP [\by_{\overline{G}_i} | x_i = a, \by_{G_i}]} 
%      \Pr_\calP [\by_{G_i} | x_i = a, \{y_{G_i}\}] 
%      \Pr_\calA[\bz | \by] d\by}
%      {\int \textcolor{red}{\Pr_\calP [\by_{\overline{G}_i} | x_i = b, \by_{G_i}]}
%      \Pr_\calP [\by_{G_i} | x_i = b, \{y_{G_i}\}]
%      \Pr_\calA[\bz | \by] d\by} \\
%      &= \frac{\int  \textcolor{red}{\Pr_\calP [\by_{\overline{G}_i} ]} \sum_{\sigma \in \mathrm{S}_m} \ \Pr_\calP [\by_{G_i} = \sigma(\{y_{G_i}\}) | x_i = a]  \ \Pr_\calA[\bz | \by] d\by_{\overline{G}_i}}
%      {\int  \textcolor{red}{\Pr_\calP [\by_{\overline{G}_i} ]} \sum_{\sigma \in \mathrm{S}_m} \ \Pr_\calP [\by_{G_i} = \sigma(\{y_{G_i}\}) | x_i = b]  \ \Pr_\calA[\bz | \by]  d\by_{\overline{G}_i}} \\
%      &\leq \sup_{ \by_{\overline{G}_i}} \frac{\sum_{\sigma \in \mathrm{S}_m} \ \Pr_\calP [\by_{G_i} = \sigma(\{y_{G_i}\}) | x_i = a]  \ \Pr_\calA[\bz | \by]}
%      {\sum_{\sigma \in \mathrm{S}_m} \ \Pr_\calP [\by_{G_i} = \sigma(\{y_{G_i}\}) | x_i = b]  \ \Pr_\calA[\bz | \by]} \\
%      &\leq \max_{\{ \sigma(\by), \sigma'(\by) : \sigma, \sigma' \in N_{G_i}\}}
%      \frac{\Pr_\calA[\bz | \sigma(\by)]}{\Pr_\calA[\bz | \sigma'(\by)]} \leq e^\alpha 
% \end{align*}

% The terms in red highlight the step where we can play with independence. The simplest assumption to add is that $\bx_{G_i} \perp \bx_{\overline{G}_i} | \calP$. We have to change what we currently have -- I had forgotten why I had changed it previously. This is close, though. 

% It's straightforward to see how we could also just condition on knowing $\by_{\overline{G}_i}$, and that would take care of the red terms. 

\subsection{Proof of Theorem \ref{thm: decision theoretic}} 

% \textbf{Theorem \ref{thm: decision theoretic}} \emph{
%   For $\mathcal{A}(\mathcal{M}(\bx))=\bz$ where $\mathcal{M}(\cdot)$ is $\epsilon$-\ldp and $\mathcal{A}(\cdot)$ is $\alpha$ - \name private, we have  
% %  \begin{align*}
% %      \Pr[\calA(\bx) = \bz, \sigma(H) \cap I < l]
% %      \geq \beta(r,k, l) e^{-(2k \epsilon + \alpha)} \Pr[\calA(\bx) = \bz, \sigma(I) \geq H]
% %  \end{align*}
%  \begin{align*}
%      \Pr[\mathcal{D}_{Adv} \text{ loses}] \geq \binom{r-k}{k} e^{-(2k\epsilon+\alpha)} \cdot \Pr[\mathcal{D}_{Adv} \text{ wins}]
%  \end{align*}
%  for any input subgroup $I \subset G_i, r = |G_i|$ and  $k < r/2$. 
%  }
 
%  \begin{proof}
%  We first focus on deterministic adversaries and then expand to randomized adversaries afterwards using the fact that randomized adversaries are mixtures of deterministic ones. 

% Our adversary $\mathcal{D}_{Adv}$ is then defined by a deterministic decision function $\eta: \calY^n \rightarrow [n]^k$. Upon observing $\bz$, $\eta(\bz)$ selects $k$ indices in $\bz$ which it believes originated from $I \subset G_i$. 
 
%  In the following, let $\Pr_{\bz}$ be the probability of events conditioned on the shuffled output sequence $\bz$, where randomness is over the $\epsilon$-\ldp mechanism $\mathcal{M}$ and the $\alpha$-\name-private shuffling mechanism $\calA$. \footnote{As an abuse of notation, we assume the output space of the \ldp randomizers, $\calY$, have outcomes with non-zero measure e.g. randomized response. The following analysis can be expanded to continuous outputs (with outcomes of zero measure) by simply replacing the output sequence $\bz \in \calY^n$ with an output event $\mathbf{Z} \subseteq \calY^n$.}
 
%  The adversary wins if it reidentifies all of the \ldp values originating from $I$. Letting $H = \eta(\bz)$ be the indices of elements in $\bz$ selected by $\eta$, we have that 
%  \begin{align*}
%      \Pr_{\bz} [\eta(\bz) \text{ wins}] = \Pr_{\bz} [\sigma(H) = I]
%  \end{align*}
%  where $\sigma$ is the shuffling permutation produced by $\calA$, $\bz = \sigma(\by)$ i.e. $z_i = y_{\sigma(i)}$. Concretely, this is equivalent to $\DO_i$ releasing $\DO_{\sigma(i)}$'s \ldp response. Since the permutation and \ldp outptus are randomized, there remains a significant probability that $\sigma(H) \cap I = \emptyset$ i.e. not a single element in $H$ originated from $I$ and $\calA (\mathcal{M}(\bx))$ still output the sequence $\bz$. 
 
%  We may rewrite the above probability by marginalizing out all possible $\ldp$ sequences $\by$. Conditioning on the output sequence $\bz$, the only possible \ldp sequences $\by$ are permutations of $\bz$. Note that the probability of any sequence $\by$ is determined by the input $\bx$ and the \ldp mechanism $\mathcal{M}$:  
%  \begin{align*}
%      \Pr_{\bz} [\sigma(H) = I]
%      &= \sum_{\sigma \in \mathrm{S}^n}
%      \Pr [\calA(\bx) = \by = \sigma^{-1}(\bz)]  \Pr [ \sigma | \by ] \Pr[\sigma(H) = I  | \by, \sigma] / \Pr[\bz] \\
%      &= \sum_{\sigma \in \mathrm{S}^n}
%      \Pr [\calA(\bx) = \by = \sigma^{-1}(\bz)]  \Pr [ \sigma | \by ] \mathbf{1}\{ \sigma(H) = I  \} / \Pr[\bz]
%  \end{align*}
%  Note that $\Pr_\bz [\sigma | \by] = \Pr_\bz [\sigma]$ for the mallows mechanism, which chooses its permutations independently of $\by$. Now consider when $\eta(\bz)$ loses. By similar arguments as above: 
% \begin{align*}
%     \Pr_\bz[\eta(\bz) \text{ loses}]
%     &= \Pr_\bz[\sigma(\bz) \cap I = \emptyset] \\
%     &= \sum_{\sigma \in \mathrm{S}^n}
%      \Pr [\calA(\bx) = \by = \sigma^{-1}(\bz)]  \Pr [ \sigma | \by ] \mathbf{1}\{ \sigma(H) \cap I = \emptyset  \} / \Pr[\bz]
% \end{align*}
% % Now, consider the set of permutations $\sigma$ such that $\eta$ reidentifies exactly $s$ \ldp responses from subgroup $I$. Let 
% % \begin{align*}
% %     C(s) &= \{ \sigma : | \sigma(H) \cap I | = s \} \\
% %     C_\emptyset &= \{ \sigma : \sigma(H) \cap I = \emptyset \} 
% % \end{align*}
% % Our argument centers on the following lemma

% Proof of the main theorem follows from the simple fact that there are many more permutations where $\eta(\bz)$ loses and they are close in probability to those where $\eta(\bz)$ wins. 

% \begin{lemma}
% Let $C = \{\sigma \in \mathrm{S}^n : \sigma(H) = I\}$. For every $\sigma \in C$, we may construct a set of $\binom{r-k}{k}$ permutations $E(\sigma)$ such that
% \begin{enumerate}
%     \item $E(\sigma_a) \cap E(\sigma_b) = \emptyset$ for any pair of permutations $\sigma_a, \sigma_b \in C$
%     \item $\sigma^{'-1} \approx_{G_i} \sigma^{-1}$ for all $\sigma' \in E(\sigma)$ 
%     \item $\Pr [\calA(\bx) = \by = \sigma^{-1}(\bz)] \leq e^{2k\epsilon} \Pr [\calA(\bx) = \by = \sigma^{'-1}(\bz)]$ for any $\sigma' \in E(\sigma)$
% \end{enumerate}
% \end{lemma}
% \begin{proof}
% For any permutation $\sigma \in C$, we construct $E(\sigma)$ by first taking its inverse permutation $\sigma^{-1}$. We know that $\sigma^{-1}(I) = H$ by definition. We then construct the permutations of $E(\sigma)$ by swapping the elements $\sigma^{-1}(I)$ with the elements $\sigma^{-1}(J)$ for any subset of $k$ elements outside $I$ but still in $G_i$, $J \subset G_i : J \cap I = \emptyset$, while preserving the relative ordering within $\sigma^{-1}(I)$ and within $\sigma^{-1}(J)$. There are $\binom{r-k}{k}$ such subsets. 

% On the first point, we know that no one permutation could be in both $E(\sigma_a)$ and in $E(\sigma_b)$ since the above operation is reversible (swap the elements $H$ back into position $I$ while preserving order). 

% On the second point, we only swapped elements within group $G_i$ to determine $\sigma^{'-1}$ from $\sigma^{-1}$, so by definition they are neighboring on $G_i$. 

% On the third point, the sequence $\by = \sigma^{-1}(\bz)$ only differs from $\sigma^{'-1}(\bz)$ on at most $2k$ indices.  
% \end{proof}

% Using the above lemma we have that 
% \begin{align*}
%     \frac{\Pr_\bz[\eta(\bz) \text{ loses}]}{\Pr_\bz[\eta(\bz) \text{ wins}]}
%     &\geq \frac{ \sum_{\sigma \in \mathrm{S}^n} \sum_{\sigma' \in E(\sigma)} \Pr [\calA(\bx) = \by = \sigma^{'-1}(\bz)]  \Pr [ \sigma' | \by ] }
%     { \sum_{\sigma \in \mathrm{S}^n} \Pr [\calA(\bx) = \by = \sigma^{-1}(\bz)]  \Pr [ \sigma | \by ] } \\
%     &\geq \max_{\sigma \in \mathrm{S}^n} \frac{\sum_{\sigma' \in E(\sigma)} \Pr [\calA(\bx) = \by = \sigma^{'-1}(\bz)]  \Pr [ \sigma' | \by ]}
%     {\Pr [\calA(\bx) = \by = \sigma^{-1}(\bz)]  \Pr [ \sigma | \by ]} \\
%     &\geq \binom{r-k}{k}e^{-(2k\epsilon + \alpha)}
% \end{align*}
% Where the final line comes from the fact 3 of the Lemma and from the fact that $\sigma^{-1}(\by)$ and $\sigma^{'-1}(\by)$ are neighboring (fact 2 of the Lemma), so the probability of the \name-private shuffler producing the permutations needed to achieve $\bz$, $\sigma$ and $\sigma'$, are close by a factor of $e^-\alpha$. 

% Since this holds conditionally for each $\bz$, it holds marginally across all $\bz$. Furthermore, we may state any randomized adversary as a mixture of deterministic adversaries $\eta(\bz)$, preserving the above bound. This completes the proof. 

% \end{proof} 

\textbf{Theorem \ref{thm: decision theoretic}} 

\emph{
  For $\mathcal{A}(\mathcal{M}(\bx))=\bz$ where $\mathcal{M}(\cdot)$ is $\epsilon$-\ldp and $\mathcal{A}(\cdot)$ is $\alpha$ - \name private, we have  
 \begin{align*}
     \Pr[\mathcal{D}_{Adv} \text{ loses}] \geq \lfloor \frac{r-k}{k} \rfloor e^{-(2k\epsilon+\alpha)} \cdot \Pr[\mathcal{D}_{Adv} \text{ wins}]
 \end{align*}
 for any input subgroup $I \subset G_i, r = |G_i|$ and  $k < r/2$. 
 }
 
 \begin{proof}
 
  We first focus on deterministic adversaries and then expand to randomized adversaries afterwards using the fact that randomized adversaries are mixtures of deterministic ones. 

Our adversary $\mathcal{D}_{Adv}$ is then defined by a deterministic decision function $\eta: \calY^n \rightarrow [n]^k$. Upon observing $\bz$, $\eta(\bz)$ selects $k$ elements in $\bz$ which it believes originated from $I \subset G_i$. 
 
 In the following, let $\Pr_{\bz}$ be the probability of events conditioned on the shuffled output sequence $\bz$, where randomness is over the $\epsilon$-\ldp mechanism $\mathcal{M}$ and the $\alpha$-\name-private shuffling mechanism $\calA$. \footnote{As an abuse of notation, we assume the output space of the \ldp randomizers, $\calY$, have outcomes with non-zero measure e.g. randomized response. The following analysis can be expanded to continuous outputs (with outcomes of zero measure) by simply replacing the output sequence $\bz \in \calY^n$ with an output event $\mathbf{Z} \subseteq \calY^n$.}
 
 The adversary wins if it reidentifies $> \frac{k}{2}$ of the \ldp values originating from $I$. Let $H = \eta(\bz)$ be the indices of elements in $\bz$ selected by $\eta$. Let $W = \{\sigma \in \mathrm{S}^n : |\sigma(H) \cap I| > \frac{k}{2}\}$ be the set of permutations where the adversary wins and let $L = \{\sigma \in \mathrm{S}^n :\sigma(H) \cap I| \leq \frac{k}{2}\} $ be the set of permutations where the adversary loses. 
 \begin{align*}
     \Pr_{\bz} [\eta(\bz) \text{ wins}] &= \Pr_{\bz} [\sigma \in W] \\
     \Pr_{\bz} [\eta(\bz) \text{ loses}] &= \Pr_{\bz} [\sigma \in L] 
 \end{align*}
 where $\sigma$ is the shuffling permutation produced by $\calA$, $\bz = \sigma(\by)$ i.e. $z_i = y_{\sigma(i)}$. Concretely, this is equivalent to $\DO_i$ releasing $\DO_{\sigma(i)}$'s \ldp response. Since the permutation and \ldp outputs are randomized, many subgroups of size $k$ in $G_i$ could have produced the \ldp values $(z_{H_1}, \dots, z_{H_k})$ and then been mapped to $H$ by a permutation. Concretely, there is a reasonable probability that Alice's household output the \ldp values of another $k$-member household in her neighborhood and they output her household's \ldp values. In the worst case, this is $e^{-2k \epsilon}$ less likely than without swapping values, by group \DP guarantees. Since both households are part of the same group $G_i$, the permutation that maps her household to elements $H$ in the output is close in probability to that which maps the other household to elements $H$ in the output. As such, we have in the worst case a $e^{-(2k\epsilon + \alpha)}$ reduction in probability of the other household having swapped \ldp values with Alice's and permuting to subset $H$. 
 
 The above provides intuition on how we could get the same output $\bz$ many different ways, and how Alice's household could or could not contribute to elements $H$. It does not, however, explain why an adversary who is given output $\bz$ has limited advantage in choosing a subset $H$ such that they recover \emph{most} of Alice's household's values. We formalize this fact as follows. 
 
 We may rewrite the probabilities of winning or losing by marginalizing out all possible $\ldp$ sequences $\by$. Conditioning on the output sequence $\bz$, the only possible \ldp sequences $\by$ are permutations of $\bz$. Note that the probability of any sequence $\by$ is determined by the input $\bx$ and the \ldp mechanism $\mathcal{M}$:  
 \begin{align*}
    \Pr_\bz[\eta(\bz) \text{ loses}]
     &= \Pr_{\bz} [\sigma \in W] \\
     &= \sum_{\sigma \in W}
     \Pr [\calA(\bx) = \by = \sigma^{-1}(\bz)]  \Pr [ \sigma | \by ] / \Pr[\bz] 
 \end{align*}
 Note that $\Pr_\bz [\sigma | \by] = \Pr_\bz [\sigma]$ for the mallows mechanism, which chooses its permutations independently of $\by$. Now consider when $\eta(\bz)$ loses. By similar arguments as above: 
\begin{align*}
    \Pr_\bz[\eta(\bz) \text{ loses}]
    &= \Pr_\bz[\sigma \in L] \\
    &= \sum_{\sigma \in L}
     \Pr [\calA(\bx) = \by = \sigma^{-1}(\bz)]  \Pr [ \sigma | \by ] / \Pr[\bz]
\end{align*}
The odds of losing versus winning is given by 
 \begin{align*}
    \frac{\Pr_\bz[\eta(\bz) \text{ loses}]}{\Pr_\bz[\eta(\bz) \text{ wins}]}
    &= \frac{ \sum_{\sigma' \in L} \Pr [\calA(\bx) = \by = \sigma^{'-1}(\bz)]  \Pr [ \sigma' | \by ] }
    { \sum_{\sigma \in W} \Pr [\calA(\bx) = \by = \sigma^{-1}(\bz)]  \Pr [ \sigma | \by ] } \\
\end{align*}
We now show that for each $\sigma$ in the denominator, we may construct $m = \lfloor \frac{r-k}{k} \rfloor$ distinct permutations $\sigma'$ in the numerator that are close in probability to it. 
\begin{lemma}
For every $\sigma \in W$ there exists a set of $m = \lfloor \frac{r-k}{k} \rfloor$ permutations, $E(\sigma)$, such that 
\begin{enumerate}
    \item $E(\sigma) \subseteq L$
    \item $\sigma^{-1} \approx_{G_i} \sigma^{'-1}$ 
    \item $E(\sigma_a) \cap E(\sigma_b) = \emptyset$ for any pair $\sigma_a, \sigma_b \in W$
    \item $\Pr [\calA(\bx) = \by = \sigma^{-1}(\bz)] \leq e^{2k\epsilon} \Pr [\calA(\bx) = \by = \sigma^{'-1}(\bz)]$ for any $\bx \in \calX^n$ and any $\bz \in \calY^n$
\end{enumerate}
\end{lemma}
\begin{proof}
Given $\sigma \in W$, we construct $E(\sigma)$ by first taking the inverse $\sigma^{-1}$. Recall that, since $\sigma \in W$, we have that $|\sigma^{-1}(I) \cap H| > \frac{k}{2}$. ($\sigma^{-1}(i) = j$ could be interpreted as data owner $i$'s \ldp value will be output at position $j$). We then divide the remainder of the group $G_i \backslash I$ into $m$ disjoint subsets of size $k$ each, $J_1, J_2, \dots, J_m$. These represent the other distinct subsets of size $k$ that Alice's household could swap \ldp values with. We then produce $m$ permutations, $\sigma^{'-1}_1, \dots, \sigma^{'-1}_m$, by making $\sigma^{'-1}_i(I) = \sigma^{-1}(J_i)$ and $\sigma^{'-1}_i(J_i) = \sigma^{-1}(I)$ (preserving order within those subsets) and $\sigma^{'-1} = \sigma^{-1}$ everywhere else. 

On the first point, we know that every $\sigma' \in E(\sigma)$ is also in $L$. We know this because $\sigma^{'-1}_i(I) = \sigma^{-1}(J_i)$. Since $\sigma \in W$, we have that $|\sigma^{-1}(J_i) \cap H| < \frac{k}{2}$ since $|\sigma^{-1}(I) \cap H| \geq \frac{k}{2}$ and $I \cap J_i = \emptyset$ by definition. Thus, $|\sigma^{'-1}_i(I) \cap H| < \frac{k}{2}$, so $|\sigma'_i(H) \cap I| < \frac{k}{2}$ and $\sigma'_i \in L$. 

On the second point, we know that the inverse permutations are neighboring $\sigma^{-1} \approx_{G_i} \sigma^{'-1}$ simply by construction -- they only differ on elements in $G_i$. 

On the third point, we know that the sets $E(\sigma_a)$ and $E(\sigma_b)$ are distinct since we can map any permutation $\sigma' \in E(\sigma_a)$ uniquely back to $\sigma_a$ for any $\sigma_a \in W$. We do so by taking its inverse $\sigma^{'-1}$, finding which subset $J_i$ has majority elements from $H$ i.e. $|\sigma^{'-1}(J_i) \cap H| > \frac{k}{2}$. Swap elements back: $\sigma^{'-1}(J_i)$ with $\sigma^{'-1}(I)$. Invert back to $\sigma_a$. 

On the fourth point, we know that $\sigma^{-1}(\bz)$ and $\sigma^{'-1}(\bz)$ differ on at most $2k$ indices. As such, by group \DP guarantees, we know that their probabilities must be close to a factor of $e^{-2k\epsilon}$ regardless of $\bz$ and $\bx$. 
\end{proof}

Using the above Lemma we may bound the odds of losing vs. winning. 

\begin{align*}
    \frac{\Pr_\bz[\eta(\bz) \text{ loses}]}{\Pr_\bz[\eta(\bz) \text{ wins}]}
    &= \frac{ \sum_{\sigma' \in L} \Pr [\calA(\bx) = \by = \sigma^{'-1}(\bz)]  \Pr [ \sigma' | \by ] }
    { \sum_{\sigma \in W} \Pr [\calA(\bx) = \by = \sigma^{-1}(\bz)]  \Pr [ \sigma | \by ] } \\
    &\geq \frac{\sum_{\sigma \in W} \sum_{\sigma' \in E(\sigma)} \Pr [\calA(\bx) = \by = \sigma^{'-1}(\bz)]  \Pr [ \sigma' | \by ]}
    {\sum_{\sigma \in W} \Pr [\calA(\bx) = \by = \sigma^{-1}(\bz)]  \Pr [ \sigma | \by ]} \\
    &\geq \min_{\sigma \in W} \frac{\sum_{\sigma' \in E(\sigma)} \Pr [\calA(\bx) = \by = \sigma^{'-1}(\bz)]  \Pr [ \sigma' | \by ]}{\Pr [\calA(\bx) = \by = \sigma^{-1}(\bz)]  \Pr [ \sigma | \by ]} \\
    &\geq \lfloor \frac{r-k}{k} \rfloor e^{-(2k\epsilon + \alpha)}
\end{align*}
where the last line follows from the fourth point of the above Lemma (for the $2k\epsilon$ term) and the fact that the inverse permutations $\sigma'^{-1}, \sigma^{-1}$ are neighboring (second point of the Lemma) so the probabilities of the mechanism to produce $\sigma$ vs. $\sigma'$ to reach $\bz$ from these neighboring permutations must be close by a factor of $e^{\alpha}$. 

Since the above holds for any $\bz$ and $\bx$, the bound holds on average across all outcomes $\bz$, thus 
\begin{align*}
    \Pr[\eta \text{ loses}] \geq \lfloor \frac{r-k}{k} \rfloor e^{-(2k\epsilon+\alpha)} \cdot \Pr[\eta \text{ wins}]
\end{align*}
for any deterministic adversary with decision function $\eta$. Finally, we may write any probabilistic adversary as mixture of decision functions. By convexity (same argument used in Lemma \ref{lem:convexity}), the above bound still holds. As such, 

\begin{align*}
    \Pr[\mathcal{D}_{Adv} \text{ loses}] \geq \lfloor \frac{r-k}{k} \rfloor e^{-(2k\epsilon+\alpha)} \cdot \Pr[\mathcal{D}_{Adv} \text{ wins}]
\end{align*}

 \end{proof}
 
 \subsection{Utility of Shuffling Mechanism}\label{app:utility}
 We now introduce a novel metric, $(\eta,\delta)$-preservation, for assessing the utility of any shuffling mechanism. Let $S\subseteq [n]$ correspond to a set of indices in $\by$. The metric is defined as follows.
%representing data owners in Alice's neighborhood for instance.  
%$(\eta,\delta)$-preservation measures how well the shuffling mechanism preserves the original indices in $S$ after shuffling, i.e. the fraction of data owners in Alice's neighborhood that still correspond to datapoints from the neighborhood after shuffling:
\begin{defn}($(\eta,\delta)$-preservation) A shuffling mechanism $\calA:\calY^n\mapsto\calY^n$ is defined to be $(\eta,\delta)$-preserving $(\eta, \delta 
\in [0,1])$ w.r.t to a given subset $S\subseteq [n]$, if \begin{gather}\Pr\big[|S_{\sigma}\cap S|\geq \eta\cdot|S|\big]\geq 1-\delta,  \sigma \in \mathrm{S}_n\end{gather} where $\bz=\calA(\by)=\sigma(\by)$ and $S_{\sigma}=\{\sigma(i)|i \in S\}$. \label{def:utility} 
% \vspace{-0.2cm}
\end{defn}
For example, consider \scalebox{0.9}{$S=\{1,4,5,7,8\}$}. If \scalebox{0.9}{$\calA(\cdot)$} permutes the output according to  \scalebox{0.9}{$\sigma=(\underline{5}\:3\:2\:\underline{6}\:\underline{7}\:9\:\underline{8}\:\underline{1}\:4\:10)$}, then  \scalebox{0.9}{$S_{\sigma}=\{5,6,7,8,1\}$}  which preserves \scalebox{0.9}{$4$} or \scalebox{0.9}{$80\%$} of its original indices.  This means that for any data sequence $\by$, at least \scalebox{0.9}{$\eta$} fraction of its data values corresponding to the subset \scalebox{0.9}{$S$} overlaps with that of shuffled sequence $\bz$ with high probability \scalebox{0.9}{$(1-\delta)$}. Assuming, \scalebox{0.9}{$\{y_S\}=\{y_{i}|i
\in S\}$} and \scalebox{0.9}{$\{z_S\}=\{z_i|i \in S\}=\{y_{\sigma(i)}| i \in S\}$} denotes the set of data values corresponding to $S$ in data sequences $\by$ and $\bz$ respectively, we  have \scalebox{0.9}{$\Pr\big[|\{y_S\}\cap \{z_S\}|\geq \eta \cdot |S|\big]\geq 1-\delta, \: \forall \by $}.
For example, let $S$ be the set of individuals from Nevada. Then, for a shuffling mechanism that provides \scalebox{0.9}{$(\eta =0.8, \delta=0.1)$}-preservation to $S$, with probability \scalebox{0.9}{$\geq 0.9$}, \scalebox{0.9}{$\geq 80\%$} of the values that are reported to be from Nevada in $\bz$ are genuinely from Nevada. The rationale behind this metric is that it captures the utility of the learning allowed by \name-privacy -- if $S$ is equal to some group \scalebox{0.9}{$G \in \calG$}, \scalebox{0.9}{$(\eta, \delta)$} preservation allows overall statistics of \scalebox{0.9}{$G$} to be captured. Note that this utility metric is \textit{agnostic of both the data distribution and the analyst's query}. Hence, it is a conservative analysis of utility which serves as a lower bound for learning from $\{z_S\}$. We suspect that with the knowledge of the data distribution and/or the query, a tighter utility analysis is possible. \\
A formal utility analysis of Alg. \ref{algo:main} is presented in App. \ref{app:utility:formal}. Empirical evaluation of $(\eta,\delta)$ - preservation is presented in App. \ref{app:extraresults}. 
\subsection{Discussion on Properties of Mallows Mechanism}\label{app:prop}

% \textbf{Property \ref{prop:1}} 
\begin{prope}
\label{prop:1}
% \emph{
For group assignment $\calG$, a  mechanism $\calA(\cdot)$ that shuffles according to a permutation sampled from the Mallows model $\mathbb{P}_{\theta,\textswab{d}}(\cdot)$, satisfies $(\alpha, \calG)$-\name privacy where
\begin{align*}
 \Delta(\sigma_0 : \textswab{d}, \calG) &= \max_{(\sigma, \sigma') \in N_\calG} |\textswab{d}(\sigma_0 \sigma, \sigma_0) - \textswab{d}(\sigma_0 \sigma', \sigma_0)|\\
 \text{and}&\\
    \alpha 
    &= \theta \cdot \Delta(\sigma_0 : \textswab{d}, \calG)
\end{align*} 
We refer to $\Delta(\sigma_0 : \textswab{d}, \calG) $ as the sensitivity of the rank-distance measure $\textswab{d}(\cdot)$
\end{prope}
% }
\begin{proof}
% Start with a useful lemma. 
% \begin{lemma}
% Consider any pair of neighboring permutations $\sigma \approx_{G_i} \sigma'$ w.r.t. group $G_i$. Consider any third permutation $\sigma^* \in \mathrm{S}_n$. Then, the permutations that turn $\sigma$ and $\sigma'$ into $\sigma^*$ are also neighboring w.r.t. $G_i$. Formally, if 
% \begin{align*}
%     \sigma^*(\by) &= \sigma_a \big( \sigma(\by) \big) \\
%     \sigma^*(\by) &= \sigma_b \big( \sigma'(\by) \big)  
% \end{align*}
% then, $\sigma_a \approx_{G_i} \sigma_b$.

Consider two permutations of the initial sequence $\by$, $\sigma_1(\by), \sigma_2(\by)$ that are neighboring w.r.t. some group $G_i \in \calG$, $\sigma_1 \approx_{G_i} \sigma_2$. Additionally consider any fixed released shuffled sequence $\bz$. Let $\Sigma_1, \Sigma_2$ be the set of permutations that turn $\sigma_1(\by), \sigma_2(\by)$ into $\bz$, respectively: 
\begin{align*}
    \Sigma_1 
    & = \{\sigma \in \mathrm{S}_n : \sigma \sigma_1(\by) = \bz \} \\
    \Sigma_2 
    & = \{\sigma \in \mathrm{S}_n : \sigma \sigma_2(\by) = \bz \} \quad .
\end{align*}
In the case that $\{y\}$ consists entirely of unique values, $\Sigma_1, \Sigma_2$ will each contain exactly one permutation, since only one permutation can map $\sigma_i(\by)$ to $\bz$. 

\begin{lemma}
For each permutation $\sigma_1' \in \Sigma_1$ there exists a permutation in $\sigma_2' \in \Sigma_2$ such that 
\begin{align*}
    \sigma_1' \approx_{G_i} \sigma_2' \quad . 
\end{align*}
\end{lemma}
Proof follows from the fact that --- since only the elements $j \in G_i$ differ in $\sigma_1(\by)$ and $\sigma_2(\by)$ --- only those elements need to differ to achieve the same output permutation. In other words, we may define $\sigma_1', \sigma_2'$ at all inputs $i \notin G_i$ identically, and then define all inputs $i \in G_i$ differently as needed. As such, they are neighboring w.r.t. $G_i$. 

Recalling that Alg. 1 applies $\sigma_0^{-1}$ to the sampled permutation, we must sample $\sigma_0\sigma_1'$ (for some $\sigma_1' \in \Sigma_1$) for the mechanism to produce $\bz$ from $\sigma_1(\by)$. Formally, since $\sigma_1' \sigma_1 (\by) = \bz$ we must sample $\sigma_0 \sigma_1'$ to get $\bz$ since we are going to apply $\sigma_0^{-1}$ to the sampled permutation. 
\begin{align*}
    \Pr\big[ \calA \big( \sigma_1(\by) \big) = \bz \big] 
    &= \mathbb{P}_{\theta,\textswab{d}}\big(\sigma_0\sigma', \sigma' \in \Sigma_1 : \sigma_0\big) \\
    \Pr\big[ \calA \big( \sigma_2(\by) \big) = \bz \big] 
    &= \mathbb{P}_{\theta,\textswab{d}}\big(\sigma_0\sigma', \sigma' \in \Sigma_2 : \sigma_0\big) 
\end{align*}

Taking the odds, we have
\begin{align*}
     \frac{\mathbb{P}_{\theta,\textswab{d}}\big(\sigma_0\sigma', \sigma' \in \Sigma_1 : \sigma_0\big)}{
     \mathbb{P}_{\theta,\textswab{d}}\big(\sigma_0\sigma'', \sigma'' \in \Sigma_2 : \sigma_0\big)} 
    &= \frac{\sum_{\sigma' \in \Sigma_1}\mathbb{P}_{\Theta,\textswab{d}}(\sigma_0\sigma' : \sigma_0)}{
    \sum_{\sigma'' \in \Sigma_2}\mathbb{P}_{\Theta,\textswab{d}}(\sigma_0\sigma'' : \sigma_0)} \\
    &=\frac{\sum_{\sigma' \in \Sigma_1} e^{-\theta \textswab{d}(\sigma_0\sigma', \sigma_0)}}{
    \sum_{\sigma'' \in \Sigma_2} e^{-\theta \textswab{d}(\sigma_0\sigma'', \sigma_0)}}\\
    &\leq \frac{e^{-\theta \textswab{d}(\sigma_0\sigma_a, \sigma_0)}}{e^{-\theta \textswab{d}(\sigma_0\sigma_b, \sigma_0)}} \\
    &\leq e^{\theta|  \textswab{d}(\sigma_0\sigma_a, \sigma_0) - \textswab{d}(\sigma_0\sigma_b, \sigma_0) |} \\
    &\leq e^{\theta \Delta}
\end{align*}
where 
\begin{align*}
    \sigma_a &= \arg \max_{\sigma' \in \Sigma_1} e^{-\theta \textswab{d}(\sigma_0 \sigma', \sigma_0)} 
    \text{ and } \\
    \sigma_a &= \arg \min_{\sigma'' \in \Sigma_2} e^{-\theta \textswab{d}(\sigma_0 \sigma'', \sigma_0)} ~.
\end{align*}
Therefore, setting $\alpha = \theta \cdot \Delta$, we achieve $(\alpha, \calG)$-\name privacy. 
\end{proof}

% \textbf{Property \ref{prop:2}}
% \emph{
\begin{prope}
\label{prope:2}
The sensitivity of a rank-distance is an increasing function of the width $\omega_{\calG}^{\sigma_0}$. For instance, for Kendall's $\tau$ distance $\textswab{d}_\tau(\cdot )$, we have 
$\Delta(\sigma_0 : \textswab{d}_\tau, \calG)
    =\frac{\omega_{\calG}^{\sigma_0}(\omega_{\calG}^{\sigma_0} + 1)}{2}$. 
% }
\end{prope}
%  \begin{lemma}
%  For any two neighboring permutations $\sigma \approx_{\calG} \sigma'$: 
%  \begin{align*}
%      \bigg| \log \frac{\mathbb{P}_{\Theta,\textswab{d}}(\sigma:\sigma_0)}{\mathbb{P}_{\Theta,\textswab{d}}(\sigma':\sigma_0)} \bigg| 
%      &\leq \theta \Delta(\sigma_0 : \textswab{d}, \calG) 
%  \end{align*}\label{lem:prop1}
%  \end{lemma}
 
%  The above lemma essentially gives the proof for Prop. \ref{prop:1}.
 
%  \begin{lemma}For Kendall's $\tau$ distance,
%  \begin{align*}
%      \bigg| \log \frac{\mathbb{P}_{\Theta,\textswab{d}}(\sigma_s:\sigma_0)}{\mathbb{P}_{\Theta,\textswab{d}}(\sigma_s':\sigma_0)} \bigg| \leq \theta \Delta(\sigma_0 : \textswab{d}, \calG)
%  \end{align*}\end{lemma}

To show the sensitivity of Kendall's $\tau$, we make use of its triangle inequality. 

 \begin{proof}
 Recall from the proof of the previous property that the expression $\textswab{d}(\sigma, \sigma_0) = \textswab{d}\big( \sigma_0\sigma, \sigma_0 \big)$, where $\textswab{d}$ is the actual rank distance measure e.g. Kendall's $\tau$. As such, we require that 
 
 \begin{align*}
     \big|  \textswab{d}(\sigma_0\sigma_a, \sigma_0) - \textswab{d}(\sigma_0\sigma_b, \sigma_0) \big|
     &\leq \frac{\omega_{\calG}^{\sigma_0}(\omega_{\calG}^{\sigma_0} + 1)}{2}
 \end{align*}

for any pair of permutations $(\sigma_a, \sigma_b) \in N_\calG$. 
 
 For any group $G_i \in \calG$, let $W_i \subseteq n$ represent the smallest contiguous subsequence of indices in $\sigma_0$ that contains all of $G_i$. 

For instance, if $\sigma_0 = [2,4,6,8,1,3,5,7]$ and $G_i = \{2,6,8\}$, then $W_i = \{2,4,6,8\}$. Then the group width width is $\omega_i = |W_i| - 1 = 3$. Now consider two permutations neighboring w.r.t. $G_i$, $\sigma_a \approx_{G_i} \sigma_b$, so only the elements of $G_i$ are shuffled between them. We want to bound 
\begin{align*}
    \big|  \textswab{d}(\sigma_0\sigma_a, \sigma_0) - \textswab{d}(\sigma_0\sigma_b, \sigma_0) \big| 
\end{align*}
For this, we use a pair of triangle inequalities: 
\begin{align*}
    \textswab{d}(\sigma_0\sigma_a, \sigma_0\sigma_b) 
    &\geq \textswab{d}(\sigma_0\sigma_a, \sigma_0) - \textswab{d}(\sigma_0\sigma_b, \sigma_0) 
    \quad \& \quad 
    \textswab{d}(\sigma_0\sigma_a, \sigma_0\sigma_b) 
    &\geq \textswab{d}(\sigma_0\sigma_b, \sigma_0) - \textswab{d}(\sigma_0\sigma_a, \sigma_0)
\end{align*}
so, 
\begin{align*}
     \big|  \textswab{d}(\sigma_0\sigma_a, \sigma_0) - \textswab{d}(\sigma_0\sigma_b, \sigma_0) \big| &\leq \textswab{d}(\sigma_0\sigma_a, \sigma_0\sigma_b)
\end{align*}


% We make use of the fact that by only permuting the contents of $W_i$ in $\sigma_0$, we can construct some $\sigma_0'$ such that $\textswab{d}(\sigma , \sigma_0) = \textswab{d}(\sigma' , \sigma_0')$. By the triangle inequality, we have
% \begin{align*}
% \textswab{d}(\sigma' , \sigma_0) &\leq \textswab{d}(\sigma' , \sigma_0') + \textswab{d}(\sigma_0 , \sigma_0') \\
% &= \textswab{d}(\sigma , \sigma_0) + \textswab{d}(\sigma_0 , \sigma_0') \\
% \textswab{d}(\sigma' , \sigma_0) - \textswab{d}(\sigma , \sigma_0) 
% &\leq \textswab{d}(\sigma_0 , \sigma_0')
% \end{align*}
% Thus,
% \begin{align*}
%     |\textswab{d}(\sigma , \sigma_0) - \textswab{d}(\sigma' , \sigma_0) |
%     &\leq |\textswab{d}(\sigma_0 , \sigma_0')|
% \end{align*}
Since $\sigma_0\sigma_a$ and $\sigma_0\sigma_b$ only differ in the contiguous subset $W_i$, the largest number of discordant pairs between them is given by the maximum Kendall's $\tau$ distance between two permutations of size $\omega_i + 1$:  
\begin{align*}
    |\textswab{d}(\sigma_0\sigma_a , \sigma_0\sigma_b)|
    &\leq \frac{\omega_i(\omega_i + 1)}{2}
\end{align*}
Since $\omega_{\calG}^{\sigma_0} \geq \omega_i$ for all $G_i \in \calG$, we have that 
\begin{align*}
    \Delta(\sigma_0 : \textswab{d}, \calG) \leq 
    \frac{\omega_{\calG}^{\sigma_0}(\omega_{\calG}^{\sigma_0} + 1)}{2}
\end{align*}
\end{proof}
 



\subsection{Hardness of Computing The Optimum Reference Permutation}\label{app:NP}
\begin{thm} The problem of finding the optimum reference permutation, i.e., $\sigma^*_0=\arg\min_{\sigma\in \mathrm{S}_n}\omega_{\calG}^{\sigma}$ is NP-hard. \label{thm:NP} \end{thm}
\begin{proof} We start with the formal
representation of the problem as follows.

\textit{Optimum Reference Permutation Problem.} Given n subsets $\calG=\{G_i\in 2^{[n]}, i \in [n]\}$,  find the permutation $\sigma^*_0=\arg\min_{\sigma\in \mathrm{S}_n}\omega_{\calG}^{\sigma}$.  

Now, consider the following job-shop scheduling problem.

\textit{Job Shop Scheduling.} There is one job $J$ with $n$ operations $o_i, i \in [n]$ and $n$ machines such that $o_i$ needs to run on machine $M_i$.  Additionally, each machine has a sequence dependent processing time $p_i$. Let $S$ be the sequence till 
There are $n$ subsets $S_i \subseteq [n]$, each corresponding to a set of operations that need to occur in contiguous machines, else the processing times incur penalty as follows. Let $p_i$ denote the processing time for the machine running the $i$-th operation scheduled. Let $\mathbb{S}_i$ be the prefix sequence  with $i$ schedulings. For instance, if the final scheduling is $ 1\:3\:4\:5\:9\:8\:10\:6\:7\:2$ then $\mathbb{S}_4=1345$. Additionally, let $P^j_{\mathbb{S}_i}$ be the shortest subsequence such of $\mathbb{S}_i$ such that it contains all the elements in $S_j \cap \{\mathbb{S}_{i}\}$. For example for $S_1=\{3,5,7\}$, $P^1_{\mathbb{S}_{4}}=345$. 
\begin{gather}p_i=\max_{i\in [n]}(|P^j_{\mathbb{S}_i}|-|S_j \cap \{\mathbb{S}_i\}|)\end{gather}
 The objective is to find a scheduling for $J$ such that it minimizes the makespan, i.e., the completion time of the job. Note that $p_n=\max_i{p_i}$, hence the problem reduces to minimizing $p_n$.

\begin{lemma}The aforementioned  job shop scheduling problem with sequence-dependent processing time is NP-hard.\end{lemma}
\begin{proof} Consider the following instantiation of the sequence-dependent job shop scheduling problem where the processing time is given by $p_i$=$p_{i-1}+w_{kl}, p_1=0$ where $\mathbb{S}_i[i-1]=k$, $\mathbb{S}_i[i]=l$ and $w_{ij}, j\in S_i$  represents some associated weight.    This problem is equivalent to the travelling salesman problem (TSP) \cite{TSP} and is therefore, NP-hard. Thus, our aforementioned job shop scheduling problem is also clearly NP-hard. \end{proof}

\textit{Reduction:}
Let the $n$ subsets $S_i$ correspond to the groups in $\calG$. Clearly, minimizing $\omega^{\sigma}_{\calG}$ minimizes $p_n$. Hence, the optimal reference permutation gives the solution to the scheduling problem as well. 

 \end{proof}
 
 
 \begin{figure}[h]
\begin{subfigure}[b]{\columnwidth}\centering
    \includegraphics[width=0.7\linewidth]{./figures/BFS_graph.png}
        \caption{Group graph}
        \label{fig:BFS:graph}
    \end{subfigure}\\
    \begin{subfigure}[b]{\columnwidth}\centering
    \includegraphics[width=0.5\columnwidth]{./figures/BFS_order.png}
        \caption{BFS reference permutation $\sigma_0$}
        \label{fig:BFS:order}
    \end{subfigure}
    \caption{Illustration of Alg. 1}
    \label{fig:alg:example}
\end{figure}
 
 
 \subsection{Illustration of Alg. 1}\label{app:alg:illustration}
 We now provide a small-scale step-by-step example of how Alg. 1 operates. 
 
 Fig. \ref{fig:BFS:graph} is an example of a grouping $\calG$ on a dataset of $n = 8$ elements. The group of $\DO_i$ includes $i$ and its neighbors. For instance, $G_8 = \{8,3,5\}$. To build a reference permutation, Alg. 1 starts at the index with the largest group, $i = 5$ (highlighted in purple), with $G_5 = \{5,2,3,8,4\}$. As shown in Figure \ref{fig:BFS:order}, the $\sigma_0$ is then constructed by following a BFS traversal from $i=5$. Each $j \in G_5$ is visited, queuing up the neighbors of each $j \in G_5$ that haven't been visited along the way, and so on. The algorithm completes after the entire graph has been visited. 
 
 The goal is to produce a reference permutation in which the width of each group in the reference permutation $\omega_i$ is small. In this case, the width of the largest group $G_5$ is as small as it can be $\omega_5 =5-1 = 4$. However, the width of $G_4 = \{4,5,7\}$ is the maximum possible since $\sigma^{-1}(5) = 1$ and $\sigma^{-1}(7) = 8$, so $\omega_4 = 7$. This is difficult to avoid when the maximum group size is large as compared to the full dataset size $n$. Realistically, we expect $n$ to be significantly larger, leading to relatively smaller groups. 
 
 With the reference permutation in place, we compute the sensitivity: 
 \begin{align*}
     \Delta(\sigma_0 : \textswab{d}, \calG)
     &= \frac{\omega_4 (\omega_4 + 1)}{2} \\
     &= 28
 \end{align*}
 Which lets us set $\theta = \frac{\alpha}{28}$ for any given $\alpha$ privacy value. To reiterate, lower $\theta$ results in more randomness in the mechanism. 
 
 We then sample the permutation $\hat{\sigma} = \mathbb{P}_{\theta, \textswab{d}}(\sigma_0)$. Suppose 
 \begin{align*}
     \hat{\sigma}
     &= [3 \ 2\ 5\ 4\ 8\ 1\ 7\ 6]
 \end{align*}
 Then, the released $\bz$ is given as 
  \begin{align*}
     \bz = \sigma^* &= \sigma^{-1} \hat{\sigma} (\by)\\
     &= [y_1 \ y_2\ y_5\ y_8\ y_3\ y_7\ y_6\ y_4]
 \end{align*}
 One can think of the above operation as follows. What was 5 in the reference permutation ($\sigma_0(1) = 5$) is 3 in the sampled permutation $(\hat{\sigma}(1) = 3)$. So, index 5 corresponding to $\DO_5$ now holds $\DO_3$'s noisy data $y_3$. As such, we shuffle mostly between members of the same group, and minimally between groups. 

%\newcommand{\calO}{\mathcal{O}}

 The runtime of this mechanism is dominated by the Repeated Insertion Model sampler \cite{RIM}, which takes $\calO(n^2)$ time. It is very possible that there are more efficient samplers available, but RIM is a standard and simple to implement for this first proposed mechanism. Additionally, the majority of this is spent computing sampling parameters which can be stored in advanced with $\calO(n^2)$ memory. Furthermore, sampling from a Mallows model with some reference permutation $\sigma_0$ is equivalent to sampling from a Mallows model with the identity permutation and applying it to $\sigma_0$. As such, permutations may be sampled in advanced, and the runtime is dominated by computation of $\sigma_0$ which takes $\calO(|V| + |E|)$ time (the number of vertices and edges in the graph). 
 
 A future direction could be exploring even better heuristics for computing $\sigma_0$. One possibility could be based on ranked enumeration of the permutations \cite{deep2021ranked,deep2021enumeration}.
 \subsection{Proof of Thm. \ref{thm:privacy}}\label{app:thm:privacy}
 \textbf{Theorem \ref{thm:privacy}}
 \emph{Alg. 1 is $(\alpha,\calG)$-\name~private. }
 \begin{proof} The proof follows from Prop. \ref{prop:1}. Having computed the sensitivity of the reference permutation $\sigma_0$, $\Delta$, and set $\theta = \alpha / \Delta$, we are guaranteed by Property \ref{prop:1} that shuffling according to the permutation $\hat{\sigma}$ guarantees $(\alpha, \calG)$-\name privacy. 
 
%  In the algorithm, we permute by $\sigma_0^{-1} \hat{\sigma}$. Since this is equivalent to first permuting by $\hat{\sigma}$ and then permuting by $\sigma_0^{-1}$, this too guarantees $(\alpha, \calG)$-\name privacy by the immunity to post-processing property (Thm. \ref{theorem:post}).
 \end{proof}.
% \arc{rearranged the proof a bit, refer to lemma \ref{lem:prop1} to prove this, should be straighforward just need to argue that $\sigma^*$ and $\hat{\sigma}$ have same probabilistic distribution (they have 1:1 mapping) which I think you try to show below. Also no need to refer to Kendall's tau here since the Alg. 1 is generic}
%  Start with a Lemma: 

%  (Proof essentially by definition of $\Delta$) 
 
%  Add a prop
%  \begin{prop}
%  If $\sigma \approx_{\calG} \sigma'$, then $\gamma \sigma \approx_{\calG} \gamma \sigma'$ for any $\gamma \in \mathrm{S}$. 
%  \end{prop}
%  \begin{proof}
%  By def'n, for some group $G_i \in \calG$, $\sigma(j) = \sigma'(j)$ for all $j \notin G_i$, then $\gamma(\sigma(j)) = \gamma(\sigma(j))$ for all $j \notin G_i$.   
%  \end{proof}
 
%  Main proof of \ref{thm: privacy} (sketch). We want to show that $\Pr[\bz | \by_{\sigma}] \leq e^\alpha \Pr[\bz | \by_{\sigma'}]$, where $\sigma \approx_{\calG} \sigma'$. Recall from the algorithm that we sample some $\sigma_s$ from Mallows given ref permutation $\sigma_0$, and apply $\sigma_0^{-1} \sigma_s$ to $\by$. Define $\sigma_a$ as the permutation that goes from $\by_{\sigma}$ to $\bz$, $\sigma_z = \sigma_a \sigma$ and $\sigma_z = \sigma_b \sigma'$ for $\by_{\sigma'}$. We then have that 
%  \begin{align*}
%      \sigma_0 \sigma_a &= \sigma_s \\
%      \sigma_0 \sigma_b &= \sigma_s'
%  \end{align*}
%  So, $\sigma_s$ must be sampled to get from $\by_\sigma$ to $\bz$ and $\sigma_s'$ needs to be sampled to get from $\by_{\sigma'}$ to $\bz$. From the above proposition we have that $\sigma_s \approx_{\calG} \sigma_s'$. It follows from the above lemma that 
%  \begin{align*}
%      \bigg| \log \frac{\mathbb{P}_{\Theta,\textswab{d}}(\sigma_s:\sigma_0)}{\mathbb{P}_{\Theta,\textswab{d}}(\sigma_s':\sigma_0)} \bigg| \leq \theta \Delta(\sigma_0 : \textswab{d}, \calG)
%  \end{align*}
% So, setting $\theta = \frac{\alpha}{\Delta(\sigma_0: \textswab{d}, \calG)}$ satisifies our privacy definition. 

% It remains to show that for Kendall $\tau$ that 
% \begin{align*}
%     \Delta(\sigma_0 : \textswab{d}, \calG)
%     &= \frac{w (w+1)}{2}
% \end{align*}

%  \begin{proof}
%  \begin{align*}
%     \bigg| \log \frac{\mathbb{P}_{\Theta,\textswab{d}}(\sigma:\sigma_0)}{\mathbb{P}_{\Theta,\textswab{d}}{(\sigma':\sigma_0)} \bigg| 
%     &\leq \theta | \textswab{d}(\sigma, \sigma_0) - \textswab{d}(\sigma', \sigma_0) |
%  \end{align*}
%  \end{proof}


 
%  The key component of the proof is that the ratio of probabilities 
%  \begin{align*}
%      \frac{\Pr[\bz | \by_{\sigma'}]}{\Pr[\bz | \by_\sigma]}
%  \end{align*}
% for any neighboring permutations $\sigma, \sigma'$ is given by 
%  \begin{align*}
%     \log \frac{\Pr[\bz | \by_{\sigma'}]}{\Pr[\bz | \by_\sigma]}
%      &= \log \frac{\exp (-\theta \textswab{d}(\sigma, \sigma_0))} 
%      {\exp (-\theta \textswab{d}(\sigma', \sigma_0)} \\
%      &\leq \theta \big| \textswab{d}(\sigma, \sigma_0) - \textswab{d}(\sigma', \sigma_0) \big| 
%  \end{align*}
 
\subsection{Proof of Thm. \ref{thm:generalized:privacy}} \label{app:thm:generalized}

\textbf{Theorem \ref{thm:generalized:privacy}} \emph{
Alg. 1 satisfies $(\alpha',\calG')$-\name privacy for any group assignment $\calG'$ where  $ \alpha'=\alpha\frac{\Delta(\sigma_0 : \textswab{d}, \calG')}{\Delta(\sigma_0 : \textswab{d}, \calG)}$ 
}
\begin{proof}
Recall from Property \ref{prop:1} that we satisfy $(\alpha, \calG)$ \name-privacy by setting $\theta = \alpha / \Delta(\sigma_0:\textswab{d}, \calG)$. Given alternative grouping $\calG'$ with sensitivity $\Delta(\sigma_0:\textswab{d}, \calG')$, this same mechanism provides 
\begin{align*}
    \alpha' 
    &= \frac{\theta}{\Delta(\sigma_0:\textswab{d}, \calG')} \\
    &= \frac{\alpha / \Delta(\sigma_0:\textswab{d}, \calG)}{\Delta(\sigma_0:\textswab{d}, \calG')} \\
    &= \alpha \frac{\Delta(\sigma_0:\textswab{d}, \calG')}{\Delta(\sigma_0:\textswab{d}, \calG)}
\end{align*}
\end{proof}


\subsection{Formal Utility Analysis of Alg. 1}\label{app:utility:formal}
\begin{thm}For a given set $S \subset [n]$ and Hamming distance metric,  $\textswab{d}_H(\cdot)$,   Alg. 1 is $(\eta,\delta)$-preserving for $\delta=\frac{1}{\psi(\theta, \textswab{d}_H)}\sum_{h=2k+1}^{n} (e^{-\theta\cdot h} \cdot c_h)$ where \scalebox{0.9}{$k=\lceil(1-\eta)\cdot |S|\rceil$} and $c_h$ is the number of permutations with hamming distance $h$ from the reference permutation that do not preserve \scalebox{0.9}{$\eta\%$} of $S$ and is given by
\begin{gather*}c_h=\sum_{j=k+1}^{\max(l_s,\lfloor h/2\rfloor)}\binom{l_s}{j}\cdot \binom{n-l_s}{j}\cdot \Bigg[\sum_{i=0}^{\min(l_s-j,h-2j)}\binom{l_s-j}{i}\\\cdot \binom{i+j}{j}\cdot f(i,j)\cdot\binom{n-l_s-j}{h-2j-i} \cdot f(h-2j-i,j)!\Bigg]\end{gather*}\begin{gather*}
f(i,0)=!i, f(0,q)=q!\\
f(i,j)=\sum_{q=0}^{\min(i,j)}\Bigg[\binom{i}{q}\cdot\binom{j}{j-q}\cdot j! \cdot f(i-q,q)\Bigg]\\l_s=|S|, k=(1-\eta)\cdot l_s, !n=\lfloor \frac{n!}{e}+\frac{1}{2}\rfloor\end{gather*}\end{thm}
\begin{proof} Let $l_s=|S|$ denote the size of the set $S$ and $k=\lceil (1-\eta)\cdot l_S\rceil$ denote the maximum number of correct values that can be missing from $S$. Now, for a given permutation $\sigma \in \mathrm{S}_n$, let $h$ denote its Hamming distance from the reference permutation $\sigma_0$, i.e, $h=\textswab{d}_H(\sigma,\sigma_0)$. This means that $\sigma$ and $\sigma_0$ differ in $h$ indices.  Now, $h$ can be analysed in the  the following two cases, 
\par
\textbf{Case I. $h\leq 2k+1$}

For $(1-\eta)$ fraction of indices to be removed from $S$, we need at least $k+1$ indices from $S$ to be replaced by $k+1$ values from outside $S$. This is clearly not possible for $h\leq 2k+1$. Hence, here $c_h=0$. 
\par
\textbf{Case II. $h > 2k$}

For the following analysis we consider we treat the permutations as strings (multi-digit numbers are treated as a single string character). Now,   
 Let $\mathbb{S}_{\sigma_0}$ denote the non-contiguous substring of $\sigma_0$ such that it consists of all the  elements of $S$, i.e., \begin{gather}|\mathbb{S}|=l_S\\
 \forall i \in [l_S], \mathbb{S}_{\sigma_0}[i] \in S \end{gather}  Let $\mathbb{S}_{\sigma}$ denote the substring corresponding to the positions occupied by $\mathbb{S}_{\sigma_0}$ in $\sigma$. Formally,
 \begin{gather}|\mathbb{S}_{\sigma}|=l_S\\
 \forall i \in [l_S], \mathbb{S}_{\sigma_0}[i] = \sigma(\sigma_0^{-1}(\mathbb{S}_{\sigma_0}[i])) \end{gather}  For example, for $\sigma_0=(1\:2\:3\:5\:4\:7\:8\:10\:9\:6), \sigma=(1\:3\:2\:7\:8\:5\:4\:6\:10\:9)$ and $S=\{2,4,5,8\}$, we have $\mathbb{S}_{\sigma_0}=2548$ and $S_{\sigma}=3784$ where $h=\textswab{d}_H(\sigma,\sigma_0)=9$.  Let $\{\mathbb{S}_{\sigma}\}$ denote the set of the elements of string $\mathbb{S}_{\sigma}$.
 Let $A$ be the set of characters in  $\mathbb{S}_{\sigma}$ such that they do not belong to $S$, i.e, $A=\{\mathbb{S}_{\sigma}[i]|\mathbb{S}_{\sigma}[i] \not \in S, i \in [l_S]\}$. Let $B$ be the set of characters in $\mathbb{S}_{\sigma}$ that belong to $S$ but differ from $\mathbb{S}_{\sigma_0}$ in position, i.e., $B=\{\mathbb{S}_{\sigma}[i]|\mathbb{S}_{\sigma}[i] \in S, \mathbb{S}_{\sigma}[i]\neq \mathbb{S}_{\sigma_0}[i], i \in [l_S]\}$. Additionally, let $C=S-\{\mathbb{S}_{\sigma}\}$. For instance, in the above example, $A=\{3,7\}, B=\{4,8\}, C=\{2,5\}$. Now consider  an initial arrangement of $p+m$ distinct objects that are subdivided into two types -- $p$ objects of Type A and m objects of Type B. Let $f(p,m)$ denote the number of permutations of these $p+m$ objects such that the $m$ Type B objects can occupy any position but no object of Type A can occupy its original position. For example, for $f(p,0)$ this becomes the number of derangements \cite{derangement} denoted as $!p=\lfloor \frac{p!}{e} +\frac{1}{2} \rfloor$. Therefore, $f(|B|,|A|)$ denotes the number of permutations of $\mathbb{S}_\sigma$ such that $\textswab{d}_H(\mathbb{S}_{\sigma_0},\mathbb{S}_{\sigma})=|A|+|B|$. This is because if elements of $B$ are allowed to occupy their original position then this will reduce the Hamming distance.  
 \par Now, let $\bar{\mathbb{S}}_{\sigma}$ ($\bar{\mathbb{S}}_{\sigma_0}$) denote the substring left out after extracting from $\mathbb{S}_{\sigma}$ ($\mathbb{S}_{\sigma_0}$) from $\sigma$ ($\sigma_0$). For example, $\bar{\mathbb{S}}_{\sigma}=1256 10 9$ and $\bar{\mathbb{S}}_{\sigma_0}=13710 9 6$ in the above example. Let $D$ be the set of elements outside of $S$  and $A$ that occupy different positions in $\bar{\mathbb{S}}_\sigma$ and $\bar{\mathbb{S}}_{\sigma_0}$ (thereby contributing to the hamming distance), i.e., $D=\{\bar{\mathbb{S}}_{\sigma_0[i]}|\bar{\mathbb{S}}_{\sigma_0[i]} \not \in S, \bar{\mathbb{S}}_{\sigma_0[i]} \neq \bar{\mathbb{S}}_{\sigma[i]}, i \in [n-l_S]\}$. For instance, in the above example $D=\{9,6,10\}$. Hence, $h=\textswab{d}_{H}(\sigma,\sigma_0)=|A|+|B|+|C|+|D|$ and clearly $f(|D|,|C|)$ represents the number of permutations of $\bar{\mathbb{S}}_{\sigma}$ such that $\textswab{d}_H(\bar{\mathbb{S}}_{\sigma},\bar{\mathbb{S}}_{\sigma_0})=|C|+|D|$. Finally, we have 
\begin{gather*}c_h=\sum_{j=k+1}^{\max(l_s,\lfloor h/2\rfloor)}\underbrace{\binom{l_s}{j}}_{\text{\# ways of selecting set $C$}}\cdot \underbrace{\binom{n-l_s}{j}}_{\text{\# ways of selecting set $A$}}\cdot \Bigg[\\\sum_{i=0}^{\min(l_s-j,h-2j)}\underbrace{\binom{l_s-j}{i}}_{\text{\# ways of selecting set $B$}}\cdot f(i,j)\\\cdot\underbrace{\binom{n-l_s-j}{h-2j-i}}_{\text{\# ways of selecting set $D$}} \cdot f(h-2j-i,j)\Bigg]\end{gather*}
Now, for $f(i,j)$ let $E$ be the set of original positions of Type A that are occupied by Type B objects in the resulting permutation. Additionally, let $F$ be the set of the original positions of Type B objects that are still occupied by some Type B object. Clearly, Type B objects can occupy these $|E
|+|F|=m$ in any way they like. However, the type A objects can only result in $f(p-q,q)$ permutations. Therefore, $f(p,m)$ is given by the following recursive function \begin{gather*}
f(p,0)=!p\\
f(0,m)=m!\\
f(p,m)=\sum_{q=0}^{\min{p,m}}\Bigg(\underbrace{\binom{p}{q}}_{\text{\# ways of selecting set $E$}}\cdot\underbrace{\binom{m}{m-q}}_{\text{\# ways of selecting set $F$}}\\\cdot m! \cdot f(p-q,q)\Bigg)\end{gather*}

Thus, the total probability of failure is given by 
\begin{gather}\delta=\frac{1}{\psi(\theta, \textswab{d}_H)}\sum_{h=2k+2}^{n} (e^{-\theta\cdot h} \cdot c_h)\end{gather}
\end{proof}






%newpage
\subsection{Additional Experimental Details}\label{app:extraresults}
\subsubsection{Evaluation of $(\eta,\delta)$-preservation}\label{app:numerical}

\begin{figure*}[ht]
    \begin{subfigure}[b]{0.33\linewidth}\centering
    \includegraphics[width=\linewidth]{./figures/eta_alpha.png}
        \caption{Variation with $\alpha$}
        \label{fig:eta:alpha}
    \end{subfigure}
    \begin{subfigure}[b]{0.33\linewidth}\centering
    \includegraphics[width=\linewidth]{./figures/eta_width.png}
        \caption{Variation with $\omega$; $\alpha = 3$}
        \label{fig:eta:width}
    \end{subfigure}
    \begin{subfigure}[b]{0.33\linewidth}\centering
    \includegraphics[width=\linewidth]{./figures/eta_subset.png}
        \caption{Variation with $l_S$; $\alpha = 3$}
        \label{fig:eta:subset}
    \end{subfigure}
        \caption{$(\eta,\delta)$-Preservation Analysis}
        \label{fig: eta delta preservation}
\end{figure*}
% \subsubsection{Twitch Dataset Details}
% \begin{figure}[h]
% \begin{subfigure}[b]{\columnwidth}\centering
%     \includegraphics[width=0.8\columnwidth]{Twitch_attack.png}
%         \caption{Twitch: Attack}
%         \label{fig:Twitch:attack:1}
%     \end{subfigure}\\
%     \begin{subfigure}[b]{\columnwidth}\centering
%     \includegraphics[width=0.8\columnwidth]{Twitch_utility.png}
%         \caption{Twitch: Utility}
%         \label{fig:Twitch:utility:1}
%     \end{subfigure}
%     \caption{Twitch dataset results.}
%     \label{fig:Twitch:dataset}
% \end{figure}


% Here, we present experimental results on an additional dataset based on the \emph{Twitch} social media platform \cite{twitch}. Here, the publicly available information is a social media graph, wherein each node represents a user and each edge represents a mutual friendship. So $t_i$ is the set of mutual friendships for data owner $\DO_i$; the $i$'th row of the graph's adjacency matrix. We have binary sensitive attributes of each user $x_i$ that indicates whether or not the user uses explicit language. The dataset includes 9,498 edges. 

% We let the distance between data owners, $d(t_i, t_j)$, be the path length between them. So if data owners $\DO_i$ and $\DO_j$ mutually follow each other, then $d(t_i, t_j) = 1$. For this dataset, we construct the reference permutation using Alg. 1 with $r = 1$, so the reference permutation $\sigma_0$ is constructed to place $\DO_i$'s friends as close as possible in the reference permutation (and thus are most likely to shuffle together). 

% We formalize the group assignment based on the distance threshold $r$ in Sec. \ref{sec:privacy:def} for the ease of exposition. Notably, the actual privacy definition (Def. \ref{def:privacy}) does not dependent on this formalization -- any arbitrary group assignment for $\calG$ is admissible. Hence, exploring other systematic group assignment policies is an interesting future direction. To this end, in this experiment we present another alternative group assignment policy, where for a fixed group size $g$,
% \begin{gather*}G_i=\{\text{Top $g$ closest neighbors of $\DO_i$ as measured via $d(\cdot)$} \}\end{gather*}
% Thus, each data owner $\DO_i$ is assigned an equal sized group $|G_i|$ consisting of its closest `friends' on Twitch. 

% The results are presented in Fig. \ref{fig:Twitch:dataset}. We see that, as we increases the group size afforded to each user, the attack efficacy and utility both gradually decline. Note that, instead of the GBDT calibration model used in the previous experiments, we simply report the empirical distribution of user $i$'s group after shuffling, $\bz_{G_i}$. The utility offered by small to medium group sizes suggests that we can maintain the distribution of sensitive attributes as it varies across the graph.




In this section, we evaluate the characteristics of the  $(\eta,\delta)$-preservation for Kendall's $\tau$ distance $\textswab{d}_\tau(\cdot, \cdot)$.

Each sweep of Fig. \ref{fig: eta delta preservation} fixes $\delta = 0.01$, and observes $\eta$. We consider a dataset of size $n = 10K$ and a subset $S$ of size $l_S$ corresponding to the indices in the middle of the reference permutation $\sigma_0$ (the actual value of the reference permutation is not significant for measuring preservation). For the rest of the discussion, we denote the width of a permutation by $\omega$ for notational brevity. For each value of the independent axis, we generate $50$ trials of the permutation $\sigma$ from a Mallows model with the appropriate $\theta$ (given the $\omega$ and $\alpha$ parameters). We then report the largest $\eta$ (fraction of subset preserved) that at least 99\% of trials satisfy. 

In Fig. \ref{fig:eta:alpha}, we see that preservation is highest for higher $\alpha$ and increases gradually with declining width $\omega$ and increasing subset size $l_s$. 

Fig. \ref{fig:eta:width} demonstrates that preservation declines with increasing width. $\Delta$ increases quadratically with width $\omega$ for $\textswab{d}_\tau$, resulting in declining $\theta$ and increasing randomness. We also see that larger subset sizes result in a more gradual decline in $\eta$. This is due to the fact that the worst-case preservation (uniform random shuffling) is better for larger subsets. i.e. we cannot do worse than $80\%$ preservation for a subset that is $80\%$ of indices. 

Finally, Fig. \ref{fig:eta:subset} demonstrates how preservation grows rapidly with increasing subset size. For large widths, we are nearly uniformly randomly permuting, so preservation will equal the size of the subset relative to the dataset size. For smaller widths, we see that preservation offers diminishing returns as we grow subset size past some critical $l_s$. For $\omega = 30$, we see that subset sizes much larger than a quarter of the dataset gain little in preservation. 

\subsubsection{Adult Dataset}
\label{app:adult experiments}
\begin{figure*}[ht]
    \begin{subfigure}[b]{0.33\linewidth}\centering
        \includegraphics[width=\linewidth]{./figures/Adult_attack_new.png}
        %\vspace{-0.15cm}
        \caption{\textit{Adult}: Attack }%($r$)}
        \label{fig:Adult:attack}
    \end{subfigure}
    \begin{subfigure}[b]{0.33\linewidth}\centering
    \includegraphics[width=\linewidth]{./figures/Adult_alpha.png}
        %   \vspace{-0.15cm} 
        \caption{\textit{Adult}: Attack ($\alpha$)}
        \label{fig:Adult:alpha}
    \end{subfigure}
    \begin{subfigure}[b]{0.33\linewidth}\centering
   \includegraphics[width=\linewidth]{./figures/Adult_utility_1.png}
        %\vspace{-0.15cm}
        \caption{\textit{Adult}: Learnability}
        \label{fig:Adult:utility}
    \end{subfigure}
        \caption{Adult dataset experiments}
        \label{fig: adult experiments}
\end{figure*}






\subsection{Additional Related Work}\label{app:related}
In this section, we discuss the relevant existing work. \par  The anonymization of noisy responses  to improve differential privacy was first proposed by Bittau et al. \cite{Bittau2017} who proposed a principled system architecture for shuffling. This model was formally studied later in \cite{shuffling1, shuffle2}. Erlingsson et al. \cite{shuffling1} showed that for arbitrary $\epsilon$-\ldp randomizers, random shuffling results in privacy amplification. Cheu et al. \cite{shuffle2} formally defined the shuffle \DP model
and analyzed the privacy guarantees of the binary randomized response in this model.
The shuffle \DP model differs from our approach in  two ways. First, it focuses completely on the \DP guarantee. %\ldp is characterized by a privacy parameter (see Section \ref{sec:background}) $\epsilon$, lower the value of $\epsilon$ stronger is the guarantee achieved. 
The privacy amplification is manifested in the from of a lower $\epsilon$ (roughly a factor of $\sqrt{n}$) when viewed in an alternative \DP model known as the central \DP model. \cite{shuffling1,shuffle2,blanket,feldman2020hiding,Bittau2017,Balcer2020SeparatingL}.  
However, our result caters to local inferential privacy. Second, the shuffle model involves an uniform random shuffling of the entire dataset. In contrast, our approach the granularity at which the data is shuffled is tunable which delineates a threshold for the learnability of the data. 
\par A steady line of work has sudied the inferential privacy setting \cite{semantics, Kifer,  IP, Dalenius:1977, dwork2010on, sok}. Kifer et al. \cite{Kifer} formally studied privacy degradation in the face of data correlations and later proposed a  privacy framework, Pufferfish \cite{Pufferfish, Song,Blowfish}, for analyzing inferential privacy. Subsequently, several other privacy definitions have also been proposed for the inferential privacy setting \cite{DDP,BDP,correlated,correlated2,CWP}. For instance, Gehrke et al.  proposed a zero-knowledge privacy \cite{ZKPrivacy,crowd} which is based on simulation semantics. Bhaskar et al. proposed  noiseless privacy \cite{noiseless, TNP} by restricting the set of prior
distributions that the adversary may have access to.  A recent work by Zhang et al. proposes attribute privacy \cite{AP} which focuses on the sensitive properties of a whole dataset. In another recent work, Ligett et al. study a relaxation of \DP that accounts for mechanisms that leak
some additional, bounded information about the database 
\cite{bounded}. Some early work in local inferential privacy include profile-based privacy \cite{profile} by Gehmke et al. where the problem setting comes with a graph of data generating distributions, whose edges encode sensitive pairs of distributions that should be made indistinguishable. In another work by Kawamoto et al., the authors propose distribution privacy \cite{takao} -- local differential privacy for probability distributions.    The major difference between our work and prior research is that we provide local inferential privacy through a new angle -- data shuffling. 

Finally, older works such as $k$-anonymity \cite{kanon},  $l$-diversity \cite{ldiv}, and Anatomy \cite{anatomy} and other \cite{older1, older2, older3, older4, older5} have studied the privacy risk of non-sensitive auxiliary information, or `quasi identifiers' (QIs). In practice, these works focus on the setting of dataset release, where we focus on dataset collection. As such, QIs can be manipulated and controlled, whereas we place no restriction on the amount or type of auxiliary information accessible to the adversary, nor do we control it. Additionally, our work offers each individual formal inferential guarantees against informed adversaries, whereas those works do not. We emphasize this last point since formalized guarantees are critical for providing meaningful privacy definitions. As established by Kifer and Lin in \emph{An Axiomatic View of Statistical Privacy and Utility} (2012), privacy definitions ought to at least satisfy post-processing and convexity properties which our formal definition does.



 
%Relations between differential privacy andthat has given rise to a set of privacy definitions \cite{DDP,BDP,correlated, correlated2, CWP} (not an exhaustive list)  including zero-knowledge privacy \cite{ZKPrivacy}, crowd-blending privacy \cite{crowd}, noiseless privacy \cite{noiseless, TNP}, Pufferfish framework \cite{Pufferfish, Blowfish, Song} and attribute privacy \cite{AP}. 
\newpage


\subsection{Evaluation of Heuristic}
\label{apx:heuristic eval}

\begin{figure}[h]
    \centering
    \includegraphics[width = \linewidth]{./figures/heuristic_optimal.png}
    \caption{Comparison of our heuristic's performance with that of an optimal reference permutation $\sigma^*_0$. An optimal $\sigma^*_0$ is generated with every group having size $w$. A graph is generated from this optimal $\sigma^*_0$ from which our heuristic (blue) attempts to reconstruct the optimal permutation. For baselining, the performance of a random $\sigma_0$ selection is plotted (orange). We observe that at worst, our heuristic picks a reference permutation with width $2.5\times$ that of the optimal reference permutation (green). See Section \ref{sec:mechanism} for definition of terms.}
    \label{fig:heuristic optimal}
\end{figure}

Algorithm \ref{algo:main} is designed to find a reference permutation $\sigma_0$ with low width $\omega_\calG^\sigma$ w.r.t. the given grouping $\calG$. A low width is desirable, since it leads to low sensitivity $\Delta(\sigma_0 : \textswab{d}, \calG)$, which in turn leads to higher dispersion parameter $\theta = \alpha / \Delta$, and thus less randomness over permutations (higher utility). Theorem \ref{thm:NP} proves that computing the optimal reference permutation (minimum width) is NP-hard. As such, we propose a BFS-based heuristic. 

\textbf{Comparison with optimal reference permutation}\\
To demonstrate the value of the heuristic used in Alg. \ref{algo:main}, we provide two evaluations of its performance. 
For our first evaluation, we compare the performance of our heuristic BFS reference permutation selection ($\sigma_0$) with that of the optimal reference permutation  and that of a random reference permutation. As identified by Theorem \ref{thm:NP}, finding the optimal reference permutation for a given grouping $\calG$ is NP-hard. For these experiments, we first create an optimal reference permutation, where each group $G_i \in \calG$ is equally sized $w$ and maximally compact. The optimal width, $\omega_\calG^\sigma$, is then $\min(n, w)$. We then generate a graph from this optimal reference permutation. Finally, we run the BFS reference permutation computation described in Alg. \ref{algo:main} attempting to approximate the optimal $\sigma^*_0$, and compute its width. 

To compare with a naive approach, we also plot the performance of a randomly chosen reference permutation. We expect the maximum width across groups $\omega_\calG^\sigma$ to be large for this technique. If one of the $n$ groups has a single entry low (near 0) in $\sigma_0$ and a single entry high (near $n$) in $\sigma_0$, the width will be near $n$. The random baseline is averaged over 10 trials with a 1 standard deviation envelope plotted (but difficult to see, since the variance is low). 

Figure \ref{fig:heuristic optimal} depicts our findings. Each plot has a different group size $w$, listed at the top, used in the optimal reference permutation. We find that the random baseline (orange) consistently chooses a reference permutation such that $\omega_\calG^\sigma$ is near $n$, as expected. Our method (blue), on the other hand, closely tracks the optimal solution (green). We find that in the worst case, our algorithm's solution has a width $\leq 2.5 \times$ larger than the optimal. Note that for $r = 0$ (upper left), all methods trivially have a width of one, since the corresponding graph has no edges. While there may be room for improvement, we find this to be sufficient for the present work. 

% \newpage


\begin{figure}[h]
    \centering
    \includegraphics[width = \linewidth]{./figures/heuristic_random.png}
    \caption{example of our heuristic's performance on randomly generated graphs. As $r$ increases, so does the connectivity of the random graphs and the average group size (green). As shown by Theorem \ref{thm:NP}, computing the optimal $\omega_\calG^\sigma$ is NP-hard. The average group size (green) in $\calG$ is a loose lower bound on the optimal $\omega_\calG^\sigma$. The performance of a random $\sigma_0$ assignment (orange) is also plotted for reference. Our heuristic BFS algorithm (blue) consistently outperforms the random baseline.}
    \label{fig:heuristic random}
\end{figure}

\textbf{Performance on randomly generated graphs}\\
For our second evaluation, we observe how well our BFS heuristic (in Algorithm \ref{algo:main}) performs on randomly generated graphs. Here, we sample $n$ points uniformly on the unit interval. We then say that the $i$th point's group, $G_i$, consists of all other points within $r$ of it. As $r$ increases, so does the groups size. Since computing the optimal reference permutation is NP-hard (Theorem \ref{thm:NP}), we do not show the optimal width. Instead, we show a loose lower bound of the optimal width (green) by plotting the average group size for a given $r$ (recall that the width is greater than or equal to the largest group size, so we expect this to be a loose lower bound, solely for reference). For comparison, we evaluate the performance of a random $\sigma_0$ choice as well. For both of these methods, we run 10 trials of generating a random graph (and picking a random $\sigma_0$) at each value of $n$ and plot the mean along with a 1 standard deviation envelope, which is difficult to see due to low variance. 

Figure \ref{fig:heuristic random} depicts our findings. We find that --- across values of $n$ and $r$ --- our heuristic (blue) significantly outperforms the random baseline (orange). Additionally, we observe the trends we expect. For a low $r$ values, our heuristic BFS algorithm chooses a $\sigma_0$ with width close to the lower bound (green) of the optimal width $\omega_{\calG}^\sigma$. As $r$ increases, the graph become significantly more connected. Both the lower bound and our heuristic move closer to the width of the random baseline. Note that for $r = 0$ (upper left), all methods trivially have a width of one, since the corresponding graph has no edges. Ultimately, these findings indicate that our heuristic for computing $\sigma_0$ significantly outperforms a naive random choice, and follows the same trend as the lower bound of the optimal. 
\graphicspath{{./chapters/chapter5/}}
\chapter{ }

\section{Appendix}
For documented code demonstrating our SDP mechanisms used to generate the plots of \textbf{Figure \ref{fig: experiments}} please visit our repo: \url{https://github.com/casey-meehan/location_trace_privacy} 

The following sections will include proofs of results, derivations of algorithms, and explanations of experimental procedures. 
\subsection{Illustrations}
\label{apx: Illustrations}

\subsubsection{NYC Mayoral Staff Member Location Trace}

\begin{figure*}[h]
	\centering
	\begin{subfigure}[b]{.32\textwidth}
		\centering
		\includegraphics[width = \linewidth]{./images/nyc_trace.png}
		\caption{}
		\label{fig:nyc_trace}
	\end{subfigure}
	\begin{subfigure}[b]{.32\textwidth}
		\centering
		\includegraphics[width = \linewidth]{./images/nyc_trace_isotropic.png}
		\caption{}
		\label{fig:nyc_trace_iso}
	\end{subfigure}
	\begin{subfigure}[b]{.32\textwidth}
		\centering
		\includegraphics[width = \linewidth]{./images/nyc_trace_opt.png}
		\caption{}
		\label{fig:nyc_trace_opt}
	\end{subfigure}
	\caption{Example of sensitive location trace of NYC mayoral staff member exposed by \citep{nyt}. (b) and (c) depict the posterior uncertainty (green) $P_{\calA,\calP}(X_i | Z)$ for each 2d location. (a) depicts three sensitive times (red with blue outline): Gracie Mansion (Mayor's home), an event on Staten Island that the mayor attended, and finally the staff member's home on long island. (b) provides an example of Approach C: adding independent Gaussian noise to each location (red dotted line). A GP posterior still maintains high confidence within a small radius along the trace, including at the sensitive times. (c) provides an example of the optimized noise of Multiple Secrets of identical aggregate MSE as (b). By focusing \textit{correlated} noise around the three sensitive times, there is high uncertainty at sensitive times and high confidence elsewhere.}
	\label{fig:nyc_example}
\end{figure*}

\subsubsection{Juxtaposition of Mechanisms' Covariance Matrices}
\label{apx: juxtaposition}
The following figures aim to illustrate the difference between the covariance matrices used in the experimental baselines (indep./uniform and indep./concentrated) and those chosen by our SDP algorithms for both the RBF and periodic prior. Note that here we presume the different dimensions of location to be independent and --- by Corollary \ref{cor: independence} --- are able to treat a 2d location trace as two 1d traces. As such, the following examples are demonstrating mechanism covariance matrices and additive noise samples used for either a single dimension of location data (for RBF kernel) or for the one dimension of temperature data (for periodic kernel). 

The first figure \textbf{(a)} shows the covariance of the Approach C baselines used in the experiments. The second figure \textbf{(b)} shows the covariance of our SDP mechanisms for the RBF kernel used on location data. The third figure \textbf{(c)} shows the covariance of our SDP mechanisms for the periodic kernel used for temperature data. 

In each figure the covariance matrix is depicted as a heat map with warmer colors indicating higher values (normalized to largest and smallest value in the covariance matrix). The drawn noise samples $G$ are plotted against their time index. So, the sequence of plotted $(x,y)$ values is $\big[(1, G_1), (2, G_2), \dots, (n, G_n)\big]$, where $n = 50$ for the RBF case and $n = 48$ for the periodic case. 

\begin{figure*}[h]
	\centering
	\begin{subfigure}[b]{1\textwidth}
		\centering
		\includegraphics[width = \linewidth]{./images/cov_table_header.png}
	\end{subfigure}
	\begin{subfigure}[b]{1\textwidth}
		\centering
		\includegraphics[width = \linewidth]{./images/cov_table_baselines.png}
		\caption{Covariance matrices and mechanism samples for the baselines used in experiments. 
		\vspace{2mm}\\
		The first figure demonstrates the uniform approach that distributes the independent Gaussian noise budget along the entire trace, regardless of $\Is$. 
		\vspace{2mm}\\
		The second and third show the concentrated approach that allocates the entire noise budget to only the sensitive locations in $\Is$: first for a basic secret (one location) and then for a compound secret of 3 evenly spaced locations.} 
		\label{fig: cov table baselines}
	\end{subfigure}
\end{figure*}

\begin{figure*}[h] \ContinuedFloat
	\begin{subfigure}[b]{1\textwidth}
		\centering
		\includegraphics[width = \linewidth]{./images/cov_table_header.png}
	\end{subfigure}
	\begin{subfigure}[b]{1\textwidth}
		\centering
		\includegraphics[width = \linewidth]{./images/cov_table_RBF_1.png}
	\end{subfigure}
	\begin{subfigure}[b]{1\textwidth}
		\centering
		\includegraphics[width = \linewidth]{./images/cov_table_RBF_2.png}
		\caption{
			Covariance matrices and mechanism samples for the median RBF prior ($\leff \approx 6$). 
			\vspace{2mm} \\
			The first noise mechanism (Mech. basic) demonstrates the covariance matrix chosen by $\text{SDP}_\text{A}$ for a basic secret of a single location $X_i$ in the middle of the trace. The uncorrelated dot in the middle of the covariance matrix, $\Sigmag_{ii}$, represents the independent noise $G_i$ added at the sensitive location to mitigate \emph{direct} loss. To mitigate \emph{inferential} loss, the SDP optimizes the remainder of the matrix to be positively correlated with maximum variance allocated to locations near $X_i$ in time. This thwarts GP inference of the true location at time $t_i$. 
			\vspace{2mm} \\
			The second mechanism (Mech. comp.) depicts the covariance chosen by $\text{SDP}_\text{A}$ to protect a compound secret of two adjacent locations in the trace (visible as the uncorrelated `$+$' through the middle consuming 2 rows/columns). Recall that a compound secret ought to protect directional information: \emph{did the user visit B first and then A, or A and then B?} That is precisely what this mechanism does by randomizing the angle of approach to the two locations in the middle with positively and negatively correlated noise. Also note that the SDP does not allocate a large share of noise budget to the actual locations themselves. This highlights the fact that protecting a compound secret does not protect its constituent basic secrets.
			\vspace{2mm} \\
			The third and final mechanism (Mech. all basic) is the noise covariance chosen by $\text{SDP}_\text{B}$ in the Multiple Secrets algorithm. To protect all basic secrets with a utility constraint, the SDP converges to a mechanism that looks similar to the uniform baseline. However, this mechanism adds a subtle degree of off-diagonal correlation along with greater noise power towards the beginning and end of the trace. The off-diagonal correlation is noticeable when the samples are compared to those of the uniform baseline in the previous figure. While this change appears to be minor, it makes a significant change in the posterior confidence of a GP adversary (as seen in \textbf{Figure \ref{fig: RBF all}}). 
			}
		\label{fig: cov table rbf}
	\end{subfigure}
\end{figure*}

\begin{figure*}[h] \ContinuedFloat
	\begin{subfigure}[b]{1\textwidth}
		\centering
		\includegraphics[width = \linewidth]{./images/cov_table_header.png}
	\end{subfigure}
	\begin{subfigure}[b]{1\textwidth}
		\centering
		\includegraphics[width = \linewidth]{./images/cov_table_PER_1.png}
	\end{subfigure}
	\begin{subfigure}[b]{1\textwidth}
		\centering
		\includegraphics[width = \linewidth]{./images/cov_table_PER_2.png}
		\caption{
			Covariance matrices and mechanism samples for the median periodic prior ($\leff \approx 1.1$), and a period of half the trace length. 
			\vspace{2mm} \\
			The first noise mechanism (Mech. Basic) shows the covariance chosen by $\text{SDP}_\text{A}$ to protect a single location (temperature) in the middle of the trace. As in the RBF case, significant noise power is allocated to the sensitive location itself, $X_i$, to limit \emph{direct} privacy loss. However, the noise added to the remainder of the trace is significantly different. It is tailored to thwart inference by a periodic prior, wherein the location one period away has correlation 1. 
			\vspace{2mm}\\
			The second noise mechanism (Mech. comp.) shows the covariance chosen by $\text{SDP}_\text{A}$ to protect a compound secret of two locations, $X_i, X_j$, 16 timesteps apart (not quite a full period). Here, we see the SDP randomize the phase of the additive noise such that periodic inference cannot tell directional information like $X_i > X_j$ or vice versa. 
			\vspace{2mm}\\
			The third noise mechanism (Mech. all basic) is identical to the all basic secrets mechanism chosen for the RBF case above, except using a periodic prior $\Sigma$. The mechanism chosen looks similar to the uniform baseline, except with slightly periodic off-diagonal correlation imitating the prior covariance. Additionally, noise power is mitigated towards the middle and ends of the trace. Again, \textbf{Figure \ref{fig: PER all}} indicates that this subtle change makes a significant difference in thwarting Bayesian adversaries. 
			}
		\label{fig: cov table rbf}
	\end{subfigure}
\end{figure*}

\clearpage

\subsection{Proof of results}
\label{apx: proofs} 
\subsubsection{Proof of Theorem \ref{thm: prior misspecification}} 
\textbf{Theorem \ref{thm: prior misspecification}} Prior-Posterior Gap:
\textit{
An $(\varepsilon, \lambda)$-CIP mechanism with conditional prior class $\Theta$ guarantees that for any event $O$ on sanitized trace $Z$
	\begin{align*}
		\bigg| \log \frac{P_{\calP, \calA}(s_i | Z \in O)}{P_{\calP, \calA}(s_j | Z \in O)} - \log \frac{P_{\calP}(s_i)}{P_{\calP}(s_j)} \bigg| \leq \varepsilon'
	\end{align*}
	for any $\calP \in \Theta$ with probability $\geq 1 - \delta$ over draws of $Z|\Xs=s_i$ or $Z|\Xs=s_j$, where $\varepsilon'$ and $\delta$ are related by
	\begin{align*}
		\varepsilon' = \varepsilon + \frac{\log \nicefrac{1}{\delta}}{\lambda - 1} \ .
	\end{align*}
	This holds under the condition that $Z|\Xs = s_i$ and $Z|\Xs = s_j$ have identical support. 
}

\begin{proof}
	This result makes use of a R\'enyi divergence property identified in \cite{renyi}: 
	\begin{lemma}
		\label{lem: renyi to eps delt}
		Let $\calP,\calQ$ be two distributions on $X$ of identical support such that  
		\begin{align*}
			\max \bigg\{ D_\lambda \binom{P_\calP(X)}{P_\calQ(X)}, 
			D_\lambda \binom{P_\calQ(X)}{P_\calP(X)} \bigg\}
			\leq \varepsilon 
		\end{align*}
		Then for any event $O$,
		\begin{align*}
			P_\calP(X \in O) \leq \max \big\{ e^{\varepsilon'} P_\calQ(X \in S), \delta \big\}
		\end{align*} 
		and
		\begin{align*}
			P_\calQ(X \in O) \leq \max \big\{ e^{\varepsilon'} P_\calP(X \in S), \delta \big\}
		\end{align*} 
		where 
		\begin{align*}
			\varepsilon' = \varepsilon + \frac{\log \nicefrac{1}{\delta}}{\lambda - 1}
		\end{align*}
	\end{lemma}
	CIP guarantees that for all $\calP \in \Theta$ and all discriminative pairs $(s_i, s_j) \in \Spairs$ (which also includes $(s_j, s_i)$) 
	\begin{align*}
		D_\lambda \binom{P_{\calP, \calA}(Z | \Xs = s_i)}{P_{\calP,\calA}(Z | \Xs = s_j)} \leq \varepsilon
	\end{align*}
	and thus by Lemma \ref{lem: renyi to eps delt} we have for any event $O$ on $Z$
	\begin{align*}
		P_{\calP, \calA}(Z \in O | \Xs = s_i) 
		\leq \max \big\{ e^{\varepsilon'} P_{\calP, \calA}(Z \in O | \Xs = s_j), \delta \big\}
	\end{align*}
	and
	\begin{align*}
		P_{\calP, \calA}(Z \in O | \Xs = s_j) 
		\leq \max \big\{ e^{\varepsilon'} P_{\calP, \calA}(Z \in O | \Xs = s_i), \delta \big\}
	\end{align*}
	As such, given that $\Xs = s_i$ the probability of some event $\{Z \in W\}$ such that 
	\begin{align*}
		P_{\calP, \calA}(Z \in W | \Xs = s_i) 
		\geq  e^{\varepsilon'} P_{\calP, \calA}(Z \in W | \Xs = s_j)
	\end{align*}
	is no more than $\delta$. The same is true swapping $s_j$ for $s_i$. So, over draws of $Z | \Xs = s_i$ or $Z | \Xs = s_j$ we have that 
	\begin{align*}
		 \frac{P_{\calP, \calA}(Z \in O | \Xs = s_i)}{P_{\calP,\calA}(Z \in O | \Xs = s_j)} \leq e^{\varepsilon'}
		 \quad \text{and} \quad
		 \frac{P_{\calP, \calA}(Z \in O | \Xs = s_j)}{P_{\calP,\calA}(Z \in O | \Xs = s_i)} \leq e^{\varepsilon'}
	\end{align*}
	with probability $\geq 1 - \delta$, which is equivalent to the statement that 
	\begin{align*}
		-\varepsilon' 
		\leq \log  \frac{P_{\calP, \calA}(Z \in O | \Xs = s_i)}{P_{\calP,\calA}(Z \in O | \Xs = s_j)}
		&\leq \varepsilon' \\
		\bigg| \log \frac{P_{\calP, \calA}(s_i | Z \in O)}{P_{\calP, \calA}(s_j | Z \in O)} - \log \frac{P_{\calP}(s_i)}{P_{\calP}(s_j)} \bigg| 
		&\leq \varepsilon'
	\end{align*}
\end{proof}

\subsubsection{Proof of Lemma \ref{lem: renyi additive loss}}
\textbf{Lemma \ref{lem: renyi additive loss}} (CIP loss for additive mechanisms)
\textit{
	For an additive noise mechanism, a fully dependent trace as in \textbf{Figure \ref{fig:condensed model}}, and any prior $\calP$ on $X$ the CIP loss may be expressed as
	\begin{align}
		&D_\lambda \binom{P_{\calA, \calP}(Z | \Xs = s_i)}{P_{\calA, \calP}(Z | \Xs = s_j)}  
		&= \sum_{i \in \Is} \bigg[ D_\lambda \binom{P_\calA(Z_i | X_i = s_i)}{P_\calA(Z_i | X_i = s_j)} \bigg]
		+ D_\lambda \binom{P_{\calA, \calP}(\Zu | \Xs = s_i)}{P_{\calA, \calP}(\Zu | \Xs = s_j)} \notag
	\end{align}
%	where $\Xu_{|x_s} \sim P_\calP(\Xu | \Xs = x_s)$ and $\Xu_{|x_s'} \sim P_\calP(\Xu | \Xs = x_s')$. 
}
\begin{proof}
\begin{align}
	D_\lambda \binom{P_{\calA, \calP}(Z | \Xs = x_s)}{P_{\calA, \calP}(Z | \Xs = x_s')} 
	&= D_\lambda \binom{P_{\calA}(\Zs | \Xs = x_s)P_{\calA, \calP}(\Zu | \Xs = x_s)}{P_{\calA}(\Zs | \Xs = x_s')P_{\calA, \calP}(\Zu | \Xs = x_s')} \tag{1} \\
	&=  D_\lambda \binom{P_{\calA}(\Zs | \Xs = x_s)}{P_{\calA}(\Zs | \Xs = x_s')} + D_\lambda \binom{P_{\calA, \calP}(\Zu | \Xs = x_s)}{P_{\calA, \calP}(\Zu | \Xs = x_s')} \tag{2} \\
	&= D_\lambda \binom{\prod_{i \in \Is } P_{\calA}(Z_i | X_i = x_i)}{\prod_{i \in \Is } P_{\calA}(Z_i | X_i = x_i')} + D_\lambda \binom{P_{\calA, \calP}(\Zu | \Xs = x_s)}{P_{\calA, \calP}(\Zu | \Xs = x_s')} \tag{3} \\
	&= \sum_{i \in \Is} \bigg[ D_\lambda \binom{P_\calA(Z_i | X_i = x_i)}{P_\calA(Z_i | X_i = x_i')} \bigg]
	+ D_\lambda \binom{P_{\calA, \calP}(\Zu | \Xs = x_s)}{P_{\calA, \calP}(\Zu | \Xs = x_s')} \tag{4}
\end{align}
Where line (1) uses the conditional independence seen in the graphical model of \textbf{Figure \ref{fig:graphical models}}. Line (2) is due to the fact that the two terms in line (1) are conditionally independent, allowing for separating into the sum of two separate divergences (which is an easily verifiable property of R\'enyi divergence evident from its definition in Equation \ref{eqn: renyi}). Line (3) is again from the conditional independence between the $Z_i$ for each $i \in \Is$ when conditioned on $\Xs$. Line (4) uses the same property of R\'enyi divergence used in Line (2): the terms in the product are conditionally independent allowing for the separation into the sum of multiple divergences. 

%We now rewrite the conditional distribution of the second term as the marginal distribution of the sum of two independent random variables: $P_{\calA, \calP}(\Zu | \Xs = x_s) = P_{\calA, \calP}(\Xu_{|x_s} + \Gu)$ where \\$\Xu_{|x_s} \sim P_\calP(\Xu | \Xs = x_s)$. Effectively this says that the distribution of the conditional random variable $\Zu | \{\Xs = x_s\}$ is identical to the distribution of independently drawing $\Xu_{|x_s} \sim P_\calP(\Xu | \Xs = x_s)$ and $\Gu \sim \calN(\mathbf{0}, \Sigma_{uu})$ and adding them together. 
%\begin{align}
%	P_{\calA, \calP}(\Zu = z_u | \Xs = x_s) 
%	&= \int_{\R^{|\Iu|}} P_{\calA, \calP}(\Zu = z_u, \Xu = x_u | \Xs = x_s) \ dx_u  \tag{5} \\
%	&=  \int_{\R^{|\Iu|}} P_{\calP}(\Xu = x_u | \Xs = x_s) P_{\calA}(\Zu = z_u | \Xu = x_u) \ dx_u  \tag{6} \\
%	&=  \int_{\R^{|\Iu|}} P_{\calP}(\Xu = x_u | \Xs = x_s) P_{\calA}(\Gu = z_u - x_u) \ dx_u \tag{7} \\
%	&= \big( P_{\calP}(\Xu | \Xs = x_s) * P_{\calA}(\Gu) \big)(z_u) \tag{8} \\
%	&= P_{\calA, \calP}(\Xu_{|x_s} + \Gu = z_u) \tag{9}
%\end{align}
%Where lines (5) and (6) are also due to the structure of conditional independence, and line (7) is simply rewriting $P_{\calA}(\Zu = z_u | \Xu = x_u)$ in terms of the density of $\Gu$. Line (8) is by definition of a convolution, and line (9) is due to the fact that the convolution of the densities of two independent random variables is the distribution of their sum. Thus, $P_{\calA, \calP}(\Zu | \Xs = x_s) = P_{\calA, \calP}(\Xu_{|x_s} + \Gu)$. Substituting this back into the second divergence in line (4), we get 
%\begin{align*}
%	D_\lambda \binom{P_{\calA, \calP}(Z | \Xs = x_s)}{P_{\calA, \calP}(Z | \Xs = x_s')} &= 
%	\sum_{i \in \Is} \bigg[ D_\lambda \binom{P_\calA(Z_i | X_i = x_i)}{P_\calA(Z_i | X_i = x_i')} \bigg]
%	+ D_\lambda \binom{P_{\calA, \calP}(\Xu_{|x_s} + \Gu)}{P_{\calA, \calP}(\Xu_{|x_s'} + \Gu)}
%\end{align*}

\end{proof}

\subsubsection{Proof of Theorem \ref{thm: prior misspecification}}
\label{apx: prior misspecification proof}
\textbf{Thoerem \ref{thm: prior misspecification}}
Robustness to Prior Misspecification 
\textit{
	Mechanism $\calA$ satisfies $\varepsilon(\lambda)$-CIP for prior class $\Theta$. Suppose the finite mean true distribution $\calQ$ is not in $\Theta$. The CIP loss of $\calA$ against prior $\calQ$ is bounded by 
	\begin{align*}
		D_\lambda \binom{P_{\calA, \calQ}(Z | \Xs = s_i)}{P_{\calA, \calQ}(Z | \Xs = s_j)} \leq \varepsilon'(\lambda)
	\end{align*}
	where
	\begin{align*}
		\varepsilon'(\lambda) 
		&= \frac{\lambda - \frac{1}{2}}{\lambda - 1} \ \Delta(2\lambda) + 
		\Delta(4\lambda - 3) +
		\frac{2\lambda - \frac{3}{2}}{2\lambda - 2} \ \varepsilon(4 \lambda -2)
	\end{align*}
	and where $\Delta(\lambda)$ is
	\begin{align*}
		\inf_{\calP \in \Theta} \sup_{s_i \in \calS} \max \bigg\{ 
		D_\lambda \binom{P_{ \calP}(\Xu | \Xs = s_i)}{P_{ \calQ}(\Xu | \Xs = s_i)}, 
		D_\lambda \binom{P_{ \calQ}(\Xu | \Xs = s_i)}{P_{ \calP}(\Xu | \Xs = s_i)}
		\bigg\}
	\end{align*}
}
\begin{proof}
By `finite mean' distribution $\calQ$, we mean that all conditionals of $\calQ$ given some $\Xs$ have finite mean. Since a conditional prior class contains conditionals of one distribution with any offset (any mean value), this guarantees that $\Delta(\lambda)$ is achieved for some $\calP \in \Theta$. Intuitively, this prevents the pathological case of $\inf_{\calP \in \Theta}$ being a limit as the mean of $\calP \rightarrow \infty$, only asymptotically approaching $\Delta(\lambda)$. If the mean of $\calQ$ is finite, then the closest $\calP \in \Theta$ (in R\'enyi divergence) must also have finite mean, since any mean is attainable in a conditional prior class $\Theta$.

With this in mind, we make use of the following triangle inequality provided in \cite{renyi}: 
\begin{lemma}
	For distributions $\calP$, $\calQ$, $\calR$ on $X$ with common support we have
	\begin{align*}
		D_\lambda \binom{P_\calP(X)}{P_\calQ(X)} \leq 
		\frac{\lambda - \frac{1}{2}}{\lambda - 1} D_{2 \lambda} \binom{P_\calP(X)}{P_\calR(X)} 
		+ D_{2\lambda - 1} \binom{P_\calR(X)}{P_\calQ(X)}
	\end{align*}
\end{lemma}
In our case, we assume that the mechanism $\calA$ gives $Z|\Xs = x_s$ identical support for all $\Is, x_s$. Using this, we have 
\begin{align*}
	D_\lambda \binom{P_{\calA, \calQ}(\Zu | \Xs = x_s)}{P_{\calA, \calQ}(\Zu | \Xs = x_s')} 
	\leq \frac{\lambda - \frac{1}{2}}{\lambda - 1} D_{2\lambda} \binom{P_{\calA, \calQ}(\Zu | \Xs = x_s)}{P_{\calA, \calP}(\Zu | \Xs = x_s)}
	+  D_{2\lambda - 1} \binom{P_{\calA, \calP}(\Zu | \Xs = x_s)}{P_{\calA, \calQ}(\Zu | \Xs = x_s')} \ \ . \\
\end{align*}
By a data processing inequality, the divergence of the first term is bounded by $\Delta(2\lambda)$ and the blue term may be bounded by a second application of the triangle inequality: 
\begin{align*}
	D_{2\lambda - 1} \binom{P_{\calA, \calP}(\Zu | \Xs = x_s)}{P_{\calA, \calQ}(\Zu | \Xs = x_s')}
	&\leq \frac{2\lambda - \frac{3}{2}}{2\lambda - 2} D_{4\lambda - 2} \binom{P_{\calA, \calP}(\Zu | \Xs = x_s)}{P_{\calA, \calP}(\Zu | \Xs = x_s')}
	+ D_{4\lambda - 3} \binom{P_{\calA, \calP}(\Zu | \Xs = x_s')}{P_{\calA, \calQ}(\Zu | \Xs = x_s')}
\end{align*}
The first divergence is bounded by $\varepsilon(4\lambda - 2)$ and the second divergence is bounded by $\Delta(4\lambda - 3)$. Putting all this together we have the following upper bound 
\begin{align*}
	D_\lambda \binom{P_{\calA, \calQ}(\Zu | \Xs = x_s)}{P_{\calA, \calQ}(\Zu | \Xs = x_s')}
	\leq 
	\frac{\lambda - \frac{1}{2}}{\lambda - 1} \ \Delta(2\lambda) + 
		\Delta(4\lambda - 3) +
		\frac{2\lambda - \frac{3}{2}}{2\lambda - 2} \ \varepsilon(4 \lambda -2)
\end{align*}
\end{proof}

\subsubsection{Proof of Theorem \ref{thm:GP bound}}
\label{apx: GP bound proof}
\textbf{Theorem \ref{thm:GP bound}}
CIP loss bound for GP conditional priors:
\emph{
Let $\Theta$ be a GP conditional prior class. Let $\Sigma$ be the covariance matrix for $X$ produced by its kernel function. Let $\calS$ be the basic or compound secret associated with $\Is$, and $S$ be the number of unique times in $\Is$. The mechanism $\calA(X) = X + G = Z$, where $G \sim \calN(\mathbf{0}, \Sigmag)$, then satisfies $(\varepsilon, \lambda)$-Conditional Inferential Privacy $(\Spairs, r, \Theta)$, where 
\begin{align*}
	\varepsilon
	&\leq \frac{\lambda}{2} S r^2 \Big(  \frac{1 }{\sigma_s^2} + \alpha^*  \Big) 
\end{align*}
where $\sigma_s^2$ is the variance of each $G_i \in \Gs$ (diagonal entries of $\Sigmag_{ss}$) and $\alpha^*$ is the maximum eigenvalue of $\Sigmaeff = \big(\Sigma_{us} \Sigma_{ss}^{-1}\big)^\intercal \big( \Sigma_{u | s} + \Sigma_{uu}^{(g)} \big)^{-1} \big(\Sigma_{us} \Sigma_{ss}^{-1}\big)$. 
}

\begin{proof}
Again, the conditional prior class $\Theta$ is defined by a kernel function $i,j \rightarrow \text{Cov}(i,j)$, which -- given the indices of the trace $X$ -- induces a covariance matrix $\Sigma$ between all $X_i, X_j$. In practice, when the sampling rate of locations is non-uniform the kernel function may use the time-stamps of the points in the trace to assign high correlation to $X_i$ that are close in time and low correlation to $X_i$ that are far apart in time. Of course, correlation between $X_i$ that are different dimension (e.g. latitude and longitude) must be designed for the given application and may be completely independent. The kernel function can encode this as well. 

Recall from Equation \ref{eqn: renyi} that the R\'enyi divergence between two mean-shifted multivariate normal distributions, $\calP_1 = \calN(\mu_1, \Sigma)$ and $\calP_2 = \calN(\mu_2, \Sigma)$ is 
\begin{align*}
	D_\lambda \binom{\calP_1}{\calP_2} = \frac{\lambda}{2} (\mu_1 - \mu_2)^\intercal \Sigma^{-1} (\mu_1 - \mu_2)
\end{align*}
Now, for any prior $\calP \in \Theta$, we have that $X \sim \calN(\mu, \Sigma)$ for some $\mu$ and for $\Sigma$ defined by the kernel function. Again, $G \sim \calN(\mathbf{0}, \Sigmag)$. $\Is$ encodes the indices of a single location basic secret or a multi-location compound secret. Then, the divergence to bound for $(\varepsilon, \lambda)$-CIP$(\Spairs, r, \Theta)$ is 
\begin{align*}
	D_\lambda \binom{P_{\calA, \calP}(Z | \Xs = s_i)}{P_{\calA, \calP}(Z | \Xs = s_j)}
\end{align*}
for any 
\begin{align*} 
	(s_i, s_j) \in \Spairs = \{(x_s, x_s'):\|x_s - x_s'\|_2 \leq 2r\}
\end{align*}
if $\Is$ encodes a basic secret, or for any
\begin{align*}
	(s_i, s_j) \in \Spairs = \Big\{\big( \{x_{s1}, x_{s2}, \dots\}, \{x_{s1}', x_{s2}', \dots\}\big): \| x_{sk} - x_{sk}' \|_2 \leq 2r, \forall \ k\Big\} 
\end{align*} 
if $\Is$ encodes a compound secret. A discriminative pair $(s_i,s_j)$ is two real valued vectors $\in \R^{|\Is|}$, representing two hypotheses about the true values of $\Xs$. We denote the $m^\text{th}$ element as ${s_i}_m, {s_j}_m$. Let $f:\Is \rightarrow [|\Is|]$ be a mapping from each index $w \in \Is$ to its corresponding position in the vector $s_i$ or $s_j$ (where the value of $X_w$ is hypothesized). By Lemma \ref{lem: renyi additive loss}, the divergence can be written as  
\begin{align*}
	D_\lambda \binom{P_{\calA, \calP}(Z | \Xs = s_i)}{P_{\calA, \calP}(Z | \Xs = s_j)}
	&= \sum_{w \in \Is} \bigg[ D_\lambda \binom{P_\calA(Z_w | X_w = {s_i}_{f(w)})}{P_\calA(Z_w | X_w = {s_j}_{f(w)})} \bigg]
	+ D_\lambda \binom{P_{\calA, \calP}(\Zu | \Xs = x_s)}{P_{\calA, \calP}(\Zu | \Xs = x_s')} 
\end{align*}
where $P_\calA(Z_w | X_w = x) = \calN(x, \sigma_s^2)$ for all $w \in \Is$. Recall from the statement of the Theorem that we assume the diagonal entries of $\Sigma_{ss}$ all equal some value $\sigma_s^2$: we add the same noise variance to each point in the secret set, which is optimal under MSE constraints. Additionally, note that for the hypothesis $\Xs = x_s$, we know the distribution of $\Xu | \Xs = x_s \sim \calN(\mu_{u|s}, \Sigma_{u|s})$, where $\mu_{u|s} = \mu_u + \Sigma_{us} \Sigma_{ss}^{-1} (x_s - \mu_s)$ and $\Sigma_{u|s} = \Sigma_{uu} - \Sigma_{us}\Sigma_{ss}^{-1} \Sigma_{su}$. Notice that only $\mu_{u|s}$ depends on the actual value of $x_s$, and $\Sigma_{u|s}$ depends only on the indices of $\Is$. Being the sum of two normally distributed variables, we have that $(\Zu | \Xs = x_s) \overset{d}{=} (\Xu|\Xs = x_s) + \Gu = \calN(\mu_{u|s}, \Sigma_{u|s} + \Sigmag_{uu})$. Substituting this into the divergences above sum of divergences: 
\begin{align}
	&D_\lambda \binom{P_{\calA, \calP}(Z | \Xs = s_i)}{P_{\calA, \calP}(Z | \Xs = s_j)}
	= \sum_{m =1}^{|\Is|} \bigg[ D_\lambda \binom{\calN({s_i}_m, \sigma_s^2)}{\calN({s_j}_m, \sigma_s^2)} \bigg]
	+ D_\lambda \binom{\calN(\mu_{u|s_i}, \Sigma_{u|s} + \Sigmag_{uu})}{\calN(\mu_{u|s_j}, \Sigma_{u|s} + \Sigmag_{uu})} \tag{1} \\
	&=  \frac{\lambda}{2} \sum_{m = 1}^{|\Is|}  \frac{1}{\sigma_s^2} ({s_i}_m - {s_j}_m)^2 
	+  \frac{\lambda}{2} (\mu_{u|s_i} - \mu_{u|s_j})^\intercal (\Sigma_{u|s} + \Sigmag_{uu})^{-1} (\mu_{u|s_i} - \mu_{u|s_j})  \tag{2} \\
	&=  \frac{\lambda}{2 \sigma_s^2}   ({s_i} - {s_j})^\intercal({s_i} - {s_j}) 
	+  \frac{\lambda}{2} \big( \Sigma_{us} \Sigma_{ss}^{-1}(s_i - s_j) \big)^\intercal (\Sigma_{u|s} + \Sigmag_{uu})^{-1} \big( \Sigma_{us} \Sigma_{ss}^{-1}(s_i - s_j) \big)  \tag{3}  \\
	&= \frac{\lambda}{2 \sigma_s^2}   ({s_i} - {s_j})^\intercal({s_i} - {s_j}) 
	+  \frac{\lambda}{2} (s_i - s_j)^\intercal \Sigma_{ss}^{-1} \Sigma_{su}  (\Sigma_{u|s} + \Sigmag_{uu})^{-1} \Sigma_{us} \Sigma_{ss}^{-1} (s_i - s_j) \tag{4} 
\end{align}
Line (1) substitutes in the normal distributions given by our mechanism and conditional prior class. Line (2) substitutes in the closed-form expression for R\'enyi divergence between two mean-shifted normal distributions given in Equation \ref{eqn: renyi}. Line (3) substitutes in the expression for $\mu_{u|s}$ given above, and simplifies. To expand out this simplification in explicit steps: 
\begin{align*}
	(\mu_{u|s_i} - \mu_{u|s_j})
	&= \big(  \mu_u + \Sigma_{us} \Sigma_{ss}^{-1} (s_i - \mu_s) -  [\mu_u + \Sigma_{us} \Sigma_{ss}^{-1} (s_j - \mu_s)] \big) \\
	&= \big(  \Sigma_{us} \Sigma_{ss}^{-1} s_i -  \Sigma_{us} \Sigma_{ss}^{-1} s_j \big) \\
	&= \Sigma_{us} \Sigma_{ss}^{-1} (s_i - s_j)
\end{align*}
Line (4) distributes the transpose in the right term of line (3): 
\begin{align*}
	\big( \Sigma_{us} \Sigma_{ss}^{-1}(s_i - s_j) \big)^\intercal
	&= (s_i - s_j)^\intercal \big(  \Sigma_{us} \Sigma_{ss}^{-1} \big)^\intercal \\
	&=  (s_i - s_j)^\intercal  \big( \Sigma_{ss}^{-1} \big)^\intercal \Sigma_{us}^\intercal   \\
	&= (s_i - s_j)^\intercal \Sigma_{ss}^{-1}  \Sigma_{su}
\end{align*}
where that final step is a consequence of $\Sigma$ being symmetric. $\Sigma_{ss}$ is also a symmetric matrix (so its inverse is symmetric) and $\Sigma_{us}^\intercal = \Sigma_{su}$. 

Returning to line (4) above, simplify this expression by substituting $\Delta = s_i - s_j$: 
\begin{align}
	D_\lambda \binom{P_{\calA, \calP}(Z | \Xs = s_i)}{P_{\calA, \calP}(Z | \Xs = s_j)}
	&= \frac{\lambda}{2 \sigma_s^2}   \Delta^\intercal \Delta 
	+  \frac{\lambda}{2} \Delta^\intercal \Sigma_{ss}^{-1} \Sigma_{su}  (\Sigma_{u|s} + \Sigmag_{uu})^{-1} \Sigma_{us} \Sigma_{ss}^{-1} \Delta \tag{5} \\
	&= \frac{\lambda}{2 \sigma_s^2}  \| \Delta \|_2^2 
	+  \frac{\lambda}{2} \Delta^\intercal \Sigmaeff \Delta \tag{6} 
\end{align}
Where $\Sigmaeff = \Sigma_{ss}^{-1} \Sigma_{su}  (\Sigma_{u|s} + \Sigmag_{uu})^{-1} \Sigma_{us} \Sigma_{ss}^{-1}$. The left term of line (6) attributes the direct loss of $\Zs$ on $\Xs$ and the right term attributes the indirect loss of $\Zu$ on $\Xs$. 

We are interested in bounding the expression of line (6) for all $(s_i, s_j) \in \Spairs$. We do this by bounding it for all vectors $\Delta \in \calD$ 
\begin{align*}
	\calD = \{ s_i - s_j : \| s_i - s_j \|_2 \leq  \sqrt{S}\  r \}
\end{align*}  
, where $S$ is the number of basic secrets (locations) contained in $\Is$ which may be a basic or compound secret set. For a basic secret ($S = 1$), this bound is tight, since $\calD = \{s_i - s_j: (s_i, s_j) \in \Spairs\}$. The set of $\Delta \in \calD$ is exactly any two hypothesis $(s_i, s_j)$ that are within any circle of radius $r$. For a compound secret, this bound is not guaranteed to be tight. Recall once again that the set of $\Spairs$ for a compound secret is given by the set of $(s_i, s_j)$ in 
\begin{align*}
	\Spairs = \Big\{\big( \{x_{s1}, x_{s2}, \dots\}, \{x_{s1}', x_{s2}', \dots\}\big): \| x_{sk} - x_{sk}' \|_2 \leq r, \forall \ k\Big\} 
\end{align*} 
For concreteness, consider the 2d location trace example in \textbf{Figure \ref{fig:nyc_example}}, where we have a compound secret of $S = 3$ locations. Here, $s_i, s_j \in \R^{6}$, where 6 comes from the fact that we have three 2d locations. So, $(s_i, s_j)$ represents a pair of hypotheses on all three locations. $s_i$'s hypothesis of the first secret location --- written as ${x_s}_1 \in \R^2$ above --- is within $r$ of the $s_j$'s hypothesis of the first secret location --- written as ${x_s}_1' \in \R^2$ above. The same goes for the second and third locations. So, the $L_2$ norm of $\Delta = s_i - s_j$ is no greater than
\begin{align*}
	\sup_{(s_i, s_j) \in \Spairs} \|s_i - s_j\|_2 
	&=  \sup_{(s_i, s_j) \in \Spairs} \sqrt{\sum_{m=1}^6 ({s_i}_m - {s_j}_m)^2} \\
	&=  \sup_{(s_i, s_j) \in \Spairs} \sqrt{\sum_{k=1}^3 \|{x_s}_k - {x_s}_k'\|_2^2} \\
	&= \sqrt{\sum_{k=1}^3 r^2} \\
	&= \sqrt{3} \ r
\end{align*}
For compound secrets, $\calD$ represents the $L_2$ ball enclosing all $\Delta \in \{s_i - s_j : (s_i, s_j) \in \Spairs \}$. However, $\calD$ also includes some values of $\Delta = s_i - s_j$ not covered by $\Spairs$. Suppose an adversary considers the hypotheses 
\begin{align*}
s_i = \{x_{s1}, x_{s2}, x_{s3}\}, s_j = \{x_{s1}', x_{s2}', x_{s3}'\}
\end{align*} 
where ${x_s}_1 = 0, {x_s}_1' = \sqrt{3} \ r, {x_s}_2 = {x_s}_2', {x_s}_3 = {x_s}_3'$. Since ${x_s}_1, {x_s}_1'$ are not within $r$ of each other, this is not in $\Spairs$. However, it is covered by $\calD$, and thus is covered by our bound on CIP loss and our mechanisms. 

With $\calD$ defined, we may return to bounding the expression in line (6): 
\begin{align}
	D_\lambda \binom{P_{\calA, \calP}(Z | \Xs = s_i)}{P_{\calA, \calP}(Z | \Xs = s_j)}
	&\leq \sup_{\Delta \in \calD} \bigg( \frac{\lambda}{2 \sigma_s^2}  \| \Delta \|_2^2 
	+  \frac{\lambda}{2} \Delta^\intercal \Sigmaeff \Delta \bigg) \tag{7} \\
	&\leq  \frac{\lambda}{2}\bigg( \frac{1}{\sigma_s^2} S r^2 + S r^2 \text{maxeig}(\Sigmaeff) \bigg) \tag{8} \\
	&= \frac{\lambda}{2} S r^2 \big( \frac{1}{\sigma_s^2} + \alpha^* \big) \tag{9}
\end{align}
where line (8) distributes the supremum. For the right term, this is given by the maximum magnitude of all $\Delta \in \calD$ times the maximum eigenvalueof $\Sigmaeff$ which equals $S r^2 \text{maxeig}(\Sigmaeff)$. Line (9) simply substitutes $\alpha^* = \text{maxeig}(\Sigmaeff)$. 

%(explain how you went from $\Sigma_{us}^\intercal$ to $\Sigma_{su}$. Also explain why $(\Sigma_{ss}^{-1})^\intercal = \Sigma_{ss}^{-1}$ (cuz inverse of symmetric matrix is symmetric). Then move to $\Delta s$ notation. Also explain that Lemma 3 isnt needed. Can show this operating on distributions of $Z|s$, $Z|s'$ alone. 
\end{proof}

\subsubsection{Proof of Corollary \ref{cor: composition}}
\textbf{Corollary \ref{cor: composition}}
Graceful Composition in Time
\textit{
	Suppose a user releases two traces $X$ and $\hat{X}$ with additive noise $G \sim \calN(\mathbf{0}, \Sigmag)$ and $\hat{G} \sim \calN(\mathbf{0}, \hat{\Sigma}^{(g)})$, respectively. Then basic or compound secret $\Xs$ of $X$ enjoys $(\bar{\varepsilon}, \lambda)$-CIP, where 
	\begin{align*}
		\bar{\varepsilon} \leq \frac{\lambda}{2} S r^2 \Big(  \frac{1 }{\sigma_s^2} + \bar{\alpha}^*  \Big) 
	\end{align*}
	and where $\bar{\alpha}$ is the maximum eigenvalue of $\bar{\Sigma}_{\text{eff}} = \big(\Sigma_{us} \Sigma_{ss}^{-1}\big)^\intercal \big( \Sigma_{u | s} + \bar{\Sigma}_{uu}^{(g)} \big)^{-1} \big(\Sigma_{us} \Sigma_{ss}^{-1}\big)$. $\Sigma$ is the covariance matrix of the joint distribution on $X, \hat{X}$ and 
	\begin{align*}
	\bar{\Sigma}^{(g)} =
		\begin{bmatrix}
			 \Sigmag & 0 \\
			 0 &  \hat{\Sigma}^{(g)} \ .
		\end{bmatrix}
	\end{align*}
}

\begin{proof}
Here, we record two traces (presumably) far apart in time 
\begin{align*}
	(X_1, \dots, X_n) \text{ and } (\hat{X}_1, \dots, \hat{X}_m)
\end{align*}
And release
\begin{align*}
	(Z_1, \dots, Z_n) = (X_1, + G_1, \dots, X_n + G_n) \text{ and } (\hat{Z}_1, \dots, \hat{Z}_m) = (\hat{X}_1, + \hat{G}_1, \dots, \hat{X}_m, + \hat{G}_m)
\end{align*}
the first trace protects secret locations $\Xs$ and the second protects $\widehat{\Xs}$, so we have that 
\begin{align*}
	D_\lambda \binom{P_{\calA, \calP}(Z | \Xs = s_i)}{P_{\calA, \calP}(Z | \Xs = s_j)} &\leq \varepsilon \\
	D_\lambda \binom{P_{\calA, \calP}(\hat{Z} | \widehat{\Xs} = \hat{s}_i)}{P_{\calA, \calP}(\hat{Z} | \widehat{\Xs} = \hat{s}_j)} &\leq \hat{\varepsilon}
\end{align*}
We aim to update the losses: 
\begin{align*}
	D_\lambda \binom{P_{\calA, \calP}(Z, \hat{Z} | \Xs = s_i)}{P_{\calA, \calP}(Z, \hat{Z} | \Xs = s_j)} &\leq \varepsilon' \\
	D_\lambda \binom{P_{\calA, \calP}(\hat{Z}, Z | \widehat{\Xs} = \hat{s}_i)}{P_{\calA, \calP}(\hat{Z}, Z | \widehat{\Xs} = \hat{s}_j)} &\leq \hat{\varepsilon}'
\end{align*}
Fortunately, our framework is pretty friendly to figuring this out, and can be done simply by updating the `inferential loss term' $\alpha^*$ and $\hat{\alpha}^*$ of each, the max eigenvalues used to compute each of $\varepsilon$ and $\hat{\varepsilon}$, respectively. Let's focus on $\varepsilon'$, since the same analysis follows for $\hat{\varepsilon}'$.  

Recall that $\alpha^*$ is given by the max eigenvalue of $\Sigmaeff$ which is 
\begin{align*}
	\Sigmaeff 
	&= \big(\Sigma_{us} \Sigma_{ss}^{-1}\big)^\intercal \big( \Sigma_{u | s} + \Sigma_{uu}^{(g)} \big)^{-1} \big(\Sigma_{us} \Sigma_{ss}^{-1}\big)
\end{align*}
Where $\Sigma$ is the covariance matrix of $X_1, \dots, X_n$ and $\Sigmag$ is the noise covariance matrix added. Simply augment $\Sigma$ to become the joint covariance matrix $\Sigma_J$ of $X, \hat{X}$, and augment $\Sigmag$ to become 
\begin{align*}
	\Sigmag_J
	&= 
	\begin{bmatrix}
		\Sigmag & 0 \\
		0 & \hat{\Sigma}^{(g)}
	\end{bmatrix}
\end{align*}
then update $\Sigmaeff$ to $\Sigma_{\text{eff}, J}$ which uses both $\Sigma_J$ and $\Sigmag_J$. Using the corresponding max eigenvalue $\alpha^*_J$ in the loss expression of Theorem 3.2 gives us $\varepsilon'$. 

Note that for kernels like RBF, $\varepsilon' \rightarrow \varepsilon$ as the traces $X$ and $\hat{X}$ move apart further and further in time. This is not the case for traces using a purely periodic kernel with not time decay, and we should expect much worse composition. 
\end{proof}


\subsubsection{Traces with Independent Dimensions}
In many cases, the different dimensions of the trace may be probabilistically independent, and it may be more convenient to make separate privacy mechanisms for each. For a 2d trace $X$, suppose $\Ix$ and $\Iy$ store the indices of the latitude points $\Xx$ and longitude points $\Xy$, such that $X = \Xx \cup \Xy$. If latitude and longitude are independent, it may be more convenient to characterize the conditional priors of $\Xx$ abd $\Xy$ separately. The question is whether privacy guarantees remain for the full trace $X$. To answer this, we provide the following corollary: 

\begin{corollary}\emph{CIP loss of independent dimensions} 
\label{cor: independence}
	Let $\Theta$ be a GP conditional prior class on a 2d trace $X$ such that the dimensions are independent. Let $\Is$ be some secret set of time indices corresponding to some basic or compound secret. For the trace $X = \Xx \cup \Xy$, the Gaussian mechanism $\calA(X) = \Zx \cup \Zy$ where $\Zx = \calA_x(\Xx) = \Xx + \Gx$ and $\Zy = \calA_y(\Xy) = \Xy + \Gy$ satisfies $(\varepsilon, \lambda)$-CIP where
	\begin{align*}
		\varepsilon \leq \frac{\lambda}{2} S r^2 \big( \frac{1}{\sigma_s^2} + \alpha^*_x + \alpha^*_y \big) 
	\end{align*} 
	when $\calA_x$ and $\calA_y$ provide $\frac{\lambda}{2} S r^2 \big( \frac{1}{\sigma_s^2} + \alpha^*_x)$ and $\frac{\lambda}{2} S r^2 \big( \frac{1}{\sigma_s^2} + \alpha^*_y)$ to $\Is \cap \Ix$ and $\Is \cap \Iy$, respectively. 
\end{corollary}
The gist of this corollary is that a mechanism can be designed to achieve the bound of Theorem \ref{thm:GP bound} to each dimension independently and released with still-meaningful privacy guarantees. The reason is that this still includes all secret pairs $\Spairs$ 
\begin{proof}
	By independence, $\Xx$ and $\Xy$ can be treated as two unconnected traces of the type seen in \textbf{Figure \ref{fig:graphical models}}. As such the privacy guarantee of Theorem \ref{thm:GP bound} can be upheld for each. The question is whether bounding CIP loss to the one-dimensional basic or compound secret associated with secret sets $\Is \cap \Ix$ and $\Is \cap \Iy$ still provides guarantees for the full secret set $\Is$. 
	
	Without loss of generality, we will demonstrate for a basic and a compound secret. Consider the basic secret set $\Is = \{X_{10}, X_{11}\}$, where $\Is \cap \Ix = \{X_{10}\}$ (latitude) and $\Is \cap \Iy = \{X_{11}\}$ (longitude). We again assume that independent gaussian noise of variance $\sigma_s^2$ is added to all $\Xs$, since this is optimal under utility constraints. We have now bounded the R\'enyi divergence when conditioning on pairs of hypotheses on latitude and longitude separately. 
	\begin{align*} 
	{\Spairs}_x = {\Spairs}_y = \{(x_s, x_s'):x_s \in \R,  \|x_s - x_s'\|_2 \leq r\}
	\end{align*}
	By independence, this also bounds the R\'enyi divergence conditioning on pairs of hypotheses on latitude and longitude jointly: 
	\begin{align*} 
	{\Spairs}_{xy} = \{(x_s, x_s'):x_s \in \R^2,  \|x_s - x_s'\|_2 \leq r\}
	\end{align*}
	In effect, we have guaranteed privacy for any pair of hypotheses $(s_i, s_j)$ in the square circumscribing the circle of radius $r$ that we with to provide. The analysis on the direct privacy loss is exactly the same as it was in the more general case. Since the R\'enyi divergences of $\Xu \cap \Xx$ and of $\Xu \cap \Xy$ add, the $\alpha^*$'s add. 
	
	The same goes for a compound secret. Consider three location compound secret pairs given by 
	\begin{align*}
		{\Spairs}_{xy} = \Big\{\big( \{x_{s1}, x_{s2}, \dots\}, \{x_{s1}', x_{s2}', \dots\}\big): x_{si} \in \R^2, \| x_{sk} - x_{sk}' \|_2 \leq r, \forall \ k\Big\} 
	\end{align*} 
	Instead, we bound privacy loss for 
	\begin{align*}
		{\Spairs}_x = {\Spairs}_y = \Big\{\big( \{x_{s1}, x_{s2}, \dots\}, \{x_{s1}', x_{s2}', \dots\} \big): x_{si} \in \R, \| x_{sk} - x_{sk}' \|_2 \leq r, \forall \ k \Big\}
	\end{align*}
	Separately, giving us $\alpha_x^*$ and $\alpha_y^*$. This again includes any two hypotheses on the three locations such that each pair of $x_{sk}, x_{sk}'$ is within a square circumscribing a circle of radius $r$. We achieve this by bounding privacy loss for all $\Delta_x$ in a 3d $L_2$ ball of radius $\sqrt{S}  \ r$, as with $\Delta_y$. 
	
	This corollary can be extended to all traces of all dimensions that are probabilistically independent. 
\end{proof}

We make use of the above proof in the Experiments section. 

\subsection{Derivation of Algorithms}
\label{apx: algorithmns}
In this section, we derive the three SDP-based algorithms of Section \ref{sec: algorithms} and their properties. 

\subsubsection{Derivation of $\text{SDP}_\text{A}$}

$\text{SDP}_\text{A}$ minimizes the privacy loss bound of Theorem \ref{thm:GP bound} for any compound or basic secret encoded by secret set $\Is$. As is clarified in its proof (Appendix \ref{apx: GP bound proof}), the bound is tight when $\Is$ encodes a basic secret. If $\Is$ encodes a compound secret, the tightness depends on the conditional prior class $\Theta$. 

Our variable for minimizing this bound is the noise covariance matrix $\Sigmag$. Due to the conditional independence exhibited by Lemma \ref{lem: renyi additive loss}, $\Gs$ and $\Gu$ may be independent. The additive noise $G_i \in \Gs$ are all independent Gaussian with variance $\sigma_s^2$. This is because --- conditioning on $\{\Xs = x_s\}$ --- $\Zs$ is independent of $\Xu$ and $\Zu$. So, $\Gs \sim \calN(\mathbf{0}, \sigma_s^2 I)$, and $\Sigmag_{ss} = \sigma_s^2 I$. The additive noise $G_i \in \Gu$ are all dependent as described by $\Sigmag_{uu}$, and $\Gu \sim \calN(\mathbf{0}, \Sigmag_{uu})$. Consequently, $\Sigmag$ is completely characterized by $\Sigmag_{uu}$ and $\sigma_s^2$. 

To see how the bound of Theorem \ref{thm:GP bound} can be redrafted as an SDP, first notice that its two terms may be written as the maximum eigenvalue of a matrix product. Here, $\Sigmaeff = A^\intercal B A$, where $A = \Sigma_{us} \Sigma_{ss}^{-1}$ and $B = \big( \Sigma_{u | s} + \Sigmag_{uu} \big)^{-1}$
\begin{align*}
	\frac{1}{\sigma_s^2} + \alpha^*
	= \text{maxeig} \big( 
	\frac{1}{\sigma_s^2} I + A^\intercal B A \big)
	= \text{maxeig} \bigg(  
	\begin{bmatrix}
		I \  A
	\end{bmatrix} 
	\begin{bmatrix}
		\frac{1}{\sigma_s^2} I \ \ \  0 \\
		\quad 0 \quad  B
	\end{bmatrix}
	\begin{bmatrix}
		I \\ A
	\end{bmatrix}
	\bigg) 
	= \text{maxeig} \big( \tilde{A}^\intercal \tilde{B} \tilde{A} \big) 
\end{align*}
This expression uses all parameters of $\Sigmag$: $\sigma_s^2$ parametrizes $\Sigmag_{ss}$ and $\Sigmag_{uu} = B^{-1} - \Sigma_{u|s}$, where $\Sigma_{u|s}$ is given by the kernel function of $\Theta$. 

Before casting this as an SDP, we provide a formal definition from \cite{SDPs}: 

\begin{definition}\emph{Semidefinite Program} 
	\label{def: SDP}
	The problem of minimizing a linear function of a variable $x \in \R^n$ subject to a matrix inequality: 
	\begin{align*}
		\min_{x \in \R^n} \ &c^\intercal x \\
		&\text{s.t. } F_0 + \sum_{i=1}^n x_i F_i \succeq 0 \\
		& \quad \ \  Ax = b
	\end{align*}
	where the $F_i \in \R^{n \times n}$ are all symmetric and $A \in \R^{p \times n}$ is a \emph{semidefinite program}, or SDP. 
\end{definition}

The task of minimizing $\text{maxeig} \big( \tilde{A}^\intercal \tilde{B} \tilde{A} \big)$ under MSE constraints can almost be formulated as an SDP: 
\begin{align*}
	\min_{B \succeq 0 , \nicefrac{1}{\sigma_s^2} \geq 0} \ &\beta^* \\
	&\text{s.t. } \beta^* I  \succeq \tilde{A}^\intercal \tilde{B} \tilde{A} \\
	& \quad \ \ B \preceq \Sigma_{u|s}^{-1} \\
	&\quad \ \ \trace(\Sigmag_{uu}) + |\Is| \sigma_s^2 \leq n o_t 
\end{align*}
Here, the first constraint guarantees that the maximum eigenvalue of $\tilde{A}^\intercal \tilde{B} \tilde{A}$ is bounded by $\beta^*$, which the objective minimizes. At program completion, we set $\Sigmag_{uu} = B^{-1} - \Sigma_{u|s}$, and the second constraints ensures that this is still PSD. The final constraint bounds the MSE of the mechanism $\Sigmag$. Note that $\trace(\Sigmag_{uu}) + |\Is| \sigma_s^2 = \trace(\Sigmag)$. The trouble lies the last constraint. Our program variable is $B$, but the final linear constraint requires $\Sigmag$, which is expressed using the inverse of $B$. This is not immediately available in the SDP framework. 

To make the final linear constraint available, we invert the above program using the observation that the maximum eigenvalue of $\tilde{A}^\intercal \tilde{B} \tilde{A}$ is the inverse of the minimum eigenvalue of $(\tilde{A}^\intercal \tilde{B} \tilde{A})^{-1}$. Instead of optimizing over $B$ and $\nicefrac{1}{\sigma_s^2}$, we optimize over $B^{-1}$ and $\sigma_s^2$. Since $B^{-1} = \Sigma_{u|s} + \Sigmag_{uu}$, we may now have a utility constraint directly on the trace of $\Sigmag$. To make $B^{-1}$ our program variable, we approximate $(\tilde{A}^\intercal \tilde{B} \tilde{A})^{-1}$ with $\tilde{A}^{-1} \tilde{B}^{-1} \tilde{A}^{-\intercal}$. First note that $\tilde{A} \in \R^{n \times |\Is|}$, and has full column rank for the covariances we work with. So, $\tilde{A}^{-1} = (\tilde{A}^\intercal \tilde{A})^{-1}\tilde{A}^\intercal \in \R^{(|\Is| \times n)}$ is the left inverse of $\tilde{A}$ and is the least squares solution to $\tilde{A}^{-1} \tilde{A} = \tilde{A}^\intercal \tilde{A}^{-\intercal}  = I$ (we denote its transpose as $\tilde{A}^{-\intercal}$). It is also the least squares solution to $\tilde{A} \tilde{A}^{-1} = \tilde{A}^{-\intercal} \tilde{A}^\intercal = I$. Thus, we have an approximation of the inverse $(\tilde{A}^\intercal \tilde{B} \tilde{A})^{-1}$: 
\begin{align*}
	(\tilde{A}^\intercal \tilde{B} \tilde{A}) \ (\tilde{A}^{-1} \tilde{B}^{-1} \tilde{A}^{-\intercal})
	&\approx \tilde{A}^\intercal \tilde{B} \tilde{B}^{-1} \tilde{A}^{-\intercal} \\
	&= \tilde{A}^\intercal \tilde{A}^{-\intercal} \\
	&\approx I
\end{align*}

We now can optimize in terms of $B^{-1}$ with the augmented matrix $\tilde{B}^{-1}$: 
\begin{align*}
	\tilde{B}^{-1} = 
	\begin{bmatrix}
		\sigma_s^2 I \ \ \  0 \\
		\quad 0 \quad  B^{-1}
	\end{bmatrix}
\end{align*}

We then optimize the following SDP: 

\begin{align*}
	\max_{B^{-1} \succeq 0 , \sigma_s^2 \geq 0} \ &\beta^* \\
	&\text{s.t. } \beta^* I  \preceq \tilde{A}^{-1} \tilde{B}^{-1} \tilde{A}^{-\intercal} \\
	& \quad \ \ B^{-1} \succeq \Sigma_{u|s} \\
	&\quad \ \ \trace(\tilde{B}) -  \trace{(\Sigma_{u|s})} \leq n o_t 
\end{align*}
Upon program completion we recover $\sigma_s^2$ and $\Sigmag_{uu} = B^{-1} - \Sigma_{u|s}$ which we know is PSD due to the second constraint. The first constraint guarantees that the minimum eigenvalue of the approximated inverse is $\geq \beta^*$, which the objective maximizes. If the minimum eigenvalue of the approximate inverse is close to that of the true inverse, then we successfully minimize the maximum eigenvalue of $\tilde{A}^\intercal \tilde{B} \tilde{A}$, and thus minimize the direct and indirect privacy loss. The third constraint limits the MSE of $\Sigmag$ since $\trace(\tilde{B}) - \trace(\Sigma_{u|s}) = (\trace(\Sigmag_{uu}) + |\Is| \sigma_s^2 + \trace(\Sigma_{u|s})) - \trace(\Sigma_{u|s}) = \trace(\Sigmag)$. By inverting $\tilde{A}^\intercal \tilde{B} \tilde{A}$, this constraint is available in the SDP framework. 

By expressing the above program in terms of the variable $\Sigmag$ instead of indirectly via $B^{-1}$ and $\sigma_s^2$, we get $\text{SDP}_\text{A}$: 

\begin{align*}
	\textbf{SDP}_\textbf{A}: \quad 
	\argmax_{\Sigmag \succeq 0}& \ \beta^* \\
	\text{s.t. }& \tilde{A}^{-1} \tilde{B}^{-1} \tilde{A}^{-\intercal} \succeq \beta^* \mathbf{I} \\
	&\trace(\Sigmag) \leq n o_t
\end{align*}
It is straightforward to write this SDP in the form seem in Definition \ref{def: SDP}. The program variables $x$ would be the diagonal and upper or lower triangular part of $\Sigmag$ along with $\beta^*$. With some linear algebra, the first constraint can be written in the form of $F_0 + \sum_{i=1}^n x_i F_i \succeq 0$, and the second constraint can be written as $Ax = b$. With the use of contemporary convex programming tools like CVXOPT \citep{cvxopt} rewriting into this form is unnecessary. 

%With the derivation of the above program, the proof of Theorem \ref{thm: SDP optimal} is clear. 
%
%\textbf{Theorem \ref{thm: SDP optimal}} SIG OPT versus isotropic:
%\emph{
%For a basic or compound secret denoted by indices $\Is$, the CIP loss bound of Equation \ref{eqn: priv bound} provided by a Gaussian noise mechanism with covariance \\$\Sigmag =$ SIG OPT$(\Is, \Sigma, o_t)$ is less than or equal to that of an isotropic mechanism of equal MSE $\Sigmag = o_t I$ if the minimum eigenvalue of $\tilde{A}^{-1} \tilde{B}^{-1} \tilde{A}^{-\intercal}$ equals that of $ (\tilde{A}^\intercal \tilde{B} \tilde{A})^{-1}$ for all $\tilde{B}$.
%}
%\begin{proof}
%	The proof is nearly by construction. If the minimum eigenvalue of $\tilde{A}^{-1} \tilde{B}^{-1} \tilde{A}^{-\intercal}$ equals that of $ (\tilde{A}^\intercal \tilde{B} \tilde{A})^{-1}$ for all $\tilde{B}$ then so do the maximum eigenvalues of their inverses. So, the $\tilde{B}$ that maximizes the minimum eigenvalue of our approximation $\tilde{A}^{-1} \tilde{B}^{-1} \tilde{A}^{-\intercal}$ also minimizes the maximum eigenvalue of $\tilde{A}^\intercal \tilde{B} \tilde{A}$ which equals the the privacy loss bound $\frac{1}{\sigma_s^2} + \alpha^*$ (constants $2 \lambda S r^2$ aside).
%	
%	Since the isotropic mechanism $\Sigmag = o_t I$ is in the feasible set of solutions, we are guaranteed that the covariance chosen by SIG OPT produces a smaller lower bound on CIP loss. 
%\end{proof}
%
%The intuition of the theorem is that if $\tilde{A}^{-1} \tilde{B}^{-1} \tilde{A}^{-\intercal}$ is a good approximation of $(\tilde{A}^\intercal \tilde{B} \tilde{A})^{-1}$, then the SDP is optimal. To show how `good' the approximation must be, consider the following. Let $f(\Sigmag) = \text{mineig}\big((\tilde{A}^\intercal \tilde{B} \tilde{A})^{-1}\big)$. Let our approximation to $f$ be $\hat{f}(\Sigmag) = \text{mineig} \big( \tilde{A}^{-1} \tilde{B}^{-1} \tilde{A}^{-\intercal} \big)$. Let the true optimal noise covariance be $\Sigmag_{\text{opt}} = \argmax_{\Sigmag \in \mathcal{T}} f(\Sigmag)$, where $\mathcal{T}$ is the set of all covariance matrices with MSE bounded by $n o_t$. Then, if for all $\Sigmag \in \mathcal{T}$ 
%\begin{align*}
%	|f(\Sigmag) - \hat{f}(\Sigmag)| \leq \delta 
%	\quad \quad \text{where} \quad \quad 
%	\delta = |f(\Sigmag_{\text{opt}}) - f(o_t I)|
%\end{align*}
%SIG OPT will return a covariance matrix that reduces the bound of Equation \ref{eqn: priv bound} better than an isotropic mechanism.

\subsubsection{Derivation of $\text{SDP}_\text{B}$ }
\label{apx: SDP B}
$\text{SDP}_\text{B}$ takes a set of covariance matrices $\calF = \{\Sigma_1, \dots, \Sigma_k\}$, each of which is designed to protect some secret set ${\Is}_i$, and returns a covariance matrix $\Sigmag$ that preserves the privacy loss bound of each $\Sigma_i$ to each ${\Is}_i$. It does so while minimizing the utility loss of $\Sigmag$. This algorithm is also expressed as an SDP. It is based on the following corollary, which we have omitted from the main text: 
\begin{corollary}\emph{More PSD, More Private: }
\label{cor: more_psd}
	For a basic or compound secret denoted by indices $\Is$, the CIP loss bound of Equation \ref{eqn: priv bound} provided by a Gaussian noise mechanism with covariance $\Sigmag$ is lower than it would be for any ${\Sigmag}' \prec \Sigmag$. 
\end{corollary}
\begin{proof}
	First note that if $\Sigmag \succ {\Sigmag}' $, then the same is true for its sub-matrices: 
	\begin{align*}
		\Sigmag_{ss} \succ {\Sigmag_{ss}}'
		\quad \quad
		\Sigmag_{uu} \succ {\Sigmag_{uu}}'
	\end{align*}
	Recall the privacy loss bound of Equation \ref{eqn: priv bound}: 
	\begin{align*}
		\varepsilon \leq \frac{\lambda}{2} S r^2 \Big(  \frac{1 }{\sigma_s^2} + \alpha^*  \Big)
	\end{align*}
	Also recall that $\Sigmag_{ss} = \sigma_s^2 I$ and ${\Sigmag_{ss}}' = {\sigma_s^2}' I$. Since $\Sigmag_{ss} \succ {\Sigmag_{ss}}'$, we already know that $\sigma_s^2 > {\sigma_s^2}'$, and thus the first term of Equation \ref{eqn: priv bound} is lower for $\Sigmag$.
	
	It remains to show that the second term is also lower, $\alpha^* < {\alpha^*}'$. Starting with what we're given, 
	\begin{align*}
		\Sigmag_{uu} &\succ {\Sigmag_{uu}}' \\
		\Sigmag_{uu} + \Sigma_{u|s} &\succ {\Sigmag_{uu}}' + \Sigma_{u|s} \\
		(\Sigmag_{uu} + \Sigma_{u|s})^{-1} &\prec ({\Sigmag_{uu}}' + \Sigma_{u|s})^{-1} \\
		B &\prec B' \\
		A^\intercal B A &\prec A^\intercal B' A \\
		\maxeig(A^\intercal B A) &< \maxeig(A^\intercal B' A) \\
		\alpha^* &< {\alpha^*}'
	\end{align*}
	Therefore $\frac{1}{\sigma_s^2} + \alpha^* < \frac{1}{{\sigma_s^2}'} + {\alpha^*}'$, and the CIP bound of Equation \ref{eqn: priv bound} is lower for $\Sigmag$ than it is for ${\Sigmag}'$. 
\end{proof}
With Corollary \ref{cor: more_psd} in mind, $\text{SDP}_\text{B}$ is natural: 

\begin{align*}
	\textbf{SDP}_\textbf{B}: \quad 
	\argmin_{\Sigmag } \  &\trace(\Sigmag) \\
	\text{s.t. }& \Sigmag \succeq \Sigmag_i , \ \forall \Sigmag_i \in \calF
\end{align*}

$\text{SDP}_\text{B}$ attempts to minimize, but does not constrain, the utility loss of the chosen $\Sigmag$. To provide an upper bound on the resulting utility loss, we provided the following claim in the main text: 

\textbf{Claim} Utility loss of $\text{SDP}_\text{B}$: 
\emph{
	The utility loss of $\Sigmag = \text{SDP}_\text{B}(\calF)$ is no greater than $\sum_{\Sigma_i \in \calF} \trace(\Sigma_i)$. 
}
\begin{proof}
	The covariance ${\Sigmag}' = \sum_{\Sigmag_i \in \calF} \Sigmag_i$ with MSE $\sum_{\Sigmag_i \in \calF} \trace(\Sigmag_i)$ is in the feasible set of $\text{SDP}_\text{B}$ problem since ${\Sigmag}' \succeq \Sigmag_i, \ \forall \Sigmag_i \in \calF$. Unless ${\Sigmag}'$ has the lowest MSE of all $\Sigmag$ in the feasible set, a covariance matrix with better utility will be chosen. 
\end{proof}

\subsubsection{Derivation of Algorithm \ref{alg: Multiple Secrets}, Multiple Secrets}

Multiple Secrets combines $\text{SDP}_\text{A}$ and $\text{SDP}_\text{B}$ to minimize the privacy loss to each basic secret within a trace. The basic mechanism is useful in cases when inferences at each time within the trace --- each basic secret --- is sensitive. 

Let ${\Is}_i$ be the secret set representing basic secret $i$, of which there are $N$ (e.g. if location is sampled at $N$ times). Then $\mathbb{I}_{\calS_b} = \{{\Is}_1, \dots, {\Is}_N\}$ contains the indices corresponding to each. Multiple Secrets works by first producing $N$ covariance matrices, $\Sigmag_i$ = $\text{SDP}_\text{A}$$({\Is}_i, \Sigma, o_t)$ on each basic secret. It then uses $\text{SDP}_\text{B}$($\calF = \{\Sigmag_1, \dots, \Sigmag_N\}$) to produce a single covariance matrix $\Sigmag$ that preserves the privacy loss to each basic secret (note that, being basic secrets, the privacy loss bound that SIG OPT optimizes is tight). 

By virtue of using $\text{SDP}_\text{B}$, the MSE of the resultant $\Sigmag$ is minimized but not constrained. To bound the MSE of the Basic Mechanism by $O$, we may simply bound the MSE of each $\Sigmag_i$ by $o_t = \nicefrac{O}{N}$. Then, by the above Claim, the MSE of the solution cannot be greater than $O$. In practice, this bound may be too loose. We hope to tighten it in future work. 

\subsection{Experimental details}
\label{apx: experiments}

We use a 2d location trace and a 1d home temperature dataset. For the location data, having observed that the correlation between latitude and longitude is low ($ \approx 0.06$) we treat each dimension as independent. By way of Corollary \ref{cor: independence}, this allows us to bound privacy loss and design mechanisms for each dimension separately. Furthermore, having observed that each dimension fits the nearly the same conditional prior, we treat our dataset of 10k 2-dimensional traces as a dataset of 20k 1-dimensional traces, where each trace represents one dimension of a 2d location trajectory. 

The one-dimensional traces of temperature and location are indexed by timestamps, for which we would use the following kernel functions: 

\begin{align}
	k_{\text{RBF}}(t_i, t_j) 
	=  \sigma_x^2 \exp \Big( -\frac{(t_i - t_j)^2}{2 l^2} \Big) 
	\quad \quad 
	k_{\text{PER}}(t_i, t_j) 
	=  \sigma_x^2 \exp \Big(  \frac{-2 \sin^2(\pi |t_i - t_j| / p)}{l^2} \Big)
\end{align}

to determine the covariance between two points sampled at times $t_i$ and $t_j$. The parameters including variance $\sigma_x^2$ and length scale $l$. The lengthscale determines the window of time in which two sampled points are highly correlated. 

\paragraph{Preprocessing of location data} We first limit the dataset to traces of under 50 locations that are between 4.5 and 5.5 minutes in duration. Caring only about the conditional dependence between locations, we then de-mean each trace and normalize its variance to one. Normalizing the variance of traces implicitly sets $\sigma_x^2 = 1$ in the above RBF kernel, in essence assuming that the adversary has a decent prior for the user's average speed in a given trace, and could do the same operation. 

\paragraph{Fitting of location data} We then find the maximum likelihood RBF kernel for each distinct trace. Having fixed the variance $\sigma_x^2$, this amounts to fitting only the length scale for each dimension, $l_x$ and $l_y$, individually. The length scale represents the average window of time during which neighboring locations are highly correlated (i.e. correlation $ > 0.8$). Relatively smooth traces will have large length scales and chaotic traces will have low length scales. However, the fact that sampling rates vary significantly between traces means that traces with equal length scales can have very different degrees of correlation. To encapsulate both of these effects, we study the empirical distribution of \emph{effective} length scale of each trace
\begin{align*}
	l_{\text{eff},x} = \frac{l_x}{P}
	\quad
	l_{\text{eff},y} = \frac{l_y}{P}
\end{align*}
where $P$ is the trace's sampling period and $l_x,l_y$ are the its optimal length scales. $l_{\text{eff},x},l_{\text{eff},y}$ tell us the average number of neighboring locations that are highly correlated, instead of time period. For instance, a given trace with an optimal $l_{\text{eff},x} = 8$ tells us that every eight neighboring location samples in the $x$ dimension have correlation $> 0.8$. The empirical distribution of effective length scales across all traces describes -- over a range of logging devices (sampling rates), users, and movement patterns -- how many neighboring points are highly correlated in location trace data. After this preprocessing, we are able to use the kernels that take indices (not time) as arguments. 

\begin{align*}
	\label{eqn: kernels}
	k_{\text{RBF}}(i, j) 
	=  \exp \Big( -\frac{(i - j)^2}{2\leff^2} \Big) 
	\quad \quad 
	k_{\text{PER}}(i, j) 
	=  \exp \Big(  \frac{-2 \sin^2(\pi |i - j| / p)}{\leff^2} \Big)
\end{align*}

In each plot we then observed a spectrum of conditional priors by sweeping the effective length scale and plotting posterior uncertainty for various noise mechanisms of equal utility loss. This ranges from a prior assuming nearly independent location samples (chaotic trace) on the left up to highly dependent location samples (traveling in a straight line or standing still) on the right. To understand how realistic these conditional prior parameters are, we displayed the middle 50\% of the empirical distribution of $l_{\text{eff}}$ ($x$ and $y$ together) from the GeoLife dataset. Note that the distribution of ${\leff}_x$ and ${\leff}_y$ are nearly identical. 

To compute posterior uncertainty, we consider a 50-point one-dimensional location trace. The basic secret is a single index in the middle of the trace, and the compound secret consists of two neighboring indices also in the middle of trace. For each value of $l_{\text{eff}}$, we compute the $\R^{50 \times 50}$ conditional prior covariance matrix $\Sigma$ using the RBF kernel above. We then compare the posterior uncertainty when $\Sigmag$ is an Approach C baseline, or an optimized covariance matrix using one of the three algorithms. We re-optimize $\Sigmag$ for each $\leff$, since each $\leff$ represents a different conditional prior class. The MSE is fixed in all figures except the two exhibiting ``All Basic Secrets'', where $\text{SDP}_\text{B}$ is used. Recall that this algorithm minimizes utility loss while maintaining a series of privacy guarantees. Here, the MSE is identical across mechanisms for each $\leff$, but changes from one $\leff$ to another. 

For the temperature data, our preprocessing steps were nearly identical, except we use the periodic kernel instead of the RBF kernel, and we did not need to remove any traces from the dataset, as the data was much cleaner. 

\paragraph{Computation of Posterior Uncertainty Interval}
Each of the plots in \textbf{Figure \ref{fig: experiments}} shows the $2\sigma$ uncertainty interval on $\Xs$ of a Gaussian process Bayesian adversary with prior covariance $\Sigma$ and any mean function

The posterior covariance is computed using standard formulas for linear Gaussian systems. Knowing that $Z = X + G$, we may write the joint precision matrix $\Lambda$ (inverse of covariance matrix) of $(X,Z)$ as 
\begin{align*}
	\Lambda^{(X,Z)}
	&= 
	\begin{bmatrix}
		\Sigma^{-1} + {\Sigmag}^{-1} & -{\Sigmag}^{-1} \\
		-{\Sigmag}^{-1} & {\Sigmag}^{-1} 
	\end{bmatrix}
\end{align*}

It is then a well known result that the conditional covariance matrix is given by 
\begin{align*}
	\Sigma_{x|z} &= \Lambda_{xx}^{-1}  \\
	&= \big(\Sigma^{-1} + {\Sigmag}^{-1}\big)^{-1}
\end{align*}
This provides the posterior covariance of all locations $X$ given any released trace $Z$ that uses a Gaussian mechanism with covariance $\Sigmag$. Note that the CIP guarantee naturally keeps posterior uncertainty large since the posterior density at any two $x_s$ close together must be similar. For these Gaussian posteriors, $2 \sigma$ tells us the adversary's 68\% confidence interval on $\Xs$ after obvserving $Z$. 

For basic secrets (one location), we simply report twice the posterior standard deviation at the sensitive index $i$, given by 
\begin{align*}
2 \sqrt{ \Sigma_{{x|z, ii}} } \ .
\end{align*}  
For compound secrets involving multiple locations the posterior distribution is a length $|\Is|$ multivariate normal with covariance $\Sigma_{x|z, ss}$. Intuitively, we wish to find the direction of the vector $\Xs$ in which the posterior interval is the \emph{shortest}. This is the worst case posterior interval on the compound secret. We do this by reporting 
\begin{align*}
	2 \sqrt{\text{mineig}\  \Sigma_{{x|z, ss}}} \ .
\end{align*}

\subsection{Discussion of GP Conditional Prior Class}
\label{apx: GP prior class}

Recall that a conditional prior class requires for any $P_{\calP_i}, P_{\calP_j} \in \Theta$ that 
\begin{align*}
	P_{\calP_i}(\Xu | \Xs = x_s)
	&= P_{\calP_j}(\Xu + c_{ij\Is}^u | \Xs = x_s + c_{ij\Is}^s)
\end{align*}
for all $x_s$. Notice that the mapping $(x_s, x_s') + c_{ij\Is}^s$ is a bijection from $\Spairs$ onto itself. As such, each pair of conditional distributions, 
\begin{align*}
	\Big(P_{\calP_j}(\Xu | \Xs = x_s), P_{\calP_j}(\Xu | \Xs = x_s')\Big)
\end{align*}
induced by $(x_s, x_s') \in \Spairs$ is a mean-shifted version of the pair of distributions 
\begin{align*}
	\Big(P_{\calP_i}(\Xu | \Xs = x_s - c_{ij\Is}^s), P_{\calP_i}(\Xu | \Xs = x_s' - c_{ij\Is}^s)\Big)
\end{align*}
induced by $(x_s , x_s' ) - c_{ij\Is}^s \in \Spairs$. Since the R\'enyi divergence between two distributions and two mean-shifted versions thereof is unchanged, we may use one additive noise mechanism for all priors in class $\Theta$.  

To see how this applies to the GP prior class, recall the formula for a conditional multivariate Gaussian distribution: 
\begin{align*}
	P(\Xu | \Xs = x_s)
	&= \calN(\mu_{u|s} , \Sigma_{u|s})
\end{align*}
where, 
\begin{align*}
		\mu_{u|s} &= \mu_u + \Sigma_{us} \Sigma_{ss}^{-1} (x_s - \mu_s) \\
		\Sigma_{u|s} &= \Sigma_{uu} - \Sigma_{us}\Sigma_{ss}^{-1} \Sigma_{su}
\end{align*}
A GP prior class includes all GP distributions with a fixed kernel $k(t_i, t_j)$ and any mean function $\mu(t)$. For a fixed set of time points, this corresponds to a fixed covariance matrix $\Sigma$ and any mean parameters $\bmu$: 
\begin{align*}
	X \sim \calN(\bmu, \Sigma)
\end{align*}

Let $P_{\calP_i} = \calN(\bar{\bmu}, \Sigma)$ and $P_{\calP_j} = \calN(\hat{\bmu}, \Sigma)$, then conditioned on some sensitive points $\Xs$ the distribution on $\Xu$ has the same covariance $\Sigma_{u|s}$ and conditional means 
\begin{align*}
	\bar{\mu}_{u|s}
	&= \bar{\mu}_u + \Sigma_{us} \Sigma_{ss}^{-1} (x_s - \bar{\mu}_s) \\
	&= (\bar{\mu}_u - \Sigma_{us} \Sigma_{ss}^{-1}\bar{\mu}_s) + \Sigma_{us} \Sigma_{ss}^{-1} x_s \\
	\hat{\mu}_{u|s}
	&= \hat{\mu}_u + \Sigma_{us} \Sigma_{ss}^{-1} (x_s - \hat{\mu}_s) \\
	&= (\hat{\mu}_u - \Sigma_{us} \Sigma_{ss}^{-1}\hat{\mu}_s) + \Sigma_{us} \Sigma_{ss}^{-1} x_s 
\end{align*}
which implies that the conditional distributions are identical up to a mean shift for the \emph{same} $x_s$ value. 
\begin{align*}
	P_{\calP_i}(\Xu | \Xs = x_s)
	&= P_{\calP_j}(\Xu + c_{ij\Is}^u | \Xs = x_s)
\end{align*}
for all $x_s$. Here, $c_{ij\Is}^u = (\bar{\mu}_u - \Sigma_{us} \Sigma_{ss}^{-1}\bar{\mu}_s) - (\hat{\mu}_u - \Sigma_{us} \Sigma_{ss}^{-1}\hat{\mu}_s)$, and $c_{ij\Is}^s = 0$. 

To see how this allows a single additive mechanism to work for all mean functions, notice that we also have 
\begin{align*}
	P_{\calP_i}(\Xu | \Xs = x_s')
	&= P_{\calP_j}(\Xu + c_{ij\Is}^u | \Xs = x_s')
\end{align*}
for $x_s'$, so the divergences 
\begin{align*}
	D_\lambda \binom{P_{\calP_i}(\Xu | \Xs = x_s)}{P_{\calP_i}(\Xu | \Xs = x_s')}
	&= D_\lambda \binom{P_{\calP_j}(\Xu + c_{ij\Is}^u | \Xs = x_s)}{P_{\calP_j}(\Xu + c_{ij\Is}^u | \Xs = x_s')} \\
	&= D_\lambda \binom{P_{\calP_j}(\Xu  | \Xs = x_s)}{P_{\calP_j}(\Xu  | \Xs = x_s')}
\end{align*}
are equal. The same goes for the noisy trace $\Xu + \Zu | \Xs = x_s$, when $Z$ is drawn independently of $X$, allowing us to bound privacy loss for all $P \in \Theta$. 









% Stuff at the end of the dissertation goes in the back matter
\backmatter
\bibliographystyle{plain} % Or whatever style you want like plainnat
\bibliography{references}

\end{document}
